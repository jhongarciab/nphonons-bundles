\begin{appendix}
\chapter{Derivación de las tasas de transición de n-fonones}\label{AnexoA}

Consideramos dos puntos cuánticos idénticos, QD$_1$ y QD$_2$, cada uno
modelado como un sistema de dos niveles con estados electrónicos $|v\rangle$
(valencia) y $|c\rangle$ (conducción), frecuencia de transición
$\omega_\sigma$ y acoplamiento electrón-fonón $\lambda$. Ambos QDs
interactúan con un único modo fonónico de cavidad de frecuencia $\omega_b$,
con operadores de escalera $b$ y $b^\dagger$. Los QDs están acoplados entre
sí por la interacción de Förster con constante $J$. Un láser coherente de
frecuencia $\omega_L$ conduce ambos QDs con la misma frecuencia de Rabi
$\Omega$. Con $\hbar = 1$ y el operador de número electrónico total
$\hat{N} \equiv \sigma_1^\dagger\sigma_1 + \sigma_2^\dagger\sigma_2$,
donde $\sigma_i = |v\rangle_i\!\langle c|$ y
$\sigma_i^\dagger = |c\rangle_i\!\langle v|$, el Hamiltoniano total es
%
\begin{equation}
	H_\mathrm{lab}
	= \omega_b b^\dagger b
	+ \omega_\sigma \hat{N}
	+ \lambda\,\hat{N}(b^\dagger + b)
	+ J\!\left(\sigma_1^\dagger\sigma_2 + \sigma_2^\dagger\sigma_1\right)
	+ \Omega\sum_{i=1,2}\!\left(e^{i\omega_L t}\sigma_i
	+ e^{-i\omega_L t}\sigma_i^\dagger\right).
	\label{eq:Hlab}
\end{equation}
%
En el marco rotante con el láser, con desintonía
$\Delta \equiv \omega_\sigma - \omega_L$,
el Hamiltoniano~\eqref{eq:Hlab} se convierte en
%
\begin{equation}
	H = \omega_b b^\dagger b
	+ \Delta\,\hat{N}
	+ \lambda\,\hat{N}(b^\dagger + b)
	+ J\!\left(\sigma_1^\dagger\sigma_2 + \sigma_2^\dagger\sigma_1\right)
	+ \Omega\sum_{i=1,2}\!\left(\sigma_i + \sigma_i^\dagger\right).
	\label{eq:H}
\end{equation}
%
El término de Förster conserva $\hat{N}$ y por tanto es independiente del
tiempo en el marco rotante. En el subespacio de una excitación electrónica
$\{|cv\rangle,|vc\rangle\}$, con $|cv\rangle \equiv |c\rangle_1|v\rangle_2$,
dicho término tiene representación matricial $J\bigl(\begin{smallmatrix}0&1\\1&0\end{smallmatrix}\bigr)$,
cuyos autoestados son el estado brillante simétrico y el estado oscuro
antisimétrico,
%
\begin{align}
	|\Psi_+\rangle &= \tfrac{1}{\sqrt{2}}\!\left(|cv\rangle + |vc\rangle\right),
	\quad E_+ = +J,
	\label{eq:bright}\\[4pt]
	|\Psi_-\rangle &= \tfrac{1}{\sqrt{2}}\!\left(|cv\rangle - |vc\rangle\right),
	\quad E_- = -J.
	\label{eq:dark}
\end{align}
%
El término de conducción sobre el vacío electrónico $|vv\rangle$ satisface
$\Omega(\sigma_1^\dagger+\sigma_2^\dagger)|vv\rangle = \sqrt{2}\,\Omega\,|\Psi_+\rangle$,
de modo que el láser acopla exclusivamente el estado brillante con frecuencia
de Rabi colectiva $\sqrt{2}\,\Omega$. El estado oscuro $|\Psi_-\rangle$ está
desacoplado del láser y no participa en la dinámica de emisión. Introduciendo
los operadores colectivos $S^\dagger = |\Psi_+\rangle\langle vv|$ y
$S = |vv\rangle\langle\Psi_+|$, y restringiendo la dinámica al subespacio
$\{|vv\rangle, |\Psi_+\rangle\}$, el Hamiltoniano~\eqref{eq:H} adopta la
forma efectiva
%
\begin{equation}
	H_\mathrm{eff} = \omega_b b^\dagger b
	+ (\Delta + J)\,S^\dagger S
	+ \lambda\,S^\dagger S(b^\dagger + b)
	+ \sqrt{2}\,\Omega\!\left(S + S^\dagger\right).
	\label{eq:Heff}
\end{equation}
%
Definimos la desintonía efectiva $\widetilde\Delta \equiv \Delta + J$ y la
frecuencia de Rabi efectiva $\widetilde\Omega \equiv \sqrt{2}\,\Omega$.
A continuación tratamos los distintos regímenes de interés.


\subsection{Régimen de conducción débil y acoplamiento débil}

En el régimen $\widetilde\Omega,\,\lambda \ll \omega_b$, los autoestados del Hamiltoniano~\eqref{eq:Heff} son
próximos a los estados producto $|m,vv\rangle$ y $|m,\Psi_+\rangle$. Cuando
el sistema se conduce a la resonancia de Stokes de $n$ fonones,
$\widetilde\Delta = -n\omega_b$, es decir
%
\begin{equation}
	\Delta_n = -n\omega_b - J,
	\label{eq:res}
\end{equation}
%
se puede considerar el espacio de Hilbert truncado generado por los
$2(n+1)$ estados producto
$\{|0,vv\rangle, |n,\Psi_+\rangle, |0,\Psi_+\rangle, |1,vv\rangle,
\ldots, |n-1,\Psi_+\rangle, |n,vv\rangle\}$.
El Hamiltoniano truncado tiene la forma de bloques
%
\begin{equation}
	H^{(n)} = \begin{pmatrix}
		\mathbf{H}^{(n)} & X^{(n)} \\[4pt]
		X^{(n)T} & R^{(n)}
	\end{pmatrix},
	\label{eq:bloques}
\end{equation}
%
donde $\mathbf{H}^{(n)}$ actúa en el subespacio de interés
$\{|0,vv\rangle,\,|n,\Psi_+\rangle\}$. En resonancia, los dos estados de
interés son degenerados y $\mathbf{H}^{(n)} = 0$. La submatriz $R^{(n)}$
es diagonal con las energías de los $2n$ estados intermedios,
$E_{|k,vv\rangle} = k\omega_b$ y
$E_{|k,\Psi_+\rangle} = -(n-k)\omega_b$ para $k=1,\ldots,n$.
Los dos subespacios quedan acoplados por
%
\begin{equation}
	X^{(n)} = \begin{pmatrix}
		\widetilde\Omega & 0 & \cdots & 0 & 0 \\
		0 & 0 & \cdots & \sqrt{n}\,\lambda & \widetilde\Omega
	\end{pmatrix},
	\label{eq:Xn}
\end{equation}
%
donde la primera (segunda) fila corresponde a $|0,vv\rangle$
($|n,\Psi_+\rangle$). Aplicando la fórmula de partición de Löwdin para
eliminar adiabáticamente los estados intermedios,
%
\begin{equation}
	H_\mathrm{eff}^{(n)} = X^{(n)}\!\left(-R^{(n)}\right)^{-1}\!X^{(n)T},
	\label{eq:Lowdin}
\end{equation}
%
la frecuencia super-Rabi efectiva de $n$ fonones queda definida como el
elemento fuera de la diagonal,
$\Omega_\mathrm{eff}^{(n)} = H_\mathrm{eff}^{(n)}(1,2)$.
\textit{Caso base $n=1$.}
El espacio truncado es
$\{|0,vv\rangle,\,|1,\Psi_+\rangle,\,|0,\Psi_+\rangle,\,|1,vv\rangle\}$.
En resonancia $\widetilde\Delta = -\omega_b$, el Hamiltoniano completo en
esta base (estados de interés primero, intermedios después) es
%
\begin{equation}
	H^{(1)} = \begin{pmatrix}
		0 & 0 & \widetilde\Omega & 0 \\[4pt]
		0 & 0 & \lambda & \widetilde\Omega \\[4pt]
		\widetilde\Omega & \lambda & -\omega_b & 0 \\[4pt]
		0 & \widetilde\Omega & 0 & \omega_b
	\end{pmatrix},
	\label{eq:H1}
\end{equation}
%
donde los elementos fuera de la diagonal provienen directamente de
$H_\mathrm{eff}$~\eqref{eq:Heff}: el acoplamiento
$\langle 0,\Psi_+|\widetilde\Omega\,S^\dagger|0,vv\rangle = \widetilde\Omega$
da las posiciones $(3,1)$ y $(1,3)$; el elemento
$\langle 1,\Psi_+|\lambda\,S^\dagger S\,b^\dagger|0,\Psi_+\rangle
= \lambda\sqrt{1} = \lambda$ da $(2,3)$ y $(3,2)$; y
$\langle 1,vv|\widetilde\Omega\,S|1,\Psi_+\rangle = \widetilde\Omega$
da $(4,2)$ y $(2,4)$.
Identificamos los bloques de~\eqref{eq:bloques}:
%
\begin{equation}
	X^{(1)} = \begin{pmatrix}
		\widetilde\Omega & 0 \\[4pt] \lambda & \widetilde\Omega
	\end{pmatrix},
	\qquad
	R^{(1)} = \begin{pmatrix}
		-\omega_b & 0 \\[4pt] 0 & \omega_b
	\end{pmatrix}.
	\label{eq:bloques1}
\end{equation}
%
Aplicando~\eqref{eq:Lowdin} explícitamente:
%
\begin{equation}
	H_\mathrm{eff}^{(1)}
	= \begin{pmatrix}\widetilde\Omega & 0\\\lambda & \widetilde\Omega\end{pmatrix}
	\begin{pmatrix}1/\omega_b & 0\\ 0 & -1/\omega_b\end{pmatrix}
	\begin{pmatrix}\widetilde\Omega & \lambda\\ 0 & \widetilde\Omega\end{pmatrix}
	= \frac{1}{\omega_b}\begin{pmatrix}
		\widetilde\Omega^2 & \widetilde\Omega\lambda \\[4pt]
		\widetilde\Omega\lambda & \lambda^2 - \widetilde\Omega^2
	\end{pmatrix},
	\label{eq:Heff1}
\end{equation}
%
de donde $\Omega_\mathrm{eff}^{(1)} = \widetilde\Omega\lambda/\omega_b$.
En términos de $\Delta = -\omega_b$ esto equivale a
$\Omega_\mathrm{eff}^{(1)} = -\widetilde\Omega\lambda/\Delta$,
que es la semilla de la recurrencia.

Para obtener la expresión general consideramos el espacio truncado
generado por los cuatro estados
$\{|0,vv\rangle, |n,\Psi_+\rangle, |n-1,\Psi_+\rangle, |n,vv\rangle\}$,
donde $\Omega_\mathrm{eff}^{(n-1)}$ es el acoplamiento efectivo ya calculado
entre $|0,vv\rangle$ y $|n-1,\Psi_+\rangle$. En la resonancia de $n$
fonones, la energía del estado intermedio $|n-1,\Psi_+\rangle$ es
$(n-1)\omega_b + \widetilde\Delta = (n-1)\omega_b - n\omega_b = -\omega_b$.
El Hamiltoniano en el ordenamiento
$(|0,vv\rangle,\,|n,\Psi_+\rangle,\,|n-1,\Psi_+\rangle,\,|n,vv\rangle)$ es
%
\begin{equation}
	H^{(n)} = \begin{pmatrix}
		0 & 0 & \Omega_\mathrm{eff}^{(n-1)} & 0 \\[4pt]
		0 & 0 & \sqrt{n}\,\lambda & \widetilde\Omega \\[4pt]
		\Omega_\mathrm{eff}^{(n-1)} & \sqrt{n}\,\lambda & -\omega_b & 0 \\[4pt]
		0 & \widetilde\Omega & 0 & n\omega_b
	\end{pmatrix},
	\label{eq:Hn}
\end{equation}
%
donde la energía del estado intermedio en la diagonal $(3,3)$ es
$-\omega_b$ (evaluada en la resonancia de $n$ fonones), no $-(n-1)\omega_b$.
El Hamiltoniano reducido adopta entonces la forma~\eqref{eq:bloques} con
%
\begin{equation}
	X^{(n)} = \begin{pmatrix}
		\Omega_\mathrm{eff}^{(n-1)} & 0 \\[4pt]
		\sqrt{n}\,\lambda & \widetilde\Omega
	\end{pmatrix},
	\qquad
	R^{(n)} = \begin{pmatrix}
		-\omega_b & 0 \\[4pt]
		0 & n\omega_b
	\end{pmatrix}.
	\label{eq:XnRn}
\end{equation}
%
Aplicando~\eqref{eq:Lowdin} al espacio reducido, el único contribuyente al
elemento $(1,2)$ del Hamiltoniano efectivo es el estado intermedio
$|n-1,\Psi_+\rangle$ (con denominador $-\omega_b$), lo que da la relación
de recurrencia
%
\begin{equation}
	\Omega_\mathrm{eff}^{(n)}
	= -\frac{\sqrt{n}\,\lambda}{\Delta+(n-1)\omega_b}\,
	\Omega_\mathrm{eff}^{(n-1)}.
	\label{eq:rec}
\end{equation}
%
Siguiendo el mismo procedimiento hacia atrás,
%
\begin{equation}
	\Omega_\mathrm{eff}^{(n-1)}
	= -\frac{\sqrt{n-1}\,\lambda}{\Delta+(n-2)\omega_b}\,
	\Omega_\mathrm{eff}^{(n-2)}
	= (-1)^{n-1}
	\frac{\sqrt{n-1}\,\lambda}{\Delta+(n-2)\omega_b}\cdots
	\frac{\sqrt{2}\,\lambda}{\Delta+\omega_b}\,
	\Omega_\mathrm{eff}^{(1)}.
	\label{eq:chain}
\end{equation}
%
Para el espacio de cuatro estados con $n=1$,
$\{|0,vv\rangle,|1,\Psi_+\rangle,|0,\Psi_+\rangle,|1,vv\rangle\}$,
la aplicación directa de~\eqref{eq:Lowdin} da
$\Omega_\mathrm{eff}^{(1)} = -\widetilde\Omega\lambda/\Delta$.
Evaluando todo en la resonancia de $n$ fonones $\Delta = -n\omega_b$, la
frecuencia super-Rabi efectiva de orden cualquiera es
%
\begin{equation}
	\Omega_\mathrm{eff}^{(n)}
	= -\frac{\sqrt{n}\,\lambda}{\Delta+(n-1)\omega_b}\,
	\Omega_\mathrm{eff}^{(n-1)}
	= (-1)^n
	\frac{\sqrt{n}\,\lambda}{\Delta+(n-1)\omega_b}\cdots
	\frac{\sqrt{2}\,\lambda}{\Delta+\omega_b}\,
	\frac{\widetilde\Omega\,\lambda}{\Delta}
	= \left(\frac{\lambda}{\omega_b}\right)^{\!n}
	\frac{\widetilde\Omega}{\sqrt{n!}},
	\label{eq:Oeff_I_pre}
\end{equation}
%
donde en el último paso se usó que los denominadores evaluados en
$\Delta = -n\omega_b$ son
$\Delta = -n\omega_b$,
$\Delta+\omega_b = -(n-1)\omega_b$,
\ldots,
$\Delta+(n-1)\omega_b = -\omega_b$,
de modo que el producto de denominadores es $(-1)^n\,\omega_b^n\,(n-1)!$ y
el producto de numeradores es $\lambda^n\sqrt{n!}$.
Sustituyendo $\widetilde\Omega = \sqrt{2}\,\Omega$, el resultado final es
%
\begin{equation}
	\boxed{
		\Omega_\mathrm{eff}^{(n)}
		= \sqrt{2}\left(\frac{\lambda}{\omega_b}\right)^{\!n}
		\frac{\Omega}{\sqrt{n!}},
	}
	\label{eq:resultado}
\end{equation}
%
válido al orden dominante en $\lambda/\omega_b$, bajo la condición de
resonancia~\eqref{eq:res}. El factor $\sqrt{2}$ es de origen puramente
colectivo: proviene del acoplamiento superradiante del estado brillante al
láser.

\subsection{Régimen de acoplamiento fuerte electrón-fonón}

En el régimen $\lambda \sim \omega_b$ con $\widetilde\Omega \ll \omega_b$,
el acoplamiento electrón-fonón no puede tratarse perturbativamente.
La estrategia es eliminarlo primero mediante una transformación unitaria
exacta, dejando al láser como único término perturbativo. Definimos el
operador de desplazamiento colectivo
%
\begin{equation}
	\widetilde{D} = \exp\!\left[
	\frac{\lambda}{\omega_b}\,S^\dagger S\,(b^\dagger - b)
	\right].
	\label{eq:D}
\end{equation}
%
Dado que $S^\dagger S$ tiene autovalores $0$ (en $|vv\rangle$) y $1$
(en $|\Psi_+\rangle$), la transformación actúa como la identidad en el
sector electrónico vacío y como el desplazamiento estándar
$D(\lambda/\omega_b)$ en el sector excitado.

Calculamos $\widetilde{D}\,H_\mathrm{eff}\,\widetilde{D}^\dagger$ término
a término. Denotando $\hat{n}_e \equiv S^\dagger S$ y usando
$D(\alpha)\,b\,D^\dagger(\alpha) = b - \alpha$ con $\alpha = \lambda/\omega_b$:
%
\begin{align}
	\widetilde{D}\,(\omega_b b^\dagger b)\,\widetilde{D}^\dagger
	&= \omega_b b^\dagger b
	- \lambda\,\hat{n}_e(b^\dagger + b)
	+ \frac{\lambda^2}{\omega_b}\,\hat{n}_e,
	\label{eq:D_fon}\\[4pt]
	\widetilde{D}\,(\widetilde\Delta\,\hat{n}_e)\,\widetilde{D}^\dagger
	&= \widetilde\Delta\,\hat{n}_e,
	\label{eq:D_elec}\\[4pt]
	\widetilde{D}\,\bigl[\lambda\,\hat{n}_e(b^\dagger+b)\bigr]\,\widetilde{D}^\dagger
	&= \lambda\,\hat{n}_e(b^\dagger+b)
	- \frac{2\lambda^2}{\omega_b}\,\hat{n}_e,
	\label{eq:D_eph}
\end{align}
%
donde en~\eqref{eq:D_fon} se usó $\hat{n}_e^2 = \hat{n}_e$ (proyector).
Sumando~\eqref{eq:D_fon}--\eqref{eq:D_eph}, los términos en
$\hat{n}_e(b^\dagger+b)$ se cancelan exactamente:
%
\begin{equation}
	\widetilde{D}\bigl[\omega_b b^\dagger b
	+ \widetilde\Delta\,\hat{n}_e
	+ \lambda\,\hat{n}_e(b^\dagger+b)\bigr]\widetilde{D}^\dagger
	= \omega_b b^\dagger b
	+ \left(\widetilde\Delta - \frac{\lambda^2}{\omega_b}\right)\hat{n}_e.
	\label{eq:D_libre}
\end{equation}
%
La frecuencia de transición renormalizada queda
$\widetilde\omega_\sigma \equiv \widetilde\Delta - \lambda^2/\omega_b
= \Delta + J - \lambda^2/\omega_b$.

Para el término del láser, $S^\dagger$ eleva $\hat{n}_e$ de $0$ a $1$,
de modo que la transformación actúa con $\hat{n}_e = 1$ a la izquierda
de $S^\dagger$ y con $\hat{n}_e = 0$ a su derecha:
%
\begin{equation}
	\widetilde{D}\,\widetilde\Omega(S+S^\dagger)\,\widetilde{D}^\dagger
	= \widetilde\Omega\!\left[
	S^\dagger\,e^{(\lambda/\omega_b)(b^\dagger-b)}
	+ S\,e^{-(\lambda/\omega_b)(b^\dagger-b)}
	\right].
	\label{eq:D_laser}
\end{equation}
%
Reuniendo~\eqref{eq:D_libre} y~\eqref{eq:D_laser}, el Hamiltoniano
transformado es
%
\begin{equation}
	\widetilde{H} = \omega_b b^\dagger b
	+ \widetilde\omega_\sigma\,S^\dagger S
	+ \widetilde\Omega\!\left[
	S^\dagger\,e^{(\lambda/\omega_b)(b^\dagger - b)}
	+ \mathrm{H.c.}
	\right].
	\label{eq:Htilde}
\end{equation}
%
Para identificar los procesos resonantes de $n$ fonones, expandimos la
exponencial en~\eqref{eq:Htilde} en orden normal mediante la identidad
de Baker-Campbell-Hausdorff,
%
\begin{equation}
	e^{(\lambda/\omega_b)(b^\dagger - b)}
	= e^{-\frac{1}{2}(\lambda/\omega_b)^2}
	\sum_{k=0}^\infty\sum_{l=0}^\infty
	\frac{(-1)^l(\lambda/\omega_b)^{k+l}}{k!\,l!}\,b^{\dagger k} b^l.
	\label{eq:BCH}
\end{equation}
%
El término $S^\dagger b^{\dagger k}b^l$ acopla $|0,vv\rangle$ con
$|k-l,\Psi_+\rangle$, de modo que el proceso Stokes de $n = k - l$
fonones tiene como condición de resonancia
$E_{|0,vv\rangle} = E_{|n,\Psi_+\rangle}$, es decir
$\widetilde\omega_\sigma = -n\omega_b$:
%
\begin{equation}
	\boxed{\Delta_n = \frac{\lambda^2}{\omega_b} - n\omega_b - J.}
	\label{eq:res_II}
\end{equation}
%
En resonancia, la frecuencia super-Rabi efectiva de $n$ fonones es el
elemento de matriz del operador $e^{(\lambda/\omega_b)(b^\dagger - b)}$
entre los estados fonónicos $|0\rangle$ y $|n\rangle$, multiplicado
por $\widetilde\Omega$. Usando la fórmula general para el elemento de
matriz de un operador de desplazamiento entre estados de Fock,
%
\begin{equation}
	\langle n|\,e^{(\lambda/\omega_b)(b^\dagger - b)}\,|0\rangle
	= e^{-\frac{1}{2}(\lambda/\omega_b)^2}
	\left(\frac{\lambda}{\omega_b}\right)^{\!n}
	\frac{1}{\sqrt{n!}}\,L_0^n\!\left(\!\left(\frac{\lambda}{\omega_b}\right)^{\!2}\right),
	\label{eq:elem_mat}
\end{equation}
%
donde $L_m^n(\varepsilon)$ es el polinomio de Laguerre asociado.
Para $m=0$, $L_0^n(\varepsilon) = 1$ para todo $n$, y por tanto:
%
\begin{equation}
	\boxed{
		\Omega_\mathrm{eff}^{(n)}
		= \sqrt{2}\,\Omega\,
		\exp\!\left[-\frac{1}{2}\!\left(\frac{\lambda}{\omega_b}\right)^{\!2}\right]
		\left(\frac{\lambda}{\omega_b}\right)^{\!n}
		\frac{1}{\sqrt{n!}},
	}
	\label{eq:resultado_II}
\end{equation}
%
bajo la condición de resonancia~\eqref{eq:res_II}. El factor gaussiano
$\exp[-(\lambda/\omega_b)^2/2]$ es el factor de Franck-Condon generado
por el desplazamiento fonónico; en el límite $\lambda \ll \omega_b$
tiende a $1$ y recupera el resultado~\eqref{eq:resultado} del Régimen~I.
Las correcciones de la molécula excitónica respecto al caso de un solo
punto cuántico son las mismas que en el Régimen~I: el factor superradiante
$\sqrt{2}$ en la amplitud y el desplazamiento de resonancia $-J$.

\subsection{Régimen de conducción fuerte}

En el régimen $\widetilde\Omega \sim \omega_b$ con $\lambda \ll \omega_b$,
el láser ya no puede tratarse perturbativamente. La estrategia es
diagonalizar primero la parte electrónica y luego tratar
$H_\mathrm{e\text{-}ph} = \lambda\,S^\dagger S(b^\dagger+b)$
perturbativamente en la base de estados vestidos resultante.
Escribimos $H_\mathrm{eff} = H_0 + H_\mathrm{e\text{-}ph}$ con
$H_0 = \omega_b b^\dagger b + H_\sigma$, donde
%
\begin{equation}
	H_\sigma = \widetilde\Delta\,S^\dagger S
	+ \widetilde\Omega\bigl(S + S^\dagger\bigr)
	= \begin{pmatrix} 0 & \widetilde\Omega \\[2pt]
		\widetilde\Omega & \widetilde\Delta \end{pmatrix}
	\label{eq:Hsigma}
\end{equation}
%
en la base $\{|vv\rangle,|\Psi_+\rangle\}$.

\textit{Diagonalización de $H_\sigma$.}
Los autovalores se obtienen de
$E^2 - \widetilde\Delta E - \widetilde\Omega^2 = 0$:
%
\begin{equation}
	E_\pm = \frac{\widetilde\Delta \pm \tilde{r}}{2},
	\qquad
	\tilde{r} \equiv \sqrt{\widetilde\Delta^2 + 4\widetilde\Omega^2}.
	\label{eq:autovalores}
\end{equation}
%
Para el autovector de $E_+$, la ecuación
$(H_\sigma - E_+\mathbb{I})\binom{\alpha}{\beta} = 0$ da
$\beta/\alpha = E_+/\widetilde\Omega
= (\widetilde\Delta + \tilde{r})/(2\widetilde\Omega) > 0$.
Normalizando con $\alpha,\beta > 0$:
%
\begin{align}
	|{+}\rangle &= \tilde{c}_+\,|vv\rangle + \tilde{c}_-\,|\Psi_+\rangle,
	\qquad E_+ = \tfrac{1}{2}(\widetilde\Delta + \tilde{r}),
	\label{eq:ket_p}\\[4pt]
	|{-}\rangle &= \tilde{c}_-\,|vv\rangle - \tilde{c}_+\,|\Psi_+\rangle,
	\qquad E_- = \tfrac{1}{2}(\widetilde\Delta - \tilde{r}),
	\label{eq:ket_m}
\end{align}
%
con coeficientes
%
\begin{equation}
	\tilde{c}_\pm = \sqrt{\frac{2\widetilde\Omega^2}
		{\widetilde\Delta^2 + 4\widetilde\Omega^2
			\pm\,\widetilde\Delta\,\tilde{r}}},
	\qquad \tilde{c}_+^2 + \tilde{c}_-^2 = 1.
	\label{eq:cpm}
\end{equation}
%
Verificación del autovector $|+\rangle$: la componente en
$|vv\rangle$ de $H_\sigma|+\rangle$ da $\tilde{c}_-\widetilde\Omega$,
que debe ser $E_+\tilde{c}_+$. De hecho,
$\tilde{c}_-/\tilde{c}_+ = E_+/\widetilde\Omega
= (\widetilde\Delta+\tilde{r})/(2\widetilde\Omega) > 0$~$\checkmark$.
Ortogonalidad: $\langle+|-\rangle = \tilde{c}_+\tilde{c}_- - \tilde{c}_-\tilde{c}_+ = 0$~$\checkmark$.

Los autoestados de $H_0$ son los productos tensor
$|n,\pm\rangle \equiv |n\rangle\otimes|\pm\rangle$
con energías $E_{|n,\pm\rangle} = n\omega_b + E_\pm$.

\textit{Inversión de la base vestida.}
Multiplicando~\eqref{eq:ket_p} por $\tilde{c}_+$
y~\eqref{eq:ket_m} por $\tilde{c}_-$ y sumando; luego
restando con los factores intercambiados; con
$\tilde{c}_+^2+\tilde{c}_-^2 = 1$:
%
\begin{equation}
	|vv\rangle = \tilde{c}_+|{+}\rangle + \tilde{c}_-|{-}\rangle,
	\qquad
	|\Psi_+\rangle = \tilde{c}_-|{+}\rangle - \tilde{c}_+|{-}\rangle.
	\label{eq:inv}
\end{equation}
%
Usando $S^\dagger S = |\Psi_+\rangle\langle\Psi_+|$
y~\eqref{eq:inv}, los elementos de matriz de $S^\dagger S$
en la base vestida son
%
\begin{equation}
	\langle{+}|S^\dagger S|{+}\rangle = \tilde{c}_-^2,\quad
	\langle{-}|S^\dagger S|{-}\rangle = \tilde{c}_+^2,\quad
	\langle{\pm}|S^\dagger S|{\mp}\rangle = -\tilde{c}_+\tilde{c}_-.
	\label{eq:SdagS}
\end{equation}

\textit{Elementos de matriz de $H_\mathrm{e\text{-}ph}$.}
Combinando~\eqref{eq:SdagS} con
$\langle m|b^\dagger+b|n\rangle = \sqrt{n+1}\,\delta_{m,n+1}+\sqrt{n}\,\delta_{m,n-1}$,
los elementos no nulos de $H_\mathrm{e\text{-}ph}$ que aumentan el
número fonónico son
%
\begin{align}
	\langle n{+}1,{+}|H_\mathrm{e\text{-}ph}|n,{+}\rangle
	&= +\tilde{c}_-^2\,\lambda\sqrt{n+1},
	\label{eq:Heph_pp}\\[4pt]
	\langle n{+}1,{-}|H_\mathrm{e\text{-}ph}|n,{-}\rangle
	&= +\tilde{c}_+^2\,\lambda\sqrt{n+1},
	\label{eq:Heph_mm}\\[4pt]
	\langle n{+}1,{-}|H_\mathrm{e\text{-}ph}|n,{+}\rangle
	&= -\tilde{c}_+\tilde{c}_-\,\lambda\sqrt{n+1},
	\label{eq:Heph_mp}\\[4pt]
	\langle n{+}1,{+}|H_\mathrm{e\text{-}ph}|n,{-}\rangle
	&= -\tilde{c}_+\tilde{c}_-\,\lambda\sqrt{n+1}.
	\label{eq:Heph_pm}
\end{align}
%
Los acoplamientos~\eqref{eq:Heph_pp}--\eqref{eq:Heph_mm} conservan
el índice vestido; los~\eqref{eq:Heph_mp}--\eqref{eq:Heph_pm}
lo invierten con amplitud $-\tilde{c}_+\tilde{c}_-\lambda$.
Son estos últimos los que median, a orden $\lambda^n$, el proceso
de $n$ fonones $|0,{+}\rangle \leftrightarrow |n,{-}\rangle$.

\textit{Condición de resonancia Stokes.}
La condición $E_{|0,+\rangle} = E_{|n,-\rangle}$ simplifica a
$\tilde{r} = n\omega_b$. Sustituyendo
$\tilde{r}^2 = (\Delta+J)^2 + 8\Omega^2$ y tomando la raíz
negativa (proceso Stokes, $\Delta < 0$):
%
\begin{equation}
	\boxed{\Delta_n = -\sqrt{n^2\omega_b^2 - 8\Omega^2} - J,}
	\label{eq:res_III}
\end{equation}
%
válida para $n\omega_b > 2\sqrt{2}\,\Omega$. En el límite
$\Omega \to 0$ se recupera $\Delta_n \to -n\omega_b - J$~$\checkmark$.
La diferencia respecto al caso de un solo punto cuántico es doble:
el término $8\Omega^2 = 4\widetilde\Omega^2$ lleva el factor
superradiante, y el desplazamiento $-J$ proviene de Förster.

En resonancia $\tilde{r} = n\omega_b$, las energías relativas de
los estados intermedios respecto a $E_0 \equiv E_{|0,+\rangle}$ son
%
\begin{equation}
	E_{|k,+\rangle} - E_0 = k\omega_b > 0,
	\qquad
	E_{|k,-\rangle} - E_0 = -(n-k)\omega_b < 0,
	\qquad 1 \leq k \leq n-1,
	\label{eq:energ_rel}
\end{equation}
%
todos no nulos y múltiplos enteros de $\omega_b$.
Los denominadores en la eliminación adiabática serán
$-k\omega_b$ (para $|k,+\rangle$) y $(n-k)\omega_b$ (para $|k,-\rangle$),
garantizando la no singularidad del proceso perturbativo cuando
$\lambda \ll \omega_b$.

\textit{Caso base $n=2$: forma matricial explícita.}
El espacio truncado es
$\mathcal{H}^{(2)} = \{|0,{+}\rangle,\,|2,{-}\rangle,\,|1,{-}\rangle,\,|1,{+}\rangle\}$.
Los estados degenerados son $|0,+\rangle$ y $|2,-\rangle$
(energía relativa $0$); los intermedios son $|1,-\rangle$
(energía relativa $-\omega_b$) y $|1,+\rangle$
(energía relativa $+\omega_b$). Calculando cada elemento
de la matriz de $H_\mathrm{eff}$ en $\mathcal{H}^{(2)}$ mediante~\eqref{eq:Heph_pp}--\eqref{eq:Heph_pm} con $n=0$ y $n=1$:
%
\begin{align}
	\langle 1,{-}|H_\mathrm{e\text{-}ph}|0,{+}\rangle
	&= -\tilde{c}_+\tilde{c}_-\lambda,
	\label{eq:h10mp}\\[2pt]
	\langle 1,{+}|H_\mathrm{e\text{-}ph}|0,{+}\rangle
	&= +\tilde{c}_-^2\lambda,
	\label{eq:h10pp}\\[2pt]
	\langle 2,{-}|H_\mathrm{e\text{-}ph}|1,{-}\rangle
	&= +\sqrt{2}\,\tilde{c}_+^2\lambda,
	\label{eq:h21mm}\\[2pt]
	\langle 2,{-}|H_\mathrm{e\text{-}ph}|1,{+}\rangle
	&= -\sqrt{2}\,\tilde{c}_+\tilde{c}_-\lambda.
	\label{eq:h21mp}
\end{align}
%
El acoplamiento directo
$\langle 2,-|H_\mathrm{e\text{-}ph}|0,+\rangle = 0$
porque $H_\mathrm{e\text{-}ph} \propto b^\dagger + b$ cambia
el número fonónico en $\pm 1$ en un solo paso.
En el ordenamiento
$\{|0,{+}\rangle,\,|2,{-}\rangle,\,|1,{-}\rangle,\,|1,{+}\rangle\}$
el Hamiltoniano completo, con $\tilde{r} = 2\omega_b$, es
%
\begin{equation}
	H^{(2)} = \begin{pmatrix}
		0 & 0
		& -\tilde{c}_+\tilde{c}_-\lambda
		& \tilde{c}_-^2\lambda \\[6pt]
		0 & 0
		& \sqrt{2}\,\tilde{c}_+^2\lambda
		& -\sqrt{2}\,\tilde{c}_+\tilde{c}_-\lambda \\[6pt]
		-\tilde{c}_+\tilde{c}_-\lambda & \sqrt{2}\,\tilde{c}_+^2\lambda
		& -\omega_b & 0 \\[6pt]
		\tilde{c}_-^2\lambda & -\sqrt{2}\,\tilde{c}_+\tilde{c}_-\lambda
		& 0 & \omega_b
	\end{pmatrix},
	\label{eq:H2_full}
\end{equation}
%
con bloques $X^{(2)}$ (acoplamiento subespacio degenerado--intermedio)
y $R^{(2)}$ (parte intermedia) identificados como
%
\begin{equation}
	X^{(2)} = \begin{pmatrix}
		-\tilde{c}_+\tilde{c}_-\lambda & \tilde{c}_-^2\lambda \\[4pt]
		\sqrt{2}\,\tilde{c}_+^2\lambda & -\sqrt{2}\,\tilde{c}_+\tilde{c}_-\lambda
	\end{pmatrix},
	\qquad
	R^{(2)} = \begin{pmatrix} -\omega_b & 0 \\[2pt] 0 & \omega_b \end{pmatrix}.
	\label{eq:XR2}
\end{equation}
%
Nótese que $R^{(2)}$ es diagonal: el acoplamiento entre
$|1,{-}\rangle$ y $|1,{+}\rangle$ es nulo porque requeriría
$\Delta n_\mathrm{fon} = 0$ con cambio de índice vestido, lo cual
es cero en $H_\mathrm{e\text{-}ph} \propto b^\dagger + b$.
Aplicando la fórmula de Löwdin~\eqref{eq:Lowdin} al elemento $(1,2)$
del subespacio degenerado, con
$[R^{(2)}]^{-1} = \mathrm{diag}(-1/\omega_b, 1/\omega_b)$:
%
\begin{align}
	\Omega_\mathrm{eff}^{(2)}
	&= \left[X^{(2)}\bigl(-R^{(2)}\bigr)^{-1}X^{(2)\,T}\right]_{12}
	\notag\\[4pt]
	&= \frac{(-\tilde{c}_+\tilde{c}_-\lambda)(\tilde{c}_-^2\lambda)}{\omega_b}
	+ \frac{\sqrt{2}\,\tilde{c}_+^2\lambda \cdot (-\sqrt{2}\,\tilde{c}_+\tilde{c}_-\lambda)}{-\omega_b}
	\notag\\[4pt]
	&= \frac{-\tilde{c}_+\tilde{c}_-^3\lambda^2}{\omega_b}
	+ \frac{2\tilde{c}_+^3\tilde{c}_-\lambda^2}{\omega_b}
	= \frac{\tilde{c}_+\tilde{c}_-\lambda^2(2\tilde{c}_+^2 - \tilde{c}_-^2)}{\omega_b},
	\label{eq:Oeff2_paso}
\end{align}
%
que con $2\tilde{c}_+^2 - \tilde{c}_-^2 = 2(1-\tilde{c}_-^2) - \tilde{c}_-^2
= 2 - 3\tilde{c}_-^2$ y $\tilde{r} = 2\omega_b$ puede reescribirse
de forma compacta como:
%
\begin{equation}
	\Omega_\mathrm{eff}^{(2)}
	= \frac{-\sqrt{2}\,\tilde{c}_+\tilde{c}_-\lambda^2
		(\omega_b - \tilde{c}_-^2\tilde{r})}
	{\omega_b(\tilde{r} - \omega_b)}.
	\label{eq:Oeff2_III}
\end{equation}

\textit{Paso recursivo.}
Para $n$ general, dado $\Omega_\mathrm{eff}^{(n-1)}$, se considera el
espacio reducido de cuatro estados
$\{|0,{+}\rangle,\,|n,{-}\rangle,\,|n-1,{+}\rangle,\,|n-1,{-}\rangle\}$.
En este espacio el Hamiltoniano, antes de ninguna transformación, es
%
\begin{equation}
	H^{(n)}_0 = \begin{pmatrix}
		0 & 0 & 0 & \Omega_\mathrm{eff}^{(n-1)} \\[4pt]
		0 & 0 & -\tilde{c}_+\tilde{c}_-\lambda\sqrt{n} &
		\tilde{c}_+^2\lambda\sqrt{n} \\[4pt]
		0 & -\tilde{c}_+\tilde{c}_-\lambda\sqrt{n} & (n-1)\omega_b & 0 \\[4pt]
		\Omega_\mathrm{eff}^{(n-1)} & \tilde{c}_+^2\lambda\sqrt{n} & 0 & -\omega_b
	\end{pmatrix},
	\label{eq:Hn0}
\end{equation}
%
en el ordenamiento
$\{|0,{+}\rangle,\,|n,{-}\rangle,\,|n-1,{+}\rangle,\,|n-1,{-}\rangle\}$,
con $\tilde{r} = n\omega_b$ ya sustituido en los elementos diagonales.
El acoplamiento cruzado entre los estados intermedios,
$-\tilde{c}_+\tilde{c}_-\lambda\sqrt{n}$, impide aplicar directamente
la fórmula de Löwdin. Se introduce el estado efectivo
%
\begin{equation}
	|n-1,{+}\rangle_\mathrm{eff}
	= |n-1,{+}\rangle + \frac{\tilde{c}_-}{\tilde{c}_+}\,|n-1,{-}\rangle,
	\label{eq:estado_eff}
\end{equation}
%
que al orden dominante en $\lambda/\omega_b$ diagonaliza el bloque
intermedio. Bajo esta transformación el Hamiltoniano reducido toma la forma
%
\begin{equation}
	H^{(n)} = \begin{pmatrix}
		0 & 0 & 0 & \Omega_\mathrm{eff}^{(n-1)} \\[4pt]
		0 & n\omega_b - \tilde{r} & 0 & \tilde{c}_+^2\lambda\sqrt{n} \\[4pt]
		0 & 0 &
		\bigl(\tilde{c}_-/\tilde{c}_+\bigr)^2(-\omega_b) + (n-1)\omega_b &
		(\tilde{c}_-/\tilde{c}_+)(-\omega_b) \\[8pt]
		\Omega_\mathrm{eff}^{(n-1)} & \tilde{c}_+^2\lambda\sqrt{n} &
		(\tilde{c}_-/\tilde{c}_+)(-\omega_b) & -\omega_b
	\end{pmatrix}.
	\label{eq:Hn_transf}
\end{equation}
%
En resonancia $\tilde{r} = n\omega_b$, el elemento $H^{(n)}_{22} = 0$,
de modo que $|n-1,{+}\rangle_\mathrm{eff}$ queda desacoplado de los
estados degenerados al orden dominante. El único intermediario que
conecta $|0,{+}\rangle$ con $|n,{-}\rangle$ es $|n-1,{-}\rangle$
con energía relativa $-\omega_b$, y el acoplamiento
$|0,{+}\rangle \leftrightarrow |n-1,{-}\rangle$ es
$\Omega_\mathrm{eff}^{(n-1)}$, mientras que
$|n-1,{-}\rangle \leftrightarrow |n,{-}\rangle$ tiene amplitud
$\tilde{c}_+^2\lambda\sqrt{n}$. Aplicando la fórmula de Löwdin al
elemento $(1,2)$ del subespacio degenerado:
%
\begin{equation}
	\Omega_\mathrm{eff}^{(n)}
	= \frac{\sqrt{n}\,\lambda\bigl[(n-1)\omega_b
		- \tilde{c}_-^2\,\tilde{r}\,\bigr]}
	{(n-1)\omega_b\,\bigl[\tilde{r} - (n-1)\omega_b\bigr]}
	\;\Omega_\mathrm{eff}^{(n-1)},
	\label{eq:rec_III}
\end{equation}
%
válida al orden dominante en $\lambda/\omega_b$.
Iterando~\eqref{eq:rec_III} desde el caso base~\eqref{eq:Oeff2_III},
el factor de cada paso $k \to k+1$ es
$\sqrt{k}\,\lambda(k-1 - n\tilde{c}_-^2)/[(k-1)(n-k+1)\omega_b]$.
El producto acumulado desde $k=3$ hasta $k=n$, multiplicado por la
semilla $\Omega_\mathrm{eff}^{(2)}$, da el resultado para orden
arbitrario:
%
\begin{equation}
	\boxed{
		\Omega_\mathrm{eff}^{(n)}
		= (-1)^n\;\frac{\sqrt{2}\,\widetilde\Omega}
		{(n-1)!\,\sqrt{n!}}
		\left(\frac{\lambda}{\omega_b}\right)^{\!n}
		\prod_{k=1}^{n-1}\bigl(n\,\tilde{c}_-^2 - k\bigr),
	}
	\label{eq:resultado_III}
\end{equation}
%
con la condición de resonancia~\eqref{eq:res_III}. Los primeros casos
explícitos son
%
\begin{align}
	\Omega_\mathrm{eff}^{(2)}
	&= \frac{\sqrt{2}\,\widetilde\Omega\lambda^2}{\omega_b^2}
	\,\tilde{c}_-^2(2\tilde{c}_-^2 - 1),
	\label{eq:Oeff2_exp}\\[6pt]
	\Omega_\mathrm{eff}^{(3)}
	&= -\frac{\sqrt{2}\,\widetilde\Omega\lambda^3}{\sqrt{6}\,\omega_b^3}
	\,\tilde{c}_-^2(3\tilde{c}_-^2-1)(3\tilde{c}_-^2-2),
	\label{eq:Oeff3_exp}\\[6pt]
	\Omega_\mathrm{eff}^{(4)}
	&= \frac{\sqrt{2}\,\widetilde\Omega\lambda^4}{2\sqrt{6}\,\omega_b^4}
	\,\tilde{c}_-^2(4\tilde{c}_-^2-1)(4\tilde{c}_-^2-2)(4\tilde{c}_-^2-3).
	\label{eq:Oeff4_exp}
\end{align}
%
El factor $(1/(n-1)!)$ adicional respecto a los regímenes anteriores
hace que $|\Omega_\mathrm{eff}^{(n)}|$ decaiga más rápido con $n$.

Los resultados de los tres regímenes se pueden resumir de forma
compacta. En todos ellos las dos correcciones de la molécula
excitónica respecto al caso de un solo punto cuántico son: un factor
superradiante $\sqrt{2}$ en la frecuencia de Rabi efectiva, originado
en el acoplamiento colectivo del estado brillante, y un desplazamiento
rígido $-J$ en la condición de resonancia, originado en el acoplamiento
de Förster:
%
\begin{equation}
	\Omega_\mathrm{eff}^{(n)}\Big|_{2\mathrm{QD}} = \!
	\begin{cases}
		\dfrac{\sqrt{2}\,\Omega}{\sqrt{n!}}
		\!\left(\dfrac{\lambda}{\omega_b}\right)^{\!n},
		& \text{Régimen I},\\[12pt]
		\dfrac{\sqrt{2}\,\Omega}{\sqrt{n!}}
		\!\left(\dfrac{\lambda}{\omega_b}\right)^{\!n}
		e^{-\lambda^2/(2\omega_b^2)},
		& \text{Régimen II},\\[12pt]
		\dfrac{(-1)^n\sqrt{2}\,\widetilde\Omega}
		{(n-1)!\sqrt{n!}}
		\!\left(\dfrac{\lambda}{\omega_b}\right)^{\!n}
		\displaystyle\prod_{k=1}^{n-1}(n\tilde{c}_-^2 - k),
		& \text{Régimen III},
	\end{cases}
	\label{eq:Oeff_resumen}
\end{equation}
%
con resonancias
$\Delta_n = -n\omega_b - J$ (I),
$\;\Delta_n = \lambda^2/\omega_b - n\omega_b - J$ (II),
$\;\Delta_n = -\sqrt{n^2\omega_b^2 - 8\Omega^2} - J$ (III).


\end{appendix}