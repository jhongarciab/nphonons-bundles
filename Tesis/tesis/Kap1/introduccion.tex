\chapter{Introducción}

La manipulación de estados cuánticos es uno de los tópicos principales de la ciencia moderna. En el caso de los fotones, una línea de investigación cada vez más popular es la física multifotónica \cite{kubanek2008,ota2011,gonzaleztudela2013,munoz2014,chang2016,kockum2017}, con aplicaciones potenciales para láseres multifotónicos \cite{gauthier1992}, superación del límite de difracción \cite{dangelo2001}, y metrología \cite{afek2010}. En particular, se ha propuesto recientemente un esquema para la generación directa de estados de $n$ fotones en el mismo modo (paquetes de $n$ fotones) bajo la plataforma de electrodinámica cuántica de cavidades (cQED) \cite{munoz2014}.\\
Además de los fotones, los fonones (los cuantos de ondas mecánicas) han emergido como candidatos fuertes para la ingeniería de dispositivos cuánticos de estado sólido y comunicaciones cuánticas en chip, con diversas ventajas distintivas. Primero, la velocidad de las ondas acústicas es significativamente más lenta que la velocidad de la luz, y por lo tanto es más adecuada para comunicaciones sobre distancias cortas, tales como unos pocos cientos de micrómetros o menos (es decir, comunicación en chip) \cite{gustafsson2014,kuzyk2018}. Segundo, dado que los fonones solo pueden propagarse en un medio material, no sufren pérdidas por radiación al vacío. Por último, las cavidades fonónicas son altamente sintonizables, con demostraciones experimentales desde gigahertz (GHz) hasta terahertz (THz) \cite{borri2001,weig2004,rozas2009,stock2011,soykal2011,fainstein2013,xu2018,kharel2018}. En particular, los fonones en el rango de los THz tienen longitudes de onda comparables a las constantes de red, lo cual tiene aplicaciones importantes para sensado e imágenes a nanoescala con resolución atómica \cite{bin2020}. Consecuentemente, la fonónica cuántica ha progresado enormemente, incluyendo la investigación de láseres fonónicos \cite{kabuss2012,maryam2013,han2015}, redes cuánticas fonónicas \cite{lemonde2018,calajo2019}, la detección de la interacción electrón-fonón en puntos cuánticos dobles \cite{hartke2018,gullans2018}, y dispositivos acústicos cuánticos \cite{schuetz2015,manenti2017}. La generación de estados cuánticos multifonónicos, como un hito fundamental en el camino de estos dispositivos, se convierte en una tarea importante de la fonónica. Por ejemplo, paquetes antibunched y estados NOON de fonones podrían ser valiosos como fuentes de $n$ fonones \cite{chu2018,satzinger2018} y para mediciones de precisión cuántica acústica \cite{toyoda2015,zhang2018}, respectivamente. Además de esto, los procesos multifonónicos también tienen aplicaciones importantes en biodetección ultrasensible \cite{suzuki2016}.\\

Recientemente, se ha demostrado teóricamente que un punto cuántico acoplado a una nanocavidad acústica puede generar paquetes de $n$ fonones mediante un proceso de Stokes ópticamente bombeado, el cual realiza oscilaciones super-Rabi \cite{strekalov2014} entre estados con grandes diferencias en su número de excitaciones \cite{bin2020}. En una línea relacionada, se han reportado oscilaciones de Rabi gigantes mediante excitones oscuros y brillantes en puntos cuánticos sin campo magnético externo \cite{vargas2022}, evidenciando que la dinámica coherente excitónica ofrece mecanismos más robustos para el control cuántico en estas plataformas. La emisión pura de paquetes se obtiene al abrir un canal disipativo que permite convertir dichas oscilaciones coherentes en emisión hacia el exterior, siempre que el bombeo óptico satisfaga una condición de resonancia adecuada entre la frecuencia del láser y las transiciones asistidas por fonones del sistema. Este resultado se enmarca en un panorama más amplio de ingeniería cuántica que incluye, además de las manipulaciones de fotones y fonones, la generación de fotones pareados con formas de onda controlables \cite{balic2005}, la transparencia electromagnéticamente inducida con pulsos de fotón único sintonizables \cite{eisaman2005}. El mecanismo fonónico difiere del esquema fotónico previo \cite{munoz2014}, que se basa en transiciones leapfrog mediante interacción de Jaynes-Cummings: aquí, el flip del QD está acompañado por una generación de $n$ fonones inducida por la interacción electrón-fonón. Sin embargo, el esquema basado en un único punto cuántico presenta limitaciones fundamentales relacionadas con la ausencia de efectos colectivos y modos híbridos que podrían emerger de la interacción coherente entre múltiples emisores, sugiriendo que el acoplamiento entre múltiples puntos cuánticos podría ofrecer ventajas significativas en términos de control, estabilidad paramétrica y escalabilidad.\\

Una extensión natural de este esquema consiste en considerar una molécula excitónica, formada por dos puntos cuánticos acoplados coherentemente mediante interacción tipo Förster \cite{forster1948}, en interacción con un modo fonónico confinado. La transferencia resonante de energía de Förster entre puntos cuánticos ha sido ampliamente estudiada en sistemas semiconductores \cite{kagan1996,crooker2003} y recientemente se ha desarrollado un tratamiento microscópico exacto que considera el acoplamiento dipolo-dipolo y el entorno fonónico compartido \cite{sirkina2025}. La presencia de dos emisores acoplados introduce nuevos grados de libertad físicos: la hibridación de los estados excitónicos, la interferencia cuántica entre trayectorias de emisión y la posibilidad de modos colectivos de emisión. Estos elementos pueden modificar sustancialmente las condiciones de resonancia de Stokes y la dinámica de emisión multifonónica.\\

En sistemas fotónicos, el acoplamiento resonante entre emisores cuánticos ha permitido observar fenómenos colectivos como estados subradiantes y superradiantes \cite{gonzaleztudela2013}, modificaciones sustanciales en las tasas de decaimiento espontáneo y corrimientos de Lamb mediante diseño de dependencias frecuenciales en átomos artificiales gigantes \cite{kockum2017}, y control acusto-óptico cuántico mediante estados ligados átomo-fotón en QED de guías de onda de baja velocidad \cite{calajo2019}. De manera análoga a las moléculas fotónicas en microcavidades ópticas \cite{vahala2003,boriskina2006}, que han demostrado capacidad de control espectral y espacial de la emisión mediante el acoplamiento entre resonadores, las moléculas excitónicas podrían ofrecer ventajas significativas en el contexto de emisión multifonónica: mayor control sobre las condiciones de resonancia de Stokes mediante la sintonización del acoplamiento tipo Förster, estabilidad paramétrica mejorada frente a fluctuaciones del sistema, y potencial acceso a regímenes de emisión colectiva de paquetes de fonones correlacionados.\\

Motivado por este panorama, el presente trabajo tiene como objetivo extender el esquema de emisión multifonónica de Bin et al. \cite{bin2020} al caso de una molécula excitónica formada por dos puntos cuánticos acoplados mediante interacción de Förster, en interacción con un modo fonónico confinado bajo condiciones de resonancia de Stokes. Este régimen ampliado, junto con la estructura de niveles más compleja, permite explorar si la emisión de paquetes de $n$ cuantos puede ser mejorada mediante efectos colectivos, y si la resonancia de Stokes ideal puede mantenerse sobre un amplio rango de parámetros en configuraciones de múltiples emisores. Se emplea el formalismo de sistemas cuánticos abiertos mediante la ecuación maestra de Lindblad \cite{lindblad1976} implementada en QuTiP \cite{johansson2011} para determinar cómo el acoplamiento de Förster afecta las funciones de correlación de orden superior $g^{(n)}$ calculadas según la teoría de coherencia de Glauber \cite{glauber1963}, las purezas de emisión \cite{munoz2018} y la estabilidad paramétrica de los paquetes generados, en comparación con el esquema de referencia de un único punto cuántico. Este estudio busca contribuir al diseño de fuentes acústicas cuánticas más estables y escalables para redes fonónicas \cite{habraken2012}, transferencia de estados mediada por fonones \cite{bienfait2019} y procesamiento de información cuántica, ampliando el ámbito de la fonónica cuántica con aplicaciones potenciales en metrología cuántica e ingeniería de dispositivos cuánticos tales como pistolas de $n$ fonones heraldadas ópticamente, dentro del marco teórico de la optomecánica de cavidades \cite{aspelmeyer2014} y sistemas híbridos luz-materia-fonón \cite{devereaux2007}.


\section{Objetivos}

\subsection*{Objetivo General}

Analizar la dinámica de emisión multifonónica en un sistema compuesto por una molécula excitónica acoplada a una cavidad acústica bajo condiciones de resonancia de Stokes, determinando cómo el acoplamiento de Förster y la hibridación excitónica modifican las propiedades estadísticas, la pureza y la estabilidad paramétrica de los paquetes de fonones generados.

\subsection*{Objetivos Específicos}

\begin{itemize}
  \item Formular el modelo teórico del sistema de dos puntos cuánticos acoplados mediante interacción de Förster que interactúan con un modo fonónico confinado, considerando excitación óptica coherente en condiciones de resonancia de Stokes de orden $n$.
  
  \item Implementar computacionalmente la ecuación maestra de Lindblad del sistema propuesto utilizando QuTiP, obteniendo la matriz densidad en el estado estacionario y verificando la convergencia numérica de las soluciones.

  \item Calcular las funciones de correlación de orden superior $g^{(n)}$ para $n=2,3,4,5$ y las purezas de emisión asociadas, caracterizando cuantitativamente la calidad de los paquetes de fonones emitidos y su dependencia con los parámetros del sistema.
  
  \item Comparar sistemáticamente los resultados obtenidos con el esquema de un único punto cuántico reportado en la literatura \cite{bin2020}, identificando ventajas, limitaciones y nuevos fenómenos físicos emergentes del acoplamiento excitónico coherente en la emisión multifonónica.
\end{itemize}