\documentclass[12pt,spanish,openany,letterpaper,pagesize]{scrbook}

\usepackage[utf8]{inputenc}
\usepackage[spanish]{babel}%escribir con acentos sin necesidad de comandos \'{} .
\usepackage[automark,headsepline]{scrlayer-scrpage}
\usepackage{epsfig}
\usepackage{epic}
\usepackage{eepic}
\usepackage{pgf}
\usepackage{graphicx}
\usepackage{amsmath}
\usepackage{amssymb}
\usepackage{amsfonts}
\usepackage{amsmath,amssymb,mathtools}
\usepackage{physics}
\usepackage{amsmath}
\usepackage{amssymb}
\usepackage{pgfplots}
\pgfplotsset{compat=1.18}
\usepackage{mathrsfs}
\usepackage{threeparttable}
\usepackage{amscd}
\usepackage{here}
\usepackage{graphicx}
\usepackage{lscape}
\usepackage{tabularx}
\usepackage{subfigure}
\usepackage{longtable}

% Comprimir citas numéricas consecutivas: [9-14] en lugar de [9,10,11,12,13,14]
\usepackage[numbers,sort&compress]{natbib}

% Hipervínculos (incluye entradas del índice/tabla de contenido)
\usepackage[hidelinks]{hyperref}
\hypersetup{linktoc=all}


\usepackage{rotating} %Para rotar texto, objetos y tablas seite. No se ve en DVI solo en PS. Seite 328 Hundebuch
                        %se usa junto con \rotate, \sidewidestable ....

\usepackage{mathrsfs}

\usepackage{mathrsfs}


\renewcommand{\theequation}{\thechapter-\arabic{equation}}
\renewcommand{\thefigure}{\textbf{\thechapter-\arabic{figure}}}
\renewcommand{\thetable}{\textbf{\thechapter-\arabic{table}}}


\textheight22.5cm \topmargin0cm \textwidth16.5cm
\oddsidemargin0.5cm \evensidemargin-0.5cm%

\clearpairofpagestyles
\ihead{\headmark}
\ohead{\pagemark}
\cfoot{}
\pagestyle{scrheadings}
\unitlength1mm %Define la unidad LE para Figuras
% Si quieres fórmulas alineadas a la izquierda, usa la opción fleqn en \documentclass o en amsmath.
% (p.ej. \documentclass[fleqn]{...} o \usepackage[fleqn]{amsmath})
% Eliminamos \mathindent porque no existe en este formato/clase.
\marginparwidth0cm
\parindent0cm %Define la distancia de la primera linea de un parrafo a la margen

%Para tablas,  redefine el backschlash en tablas donde se define la posici\'{o}n del texto en las
%casillas (con \centering \raggedright o \raggedleft)
\newcommand{\PreserveBackslash}[1]{\let\temp=\\#1\let\\=\temp}
\let\PBS=\PreserveBackslash

%Espacio entre lineas
\renewcommand{\baselinestretch}{1.1}

%Neuer Befehl f\"{u}r die Tabelle Eigenschaften der Aktivkohlen
\newcommand{\arr}[1]{\raisebox{1.5ex}[0cm][0cm]{#1}}

%Neue Kommandos
\usepackage{Befehle}
%Inicio del documento. Tener en cuenta que hay archivos auxiliares

\begin{document}
\pagenumbering{roman}
\newpage{\pagestyle{empty}\cleardoublepage}

\newpage
\begin{center}
\thispagestyle{empty} \vspace*{0cm} \textbf{\huge
Emisión de paquetes de n-fonones en una molécula excitónica mediante procesos de Stokes}\\[3.0cm]
\Large\textbf{Jhon Sebastian García Barrera}\\[3.5cm]
\small Trabajo de grado presentado como requisito para optar al
t\'{\i}tulo de:\\
\textbf{Físico}\\[3.0cm]
Director:\\
Ph.D. Edgar A. Gómez G.\\
Codirector:\\
Ph.D. Santiago Echeverri Arteaga\\[3.0cm]
Universidad del Quindío\\
Facultad de Ciencias Básicas y Tecnologías\\
Programa de Física\\
Armenia, Colombia\\
2026\\
\end{center}

\newpage{\pagestyle{empty}\cleardoublepage}

\newpage
\thispagestyle{empty} \textbf{}\normalsize
\\\\\\%
[4.0cm]

\begin{flushright}
\begin{minipage}{10cm}
    \noindent
        \small
        La sabiduría no se puede transmitir. La sabiduría que un sabio intenta comunicar siempre suena a locura para otro. El conocimiento se puede comunicar, pero la sabiduría no. Se puede encontrar, vivir, experimentar. Se puede expresar, pero no enseñar. Por eso cada camino es propio, y cada búsqueda verdadera exige tiempo, paciencia y fidelidad a uno mismo.\\\\
        \textbf{Hermann Hesse}\\
\end{minipage}
\end{flushright}

\newpage{\pagestyle{empty}\cleardoublepage}

\newpage
\thispagestyle{empty} \textbf{}\normalsize
\\\\\\%
\textbf{\LARGE Agradecimientos}
\addcontentsline{toc}{chapter}{\numberline{}Agradecimientos}\\\\


\newpage{\pagestyle{empty}\cleardoublepage}

\newpage
\textbf{\LARGE Resumen}
\addcontentsline{toc}{chapter}{\numberline{}Resumen}\\\\

\textbf{\small Palabras clave: }\\



\textbf{\LARGE Abstract}\\\\
\textbf{\small Key words: }\\

\renewcommand{\tablename}{\textbf{Tabla}}
\renewcommand{\figurename}{\textbf{Figura}}
\renewcommand{\listtablename}{Lista de Tablas}
\renewcommand{\listfigurename}{Lista de Figuras}
\renewcommand{\contentsname}{Contenido}

\tableofcontents
\cleardoublepage

%\include{Tab_Simbolos/TabSimbolosMSc}
%\include{Resumen}%\newcommand{\clearemptydoublepage}{\newpage{\pagestyle{empty}\cleardoublepage}}
\pagenumbering{arabic}
\chapter{Introducción}

La manipulación de estados cuánticos es uno de los tópicos principales de la 
ciencia moderna. En el caso de los fotones, una línea de investigación cada vez 
más popular es la física multifotónica 
\cite{kubanek2008,ota2011,gonzaleztudela2013,munoz2014,chang2016,kockum2017}, con 
aplicaciones potenciales para láseres multifotónicos \cite{gauthier1992}, 
superación del límite de difracción \cite{dangelo2001}, y metrología 
\cite{afek2010}. En particular, Muñoz et al. propusieron un esquema para la generación directa 
de estados de $n$ fotones en el mismo modo (paquetes de $n$ fotones) bajo la 
plataforma de electrodinámica cuántica de cavidades (cQED) \cite{munoz2014}.

Además de los fotones, los fonones (los cuantos de ondas mecánicas) han emergido 
como candidatos fuertes para la ingeniería de dispositivos cuánticos de estado 
sólido y comunicaciones cuánticas en chip 
\cite{balandin2005,manenti2017,noguchi2017,schuetz2015,arrangoiz2018}, con 
diversas ventajas distintivas. Primero, la velocidad de las ondas acústicas es 
significativamente más lenta que la velocidad de la luz, y por lo tanto es más 
adecuada para comunicaciones sobre distancias cortas, tales como unos pocos 
cientos de micrómetros o menos (es decir, comunicación en chip) 
\cite{gustafsson2014,kuzyk2018}; segundo, dado que los fonones solo pueden 
propagarse en un medio material, no sufren pérdidas por radiación al vacío; 
tercero, las cavidades fonónicas son altamente sintonizables, con demostraciones 
experimentales desde gigahertz (GHz) hasta terahertz (THz) 
\cite{weig2004,rozas2009,soykal2011,fainstein2013,xu2018,kharel2018,oconnell2010,
	macCabe2020,bolgar2018,wigger2021,kettler2021,chu2017,pirkkalainen2013,gokhale2020,
	zalalutdinov2021}; y cuarto, 
las numerosas técnicas experimentales desarrolladas por físicos de estado sólido 
\cite{ask2019,kharel2018,xu2018,akimov2017} se vuelven disponibles para tareas de 
procesamiento de información cuántica con fonones \cite{manenti2017,wan2021}. 
Consecuentemente, la fonónica cuántica ha progresado enormemente, incluyendo la 
investigación de láseres fonónicos \cite{kabuss2012,maryam2013,han2015}, redes 
cuánticas fonónicas \cite{lemonde2018,calajo2019}, la detección de la interacción 
electrón-fonón en puntos cuánticos dobles \cite{hartke2018,gullans2018}, y 
dispositivos acústicos cuánticos \cite{chu2017,manenti2017}.

La generación de estados cuánticos multifonónicos se convierte en una tarea 
importante de la fonónica, siendo requeridos para fuentes altamente no clásicas 
útiles en sensado y metrología cuántica \cite{zhang2018,chu2018,satzinger2018}, 
tecnologías cuánticas tales como memorias y transductores cuánticos 
\cite{schuetz2015,noguchi2017,arrangoiz2018}, y mediciones de precisión cuántica 
acústica mediante paquetes antibunched y estados NOON de fonones 
\cite{toyoda2015,zhang2018}.\\

Recientemente, se ha demostrado teóricamente que un punto cuántico acoplado a una nanocavidad acústica puede generar paquetes de $n$ fonones mediante un proceso de Stokes ópticamente bombeado, el cual realiza oscilaciones super-Rabi entre estados con grandes diferencias en su número de excitaciones \cite{bin2020,strekalov2014}. La emisión pura de paquetes se obtiene al abrir un canal disipativo que convierte las oscilaciones super-Rabi en emisión hacia el exterior, siempre que el bombeo óptico satisfaga la condición de resonancia entre la frecuencia del láser y las transiciones asistidas por fonones del sistema. El mecanismo fonónico difiere del esquema fotónico previo \cite{munoz2014}, que se basa en transiciones leapfrog mediante interacción de Jaynes-Cummings: aquí, el flip del QD está acompañado por una generación de $n$ fonones inducida por la interacción electrón-fonón. Este resultado enriquece la ingeniería cuántica de procesos de Stokes no lineales, más allá de las manipulaciones de fotones y fonones ampliamente aplicadas, tales como la generación de fotones pareados con formas de onda controlables \cite{balic2005}, la transparencia electromagnéticamente inducida con pulsos de fotón único sintonizables \cite{eisaman2005}, el enfriamiento por bandas laterales de osciladores mecánicos \cite{aspelmeyer2014}, y la emisión fonónica colectiva en sistemas de múltiples emisores \cite{droenner2017,devereaux2007}. Extendiendo este marco al considerar la estructura interna del excitón, se han reportado oscilaciones de Rabi gigantes entre estados excitónicos oscuros y brillantes en puntos cuánticos sin campo magnético externo \cite{vargas2022}, lo que enriquece el espacio de control coherente disponible en estas plataformas. Sin embargo, el esquema basado en un único punto cuántico presenta limitaciones inherentes relacionadas con la ausencia de efectos colectivos y modos híbridos que podrían emerger de la interacción coherente entre múltiples emisores, sugiriendo que el acoplamiento entre múltiples puntos cuánticos podría ofrecer ventajas significativas en términos de control, estabilidad paramétrica y escalabilidad.

Una extensión natural de este esquema consiste en considerar una molécula excitónica, formada por dos puntos cuánticos acoplados coherentemente mediante interacción tipo Förster \cite{forster1948}, en interacción con un modo fonónico confinado. > La transferencia resonante de energía de Förster entre puntos cuánticos ha sido ampliamente estudiada en sistemas semiconductores \cite{kagan1996,crooker2002} y recientemente se ha desarrollado un tratamiento microscópico exacto que considera el acoplamiento dipolo-dipolo y el entorno fonónico compartido \cite{hall2025}. La presencia de dos emisores acoplados introduce nuevos grados de libertad físicos: la hibridación de los estados excitónicos, la interferencia cuántica entre trayectorias de emisión y la posibilidad de modos colectivos de emisión. Estos elementos pueden modificar sustancialmente las condiciones de resonancia de Stokes y la dinámica de emisión multifonónica.\\

En sistemas fotónicos, el acoplamiento resonante entre emisores cuánticos ha permitido observar fenómenos colectivos como estados subradiantes y superradiantes \cite{gonzaleztudela2013}, modificaciones sustanciales en las tasas de decaimiento espontáneo y corrimientos de Lamb mediante diseño de dependencias frecuenciales en átomos artificiales gigantes \cite{kockum2017}, y control acusto-óptico cuántico mediante estados ligados átomo-fotón en QED de guías de onda de baja velocidad \cite{calajo2019}. De manera análoga a las moléculas fotónicas en microcavidades ópticas \cite{vahala2003,boriskina2006}, que han demostrado capacidad de control espectral y espacial de la emisión mediante el acoplamiento entre resonadores, las moléculas excitónicas podrían ofrecer ventajas significativas en el contexto de emisión multifonónica: mayor control sobre las condiciones de resonancia de Stokes mediante la sintonización del acoplamiento tipo Förster, estabilidad paramétrica mejorada frente a fluctuaciones del sistema, y potencial acceso a regímenes de emisión colectiva de paquetes de fonones correlacionados.\\

Motivado por este panorama, el presente trabajo tiene como objetivo extender el esquema de emisión multifonónica de Bin et al. \cite{bin2020} al caso de una molécula excitónica formada por dos puntos cuánticos acoplados mediante interacción de Förster, en interacción con un modo fonónico confinado bajo condiciones de resonancia de Stokes. Este régimen ampliado, junto con la estructura de niveles más compleja, permite explorar si la emisión de paquetes de $n$ cuantos puede ser mejorada mediante efectos colectivos, y si la resonancia de Stokes ideal puede mantenerse sobre un amplio rango de parámetros en configuraciones de múltiples emisores. Se emplea el formalismo de sistemas cuánticos abiertos mediante la ecuación maestra de Lindblad \cite{lindblad1976} implementada en QuTiP \cite{johansson2011} para determinar cómo el acoplamiento de Förster afecta las funciones de correlación de orden superior $g^{(n)}$ calculadas según la teoría de coherencia de Glauber \cite{glauber1963}, las purezas de emisión \cite{munoz2018} y la estabilidad paramétrica de los paquetes generados, en comparación con el esquema de referencia de un único punto cuántico. Este estudio busca contribuir al diseño de fuentes acústicas cuánticas más estables y escalables para redes fonónicas \cite{habraken2012}, transferencia de estados mediada por fonones \cite{bienfait2019} y procesamiento de información cuántica, ampliando el ámbito de la fonónica cuántica con aplicaciones potenciales en metrología cuántica e ingeniería de dispositivos cuánticos tales como pistolas de $n$ fonones heraldadas ópticamente, dentro del marco teórico de la optomecánica de cavidades \cite{aspelmeyer2014} y sistemas híbridos luz-materia-fonón \cite{devereaux2007}.


\section{Objetivos}

\subsection*{Objetivo General}

Analizar la dinámica de emisión multifonónica en un sistema compuesto por una molécula excitónica acoplada a una cavidad acústica bajo condiciones de resonancia de Stokes, determinando cómo el acoplamiento de Förster y la hibridación excitónica modifican las propiedades estadísticas, la pureza y la estabilidad paramétrica de los paquetes de fonones generados.

\subsection*{Objetivos Específicos}

\begin{itemize}
  \item Formular el modelo teórico del sistema de dos puntos cuánticos acoplados mediante interacción de Förster que interactúan con un modo fonónico confinado, considerando excitación óptica coherente en condiciones de resonancia de Stokes de orden $n$.
  
  \item Implementar computacionalmente la ecuación maestra de Lindblad del sistema propuesto utilizando QuTiP, obteniendo la matriz densidad en el estado estacionario y verificando la convergencia numérica de las soluciones.

  \item Calcular las funciones de correlación de orden superior $g^{(n)}$ para $n=2,3,4,5$ y las purezas de emisión asociadas, caracterizando cuantitativamente la calidad de los paquetes de fonones emitidos y su dependencia con los parámetros del sistema.
  
  \item Comparar sistemáticamente los resultados obtenidos con el esquema de un único punto cuántico reportado en la literatura \cite{bin2020}, identificando ventajas, limitaciones y nuevos fenómenos físicos emergentes del acoplamiento excitónico coherente en la emisión multifonónica.
\end{itemize}
\chapter{Marco Teórico}
El presente capítulo desarrolla el marco teórico necesario para describir la emisión de paquetes de $n$ fonones en una molécula excitónica acoplada a una cavidad acústica. Los puntos cuánticos (QDs, del inglés: quantum dot) son nanoestructuras semiconductoras que confinan portadores de carga en las tres direcciones espaciales \cite{vargas2022}, produciendo un espectro discreto de niveles de energía que permite modelarlos como emisores cuánticos artificiales de dos niveles, casi independientes del material específico. Gracias a esta estructura, los QDs constituyen plataformas privilegiadas para estudiar fenómenos de interacción luz--materia y luz--vibración, desde el intercambio coherente de excitación hasta regímenes de acoplamiento fuerte con campos ópticos o mecánicos \cite{borri2001,aspelmeyer2014}. Cuando dos QDs se encuentran suficientemente próximos, su interacción dipolo--dipolo da lugar al acoplamiento de Förster \cite{forster1948}, mediante el cual la excitación puede transferirse coherentemente de un emisor al otro, hibridando los estados excitónicos individuales en modos colectivos simétrico y antisimétrico. Esta hibridación modifica sustancialmente las condiciones de resonancia del sistema y abre la posibilidad de fenómenos colectivos de emisión \cite{kagan1996,crooker2003}.

En estas condiciones, un sistema de dos niveles acoplado a un modo fonónico confinado puede habilitar procesos multicuánticos; en particular, cuando se satisface una condición de resonancia de Stokes de orden $n$, es decir, una desintonía del láser igual a un múltiplo entero de la frecuencia del modo fonónico, se posibilita la emisión coherente de paquetes de $n$ fonones \cite{bin2020,devereaux2007}. La descripción completa de estas dinámicas requiere incluir los mecanismos de disipación del sistema, tales como las pérdidas de la cavidad, el decaimiento del QD y los procesos de decoherencia. Para ello, se emplea la ecuación maestra en forma de Lindblad, que en el marco de Born--Markov permite modelar la evolución de sistemas cuánticos abiertos y cuantificar la competencia entre procesos coherentes y disipativos \cite{lindblad1976,breuerpetruccione2002}.

\section{Dinámica coherente de un sistema de dos niveles}
\label{sec:dinamica_coherente_tls}

\subsection{Sistema de dos niveles}
Un sistema de dos niveles (TLS, del inglés: two-level system) es el modelo más fundamental en óptica cuántica para describir la interacción entre un emisor cuántico y un campo de luz \cite{gerry2005,fox2006}. En él, la estructura interna del emisor se reduce a dos estados relevantes: un estado base $|g\rangle$ y un estado excitado $|e\rangle$, los cuales pueden representarse en la base canónica de dos dimensiones como
%
\begin{equation}
    |g\rangle = \begin{pmatrix} 0 \\ 1 \end{pmatrix}, \qquad 
    |e\rangle = \begin{pmatrix} 1 \\ 0 \end{pmatrix}.
\end{equation}
%
En este marco, cualquier operador hermítico que actúe sobre el TLS puede representarse mediante una matriz hermítica $2\times 2$. El conjunto completo de operadores que generan el álgebra correspondiente está dado por $\{\hat{I}, \hat{\sigma}_+, \hat{\sigma}_-, \hat{\sigma}_z\}$, definidos como
%
\begin{equation}
    \hat{I} = \begin{pmatrix} 1 & 0 \\ 0 & 1 \end{pmatrix}, \quad
    \hat{\sigma}_+ = \begin{pmatrix} 0 & 1 \\ 0 & 0 \end{pmatrix}, \quad
    \hat{\sigma}_- = \begin{pmatrix} 0 & 0 \\ 1 & 0 \end{pmatrix}, \quad
    \hat{\sigma}_z = \begin{pmatrix} 1 & 0 \\ 0 & -1 \end{pmatrix},
\end{equation}
%
donde $\hat{\sigma}_+= |e\rangle\langle g|$ y $\hat{\sigma}_- = |g\rangle\langle e|$ son los operadores de subida y bajada que inducen transiciones entre los estados $|g\rangle$ y $|e\rangle$, y $\hat{\sigma}_z$ es la tercera componente del vector de Pauli, que describe la diferencia de población entre el estado excitado y el estado base \cite{gerry2005}.

El Hamiltoniano que describe el sistema libre, tomando el estado base como origen de energías, es
%
\begin{equation}
    \hat{H}_S = \hbar\omega_0\,|e\rangle\langle e| = 
    \frac{\hbar\omega_0}{2}\left(\hat{\sigma}_z + \hat{I}\right),
    \label{eq:HS}
\end{equation}
%
donde $\omega_0$ es la frecuencia de transición entre los niveles del sistema. El término proporcional a $\hat{I}$ representa un desplazamiento global de energía sin consecuencias físicas observables, por lo que habitualmente se omite. Este formalismo permite describir de manera general una amplia clase de sistemas cuánticos que exhiben transiciones discretas entre dos estados bien definidos, como átomos, iones atrapados, cúbits superconductores y, en particular, excitones en puntos cuánticos semiconductores \cite{fox2006}.

\subsection{Hamiltoniano de interacción y aproximación de onda rotante}

La interacción entre el TLS y un campo electromagnético se describe considerando al primero como un dipolo eléctrico. Bajo esta consideración, el Hamiltoniano de interacción que representa la energía potencial eléctrica se expande a su primer término no nulo como
%
\begin{equation}
    \hat{H}_\text{int} = -\hat{\mathbf{d}}\cdot\hat{\mathbf{E}},
    \label{eq:Hint_dipolo}
\end{equation}
%
donde $\hat{\mathbf{d}}$ es el operador de momento dipolar del TLS y $\hat{\mathbf{E}}$ es el operador del campo eléctrico. Dado que la paridad de los estados garantiza que los elementos diagonales de la matriz de momento dipolar son nulos, el operador $\hat{\mathbf{d}}$ puede descomponerse como
%
\begin{equation}
    \hat{\mathbf{d}} = \mathbf{d}\,\hat{\sigma}_+ + \mathbf{d}^*\hat{\sigma}_-,
\end{equation}
%
donde $\mathbf{d} = \langle e|\hat{\mathbf{d}}|g\rangle$ es el elemento matricial del momento dipolar entre los estados base y excitado. Para un campo clásico de amplitud $\mathcal{E}_0$ y frecuencia $\omega_L$, el campo eléctrico se expresa como una superposición de ondas de la forma $\mathcal{E}_0\,e^{\pm i\omega_L t}$. Incluyendo la dependencia temporal de los operadores $\hat{\sigma}_\pm$ en la imagen de interacción, dada por las ecuaciones de movimiento de Heisenberg,
%
\begin{equation}
    \frac{d\hat{\sigma}_+}{dt} = i\omega_0\,\hat{\sigma}_+, \qquad 
    \frac{d\hat{\sigma}_-}{dt} = -i\omega_0\,\hat{\sigma}_-,
\end{equation}
%
y sustituyendo en~\eqref{eq:Hint_dipolo}, el Hamiltoniano de interacción adquiere cuatro términos:
%
\begin{equation}
    \hat{H}_\text{int} = \hbar\left[
    g\,\hat{\sigma}_+\,e^{i(\omega_0-\omega_L)t} + 
    g^*\hat{\sigma}_-\,e^{-i(\omega_0-\omega_L)t} - 
    g\,\hat{\sigma}_+\,e^{i(\omega_0+\omega_L)t} - 
    g^*\hat{\sigma}_-\,e^{-i(\omega_0+\omega_L)t}
    \right],
\end{equation}
%
donde $g = -d\mathcal{E}_0/(2\hbar)$ es la constante de acoplamiento entre el TLS y el campo. Los dos primeros términos oscilan a la frecuencia de diferencia $|\omega_L - \omega_0| = |\Delta|$, mientras que los dos últimos oscilan a la frecuencia de suma $\omega_0 + \omega_L$. La aproximación de onda rotante (RWA, del inglés: Rotating Wave Approximation) consiste en despreciar los términos con frecuencia de suma, porque oscilan tan rápido que, en promedio, no tienen efecto apreciable en la dinámica del sistema promediada en tiempos significativos \cite{gerry2005,steck2007}. Esto se justifica cuando $|\Delta| = |\omega_0 - \omega_L|$ es mucho menor que $\omega_0 + \omega_L$, condición que se satisface cerca de resonancia. Al eliminar los términos no resonantes, que inducen procesos de intercambio de energía de dos fotones y rompen la conservación del número de excitaciones, el Hamiltoniano de interacción se simplifica a
%
\begin{equation}
    \hat{H}_\text{int} = \frac{\hbar\Omega}{2}\left(
    \hat{\sigma}_-\,e^{i\omega_L t} + \hat{\sigma}_+\,e^{-i\omega_L t}
    \right),
    \label{eq:Hint_RWA}
\end{equation}
%
donde $\Omega$ es la frecuencia de Rabi, que cuantifica la intensidad del acoplamiento entre el campo y el emisor \cite{gerry2005}. El Hamiltoniano total del sistema en el marco rotante es entonces
%
\begin{equation}
    \hat{H} = -\hbar\Delta\,|e\rangle\langle e| + 
    \frac{\hbar\Omega}{2}\left(\hat{\sigma}_+ + \hat{\sigma}_-\right),
    \label{eq:H_RWA}
\end{equation}
%
donde $\Delta = \omega_L - \omega_0$ es la desintonía entre la frecuencia del láser y la frecuencia de transición del sistema.

\subsection{Oscilaciones de Rabi}

El sistema descrito por $\hat{H}_S + \hat{H}_\text{int}$ experimenta oscilaciones entre los estados $|g\rangle$ y $|e\rangle$, conocidas como oscilaciones de Rabi. Para un sistema que parte del estado base, la probabilidad de encontrarlo en el estado excitado en el tiempo $t$ es \cite{gerry2005,fox2006}
%
\begin{equation}
    P_e(t) = \frac{\Omega^2}{\Omega_R^2}\,\sin^2\!\left(\frac{\Omega_R\,t}{2}\right),
    \label{eq:Pe}
\end{equation}
%
donde $\Omega_R = \sqrt{\Omega^2 + \Delta^2}$ es la frecuencia de Rabi generalizada. La amplitud de las oscilaciones depende directamente del valor de la desintonía $\Delta$ entre la frecuencia del láser $\omega_L$ y la frecuencia de transición $\omega_0$, como se ilustra en la figura~\ref{fig:rabi_oscilaciones}. Precisamente en resonancia ($\Delta = 0$), el estado excitado y el estado base son degenerados en el marco rotante, la transferencia de población entre $|g\rangle$ y $|e\rangle$ es completa y la frecuencia de oscilación es mínima. Este punto de resonancia es la clave para producir oscilaciones de Rabi en cualquier sistema \cite{fox2006}.

\begin{figure}[H]
    \centering
    \includegraphics[width=0.7\textwidth]{Kap2/rabi.png}
    \caption{Probabilidad de encontrar el sistema en el estado excitado $P_e(t)$ cuando inicialmente está en el estado base, para diferentes valores de desintonía $\Delta = \omega_L - \omega_0$. Tomado de \cite{steck2007}.}
    \label{fig:rabi_oscilaciones}
\end{figure}

Es importante destacar que las oscilaciones de Rabi son un fenómeno puramente coherente que no involucra disipación. En sistemas reales, los mecanismos de decaimiento espontáneo y decoherencia amortiguan estas oscilaciones en escalas temporales características, cuyo tratamiento se aborda en la sección~\ref{sec:sistemas_abiertos}. En el contexto del presente trabajo, el resultado~\eqref{eq:Pe} constituye el punto de partida para comprender las oscilaciones super-Rabi que emergen en el sistema electrón-fonón bajo condiciones de resonancia de Stokes \cite{bin2020,vargas2022}, donde el intercambio coherente ocurre entre estados separados por $n$ cuantos de excitación a través de un mecanismo no lineal mediado por el acoplamiento con el modo fonónico de la cavidad acústica, como se desarrolla en la sección~\ref{sec:stokes}.

\section{Interacción electrón--fonón}
\label{sec:electron_fonon}

\subsection{Acoplamiento electrón--fonón en puntos cuánticos}

En la sección anterior se describió la dinámica coherente de un TLS bajo la acción de un campo óptico clásico. Sin embargo, en un punto cuántico embebido en una cavidad acústica, el excitón no solo interactúa con el campo de luz: también lo hace con los modos vibracionales del cristal a través del acoplamiento electrón--fonón. Este acoplamiento surge de la dependencia de la energía de banda del semiconductor con la deformación de la red, de manera que las vibraciones del cristal modulan la energía de los portadores confinados en el QD \cite{borri2001,stock2011}.

En el contexto de este trabajo, el QD se modela como un TLS con un estado de valencia $|v\rangle$ y un estado de conducción $|c\rangle$, los cuales corresponden, respectivamente, al estado base $|g\rangle$ y al estado excitado $|e\rangle$ introducidos en la sección~\ref{sec:dinamica_coherente_tls}. El acoplamiento electrón--fonón relevante en este sistema es diagonal en el espacio electrónico: solo el estado de conducción $|c\rangle$ desplaza el potencial del oscilador fonónico, mientras que el estado de valencia $|v\rangle$ no lo hace. Físicamente, esto refleja el hecho de que la presencia del electrón en la banda de conducción deforma la red cristalina, induciendo la emisión o absorción de fonones. El acoplamiento queda entonces gobernado por el proyector sobre el estado excitado, $\hat{\sigma}^\dagger\hat{\sigma} = |c\rangle\langle c|$, y no por el operador de inversión de población $\hat{\sigma}_z$, cuya acción es simétrica respecto a ambos estados \cite{bin2020,soykal2011}. La intensidad de este proceso queda caracterizada por el parámetro adimensional $\lambda/\omega_b$, que cuantifica el desplazamiento del modo fonónico inducido por la ocupación del estado excitado \cite{bin2020,alkauskas2014}.

\subsection{Molécula excitónica: acoplamiento de Förster e hibridación}

La extensión natural del modelo anterior consiste en considerar dos puntos cuánticos idénticos acoplados a la misma cavidad acústica. Cuando dos QDs se encuentran suficientemente próximos, su interacción dipolo--dipolo da lugar al acoplamiento de Förster \cite{forster1948}, un mecanismo de transferencia resonante de energía de naturaleza coherente, cuya constante de acoplamiento $J$ depende de la distancia entre los emisores y del solapamiento de sus respectivos momentos dipolares de transición \cite{kagan1996,crooker2003,sirkina2025}. En el límite en que la separación entre los QDs es pequeña comparada con la longitud de onda del campo óptico, este acoplamiento puede describirse mediante el término
%
\begin{equation}
    \hat{H}_J = J\left(\hat{\sigma}_1^\dagger\hat{\sigma}_2 + 
    \hat{\sigma}_2^\dagger\hat{\sigma}_1\right),
    \label{eq:Forster}
\end{equation}
%
donde $\hat{\sigma}_i^\dagger$ ($\hat{\sigma}_i$) es el operador de creación (aniquilación) del excitón en el $i$-ésimo punto cuántico. El Hamiltoniano total de la molécula excitónica acoplada a la cavidad acústica es entonces
%
\begin{equation}
    \hat{H}_\text{tot} = \omega_b\hat{b}^\dagger\hat{b} 
    - \Delta\sum_{i=1}^{2}\hat{\sigma}_i^\dagger\hat{\sigma}_i 
    + \lambda\sum_{i=1}^{2}\hat{\sigma}_i^\dagger\hat{\sigma}_i(\hat{b}^\dagger + \hat{b}) 
    + \Omega\sum_{i=1}^{2}\left(\hat{\sigma}_i + \hat{\sigma}_i^\dagger\right)
    + J\left(\hat{\sigma}_1^\dagger\hat{\sigma}_2 + \hat{\sigma}_2^\dagger\hat{\sigma}_1\right),
    \label{eq:H_total}
\end{equation}
%
donde se ha asumido que ambos QDs son idénticos, comparten la misma desintonía $\Delta$, y están acoplados al modo fonónico con la misma constante $\lambda$.

La presencia del término de Förster \eqref{eq:Forster} tiene una consecuencia inmediata sobre la estructura de estados del sistema: los estados excitónicos individuales $|c\rangle_1|v\rangle_2$ y $|v\rangle_1|c\rangle_2$ dejan de ser eigenstados del Hamiltoniano. En su lugar, la diagonalización del sector de una excitación electrónica produce dos nuevos eigenstados colectivos
%
\begin{equation}
    |+\rangle = \frac{1}{\sqrt{2}}\left(|c\rangle_1|v\rangle_2 
    + |v\rangle_1|c\rangle_2\right), \qquad
    |-\rangle = \frac{1}{\sqrt{2}}\left(|c\rangle_1|v\rangle_2 
    - |v\rangle_1|c\rangle_2\right),
    \label{eq:estados_hibridados}
\end{equation}
%
conocidos como el estado \textit{simétrico} $|+\rangle$ y el estado 
\textit{antisimétrico} $|-\rangle$, con energías $\omega_\sigma + J$ y $\omega_\sigma - J$, respectivamente. El primero acopla constructivamente al campo óptico externo y al modo fonónico, mientras que el segundo lo hace de forma destructiva: en el límite de QDs idénticos queda desacoplado del láser y se comporta como un estado oscuro desde el punto de vista óptico \cite{gonzaleztudela2013}. Esta hibridación modifica de manera fundamental el espectro de resonancias del sistema respecto al caso d un único emisor: el acoplamiento de Förster desdobla los niveles excitónicos en dos ramas separadas por $2J$, lo cual altera cualitativamente las condiciones bajo las que puede ocurrir la emisión coherente de fonones. El análisis detallado de cómo esta estructura de niveles determina las resonancias multifonónicas del sistema se desarrolla en la sección~\ref{sec:stokes}.

\section{Sistemas cu\'anticos abiertos}
\label{sec:sistemas_abiertos}
\subsection{Aproximaci\'on de Born--Markov}
\subsection{Ecuaci\'on maestra en forma de Lindblad}

\section{Procesos de Stokes y emisi\'on multifon\'onica}
\label{sec:stokes}
\subsection{Resonancias de Stokes asistidas por fonones}
\subsection{Oscilaciones super-Rabi y emisi\'on de paquetes de fonones}

\section{Observables de emisi\'on multifon\'onica}
\subsection{Teor\'ia de coherencia de Glauber}
\subsection{Funciones de correlaci\'on de orden superior $g^{(n)}$}
\subsection{Pureza de emisi\'on}

\chapter{Resultados y Discusión}

\begin{figure}[H]
	\centering
	\resizebox{0.95\textwidth}{!}{%
		%% Creator: Matplotlib, PGF backend
%%
%% To include the figure in your LaTeX document, write
%%   \input{<filename>.pgf}
%%
%% Make sure the required packages are loaded in your preamble
%%   \usepackage{pgf}
%%
%% Also ensure that all the required font packages are loaded; for instance,
%% the lmodern package is sometimes necessary when using math font.
%%   \usepackage{lmodern}
%%
%% Figures using additional raster images can only be included by \input if
%% they are in the same directory as the main LaTeX file. For loading figures
%% from other directories you can use the `import` package
%%   \usepackage{import}
%%
%% and then include the figures with
%%   \import{<path to file>}{<filename>.pgf}
%%
%% Matplotlib used the following preamble
%%   \def\mathdefault#1{#1}
%%   \everymath=\expandafter{\the\everymath\displaystyle}
%%   \IfFileExists{scrextend.sty}{
%%     \usepackage[fontsize=13.000000pt]{scrextend}
%%   }{
%%     \renewcommand{\normalsize}{\fontsize{13.000000}{15.600000}\selectfont}
%%     \normalsize
%%   }
%%   
%%   \makeatletter\@ifpackageloaded{underscore}{}{\usepackage[strings]{underscore}}\makeatother
%%
\begingroup%
\makeatletter%
\begin{pgfpicture}%
\pgfpathrectangle{\pgfpointorigin}{\pgfqpoint{11.000000in}{7.500000in}}%
\pgfusepath{use as bounding box, clip}%
\begin{pgfscope}%
\pgfsetbuttcap%
\pgfsetmiterjoin%
\definecolor{currentfill}{rgb}{1.000000,1.000000,1.000000}%
\pgfsetfillcolor{currentfill}%
\pgfsetlinewidth{0.000000pt}%
\definecolor{currentstroke}{rgb}{1.000000,1.000000,1.000000}%
\pgfsetstrokecolor{currentstroke}%
\pgfsetdash{}{0pt}%
\pgfpathmoveto{\pgfqpoint{0.000000in}{0.000000in}}%
\pgfpathlineto{\pgfqpoint{11.000000in}{0.000000in}}%
\pgfpathlineto{\pgfqpoint{11.000000in}{7.500000in}}%
\pgfpathlineto{\pgfqpoint{0.000000in}{7.500000in}}%
\pgfpathlineto{\pgfqpoint{0.000000in}{0.000000in}}%
\pgfpathclose%
\pgfusepath{fill}%
\end{pgfscope}%
\begin{pgfscope}%
\pgfsetbuttcap%
\pgfsetmiterjoin%
\definecolor{currentfill}{rgb}{1.000000,1.000000,1.000000}%
\pgfsetfillcolor{currentfill}%
\pgfsetlinewidth{0.000000pt}%
\definecolor{currentstroke}{rgb}{0.000000,0.000000,0.000000}%
\pgfsetstrokecolor{currentstroke}%
\pgfsetstrokeopacity{0.000000}%
\pgfsetdash{}{0pt}%
\pgfpathmoveto{\pgfqpoint{0.390138in}{5.350738in}}%
\pgfpathlineto{\pgfqpoint{10.805000in}{5.350738in}}%
\pgfpathlineto{\pgfqpoint{10.805000in}{7.235556in}}%
\pgfpathlineto{\pgfqpoint{0.390138in}{7.235556in}}%
\pgfpathlineto{\pgfqpoint{0.390138in}{5.350738in}}%
\pgfpathclose%
\pgfusepath{fill}%
\end{pgfscope}%
\begin{pgfscope}%
\pgfsetbuttcap%
\pgfsetroundjoin%
\definecolor{currentfill}{rgb}{0.000000,0.000000,0.000000}%
\pgfsetfillcolor{currentfill}%
\pgfsetlinewidth{0.803000pt}%
\definecolor{currentstroke}{rgb}{0.000000,0.000000,0.000000}%
\pgfsetstrokecolor{currentstroke}%
\pgfsetdash{}{0pt}%
\pgfsys@defobject{currentmarker}{\pgfqpoint{0.000000in}{-0.048611in}}{\pgfqpoint{0.000000in}{0.000000in}}{%
\pgfpathmoveto{\pgfqpoint{0.000000in}{0.000000in}}%
\pgfpathlineto{\pgfqpoint{0.000000in}{-0.048611in}}%
\pgfusepath{stroke,fill}%
}%
\begin{pgfscope}%
\pgfsys@transformshift{2.371931in}{5.350738in}%
\pgfsys@useobject{currentmarker}{}%
\end{pgfscope}%
\end{pgfscope}%
\begin{pgfscope}%
\pgfsetbuttcap%
\pgfsetroundjoin%
\definecolor{currentfill}{rgb}{0.000000,0.000000,0.000000}%
\pgfsetfillcolor{currentfill}%
\pgfsetlinewidth{0.803000pt}%
\definecolor{currentstroke}{rgb}{0.000000,0.000000,0.000000}%
\pgfsetstrokecolor{currentstroke}%
\pgfsetdash{}{0pt}%
\pgfsys@defobject{currentmarker}{\pgfqpoint{0.000000in}{-0.048611in}}{\pgfqpoint{0.000000in}{0.000000in}}{%
\pgfpathmoveto{\pgfqpoint{0.000000in}{0.000000in}}%
\pgfpathlineto{\pgfqpoint{0.000000in}{-0.048611in}}%
\pgfusepath{stroke,fill}%
}%
\begin{pgfscope}%
\pgfsys@transformshift{6.335517in}{5.350738in}%
\pgfsys@useobject{currentmarker}{}%
\end{pgfscope}%
\end{pgfscope}%
\begin{pgfscope}%
\pgfsetbuttcap%
\pgfsetroundjoin%
\definecolor{currentfill}{rgb}{0.000000,0.000000,0.000000}%
\pgfsetfillcolor{currentfill}%
\pgfsetlinewidth{0.803000pt}%
\definecolor{currentstroke}{rgb}{0.000000,0.000000,0.000000}%
\pgfsetstrokecolor{currentstroke}%
\pgfsetdash{}{0pt}%
\pgfsys@defobject{currentmarker}{\pgfqpoint{0.000000in}{-0.048611in}}{\pgfqpoint{0.000000in}{0.000000in}}{%
\pgfpathmoveto{\pgfqpoint{0.000000in}{0.000000in}}%
\pgfpathlineto{\pgfqpoint{0.000000in}{-0.048611in}}%
\pgfusepath{stroke,fill}%
}%
\begin{pgfscope}%
\pgfsys@transformshift{10.299103in}{5.350738in}%
\pgfsys@useobject{currentmarker}{}%
\end{pgfscope}%
\end{pgfscope}%
\begin{pgfscope}%
\pgfsetbuttcap%
\pgfsetroundjoin%
\definecolor{currentfill}{rgb}{0.000000,0.000000,0.000000}%
\pgfsetfillcolor{currentfill}%
\pgfsetlinewidth{0.602250pt}%
\definecolor{currentstroke}{rgb}{0.000000,0.000000,0.000000}%
\pgfsetstrokecolor{currentstroke}%
\pgfsetdash{}{0pt}%
\pgfsys@defobject{currentmarker}{\pgfqpoint{0.000000in}{-0.027778in}}{\pgfqpoint{0.000000in}{0.000000in}}{%
\pgfpathmoveto{\pgfqpoint{0.000000in}{0.000000in}}%
\pgfpathlineto{\pgfqpoint{0.000000in}{-0.027778in}}%
\pgfusepath{stroke,fill}%
}%
\begin{pgfscope}%
\pgfsys@transformshift{0.986717in}{5.350738in}%
\pgfsys@useobject{currentmarker}{}%
\end{pgfscope}%
\end{pgfscope}%
\begin{pgfscope}%
\pgfsetbuttcap%
\pgfsetroundjoin%
\definecolor{currentfill}{rgb}{0.000000,0.000000,0.000000}%
\pgfsetfillcolor{currentfill}%
\pgfsetlinewidth{0.602250pt}%
\definecolor{currentstroke}{rgb}{0.000000,0.000000,0.000000}%
\pgfsetstrokecolor{currentstroke}%
\pgfsetdash{}{0pt}%
\pgfsys@defobject{currentmarker}{\pgfqpoint{0.000000in}{-0.027778in}}{\pgfqpoint{0.000000in}{0.000000in}}{%
\pgfpathmoveto{\pgfqpoint{0.000000in}{0.000000in}}%
\pgfpathlineto{\pgfqpoint{0.000000in}{-0.027778in}}%
\pgfusepath{stroke,fill}%
}%
\begin{pgfscope}%
\pgfsys@transformshift{1.335693in}{5.350738in}%
\pgfsys@useobject{currentmarker}{}%
\end{pgfscope}%
\end{pgfscope}%
\begin{pgfscope}%
\pgfsetbuttcap%
\pgfsetroundjoin%
\definecolor{currentfill}{rgb}{0.000000,0.000000,0.000000}%
\pgfsetfillcolor{currentfill}%
\pgfsetlinewidth{0.602250pt}%
\definecolor{currentstroke}{rgb}{0.000000,0.000000,0.000000}%
\pgfsetstrokecolor{currentstroke}%
\pgfsetdash{}{0pt}%
\pgfsys@defobject{currentmarker}{\pgfqpoint{0.000000in}{-0.027778in}}{\pgfqpoint{0.000000in}{0.000000in}}{%
\pgfpathmoveto{\pgfqpoint{0.000000in}{0.000000in}}%
\pgfpathlineto{\pgfqpoint{0.000000in}{-0.027778in}}%
\pgfusepath{stroke,fill}%
}%
\begin{pgfscope}%
\pgfsys@transformshift{1.583296in}{5.350738in}%
\pgfsys@useobject{currentmarker}{}%
\end{pgfscope}%
\end{pgfscope}%
\begin{pgfscope}%
\pgfsetbuttcap%
\pgfsetroundjoin%
\definecolor{currentfill}{rgb}{0.000000,0.000000,0.000000}%
\pgfsetfillcolor{currentfill}%
\pgfsetlinewidth{0.602250pt}%
\definecolor{currentstroke}{rgb}{0.000000,0.000000,0.000000}%
\pgfsetstrokecolor{currentstroke}%
\pgfsetdash{}{0pt}%
\pgfsys@defobject{currentmarker}{\pgfqpoint{0.000000in}{-0.027778in}}{\pgfqpoint{0.000000in}{0.000000in}}{%
\pgfpathmoveto{\pgfqpoint{0.000000in}{0.000000in}}%
\pgfpathlineto{\pgfqpoint{0.000000in}{-0.027778in}}%
\pgfusepath{stroke,fill}%
}%
\begin{pgfscope}%
\pgfsys@transformshift{1.775352in}{5.350738in}%
\pgfsys@useobject{currentmarker}{}%
\end{pgfscope}%
\end{pgfscope}%
\begin{pgfscope}%
\pgfsetbuttcap%
\pgfsetroundjoin%
\definecolor{currentfill}{rgb}{0.000000,0.000000,0.000000}%
\pgfsetfillcolor{currentfill}%
\pgfsetlinewidth{0.602250pt}%
\definecolor{currentstroke}{rgb}{0.000000,0.000000,0.000000}%
\pgfsetstrokecolor{currentstroke}%
\pgfsetdash{}{0pt}%
\pgfsys@defobject{currentmarker}{\pgfqpoint{0.000000in}{-0.027778in}}{\pgfqpoint{0.000000in}{0.000000in}}{%
\pgfpathmoveto{\pgfqpoint{0.000000in}{0.000000in}}%
\pgfpathlineto{\pgfqpoint{0.000000in}{-0.027778in}}%
\pgfusepath{stroke,fill}%
}%
\begin{pgfscope}%
\pgfsys@transformshift{1.932272in}{5.350738in}%
\pgfsys@useobject{currentmarker}{}%
\end{pgfscope}%
\end{pgfscope}%
\begin{pgfscope}%
\pgfsetbuttcap%
\pgfsetroundjoin%
\definecolor{currentfill}{rgb}{0.000000,0.000000,0.000000}%
\pgfsetfillcolor{currentfill}%
\pgfsetlinewidth{0.602250pt}%
\definecolor{currentstroke}{rgb}{0.000000,0.000000,0.000000}%
\pgfsetstrokecolor{currentstroke}%
\pgfsetdash{}{0pt}%
\pgfsys@defobject{currentmarker}{\pgfqpoint{0.000000in}{-0.027778in}}{\pgfqpoint{0.000000in}{0.000000in}}{%
\pgfpathmoveto{\pgfqpoint{0.000000in}{0.000000in}}%
\pgfpathlineto{\pgfqpoint{0.000000in}{-0.027778in}}%
\pgfusepath{stroke,fill}%
}%
\begin{pgfscope}%
\pgfsys@transformshift{2.064947in}{5.350738in}%
\pgfsys@useobject{currentmarker}{}%
\end{pgfscope}%
\end{pgfscope}%
\begin{pgfscope}%
\pgfsetbuttcap%
\pgfsetroundjoin%
\definecolor{currentfill}{rgb}{0.000000,0.000000,0.000000}%
\pgfsetfillcolor{currentfill}%
\pgfsetlinewidth{0.602250pt}%
\definecolor{currentstroke}{rgb}{0.000000,0.000000,0.000000}%
\pgfsetstrokecolor{currentstroke}%
\pgfsetdash{}{0pt}%
\pgfsys@defobject{currentmarker}{\pgfqpoint{0.000000in}{-0.027778in}}{\pgfqpoint{0.000000in}{0.000000in}}{%
\pgfpathmoveto{\pgfqpoint{0.000000in}{0.000000in}}%
\pgfpathlineto{\pgfqpoint{0.000000in}{-0.027778in}}%
\pgfusepath{stroke,fill}%
}%
\begin{pgfscope}%
\pgfsys@transformshift{2.179875in}{5.350738in}%
\pgfsys@useobject{currentmarker}{}%
\end{pgfscope}%
\end{pgfscope}%
\begin{pgfscope}%
\pgfsetbuttcap%
\pgfsetroundjoin%
\definecolor{currentfill}{rgb}{0.000000,0.000000,0.000000}%
\pgfsetfillcolor{currentfill}%
\pgfsetlinewidth{0.602250pt}%
\definecolor{currentstroke}{rgb}{0.000000,0.000000,0.000000}%
\pgfsetstrokecolor{currentstroke}%
\pgfsetdash{}{0pt}%
\pgfsys@defobject{currentmarker}{\pgfqpoint{0.000000in}{-0.027778in}}{\pgfqpoint{0.000000in}{0.000000in}}{%
\pgfpathmoveto{\pgfqpoint{0.000000in}{0.000000in}}%
\pgfpathlineto{\pgfqpoint{0.000000in}{-0.027778in}}%
\pgfusepath{stroke,fill}%
}%
\begin{pgfscope}%
\pgfsys@transformshift{2.281249in}{5.350738in}%
\pgfsys@useobject{currentmarker}{}%
\end{pgfscope}%
\end{pgfscope}%
\begin{pgfscope}%
\pgfsetbuttcap%
\pgfsetroundjoin%
\definecolor{currentfill}{rgb}{0.000000,0.000000,0.000000}%
\pgfsetfillcolor{currentfill}%
\pgfsetlinewidth{0.602250pt}%
\definecolor{currentstroke}{rgb}{0.000000,0.000000,0.000000}%
\pgfsetstrokecolor{currentstroke}%
\pgfsetdash{}{0pt}%
\pgfsys@defobject{currentmarker}{\pgfqpoint{0.000000in}{-0.027778in}}{\pgfqpoint{0.000000in}{0.000000in}}{%
\pgfpathmoveto{\pgfqpoint{0.000000in}{0.000000in}}%
\pgfpathlineto{\pgfqpoint{0.000000in}{-0.027778in}}%
\pgfusepath{stroke,fill}%
}%
\begin{pgfscope}%
\pgfsys@transformshift{2.968510in}{5.350738in}%
\pgfsys@useobject{currentmarker}{}%
\end{pgfscope}%
\end{pgfscope}%
\begin{pgfscope}%
\pgfsetbuttcap%
\pgfsetroundjoin%
\definecolor{currentfill}{rgb}{0.000000,0.000000,0.000000}%
\pgfsetfillcolor{currentfill}%
\pgfsetlinewidth{0.602250pt}%
\definecolor{currentstroke}{rgb}{0.000000,0.000000,0.000000}%
\pgfsetstrokecolor{currentstroke}%
\pgfsetdash{}{0pt}%
\pgfsys@defobject{currentmarker}{\pgfqpoint{0.000000in}{-0.027778in}}{\pgfqpoint{0.000000in}{0.000000in}}{%
\pgfpathmoveto{\pgfqpoint{0.000000in}{0.000000in}}%
\pgfpathlineto{\pgfqpoint{0.000000in}{-0.027778in}}%
\pgfusepath{stroke,fill}%
}%
\begin{pgfscope}%
\pgfsys@transformshift{3.317486in}{5.350738in}%
\pgfsys@useobject{currentmarker}{}%
\end{pgfscope}%
\end{pgfscope}%
\begin{pgfscope}%
\pgfsetbuttcap%
\pgfsetroundjoin%
\definecolor{currentfill}{rgb}{0.000000,0.000000,0.000000}%
\pgfsetfillcolor{currentfill}%
\pgfsetlinewidth{0.602250pt}%
\definecolor{currentstroke}{rgb}{0.000000,0.000000,0.000000}%
\pgfsetstrokecolor{currentstroke}%
\pgfsetdash{}{0pt}%
\pgfsys@defobject{currentmarker}{\pgfqpoint{0.000000in}{-0.027778in}}{\pgfqpoint{0.000000in}{0.000000in}}{%
\pgfpathmoveto{\pgfqpoint{0.000000in}{0.000000in}}%
\pgfpathlineto{\pgfqpoint{0.000000in}{-0.027778in}}%
\pgfusepath{stroke,fill}%
}%
\begin{pgfscope}%
\pgfsys@transformshift{3.565089in}{5.350738in}%
\pgfsys@useobject{currentmarker}{}%
\end{pgfscope}%
\end{pgfscope}%
\begin{pgfscope}%
\pgfsetbuttcap%
\pgfsetroundjoin%
\definecolor{currentfill}{rgb}{0.000000,0.000000,0.000000}%
\pgfsetfillcolor{currentfill}%
\pgfsetlinewidth{0.602250pt}%
\definecolor{currentstroke}{rgb}{0.000000,0.000000,0.000000}%
\pgfsetstrokecolor{currentstroke}%
\pgfsetdash{}{0pt}%
\pgfsys@defobject{currentmarker}{\pgfqpoint{0.000000in}{-0.027778in}}{\pgfqpoint{0.000000in}{0.000000in}}{%
\pgfpathmoveto{\pgfqpoint{0.000000in}{0.000000in}}%
\pgfpathlineto{\pgfqpoint{0.000000in}{-0.027778in}}%
\pgfusepath{stroke,fill}%
}%
\begin{pgfscope}%
\pgfsys@transformshift{3.757145in}{5.350738in}%
\pgfsys@useobject{currentmarker}{}%
\end{pgfscope}%
\end{pgfscope}%
\begin{pgfscope}%
\pgfsetbuttcap%
\pgfsetroundjoin%
\definecolor{currentfill}{rgb}{0.000000,0.000000,0.000000}%
\pgfsetfillcolor{currentfill}%
\pgfsetlinewidth{0.602250pt}%
\definecolor{currentstroke}{rgb}{0.000000,0.000000,0.000000}%
\pgfsetstrokecolor{currentstroke}%
\pgfsetdash{}{0pt}%
\pgfsys@defobject{currentmarker}{\pgfqpoint{0.000000in}{-0.027778in}}{\pgfqpoint{0.000000in}{0.000000in}}{%
\pgfpathmoveto{\pgfqpoint{0.000000in}{0.000000in}}%
\pgfpathlineto{\pgfqpoint{0.000000in}{-0.027778in}}%
\pgfusepath{stroke,fill}%
}%
\begin{pgfscope}%
\pgfsys@transformshift{3.914065in}{5.350738in}%
\pgfsys@useobject{currentmarker}{}%
\end{pgfscope}%
\end{pgfscope}%
\begin{pgfscope}%
\pgfsetbuttcap%
\pgfsetroundjoin%
\definecolor{currentfill}{rgb}{0.000000,0.000000,0.000000}%
\pgfsetfillcolor{currentfill}%
\pgfsetlinewidth{0.602250pt}%
\definecolor{currentstroke}{rgb}{0.000000,0.000000,0.000000}%
\pgfsetstrokecolor{currentstroke}%
\pgfsetdash{}{0pt}%
\pgfsys@defobject{currentmarker}{\pgfqpoint{0.000000in}{-0.027778in}}{\pgfqpoint{0.000000in}{0.000000in}}{%
\pgfpathmoveto{\pgfqpoint{0.000000in}{0.000000in}}%
\pgfpathlineto{\pgfqpoint{0.000000in}{-0.027778in}}%
\pgfusepath{stroke,fill}%
}%
\begin{pgfscope}%
\pgfsys@transformshift{4.046740in}{5.350738in}%
\pgfsys@useobject{currentmarker}{}%
\end{pgfscope}%
\end{pgfscope}%
\begin{pgfscope}%
\pgfsetbuttcap%
\pgfsetroundjoin%
\definecolor{currentfill}{rgb}{0.000000,0.000000,0.000000}%
\pgfsetfillcolor{currentfill}%
\pgfsetlinewidth{0.602250pt}%
\definecolor{currentstroke}{rgb}{0.000000,0.000000,0.000000}%
\pgfsetstrokecolor{currentstroke}%
\pgfsetdash{}{0pt}%
\pgfsys@defobject{currentmarker}{\pgfqpoint{0.000000in}{-0.027778in}}{\pgfqpoint{0.000000in}{0.000000in}}{%
\pgfpathmoveto{\pgfqpoint{0.000000in}{0.000000in}}%
\pgfpathlineto{\pgfqpoint{0.000000in}{-0.027778in}}%
\pgfusepath{stroke,fill}%
}%
\begin{pgfscope}%
\pgfsys@transformshift{4.161668in}{5.350738in}%
\pgfsys@useobject{currentmarker}{}%
\end{pgfscope}%
\end{pgfscope}%
\begin{pgfscope}%
\pgfsetbuttcap%
\pgfsetroundjoin%
\definecolor{currentfill}{rgb}{0.000000,0.000000,0.000000}%
\pgfsetfillcolor{currentfill}%
\pgfsetlinewidth{0.602250pt}%
\definecolor{currentstroke}{rgb}{0.000000,0.000000,0.000000}%
\pgfsetstrokecolor{currentstroke}%
\pgfsetdash{}{0pt}%
\pgfsys@defobject{currentmarker}{\pgfqpoint{0.000000in}{-0.027778in}}{\pgfqpoint{0.000000in}{0.000000in}}{%
\pgfpathmoveto{\pgfqpoint{0.000000in}{0.000000in}}%
\pgfpathlineto{\pgfqpoint{0.000000in}{-0.027778in}}%
\pgfusepath{stroke,fill}%
}%
\begin{pgfscope}%
\pgfsys@transformshift{4.263042in}{5.350738in}%
\pgfsys@useobject{currentmarker}{}%
\end{pgfscope}%
\end{pgfscope}%
\begin{pgfscope}%
\pgfsetbuttcap%
\pgfsetroundjoin%
\definecolor{currentfill}{rgb}{0.000000,0.000000,0.000000}%
\pgfsetfillcolor{currentfill}%
\pgfsetlinewidth{0.602250pt}%
\definecolor{currentstroke}{rgb}{0.000000,0.000000,0.000000}%
\pgfsetstrokecolor{currentstroke}%
\pgfsetdash{}{0pt}%
\pgfsys@defobject{currentmarker}{\pgfqpoint{0.000000in}{-0.027778in}}{\pgfqpoint{0.000000in}{0.000000in}}{%
\pgfpathmoveto{\pgfqpoint{0.000000in}{0.000000in}}%
\pgfpathlineto{\pgfqpoint{0.000000in}{-0.027778in}}%
\pgfusepath{stroke,fill}%
}%
\begin{pgfscope}%
\pgfsys@transformshift{4.950303in}{5.350738in}%
\pgfsys@useobject{currentmarker}{}%
\end{pgfscope}%
\end{pgfscope}%
\begin{pgfscope}%
\pgfsetbuttcap%
\pgfsetroundjoin%
\definecolor{currentfill}{rgb}{0.000000,0.000000,0.000000}%
\pgfsetfillcolor{currentfill}%
\pgfsetlinewidth{0.602250pt}%
\definecolor{currentstroke}{rgb}{0.000000,0.000000,0.000000}%
\pgfsetstrokecolor{currentstroke}%
\pgfsetdash{}{0pt}%
\pgfsys@defobject{currentmarker}{\pgfqpoint{0.000000in}{-0.027778in}}{\pgfqpoint{0.000000in}{0.000000in}}{%
\pgfpathmoveto{\pgfqpoint{0.000000in}{0.000000in}}%
\pgfpathlineto{\pgfqpoint{0.000000in}{-0.027778in}}%
\pgfusepath{stroke,fill}%
}%
\begin{pgfscope}%
\pgfsys@transformshift{5.299279in}{5.350738in}%
\pgfsys@useobject{currentmarker}{}%
\end{pgfscope}%
\end{pgfscope}%
\begin{pgfscope}%
\pgfsetbuttcap%
\pgfsetroundjoin%
\definecolor{currentfill}{rgb}{0.000000,0.000000,0.000000}%
\pgfsetfillcolor{currentfill}%
\pgfsetlinewidth{0.602250pt}%
\definecolor{currentstroke}{rgb}{0.000000,0.000000,0.000000}%
\pgfsetstrokecolor{currentstroke}%
\pgfsetdash{}{0pt}%
\pgfsys@defobject{currentmarker}{\pgfqpoint{0.000000in}{-0.027778in}}{\pgfqpoint{0.000000in}{0.000000in}}{%
\pgfpathmoveto{\pgfqpoint{0.000000in}{0.000000in}}%
\pgfpathlineto{\pgfqpoint{0.000000in}{-0.027778in}}%
\pgfusepath{stroke,fill}%
}%
\begin{pgfscope}%
\pgfsys@transformshift{5.546882in}{5.350738in}%
\pgfsys@useobject{currentmarker}{}%
\end{pgfscope}%
\end{pgfscope}%
\begin{pgfscope}%
\pgfsetbuttcap%
\pgfsetroundjoin%
\definecolor{currentfill}{rgb}{0.000000,0.000000,0.000000}%
\pgfsetfillcolor{currentfill}%
\pgfsetlinewidth{0.602250pt}%
\definecolor{currentstroke}{rgb}{0.000000,0.000000,0.000000}%
\pgfsetstrokecolor{currentstroke}%
\pgfsetdash{}{0pt}%
\pgfsys@defobject{currentmarker}{\pgfqpoint{0.000000in}{-0.027778in}}{\pgfqpoint{0.000000in}{0.000000in}}{%
\pgfpathmoveto{\pgfqpoint{0.000000in}{0.000000in}}%
\pgfpathlineto{\pgfqpoint{0.000000in}{-0.027778in}}%
\pgfusepath{stroke,fill}%
}%
\begin{pgfscope}%
\pgfsys@transformshift{5.738938in}{5.350738in}%
\pgfsys@useobject{currentmarker}{}%
\end{pgfscope}%
\end{pgfscope}%
\begin{pgfscope}%
\pgfsetbuttcap%
\pgfsetroundjoin%
\definecolor{currentfill}{rgb}{0.000000,0.000000,0.000000}%
\pgfsetfillcolor{currentfill}%
\pgfsetlinewidth{0.602250pt}%
\definecolor{currentstroke}{rgb}{0.000000,0.000000,0.000000}%
\pgfsetstrokecolor{currentstroke}%
\pgfsetdash{}{0pt}%
\pgfsys@defobject{currentmarker}{\pgfqpoint{0.000000in}{-0.027778in}}{\pgfqpoint{0.000000in}{0.000000in}}{%
\pgfpathmoveto{\pgfqpoint{0.000000in}{0.000000in}}%
\pgfpathlineto{\pgfqpoint{0.000000in}{-0.027778in}}%
\pgfusepath{stroke,fill}%
}%
\begin{pgfscope}%
\pgfsys@transformshift{5.895858in}{5.350738in}%
\pgfsys@useobject{currentmarker}{}%
\end{pgfscope}%
\end{pgfscope}%
\begin{pgfscope}%
\pgfsetbuttcap%
\pgfsetroundjoin%
\definecolor{currentfill}{rgb}{0.000000,0.000000,0.000000}%
\pgfsetfillcolor{currentfill}%
\pgfsetlinewidth{0.602250pt}%
\definecolor{currentstroke}{rgb}{0.000000,0.000000,0.000000}%
\pgfsetstrokecolor{currentstroke}%
\pgfsetdash{}{0pt}%
\pgfsys@defobject{currentmarker}{\pgfqpoint{0.000000in}{-0.027778in}}{\pgfqpoint{0.000000in}{0.000000in}}{%
\pgfpathmoveto{\pgfqpoint{0.000000in}{0.000000in}}%
\pgfpathlineto{\pgfqpoint{0.000000in}{-0.027778in}}%
\pgfusepath{stroke,fill}%
}%
\begin{pgfscope}%
\pgfsys@transformshift{6.028533in}{5.350738in}%
\pgfsys@useobject{currentmarker}{}%
\end{pgfscope}%
\end{pgfscope}%
\begin{pgfscope}%
\pgfsetbuttcap%
\pgfsetroundjoin%
\definecolor{currentfill}{rgb}{0.000000,0.000000,0.000000}%
\pgfsetfillcolor{currentfill}%
\pgfsetlinewidth{0.602250pt}%
\definecolor{currentstroke}{rgb}{0.000000,0.000000,0.000000}%
\pgfsetstrokecolor{currentstroke}%
\pgfsetdash{}{0pt}%
\pgfsys@defobject{currentmarker}{\pgfqpoint{0.000000in}{-0.027778in}}{\pgfqpoint{0.000000in}{0.000000in}}{%
\pgfpathmoveto{\pgfqpoint{0.000000in}{0.000000in}}%
\pgfpathlineto{\pgfqpoint{0.000000in}{-0.027778in}}%
\pgfusepath{stroke,fill}%
}%
\begin{pgfscope}%
\pgfsys@transformshift{6.143461in}{5.350738in}%
\pgfsys@useobject{currentmarker}{}%
\end{pgfscope}%
\end{pgfscope}%
\begin{pgfscope}%
\pgfsetbuttcap%
\pgfsetroundjoin%
\definecolor{currentfill}{rgb}{0.000000,0.000000,0.000000}%
\pgfsetfillcolor{currentfill}%
\pgfsetlinewidth{0.602250pt}%
\definecolor{currentstroke}{rgb}{0.000000,0.000000,0.000000}%
\pgfsetstrokecolor{currentstroke}%
\pgfsetdash{}{0pt}%
\pgfsys@defobject{currentmarker}{\pgfqpoint{0.000000in}{-0.027778in}}{\pgfqpoint{0.000000in}{0.000000in}}{%
\pgfpathmoveto{\pgfqpoint{0.000000in}{0.000000in}}%
\pgfpathlineto{\pgfqpoint{0.000000in}{-0.027778in}}%
\pgfusepath{stroke,fill}%
}%
\begin{pgfscope}%
\pgfsys@transformshift{6.244835in}{5.350738in}%
\pgfsys@useobject{currentmarker}{}%
\end{pgfscope}%
\end{pgfscope}%
\begin{pgfscope}%
\pgfsetbuttcap%
\pgfsetroundjoin%
\definecolor{currentfill}{rgb}{0.000000,0.000000,0.000000}%
\pgfsetfillcolor{currentfill}%
\pgfsetlinewidth{0.602250pt}%
\definecolor{currentstroke}{rgb}{0.000000,0.000000,0.000000}%
\pgfsetstrokecolor{currentstroke}%
\pgfsetdash{}{0pt}%
\pgfsys@defobject{currentmarker}{\pgfqpoint{0.000000in}{-0.027778in}}{\pgfqpoint{0.000000in}{0.000000in}}{%
\pgfpathmoveto{\pgfqpoint{0.000000in}{0.000000in}}%
\pgfpathlineto{\pgfqpoint{0.000000in}{-0.027778in}}%
\pgfusepath{stroke,fill}%
}%
\begin{pgfscope}%
\pgfsys@transformshift{6.932096in}{5.350738in}%
\pgfsys@useobject{currentmarker}{}%
\end{pgfscope}%
\end{pgfscope}%
\begin{pgfscope}%
\pgfsetbuttcap%
\pgfsetroundjoin%
\definecolor{currentfill}{rgb}{0.000000,0.000000,0.000000}%
\pgfsetfillcolor{currentfill}%
\pgfsetlinewidth{0.602250pt}%
\definecolor{currentstroke}{rgb}{0.000000,0.000000,0.000000}%
\pgfsetstrokecolor{currentstroke}%
\pgfsetdash{}{0pt}%
\pgfsys@defobject{currentmarker}{\pgfqpoint{0.000000in}{-0.027778in}}{\pgfqpoint{0.000000in}{0.000000in}}{%
\pgfpathmoveto{\pgfqpoint{0.000000in}{0.000000in}}%
\pgfpathlineto{\pgfqpoint{0.000000in}{-0.027778in}}%
\pgfusepath{stroke,fill}%
}%
\begin{pgfscope}%
\pgfsys@transformshift{7.281072in}{5.350738in}%
\pgfsys@useobject{currentmarker}{}%
\end{pgfscope}%
\end{pgfscope}%
\begin{pgfscope}%
\pgfsetbuttcap%
\pgfsetroundjoin%
\definecolor{currentfill}{rgb}{0.000000,0.000000,0.000000}%
\pgfsetfillcolor{currentfill}%
\pgfsetlinewidth{0.602250pt}%
\definecolor{currentstroke}{rgb}{0.000000,0.000000,0.000000}%
\pgfsetstrokecolor{currentstroke}%
\pgfsetdash{}{0pt}%
\pgfsys@defobject{currentmarker}{\pgfqpoint{0.000000in}{-0.027778in}}{\pgfqpoint{0.000000in}{0.000000in}}{%
\pgfpathmoveto{\pgfqpoint{0.000000in}{0.000000in}}%
\pgfpathlineto{\pgfqpoint{0.000000in}{-0.027778in}}%
\pgfusepath{stroke,fill}%
}%
\begin{pgfscope}%
\pgfsys@transformshift{7.528675in}{5.350738in}%
\pgfsys@useobject{currentmarker}{}%
\end{pgfscope}%
\end{pgfscope}%
\begin{pgfscope}%
\pgfsetbuttcap%
\pgfsetroundjoin%
\definecolor{currentfill}{rgb}{0.000000,0.000000,0.000000}%
\pgfsetfillcolor{currentfill}%
\pgfsetlinewidth{0.602250pt}%
\definecolor{currentstroke}{rgb}{0.000000,0.000000,0.000000}%
\pgfsetstrokecolor{currentstroke}%
\pgfsetdash{}{0pt}%
\pgfsys@defobject{currentmarker}{\pgfqpoint{0.000000in}{-0.027778in}}{\pgfqpoint{0.000000in}{0.000000in}}{%
\pgfpathmoveto{\pgfqpoint{0.000000in}{0.000000in}}%
\pgfpathlineto{\pgfqpoint{0.000000in}{-0.027778in}}%
\pgfusepath{stroke,fill}%
}%
\begin{pgfscope}%
\pgfsys@transformshift{7.720731in}{5.350738in}%
\pgfsys@useobject{currentmarker}{}%
\end{pgfscope}%
\end{pgfscope}%
\begin{pgfscope}%
\pgfsetbuttcap%
\pgfsetroundjoin%
\definecolor{currentfill}{rgb}{0.000000,0.000000,0.000000}%
\pgfsetfillcolor{currentfill}%
\pgfsetlinewidth{0.602250pt}%
\definecolor{currentstroke}{rgb}{0.000000,0.000000,0.000000}%
\pgfsetstrokecolor{currentstroke}%
\pgfsetdash{}{0pt}%
\pgfsys@defobject{currentmarker}{\pgfqpoint{0.000000in}{-0.027778in}}{\pgfqpoint{0.000000in}{0.000000in}}{%
\pgfpathmoveto{\pgfqpoint{0.000000in}{0.000000in}}%
\pgfpathlineto{\pgfqpoint{0.000000in}{-0.027778in}}%
\pgfusepath{stroke,fill}%
}%
\begin{pgfscope}%
\pgfsys@transformshift{7.877651in}{5.350738in}%
\pgfsys@useobject{currentmarker}{}%
\end{pgfscope}%
\end{pgfscope}%
\begin{pgfscope}%
\pgfsetbuttcap%
\pgfsetroundjoin%
\definecolor{currentfill}{rgb}{0.000000,0.000000,0.000000}%
\pgfsetfillcolor{currentfill}%
\pgfsetlinewidth{0.602250pt}%
\definecolor{currentstroke}{rgb}{0.000000,0.000000,0.000000}%
\pgfsetstrokecolor{currentstroke}%
\pgfsetdash{}{0pt}%
\pgfsys@defobject{currentmarker}{\pgfqpoint{0.000000in}{-0.027778in}}{\pgfqpoint{0.000000in}{0.000000in}}{%
\pgfpathmoveto{\pgfqpoint{0.000000in}{0.000000in}}%
\pgfpathlineto{\pgfqpoint{0.000000in}{-0.027778in}}%
\pgfusepath{stroke,fill}%
}%
\begin{pgfscope}%
\pgfsys@transformshift{8.010326in}{5.350738in}%
\pgfsys@useobject{currentmarker}{}%
\end{pgfscope}%
\end{pgfscope}%
\begin{pgfscope}%
\pgfsetbuttcap%
\pgfsetroundjoin%
\definecolor{currentfill}{rgb}{0.000000,0.000000,0.000000}%
\pgfsetfillcolor{currentfill}%
\pgfsetlinewidth{0.602250pt}%
\definecolor{currentstroke}{rgb}{0.000000,0.000000,0.000000}%
\pgfsetstrokecolor{currentstroke}%
\pgfsetdash{}{0pt}%
\pgfsys@defobject{currentmarker}{\pgfqpoint{0.000000in}{-0.027778in}}{\pgfqpoint{0.000000in}{0.000000in}}{%
\pgfpathmoveto{\pgfqpoint{0.000000in}{0.000000in}}%
\pgfpathlineto{\pgfqpoint{0.000000in}{-0.027778in}}%
\pgfusepath{stroke,fill}%
}%
\begin{pgfscope}%
\pgfsys@transformshift{8.125254in}{5.350738in}%
\pgfsys@useobject{currentmarker}{}%
\end{pgfscope}%
\end{pgfscope}%
\begin{pgfscope}%
\pgfsetbuttcap%
\pgfsetroundjoin%
\definecolor{currentfill}{rgb}{0.000000,0.000000,0.000000}%
\pgfsetfillcolor{currentfill}%
\pgfsetlinewidth{0.602250pt}%
\definecolor{currentstroke}{rgb}{0.000000,0.000000,0.000000}%
\pgfsetstrokecolor{currentstroke}%
\pgfsetdash{}{0pt}%
\pgfsys@defobject{currentmarker}{\pgfqpoint{0.000000in}{-0.027778in}}{\pgfqpoint{0.000000in}{0.000000in}}{%
\pgfpathmoveto{\pgfqpoint{0.000000in}{0.000000in}}%
\pgfpathlineto{\pgfqpoint{0.000000in}{-0.027778in}}%
\pgfusepath{stroke,fill}%
}%
\begin{pgfscope}%
\pgfsys@transformshift{8.226628in}{5.350738in}%
\pgfsys@useobject{currentmarker}{}%
\end{pgfscope}%
\end{pgfscope}%
\begin{pgfscope}%
\pgfsetbuttcap%
\pgfsetroundjoin%
\definecolor{currentfill}{rgb}{0.000000,0.000000,0.000000}%
\pgfsetfillcolor{currentfill}%
\pgfsetlinewidth{0.602250pt}%
\definecolor{currentstroke}{rgb}{0.000000,0.000000,0.000000}%
\pgfsetstrokecolor{currentstroke}%
\pgfsetdash{}{0pt}%
\pgfsys@defobject{currentmarker}{\pgfqpoint{0.000000in}{-0.027778in}}{\pgfqpoint{0.000000in}{0.000000in}}{%
\pgfpathmoveto{\pgfqpoint{0.000000in}{0.000000in}}%
\pgfpathlineto{\pgfqpoint{0.000000in}{-0.027778in}}%
\pgfusepath{stroke,fill}%
}%
\begin{pgfscope}%
\pgfsys@transformshift{8.913889in}{5.350738in}%
\pgfsys@useobject{currentmarker}{}%
\end{pgfscope}%
\end{pgfscope}%
\begin{pgfscope}%
\pgfsetbuttcap%
\pgfsetroundjoin%
\definecolor{currentfill}{rgb}{0.000000,0.000000,0.000000}%
\pgfsetfillcolor{currentfill}%
\pgfsetlinewidth{0.602250pt}%
\definecolor{currentstroke}{rgb}{0.000000,0.000000,0.000000}%
\pgfsetstrokecolor{currentstroke}%
\pgfsetdash{}{0pt}%
\pgfsys@defobject{currentmarker}{\pgfqpoint{0.000000in}{-0.027778in}}{\pgfqpoint{0.000000in}{0.000000in}}{%
\pgfpathmoveto{\pgfqpoint{0.000000in}{0.000000in}}%
\pgfpathlineto{\pgfqpoint{0.000000in}{-0.027778in}}%
\pgfusepath{stroke,fill}%
}%
\begin{pgfscope}%
\pgfsys@transformshift{9.262865in}{5.350738in}%
\pgfsys@useobject{currentmarker}{}%
\end{pgfscope}%
\end{pgfscope}%
\begin{pgfscope}%
\pgfsetbuttcap%
\pgfsetroundjoin%
\definecolor{currentfill}{rgb}{0.000000,0.000000,0.000000}%
\pgfsetfillcolor{currentfill}%
\pgfsetlinewidth{0.602250pt}%
\definecolor{currentstroke}{rgb}{0.000000,0.000000,0.000000}%
\pgfsetstrokecolor{currentstroke}%
\pgfsetdash{}{0pt}%
\pgfsys@defobject{currentmarker}{\pgfqpoint{0.000000in}{-0.027778in}}{\pgfqpoint{0.000000in}{0.000000in}}{%
\pgfpathmoveto{\pgfqpoint{0.000000in}{0.000000in}}%
\pgfpathlineto{\pgfqpoint{0.000000in}{-0.027778in}}%
\pgfusepath{stroke,fill}%
}%
\begin{pgfscope}%
\pgfsys@transformshift{9.510468in}{5.350738in}%
\pgfsys@useobject{currentmarker}{}%
\end{pgfscope}%
\end{pgfscope}%
\begin{pgfscope}%
\pgfsetbuttcap%
\pgfsetroundjoin%
\definecolor{currentfill}{rgb}{0.000000,0.000000,0.000000}%
\pgfsetfillcolor{currentfill}%
\pgfsetlinewidth{0.602250pt}%
\definecolor{currentstroke}{rgb}{0.000000,0.000000,0.000000}%
\pgfsetstrokecolor{currentstroke}%
\pgfsetdash{}{0pt}%
\pgfsys@defobject{currentmarker}{\pgfqpoint{0.000000in}{-0.027778in}}{\pgfqpoint{0.000000in}{0.000000in}}{%
\pgfpathmoveto{\pgfqpoint{0.000000in}{0.000000in}}%
\pgfpathlineto{\pgfqpoint{0.000000in}{-0.027778in}}%
\pgfusepath{stroke,fill}%
}%
\begin{pgfscope}%
\pgfsys@transformshift{9.702524in}{5.350738in}%
\pgfsys@useobject{currentmarker}{}%
\end{pgfscope}%
\end{pgfscope}%
\begin{pgfscope}%
\pgfsetbuttcap%
\pgfsetroundjoin%
\definecolor{currentfill}{rgb}{0.000000,0.000000,0.000000}%
\pgfsetfillcolor{currentfill}%
\pgfsetlinewidth{0.602250pt}%
\definecolor{currentstroke}{rgb}{0.000000,0.000000,0.000000}%
\pgfsetstrokecolor{currentstroke}%
\pgfsetdash{}{0pt}%
\pgfsys@defobject{currentmarker}{\pgfqpoint{0.000000in}{-0.027778in}}{\pgfqpoint{0.000000in}{0.000000in}}{%
\pgfpathmoveto{\pgfqpoint{0.000000in}{0.000000in}}%
\pgfpathlineto{\pgfqpoint{0.000000in}{-0.027778in}}%
\pgfusepath{stroke,fill}%
}%
\begin{pgfscope}%
\pgfsys@transformshift{9.859444in}{5.350738in}%
\pgfsys@useobject{currentmarker}{}%
\end{pgfscope}%
\end{pgfscope}%
\begin{pgfscope}%
\pgfsetbuttcap%
\pgfsetroundjoin%
\definecolor{currentfill}{rgb}{0.000000,0.000000,0.000000}%
\pgfsetfillcolor{currentfill}%
\pgfsetlinewidth{0.602250pt}%
\definecolor{currentstroke}{rgb}{0.000000,0.000000,0.000000}%
\pgfsetstrokecolor{currentstroke}%
\pgfsetdash{}{0pt}%
\pgfsys@defobject{currentmarker}{\pgfqpoint{0.000000in}{-0.027778in}}{\pgfqpoint{0.000000in}{0.000000in}}{%
\pgfpathmoveto{\pgfqpoint{0.000000in}{0.000000in}}%
\pgfpathlineto{\pgfqpoint{0.000000in}{-0.027778in}}%
\pgfusepath{stroke,fill}%
}%
\begin{pgfscope}%
\pgfsys@transformshift{9.992119in}{5.350738in}%
\pgfsys@useobject{currentmarker}{}%
\end{pgfscope}%
\end{pgfscope}%
\begin{pgfscope}%
\pgfsetbuttcap%
\pgfsetroundjoin%
\definecolor{currentfill}{rgb}{0.000000,0.000000,0.000000}%
\pgfsetfillcolor{currentfill}%
\pgfsetlinewidth{0.602250pt}%
\definecolor{currentstroke}{rgb}{0.000000,0.000000,0.000000}%
\pgfsetstrokecolor{currentstroke}%
\pgfsetdash{}{0pt}%
\pgfsys@defobject{currentmarker}{\pgfqpoint{0.000000in}{-0.027778in}}{\pgfqpoint{0.000000in}{0.000000in}}{%
\pgfpathmoveto{\pgfqpoint{0.000000in}{0.000000in}}%
\pgfpathlineto{\pgfqpoint{0.000000in}{-0.027778in}}%
\pgfusepath{stroke,fill}%
}%
\begin{pgfscope}%
\pgfsys@transformshift{10.107047in}{5.350738in}%
\pgfsys@useobject{currentmarker}{}%
\end{pgfscope}%
\end{pgfscope}%
\begin{pgfscope}%
\pgfsetbuttcap%
\pgfsetroundjoin%
\definecolor{currentfill}{rgb}{0.000000,0.000000,0.000000}%
\pgfsetfillcolor{currentfill}%
\pgfsetlinewidth{0.602250pt}%
\definecolor{currentstroke}{rgb}{0.000000,0.000000,0.000000}%
\pgfsetstrokecolor{currentstroke}%
\pgfsetdash{}{0pt}%
\pgfsys@defobject{currentmarker}{\pgfqpoint{0.000000in}{-0.027778in}}{\pgfqpoint{0.000000in}{0.000000in}}{%
\pgfpathmoveto{\pgfqpoint{0.000000in}{0.000000in}}%
\pgfpathlineto{\pgfqpoint{0.000000in}{-0.027778in}}%
\pgfusepath{stroke,fill}%
}%
\begin{pgfscope}%
\pgfsys@transformshift{10.208421in}{5.350738in}%
\pgfsys@useobject{currentmarker}{}%
\end{pgfscope}%
\end{pgfscope}%
\begin{pgfscope}%
\pgfsetbuttcap%
\pgfsetroundjoin%
\definecolor{currentfill}{rgb}{0.000000,0.000000,0.000000}%
\pgfsetfillcolor{currentfill}%
\pgfsetlinewidth{0.803000pt}%
\definecolor{currentstroke}{rgb}{0.000000,0.000000,0.000000}%
\pgfsetstrokecolor{currentstroke}%
\pgfsetdash{}{0pt}%
\pgfsys@defobject{currentmarker}{\pgfqpoint{-0.048611in}{0.000000in}}{\pgfqpoint{-0.000000in}{0.000000in}}{%
\pgfpathmoveto{\pgfqpoint{-0.000000in}{0.000000in}}%
\pgfpathlineto{\pgfqpoint{-0.048611in}{0.000000in}}%
\pgfusepath{stroke,fill}%
}%
\begin{pgfscope}%
\pgfsys@transformshift{0.390138in}{5.350738in}%
\pgfsys@useobject{currentmarker}{}%
\end{pgfscope}%
\end{pgfscope}%
\begin{pgfscope}%
\definecolor{textcolor}{rgb}{0.000000,0.000000,0.000000}%
\pgfsetstrokecolor{textcolor}%
\pgfsetfillcolor{textcolor}%
\pgftext[x=0.195000in, y=5.281294in, left, base]{\color{textcolor}{\rmfamily\fontsize{15.000000}{18.000000}\selectfont\catcode`\^=\active\def^{\ifmmode\sp\else\^{}\fi}\catcode`\%=\active\def%{\%}0}}%
\end{pgfscope}%
\begin{pgfscope}%
\pgfsetbuttcap%
\pgfsetroundjoin%
\definecolor{currentfill}{rgb}{0.000000,0.000000,0.000000}%
\pgfsetfillcolor{currentfill}%
\pgfsetlinewidth{0.803000pt}%
\definecolor{currentstroke}{rgb}{0.000000,0.000000,0.000000}%
\pgfsetstrokecolor{currentstroke}%
\pgfsetdash{}{0pt}%
\pgfsys@defobject{currentmarker}{\pgfqpoint{-0.048611in}{0.000000in}}{\pgfqpoint{-0.000000in}{0.000000in}}{%
\pgfpathmoveto{\pgfqpoint{-0.000000in}{0.000000in}}%
\pgfpathlineto{\pgfqpoint{-0.048611in}{0.000000in}}%
\pgfusepath{stroke,fill}%
}%
\begin{pgfscope}%
\pgfsys@transformshift{0.390138in}{7.235556in}%
\pgfsys@useobject{currentmarker}{}%
\end{pgfscope}%
\end{pgfscope}%
\begin{pgfscope}%
\definecolor{textcolor}{rgb}{0.000000,0.000000,0.000000}%
\pgfsetstrokecolor{textcolor}%
\pgfsetfillcolor{textcolor}%
\pgftext[x=0.195000in, y=7.166111in, left, base]{\color{textcolor}{\rmfamily\fontsize{15.000000}{18.000000}\selectfont\catcode`\^=\active\def^{\ifmmode\sp\else\^{}\fi}\catcode`\%=\active\def%{\%}1}}%
\end{pgfscope}%
\begin{pgfscope}%
\pgfpathrectangle{\pgfqpoint{0.390138in}{5.350738in}}{\pgfqpoint{10.414862in}{1.884818in}}%
\pgfusepath{clip}%
\pgfsetrectcap%
\pgfsetroundjoin%
\pgfsetlinewidth{1.606000pt}%
\definecolor{currentstroke}{rgb}{0.000000,0.000000,0.000000}%
\pgfsetstrokecolor{currentstroke}%
\pgfsetdash{}{0pt}%
\pgfpathmoveto{\pgfqpoint{0.390138in}{7.235556in}}%
\pgfpathlineto{\pgfqpoint{6.243573in}{7.234446in}}%
\pgfpathlineto{\pgfqpoint{6.737293in}{7.232063in}}%
\pgfpathlineto{\pgfqpoint{7.040761in}{7.228490in}}%
\pgfpathlineto{\pgfqpoint{7.261359in}{7.223768in}}%
\pgfpathlineto{\pgfqpoint{7.436437in}{7.217868in}}%
\pgfpathlineto{\pgfqpoint{7.581168in}{7.210828in}}%
\pgfpathlineto{\pgfqpoint{7.704890in}{7.202640in}}%
\pgfpathlineto{\pgfqpoint{7.813438in}{7.193267in}}%
\pgfpathlineto{\pgfqpoint{7.910314in}{7.182692in}}%
\pgfpathlineto{\pgfqpoint{7.996686in}{7.171078in}}%
\pgfpathlineto{\pgfqpoint{8.076055in}{7.158200in}}%
\pgfpathlineto{\pgfqpoint{8.149587in}{7.144023in}}%
\pgfpathlineto{\pgfqpoint{8.217284in}{7.128731in}}%
\pgfpathlineto{\pgfqpoint{8.280312in}{7.112258in}}%
\pgfpathlineto{\pgfqpoint{8.339839in}{7.094437in}}%
\pgfpathlineto{\pgfqpoint{8.395864in}{7.075390in}}%
\pgfpathlineto{\pgfqpoint{8.449554in}{7.054814in}}%
\pgfpathlineto{\pgfqpoint{8.500910in}{7.032762in}}%
\pgfpathlineto{\pgfqpoint{8.549932in}{7.009320in}}%
\pgfpathlineto{\pgfqpoint{8.597787in}{6.983959in}}%
\pgfpathlineto{\pgfqpoint{8.643307in}{6.957353in}}%
\pgfpathlineto{\pgfqpoint{8.687660in}{6.928900in}}%
\pgfpathlineto{\pgfqpoint{8.730846in}{6.898605in}}%
\pgfpathlineto{\pgfqpoint{8.774032in}{6.865560in}}%
\pgfpathlineto{\pgfqpoint{8.816050in}{6.830583in}}%
\pgfpathlineto{\pgfqpoint{8.856902in}{6.793732in}}%
\pgfpathlineto{\pgfqpoint{8.897753in}{6.753912in}}%
\pgfpathlineto{\pgfqpoint{8.938605in}{6.710964in}}%
\pgfpathlineto{\pgfqpoint{8.979456in}{6.664739in}}%
\pgfpathlineto{\pgfqpoint{9.020308in}{6.615104in}}%
\pgfpathlineto{\pgfqpoint{9.061159in}{6.561950in}}%
\pgfpathlineto{\pgfqpoint{9.103178in}{6.503522in}}%
\pgfpathlineto{\pgfqpoint{9.145197in}{6.441245in}}%
\pgfpathlineto{\pgfqpoint{9.188383in}{6.373244in}}%
\pgfpathlineto{\pgfqpoint{9.233903in}{6.297297in}}%
\pgfpathlineto{\pgfqpoint{9.281757in}{6.212992in}}%
\pgfpathlineto{\pgfqpoint{9.334281in}{6.115759in}}%
\pgfpathlineto{\pgfqpoint{9.394974in}{5.998520in}}%
\pgfpathlineto{\pgfqpoint{9.586393in}{5.624763in}}%
\pgfpathlineto{\pgfqpoint{9.623743in}{5.559029in}}%
\pgfpathlineto{\pgfqpoint{9.655257in}{5.507904in}}%
\pgfpathlineto{\pgfqpoint{9.682102in}{5.468330in}}%
\pgfpathlineto{\pgfqpoint{9.706613in}{5.436033in}}%
\pgfpathlineto{\pgfqpoint{9.728789in}{5.410463in}}%
\pgfpathlineto{\pgfqpoint{9.748632in}{5.390903in}}%
\pgfpathlineto{\pgfqpoint{9.767307in}{5.375667in}}%
\pgfpathlineto{\pgfqpoint{9.783647in}{5.365087in}}%
\pgfpathlineto{\pgfqpoint{9.798821in}{5.357738in}}%
\pgfpathlineto{\pgfqpoint{9.812827in}{5.353208in}}%
\pgfpathlineto{\pgfqpoint{9.826833in}{5.350967in}}%
\pgfpathlineto{\pgfqpoint{9.839672in}{5.351028in}}%
\pgfpathlineto{\pgfqpoint{9.852511in}{5.353208in}}%
\pgfpathlineto{\pgfqpoint{9.865350in}{5.357596in}}%
\pgfpathlineto{\pgfqpoint{9.878189in}{5.364282in}}%
\pgfpathlineto{\pgfqpoint{9.891028in}{5.373352in}}%
\pgfpathlineto{\pgfqpoint{9.905034in}{5.386061in}}%
\pgfpathlineto{\pgfqpoint{9.919041in}{5.401804in}}%
\pgfpathlineto{\pgfqpoint{9.934214in}{5.422389in}}%
\pgfpathlineto{\pgfqpoint{9.949388in}{5.446750in}}%
\pgfpathlineto{\pgfqpoint{9.965728in}{5.477309in}}%
\pgfpathlineto{\pgfqpoint{9.983236in}{5.515127in}}%
\pgfpathlineto{\pgfqpoint{10.000744in}{5.558272in}}%
\pgfpathlineto{\pgfqpoint{10.019419in}{5.610201in}}%
\pgfpathlineto{\pgfqpoint{10.039261in}{5.672039in}}%
\pgfpathlineto{\pgfqpoint{10.060270in}{5.744877in}}%
\pgfpathlineto{\pgfqpoint{10.082447in}{5.829702in}}%
\pgfpathlineto{\pgfqpoint{10.106958in}{5.932381in}}%
\pgfpathlineto{\pgfqpoint{10.133803in}{6.054530in}}%
\pgfpathlineto{\pgfqpoint{10.165317in}{6.208642in}}%
\pgfpathlineto{\pgfqpoint{10.208503in}{6.432183in}}%
\pgfpathlineto{\pgfqpoint{10.275032in}{6.776608in}}%
\pgfpathlineto{\pgfqpoint{10.303045in}{6.909154in}}%
\pgfpathlineto{\pgfqpoint{10.325221in}{7.003638in}}%
\pgfpathlineto{\pgfqpoint{10.343896in}{7.073757in}}%
\pgfpathlineto{\pgfqpoint{10.360237in}{7.126576in}}%
\pgfpathlineto{\pgfqpoint{10.374243in}{7.164590in}}%
\pgfpathlineto{\pgfqpoint{10.385915in}{7.190605in}}%
\pgfpathlineto{\pgfqpoint{10.396420in}{7.209267in}}%
\pgfpathlineto{\pgfqpoint{10.405757in}{7.221843in}}%
\pgfpathlineto{\pgfqpoint{10.413927in}{7.229598in}}%
\pgfpathlineto{\pgfqpoint{10.422098in}{7.234200in}}%
\pgfpathlineto{\pgfqpoint{10.429101in}{7.235552in}}%
\pgfpathlineto{\pgfqpoint{10.436104in}{7.234445in}}%
\pgfpathlineto{\pgfqpoint{10.443107in}{7.230820in}}%
\pgfpathlineto{\pgfqpoint{10.450110in}{7.224626in}}%
\pgfpathlineto{\pgfqpoint{10.458280in}{7.214093in}}%
\pgfpathlineto{\pgfqpoint{10.466451in}{7.199941in}}%
\pgfpathlineto{\pgfqpoint{10.475788in}{7.179276in}}%
\pgfpathlineto{\pgfqpoint{10.486293in}{7.150249in}}%
\pgfpathlineto{\pgfqpoint{10.497965in}{7.110797in}}%
\pgfpathlineto{\pgfqpoint{10.510804in}{7.058695in}}%
\pgfpathlineto{\pgfqpoint{10.524810in}{6.991648in}}%
\pgfpathlineto{\pgfqpoint{10.539983in}{6.907438in}}%
\pgfpathlineto{\pgfqpoint{10.556324in}{6.804132in}}%
\pgfpathlineto{\pgfqpoint{10.574999in}{6.671702in}}%
\pgfpathlineto{\pgfqpoint{10.597176in}{6.498028in}}%
\pgfpathlineto{\pgfqpoint{10.628690in}{6.231388in}}%
\pgfpathlineto{\pgfqpoint{10.676544in}{5.826204in}}%
\pgfpathlineto{\pgfqpoint{10.697553in}{5.668011in}}%
\pgfpathlineto{\pgfqpoint{10.713894in}{5.561000in}}%
\pgfpathlineto{\pgfqpoint{10.727900in}{5.483690in}}%
\pgfpathlineto{\pgfqpoint{10.739572in}{5.431225in}}%
\pgfpathlineto{\pgfqpoint{10.748910in}{5.397996in}}%
\pgfpathlineto{\pgfqpoint{10.757080in}{5.375806in}}%
\pgfpathlineto{\pgfqpoint{10.764083in}{5.362190in}}%
\pgfpathlineto{\pgfqpoint{10.769919in}{5.354810in}}%
\pgfpathlineto{\pgfqpoint{10.774588in}{5.351579in}}%
\pgfpathlineto{\pgfqpoint{10.779256in}{5.350774in}}%
\pgfpathlineto{\pgfqpoint{10.783925in}{5.352440in}}%
\pgfpathlineto{\pgfqpoint{10.788594in}{5.356611in}}%
\pgfpathlineto{\pgfqpoint{10.794430in}{5.365392in}}%
\pgfpathlineto{\pgfqpoint{10.800266in}{5.378177in}}%
\pgfpathlineto{\pgfqpoint{10.806102in}{5.394992in}}%
\pgfpathlineto{\pgfqpoint{10.806102in}{5.394992in}}%
\pgfusepath{stroke}%
\end{pgfscope}%
\begin{pgfscope}%
\pgfpathrectangle{\pgfqpoint{0.390138in}{5.350738in}}{\pgfqpoint{10.414862in}{1.884818in}}%
\pgfusepath{clip}%
\pgfsetrectcap%
\pgfsetroundjoin%
\pgfsetlinewidth{1.606000pt}%
\definecolor{currentstroke}{rgb}{0.000000,0.000000,1.000000}%
\pgfsetstrokecolor{currentstroke}%
\pgfsetdash{}{0pt}%
\pgfpathmoveto{\pgfqpoint{0.390138in}{5.350738in}}%
\pgfpathlineto{\pgfqpoint{6.243573in}{5.351847in}}%
\pgfpathlineto{\pgfqpoint{6.737293in}{5.354231in}}%
\pgfpathlineto{\pgfqpoint{7.040761in}{5.357804in}}%
\pgfpathlineto{\pgfqpoint{7.261359in}{5.362526in}}%
\pgfpathlineto{\pgfqpoint{7.436437in}{5.368426in}}%
\pgfpathlineto{\pgfqpoint{7.581168in}{5.375466in}}%
\pgfpathlineto{\pgfqpoint{7.704890in}{5.383654in}}%
\pgfpathlineto{\pgfqpoint{7.813438in}{5.393026in}}%
\pgfpathlineto{\pgfqpoint{7.910314in}{5.403602in}}%
\pgfpathlineto{\pgfqpoint{7.996686in}{5.415216in}}%
\pgfpathlineto{\pgfqpoint{8.076055in}{5.428094in}}%
\pgfpathlineto{\pgfqpoint{8.149587in}{5.442270in}}%
\pgfpathlineto{\pgfqpoint{8.217284in}{5.457563in}}%
\pgfpathlineto{\pgfqpoint{8.280312in}{5.474035in}}%
\pgfpathlineto{\pgfqpoint{8.339839in}{5.491856in}}%
\pgfpathlineto{\pgfqpoint{8.395864in}{5.510903in}}%
\pgfpathlineto{\pgfqpoint{8.449554in}{5.531480in}}%
\pgfpathlineto{\pgfqpoint{8.500910in}{5.553532in}}%
\pgfpathlineto{\pgfqpoint{8.549932in}{5.576973in}}%
\pgfpathlineto{\pgfqpoint{8.597787in}{5.602334in}}%
\pgfpathlineto{\pgfqpoint{8.643307in}{5.628940in}}%
\pgfpathlineto{\pgfqpoint{8.687660in}{5.657393in}}%
\pgfpathlineto{\pgfqpoint{8.730846in}{5.687689in}}%
\pgfpathlineto{\pgfqpoint{8.774032in}{5.720734in}}%
\pgfpathlineto{\pgfqpoint{8.816050in}{5.755711in}}%
\pgfpathlineto{\pgfqpoint{8.856902in}{5.792561in}}%
\pgfpathlineto{\pgfqpoint{8.897753in}{5.832381in}}%
\pgfpathlineto{\pgfqpoint{8.938605in}{5.875330in}}%
\pgfpathlineto{\pgfqpoint{8.979456in}{5.921555in}}%
\pgfpathlineto{\pgfqpoint{9.020308in}{5.971189in}}%
\pgfpathlineto{\pgfqpoint{9.061159in}{6.024343in}}%
\pgfpathlineto{\pgfqpoint{9.103178in}{6.082771in}}%
\pgfpathlineto{\pgfqpoint{9.145197in}{6.145049in}}%
\pgfpathlineto{\pgfqpoint{9.188383in}{6.213049in}}%
\pgfpathlineto{\pgfqpoint{9.233903in}{6.288997in}}%
\pgfpathlineto{\pgfqpoint{9.281757in}{6.373302in}}%
\pgfpathlineto{\pgfqpoint{9.334281in}{6.470535in}}%
\pgfpathlineto{\pgfqpoint{9.394974in}{6.587773in}}%
\pgfpathlineto{\pgfqpoint{9.586393in}{6.961531in}}%
\pgfpathlineto{\pgfqpoint{9.623743in}{7.027265in}}%
\pgfpathlineto{\pgfqpoint{9.655257in}{7.078390in}}%
\pgfpathlineto{\pgfqpoint{9.682102in}{7.117964in}}%
\pgfpathlineto{\pgfqpoint{9.706613in}{7.150261in}}%
\pgfpathlineto{\pgfqpoint{9.728789in}{7.175831in}}%
\pgfpathlineto{\pgfqpoint{9.748632in}{7.195390in}}%
\pgfpathlineto{\pgfqpoint{9.767307in}{7.210627in}}%
\pgfpathlineto{\pgfqpoint{9.783647in}{7.221207in}}%
\pgfpathlineto{\pgfqpoint{9.798821in}{7.228556in}}%
\pgfpathlineto{\pgfqpoint{9.812827in}{7.233086in}}%
\pgfpathlineto{\pgfqpoint{9.826833in}{7.235326in}}%
\pgfpathlineto{\pgfqpoint{9.839672in}{7.235265in}}%
\pgfpathlineto{\pgfqpoint{9.852511in}{7.233086in}}%
\pgfpathlineto{\pgfqpoint{9.865350in}{7.228698in}}%
\pgfpathlineto{\pgfqpoint{9.878189in}{7.222011in}}%
\pgfpathlineto{\pgfqpoint{9.891028in}{7.212942in}}%
\pgfpathlineto{\pgfqpoint{9.905034in}{7.200233in}}%
\pgfpathlineto{\pgfqpoint{9.919041in}{7.184490in}}%
\pgfpathlineto{\pgfqpoint{9.934214in}{7.163904in}}%
\pgfpathlineto{\pgfqpoint{9.949388in}{7.139544in}}%
\pgfpathlineto{\pgfqpoint{9.965728in}{7.108985in}}%
\pgfpathlineto{\pgfqpoint{9.983236in}{7.071166in}}%
\pgfpathlineto{\pgfqpoint{10.000744in}{7.028022in}}%
\pgfpathlineto{\pgfqpoint{10.019419in}{6.976093in}}%
\pgfpathlineto{\pgfqpoint{10.039261in}{6.914255in}}%
\pgfpathlineto{\pgfqpoint{10.060270in}{6.841417in}}%
\pgfpathlineto{\pgfqpoint{10.082447in}{6.756592in}}%
\pgfpathlineto{\pgfqpoint{10.106958in}{6.653913in}}%
\pgfpathlineto{\pgfqpoint{10.133803in}{6.531764in}}%
\pgfpathlineto{\pgfqpoint{10.165317in}{6.377652in}}%
\pgfpathlineto{\pgfqpoint{10.208503in}{6.154111in}}%
\pgfpathlineto{\pgfqpoint{10.275032in}{5.809686in}}%
\pgfpathlineto{\pgfqpoint{10.303045in}{5.677139in}}%
\pgfpathlineto{\pgfqpoint{10.325221in}{5.582656in}}%
\pgfpathlineto{\pgfqpoint{10.343896in}{5.512537in}}%
\pgfpathlineto{\pgfqpoint{10.360237in}{5.459718in}}%
\pgfpathlineto{\pgfqpoint{10.374243in}{5.421704in}}%
\pgfpathlineto{\pgfqpoint{10.385915in}{5.395688in}}%
\pgfpathlineto{\pgfqpoint{10.396420in}{5.377027in}}%
\pgfpathlineto{\pgfqpoint{10.405757in}{5.364451in}}%
\pgfpathlineto{\pgfqpoint{10.413927in}{5.356695in}}%
\pgfpathlineto{\pgfqpoint{10.422098in}{5.352094in}}%
\pgfpathlineto{\pgfqpoint{10.429101in}{5.350742in}}%
\pgfpathlineto{\pgfqpoint{10.436104in}{5.351849in}}%
\pgfpathlineto{\pgfqpoint{10.443107in}{5.355474in}}%
\pgfpathlineto{\pgfqpoint{10.450110in}{5.361667in}}%
\pgfpathlineto{\pgfqpoint{10.458280in}{5.372201in}}%
\pgfpathlineto{\pgfqpoint{10.466451in}{5.386353in}}%
\pgfpathlineto{\pgfqpoint{10.475788in}{5.407018in}}%
\pgfpathlineto{\pgfqpoint{10.486293in}{5.436045in}}%
\pgfpathlineto{\pgfqpoint{10.497965in}{5.475497in}}%
\pgfpathlineto{\pgfqpoint{10.510804in}{5.527599in}}%
\pgfpathlineto{\pgfqpoint{10.524810in}{5.594646in}}%
\pgfpathlineto{\pgfqpoint{10.539983in}{5.678856in}}%
\pgfpathlineto{\pgfqpoint{10.556324in}{5.782162in}}%
\pgfpathlineto{\pgfqpoint{10.574999in}{5.914592in}}%
\pgfpathlineto{\pgfqpoint{10.597176in}{6.088265in}}%
\pgfpathlineto{\pgfqpoint{10.628690in}{6.354905in}}%
\pgfpathlineto{\pgfqpoint{10.676544in}{6.760090in}}%
\pgfpathlineto{\pgfqpoint{10.697553in}{6.918282in}}%
\pgfpathlineto{\pgfqpoint{10.713894in}{7.025294in}}%
\pgfpathlineto{\pgfqpoint{10.727900in}{7.102604in}}%
\pgfpathlineto{\pgfqpoint{10.739572in}{7.155069in}}%
\pgfpathlineto{\pgfqpoint{10.748910in}{7.188298in}}%
\pgfpathlineto{\pgfqpoint{10.757080in}{7.210487in}}%
\pgfpathlineto{\pgfqpoint{10.764083in}{7.224103in}}%
\pgfpathlineto{\pgfqpoint{10.769919in}{7.231484in}}%
\pgfpathlineto{\pgfqpoint{10.774588in}{7.234715in}}%
\pgfpathlineto{\pgfqpoint{10.779256in}{7.235519in}}%
\pgfpathlineto{\pgfqpoint{10.783925in}{7.233854in}}%
\pgfpathlineto{\pgfqpoint{10.788594in}{7.229683in}}%
\pgfpathlineto{\pgfqpoint{10.794430in}{7.220902in}}%
\pgfpathlineto{\pgfqpoint{10.800266in}{7.208116in}}%
\pgfpathlineto{\pgfqpoint{10.806102in}{7.191301in}}%
\pgfpathlineto{\pgfqpoint{10.806102in}{7.191301in}}%
\pgfusepath{stroke}%
\end{pgfscope}%
\begin{pgfscope}%
\pgfsetrectcap%
\pgfsetmiterjoin%
\pgfsetlinewidth{0.803000pt}%
\definecolor{currentstroke}{rgb}{0.000000,0.000000,0.000000}%
\pgfsetstrokecolor{currentstroke}%
\pgfsetdash{}{0pt}%
\pgfpathmoveto{\pgfqpoint{0.390138in}{5.350738in}}%
\pgfpathlineto{\pgfqpoint{0.390138in}{7.235556in}}%
\pgfusepath{stroke}%
\end{pgfscope}%
\begin{pgfscope}%
\pgfsetrectcap%
\pgfsetmiterjoin%
\pgfsetlinewidth{0.803000pt}%
\definecolor{currentstroke}{rgb}{0.000000,0.000000,0.000000}%
\pgfsetstrokecolor{currentstroke}%
\pgfsetdash{}{0pt}%
\pgfpathmoveto{\pgfqpoint{10.805000in}{5.350738in}}%
\pgfpathlineto{\pgfqpoint{10.805000in}{7.235556in}}%
\pgfusepath{stroke}%
\end{pgfscope}%
\begin{pgfscope}%
\pgfsetrectcap%
\pgfsetmiterjoin%
\pgfsetlinewidth{0.803000pt}%
\definecolor{currentstroke}{rgb}{0.000000,0.000000,0.000000}%
\pgfsetstrokecolor{currentstroke}%
\pgfsetdash{}{0pt}%
\pgfpathmoveto{\pgfqpoint{0.390138in}{5.350738in}}%
\pgfpathlineto{\pgfqpoint{10.805000in}{5.350738in}}%
\pgfusepath{stroke}%
\end{pgfscope}%
\begin{pgfscope}%
\pgfsetrectcap%
\pgfsetmiterjoin%
\pgfsetlinewidth{0.803000pt}%
\definecolor{currentstroke}{rgb}{0.000000,0.000000,0.000000}%
\pgfsetstrokecolor{currentstroke}%
\pgfsetdash{}{0pt}%
\pgfpathmoveto{\pgfqpoint{0.390138in}{7.235556in}}%
\pgfpathlineto{\pgfqpoint{10.805000in}{7.235556in}}%
\pgfusepath{stroke}%
\end{pgfscope}%
\begin{pgfscope}%
\definecolor{textcolor}{rgb}{0.000000,0.000000,0.000000}%
\pgfsetstrokecolor{textcolor}%
\pgfsetfillcolor{textcolor}%
\pgftext[x=8.125254in,y=6.764351in,left,base]{\color{textcolor}{\rmfamily\fontsize{16.000000}{19.200000}\selectfont\catcode`\^=\active\def^{\ifmmode\sp\else\^{}\fi}\catcode`\%=\active\def%{\%}$P_{0v}$}}%
\end{pgfscope}%
\begin{pgfscope}%
\definecolor{textcolor}{rgb}{0.000000,0.000000,1.000000}%
\pgfsetstrokecolor{textcolor}%
\pgfsetfillcolor{textcolor}%
\pgftext[x=8.125254in,y=5.727702in,left,base]{\color{textcolor}{\rmfamily\fontsize{16.000000}{19.200000}\selectfont\catcode`\^=\active\def^{\ifmmode\sp\else\^{}\fi}\catcode`\%=\active\def%{\%}$P_{2c}$}}%
\end{pgfscope}%
\begin{pgfscope}%
\definecolor{textcolor}{rgb}{0.000000,0.000000,0.000000}%
\pgfsetstrokecolor{textcolor}%
\pgfsetfillcolor{textcolor}%
\pgftext[x=0.494286in,y=7.141315in,left,top]{\color{textcolor}{\rmfamily\fontsize{16.000000}{19.200000}\selectfont\catcode`\^=\active\def^{\ifmmode\sp\else\^{}\fi}\catcode`\%=\active\def%{\%}$(a)$}}%
\end{pgfscope}%
\begin{pgfscope}%
\pgfsetbuttcap%
\pgfsetmiterjoin%
\definecolor{currentfill}{rgb}{1.000000,1.000000,1.000000}%
\pgfsetfillcolor{currentfill}%
\pgfsetlinewidth{0.000000pt}%
\definecolor{currentstroke}{rgb}{0.000000,0.000000,0.000000}%
\pgfsetstrokecolor{currentstroke}%
\pgfsetstrokeopacity{0.000000}%
\pgfsetdash{}{0pt}%
\pgfpathmoveto{\pgfqpoint{0.390138in}{3.093143in}}%
\pgfpathlineto{\pgfqpoint{10.805000in}{3.093143in}}%
\pgfpathlineto{\pgfqpoint{10.805000in}{4.977961in}}%
\pgfpathlineto{\pgfqpoint{0.390138in}{4.977961in}}%
\pgfpathlineto{\pgfqpoint{0.390138in}{3.093143in}}%
\pgfpathclose%
\pgfusepath{fill}%
\end{pgfscope}%
\begin{pgfscope}%
\pgfsetbuttcap%
\pgfsetroundjoin%
\definecolor{currentfill}{rgb}{0.000000,0.000000,0.000000}%
\pgfsetfillcolor{currentfill}%
\pgfsetlinewidth{0.803000pt}%
\definecolor{currentstroke}{rgb}{0.000000,0.000000,0.000000}%
\pgfsetstrokecolor{currentstroke}%
\pgfsetdash{}{0pt}%
\pgfsys@defobject{currentmarker}{\pgfqpoint{0.000000in}{-0.048611in}}{\pgfqpoint{0.000000in}{0.000000in}}{%
\pgfpathmoveto{\pgfqpoint{0.000000in}{0.000000in}}%
\pgfpathlineto{\pgfqpoint{0.000000in}{-0.048611in}}%
\pgfusepath{stroke,fill}%
}%
\begin{pgfscope}%
\pgfsys@transformshift{2.371931in}{3.093143in}%
\pgfsys@useobject{currentmarker}{}%
\end{pgfscope}%
\end{pgfscope}%
\begin{pgfscope}%
\pgfsetbuttcap%
\pgfsetroundjoin%
\definecolor{currentfill}{rgb}{0.000000,0.000000,0.000000}%
\pgfsetfillcolor{currentfill}%
\pgfsetlinewidth{0.803000pt}%
\definecolor{currentstroke}{rgb}{0.000000,0.000000,0.000000}%
\pgfsetstrokecolor{currentstroke}%
\pgfsetdash{}{0pt}%
\pgfsys@defobject{currentmarker}{\pgfqpoint{0.000000in}{-0.048611in}}{\pgfqpoint{0.000000in}{0.000000in}}{%
\pgfpathmoveto{\pgfqpoint{0.000000in}{0.000000in}}%
\pgfpathlineto{\pgfqpoint{0.000000in}{-0.048611in}}%
\pgfusepath{stroke,fill}%
}%
\begin{pgfscope}%
\pgfsys@transformshift{6.335517in}{3.093143in}%
\pgfsys@useobject{currentmarker}{}%
\end{pgfscope}%
\end{pgfscope}%
\begin{pgfscope}%
\pgfsetbuttcap%
\pgfsetroundjoin%
\definecolor{currentfill}{rgb}{0.000000,0.000000,0.000000}%
\pgfsetfillcolor{currentfill}%
\pgfsetlinewidth{0.803000pt}%
\definecolor{currentstroke}{rgb}{0.000000,0.000000,0.000000}%
\pgfsetstrokecolor{currentstroke}%
\pgfsetdash{}{0pt}%
\pgfsys@defobject{currentmarker}{\pgfqpoint{0.000000in}{-0.048611in}}{\pgfqpoint{0.000000in}{0.000000in}}{%
\pgfpathmoveto{\pgfqpoint{0.000000in}{0.000000in}}%
\pgfpathlineto{\pgfqpoint{0.000000in}{-0.048611in}}%
\pgfusepath{stroke,fill}%
}%
\begin{pgfscope}%
\pgfsys@transformshift{10.299103in}{3.093143in}%
\pgfsys@useobject{currentmarker}{}%
\end{pgfscope}%
\end{pgfscope}%
\begin{pgfscope}%
\pgfsetbuttcap%
\pgfsetroundjoin%
\definecolor{currentfill}{rgb}{0.000000,0.000000,0.000000}%
\pgfsetfillcolor{currentfill}%
\pgfsetlinewidth{0.602250pt}%
\definecolor{currentstroke}{rgb}{0.000000,0.000000,0.000000}%
\pgfsetstrokecolor{currentstroke}%
\pgfsetdash{}{0pt}%
\pgfsys@defobject{currentmarker}{\pgfqpoint{0.000000in}{-0.027778in}}{\pgfqpoint{0.000000in}{0.000000in}}{%
\pgfpathmoveto{\pgfqpoint{0.000000in}{0.000000in}}%
\pgfpathlineto{\pgfqpoint{0.000000in}{-0.027778in}}%
\pgfusepath{stroke,fill}%
}%
\begin{pgfscope}%
\pgfsys@transformshift{0.986717in}{3.093143in}%
\pgfsys@useobject{currentmarker}{}%
\end{pgfscope}%
\end{pgfscope}%
\begin{pgfscope}%
\pgfsetbuttcap%
\pgfsetroundjoin%
\definecolor{currentfill}{rgb}{0.000000,0.000000,0.000000}%
\pgfsetfillcolor{currentfill}%
\pgfsetlinewidth{0.602250pt}%
\definecolor{currentstroke}{rgb}{0.000000,0.000000,0.000000}%
\pgfsetstrokecolor{currentstroke}%
\pgfsetdash{}{0pt}%
\pgfsys@defobject{currentmarker}{\pgfqpoint{0.000000in}{-0.027778in}}{\pgfqpoint{0.000000in}{0.000000in}}{%
\pgfpathmoveto{\pgfqpoint{0.000000in}{0.000000in}}%
\pgfpathlineto{\pgfqpoint{0.000000in}{-0.027778in}}%
\pgfusepath{stroke,fill}%
}%
\begin{pgfscope}%
\pgfsys@transformshift{1.335693in}{3.093143in}%
\pgfsys@useobject{currentmarker}{}%
\end{pgfscope}%
\end{pgfscope}%
\begin{pgfscope}%
\pgfsetbuttcap%
\pgfsetroundjoin%
\definecolor{currentfill}{rgb}{0.000000,0.000000,0.000000}%
\pgfsetfillcolor{currentfill}%
\pgfsetlinewidth{0.602250pt}%
\definecolor{currentstroke}{rgb}{0.000000,0.000000,0.000000}%
\pgfsetstrokecolor{currentstroke}%
\pgfsetdash{}{0pt}%
\pgfsys@defobject{currentmarker}{\pgfqpoint{0.000000in}{-0.027778in}}{\pgfqpoint{0.000000in}{0.000000in}}{%
\pgfpathmoveto{\pgfqpoint{0.000000in}{0.000000in}}%
\pgfpathlineto{\pgfqpoint{0.000000in}{-0.027778in}}%
\pgfusepath{stroke,fill}%
}%
\begin{pgfscope}%
\pgfsys@transformshift{1.583296in}{3.093143in}%
\pgfsys@useobject{currentmarker}{}%
\end{pgfscope}%
\end{pgfscope}%
\begin{pgfscope}%
\pgfsetbuttcap%
\pgfsetroundjoin%
\definecolor{currentfill}{rgb}{0.000000,0.000000,0.000000}%
\pgfsetfillcolor{currentfill}%
\pgfsetlinewidth{0.602250pt}%
\definecolor{currentstroke}{rgb}{0.000000,0.000000,0.000000}%
\pgfsetstrokecolor{currentstroke}%
\pgfsetdash{}{0pt}%
\pgfsys@defobject{currentmarker}{\pgfqpoint{0.000000in}{-0.027778in}}{\pgfqpoint{0.000000in}{0.000000in}}{%
\pgfpathmoveto{\pgfqpoint{0.000000in}{0.000000in}}%
\pgfpathlineto{\pgfqpoint{0.000000in}{-0.027778in}}%
\pgfusepath{stroke,fill}%
}%
\begin{pgfscope}%
\pgfsys@transformshift{1.775352in}{3.093143in}%
\pgfsys@useobject{currentmarker}{}%
\end{pgfscope}%
\end{pgfscope}%
\begin{pgfscope}%
\pgfsetbuttcap%
\pgfsetroundjoin%
\definecolor{currentfill}{rgb}{0.000000,0.000000,0.000000}%
\pgfsetfillcolor{currentfill}%
\pgfsetlinewidth{0.602250pt}%
\definecolor{currentstroke}{rgb}{0.000000,0.000000,0.000000}%
\pgfsetstrokecolor{currentstroke}%
\pgfsetdash{}{0pt}%
\pgfsys@defobject{currentmarker}{\pgfqpoint{0.000000in}{-0.027778in}}{\pgfqpoint{0.000000in}{0.000000in}}{%
\pgfpathmoveto{\pgfqpoint{0.000000in}{0.000000in}}%
\pgfpathlineto{\pgfqpoint{0.000000in}{-0.027778in}}%
\pgfusepath{stroke,fill}%
}%
\begin{pgfscope}%
\pgfsys@transformshift{1.932272in}{3.093143in}%
\pgfsys@useobject{currentmarker}{}%
\end{pgfscope}%
\end{pgfscope}%
\begin{pgfscope}%
\pgfsetbuttcap%
\pgfsetroundjoin%
\definecolor{currentfill}{rgb}{0.000000,0.000000,0.000000}%
\pgfsetfillcolor{currentfill}%
\pgfsetlinewidth{0.602250pt}%
\definecolor{currentstroke}{rgb}{0.000000,0.000000,0.000000}%
\pgfsetstrokecolor{currentstroke}%
\pgfsetdash{}{0pt}%
\pgfsys@defobject{currentmarker}{\pgfqpoint{0.000000in}{-0.027778in}}{\pgfqpoint{0.000000in}{0.000000in}}{%
\pgfpathmoveto{\pgfqpoint{0.000000in}{0.000000in}}%
\pgfpathlineto{\pgfqpoint{0.000000in}{-0.027778in}}%
\pgfusepath{stroke,fill}%
}%
\begin{pgfscope}%
\pgfsys@transformshift{2.064947in}{3.093143in}%
\pgfsys@useobject{currentmarker}{}%
\end{pgfscope}%
\end{pgfscope}%
\begin{pgfscope}%
\pgfsetbuttcap%
\pgfsetroundjoin%
\definecolor{currentfill}{rgb}{0.000000,0.000000,0.000000}%
\pgfsetfillcolor{currentfill}%
\pgfsetlinewidth{0.602250pt}%
\definecolor{currentstroke}{rgb}{0.000000,0.000000,0.000000}%
\pgfsetstrokecolor{currentstroke}%
\pgfsetdash{}{0pt}%
\pgfsys@defobject{currentmarker}{\pgfqpoint{0.000000in}{-0.027778in}}{\pgfqpoint{0.000000in}{0.000000in}}{%
\pgfpathmoveto{\pgfqpoint{0.000000in}{0.000000in}}%
\pgfpathlineto{\pgfqpoint{0.000000in}{-0.027778in}}%
\pgfusepath{stroke,fill}%
}%
\begin{pgfscope}%
\pgfsys@transformshift{2.179875in}{3.093143in}%
\pgfsys@useobject{currentmarker}{}%
\end{pgfscope}%
\end{pgfscope}%
\begin{pgfscope}%
\pgfsetbuttcap%
\pgfsetroundjoin%
\definecolor{currentfill}{rgb}{0.000000,0.000000,0.000000}%
\pgfsetfillcolor{currentfill}%
\pgfsetlinewidth{0.602250pt}%
\definecolor{currentstroke}{rgb}{0.000000,0.000000,0.000000}%
\pgfsetstrokecolor{currentstroke}%
\pgfsetdash{}{0pt}%
\pgfsys@defobject{currentmarker}{\pgfqpoint{0.000000in}{-0.027778in}}{\pgfqpoint{0.000000in}{0.000000in}}{%
\pgfpathmoveto{\pgfqpoint{0.000000in}{0.000000in}}%
\pgfpathlineto{\pgfqpoint{0.000000in}{-0.027778in}}%
\pgfusepath{stroke,fill}%
}%
\begin{pgfscope}%
\pgfsys@transformshift{2.281249in}{3.093143in}%
\pgfsys@useobject{currentmarker}{}%
\end{pgfscope}%
\end{pgfscope}%
\begin{pgfscope}%
\pgfsetbuttcap%
\pgfsetroundjoin%
\definecolor{currentfill}{rgb}{0.000000,0.000000,0.000000}%
\pgfsetfillcolor{currentfill}%
\pgfsetlinewidth{0.602250pt}%
\definecolor{currentstroke}{rgb}{0.000000,0.000000,0.000000}%
\pgfsetstrokecolor{currentstroke}%
\pgfsetdash{}{0pt}%
\pgfsys@defobject{currentmarker}{\pgfqpoint{0.000000in}{-0.027778in}}{\pgfqpoint{0.000000in}{0.000000in}}{%
\pgfpathmoveto{\pgfqpoint{0.000000in}{0.000000in}}%
\pgfpathlineto{\pgfqpoint{0.000000in}{-0.027778in}}%
\pgfusepath{stroke,fill}%
}%
\begin{pgfscope}%
\pgfsys@transformshift{2.968510in}{3.093143in}%
\pgfsys@useobject{currentmarker}{}%
\end{pgfscope}%
\end{pgfscope}%
\begin{pgfscope}%
\pgfsetbuttcap%
\pgfsetroundjoin%
\definecolor{currentfill}{rgb}{0.000000,0.000000,0.000000}%
\pgfsetfillcolor{currentfill}%
\pgfsetlinewidth{0.602250pt}%
\definecolor{currentstroke}{rgb}{0.000000,0.000000,0.000000}%
\pgfsetstrokecolor{currentstroke}%
\pgfsetdash{}{0pt}%
\pgfsys@defobject{currentmarker}{\pgfqpoint{0.000000in}{-0.027778in}}{\pgfqpoint{0.000000in}{0.000000in}}{%
\pgfpathmoveto{\pgfqpoint{0.000000in}{0.000000in}}%
\pgfpathlineto{\pgfqpoint{0.000000in}{-0.027778in}}%
\pgfusepath{stroke,fill}%
}%
\begin{pgfscope}%
\pgfsys@transformshift{3.317486in}{3.093143in}%
\pgfsys@useobject{currentmarker}{}%
\end{pgfscope}%
\end{pgfscope}%
\begin{pgfscope}%
\pgfsetbuttcap%
\pgfsetroundjoin%
\definecolor{currentfill}{rgb}{0.000000,0.000000,0.000000}%
\pgfsetfillcolor{currentfill}%
\pgfsetlinewidth{0.602250pt}%
\definecolor{currentstroke}{rgb}{0.000000,0.000000,0.000000}%
\pgfsetstrokecolor{currentstroke}%
\pgfsetdash{}{0pt}%
\pgfsys@defobject{currentmarker}{\pgfqpoint{0.000000in}{-0.027778in}}{\pgfqpoint{0.000000in}{0.000000in}}{%
\pgfpathmoveto{\pgfqpoint{0.000000in}{0.000000in}}%
\pgfpathlineto{\pgfqpoint{0.000000in}{-0.027778in}}%
\pgfusepath{stroke,fill}%
}%
\begin{pgfscope}%
\pgfsys@transformshift{3.565089in}{3.093143in}%
\pgfsys@useobject{currentmarker}{}%
\end{pgfscope}%
\end{pgfscope}%
\begin{pgfscope}%
\pgfsetbuttcap%
\pgfsetroundjoin%
\definecolor{currentfill}{rgb}{0.000000,0.000000,0.000000}%
\pgfsetfillcolor{currentfill}%
\pgfsetlinewidth{0.602250pt}%
\definecolor{currentstroke}{rgb}{0.000000,0.000000,0.000000}%
\pgfsetstrokecolor{currentstroke}%
\pgfsetdash{}{0pt}%
\pgfsys@defobject{currentmarker}{\pgfqpoint{0.000000in}{-0.027778in}}{\pgfqpoint{0.000000in}{0.000000in}}{%
\pgfpathmoveto{\pgfqpoint{0.000000in}{0.000000in}}%
\pgfpathlineto{\pgfqpoint{0.000000in}{-0.027778in}}%
\pgfusepath{stroke,fill}%
}%
\begin{pgfscope}%
\pgfsys@transformshift{3.757145in}{3.093143in}%
\pgfsys@useobject{currentmarker}{}%
\end{pgfscope}%
\end{pgfscope}%
\begin{pgfscope}%
\pgfsetbuttcap%
\pgfsetroundjoin%
\definecolor{currentfill}{rgb}{0.000000,0.000000,0.000000}%
\pgfsetfillcolor{currentfill}%
\pgfsetlinewidth{0.602250pt}%
\definecolor{currentstroke}{rgb}{0.000000,0.000000,0.000000}%
\pgfsetstrokecolor{currentstroke}%
\pgfsetdash{}{0pt}%
\pgfsys@defobject{currentmarker}{\pgfqpoint{0.000000in}{-0.027778in}}{\pgfqpoint{0.000000in}{0.000000in}}{%
\pgfpathmoveto{\pgfqpoint{0.000000in}{0.000000in}}%
\pgfpathlineto{\pgfqpoint{0.000000in}{-0.027778in}}%
\pgfusepath{stroke,fill}%
}%
\begin{pgfscope}%
\pgfsys@transformshift{3.914065in}{3.093143in}%
\pgfsys@useobject{currentmarker}{}%
\end{pgfscope}%
\end{pgfscope}%
\begin{pgfscope}%
\pgfsetbuttcap%
\pgfsetroundjoin%
\definecolor{currentfill}{rgb}{0.000000,0.000000,0.000000}%
\pgfsetfillcolor{currentfill}%
\pgfsetlinewidth{0.602250pt}%
\definecolor{currentstroke}{rgb}{0.000000,0.000000,0.000000}%
\pgfsetstrokecolor{currentstroke}%
\pgfsetdash{}{0pt}%
\pgfsys@defobject{currentmarker}{\pgfqpoint{0.000000in}{-0.027778in}}{\pgfqpoint{0.000000in}{0.000000in}}{%
\pgfpathmoveto{\pgfqpoint{0.000000in}{0.000000in}}%
\pgfpathlineto{\pgfqpoint{0.000000in}{-0.027778in}}%
\pgfusepath{stroke,fill}%
}%
\begin{pgfscope}%
\pgfsys@transformshift{4.046740in}{3.093143in}%
\pgfsys@useobject{currentmarker}{}%
\end{pgfscope}%
\end{pgfscope}%
\begin{pgfscope}%
\pgfsetbuttcap%
\pgfsetroundjoin%
\definecolor{currentfill}{rgb}{0.000000,0.000000,0.000000}%
\pgfsetfillcolor{currentfill}%
\pgfsetlinewidth{0.602250pt}%
\definecolor{currentstroke}{rgb}{0.000000,0.000000,0.000000}%
\pgfsetstrokecolor{currentstroke}%
\pgfsetdash{}{0pt}%
\pgfsys@defobject{currentmarker}{\pgfqpoint{0.000000in}{-0.027778in}}{\pgfqpoint{0.000000in}{0.000000in}}{%
\pgfpathmoveto{\pgfqpoint{0.000000in}{0.000000in}}%
\pgfpathlineto{\pgfqpoint{0.000000in}{-0.027778in}}%
\pgfusepath{stroke,fill}%
}%
\begin{pgfscope}%
\pgfsys@transformshift{4.161668in}{3.093143in}%
\pgfsys@useobject{currentmarker}{}%
\end{pgfscope}%
\end{pgfscope}%
\begin{pgfscope}%
\pgfsetbuttcap%
\pgfsetroundjoin%
\definecolor{currentfill}{rgb}{0.000000,0.000000,0.000000}%
\pgfsetfillcolor{currentfill}%
\pgfsetlinewidth{0.602250pt}%
\definecolor{currentstroke}{rgb}{0.000000,0.000000,0.000000}%
\pgfsetstrokecolor{currentstroke}%
\pgfsetdash{}{0pt}%
\pgfsys@defobject{currentmarker}{\pgfqpoint{0.000000in}{-0.027778in}}{\pgfqpoint{0.000000in}{0.000000in}}{%
\pgfpathmoveto{\pgfqpoint{0.000000in}{0.000000in}}%
\pgfpathlineto{\pgfqpoint{0.000000in}{-0.027778in}}%
\pgfusepath{stroke,fill}%
}%
\begin{pgfscope}%
\pgfsys@transformshift{4.263042in}{3.093143in}%
\pgfsys@useobject{currentmarker}{}%
\end{pgfscope}%
\end{pgfscope}%
\begin{pgfscope}%
\pgfsetbuttcap%
\pgfsetroundjoin%
\definecolor{currentfill}{rgb}{0.000000,0.000000,0.000000}%
\pgfsetfillcolor{currentfill}%
\pgfsetlinewidth{0.602250pt}%
\definecolor{currentstroke}{rgb}{0.000000,0.000000,0.000000}%
\pgfsetstrokecolor{currentstroke}%
\pgfsetdash{}{0pt}%
\pgfsys@defobject{currentmarker}{\pgfqpoint{0.000000in}{-0.027778in}}{\pgfqpoint{0.000000in}{0.000000in}}{%
\pgfpathmoveto{\pgfqpoint{0.000000in}{0.000000in}}%
\pgfpathlineto{\pgfqpoint{0.000000in}{-0.027778in}}%
\pgfusepath{stroke,fill}%
}%
\begin{pgfscope}%
\pgfsys@transformshift{4.950303in}{3.093143in}%
\pgfsys@useobject{currentmarker}{}%
\end{pgfscope}%
\end{pgfscope}%
\begin{pgfscope}%
\pgfsetbuttcap%
\pgfsetroundjoin%
\definecolor{currentfill}{rgb}{0.000000,0.000000,0.000000}%
\pgfsetfillcolor{currentfill}%
\pgfsetlinewidth{0.602250pt}%
\definecolor{currentstroke}{rgb}{0.000000,0.000000,0.000000}%
\pgfsetstrokecolor{currentstroke}%
\pgfsetdash{}{0pt}%
\pgfsys@defobject{currentmarker}{\pgfqpoint{0.000000in}{-0.027778in}}{\pgfqpoint{0.000000in}{0.000000in}}{%
\pgfpathmoveto{\pgfqpoint{0.000000in}{0.000000in}}%
\pgfpathlineto{\pgfqpoint{0.000000in}{-0.027778in}}%
\pgfusepath{stroke,fill}%
}%
\begin{pgfscope}%
\pgfsys@transformshift{5.299279in}{3.093143in}%
\pgfsys@useobject{currentmarker}{}%
\end{pgfscope}%
\end{pgfscope}%
\begin{pgfscope}%
\pgfsetbuttcap%
\pgfsetroundjoin%
\definecolor{currentfill}{rgb}{0.000000,0.000000,0.000000}%
\pgfsetfillcolor{currentfill}%
\pgfsetlinewidth{0.602250pt}%
\definecolor{currentstroke}{rgb}{0.000000,0.000000,0.000000}%
\pgfsetstrokecolor{currentstroke}%
\pgfsetdash{}{0pt}%
\pgfsys@defobject{currentmarker}{\pgfqpoint{0.000000in}{-0.027778in}}{\pgfqpoint{0.000000in}{0.000000in}}{%
\pgfpathmoveto{\pgfqpoint{0.000000in}{0.000000in}}%
\pgfpathlineto{\pgfqpoint{0.000000in}{-0.027778in}}%
\pgfusepath{stroke,fill}%
}%
\begin{pgfscope}%
\pgfsys@transformshift{5.546882in}{3.093143in}%
\pgfsys@useobject{currentmarker}{}%
\end{pgfscope}%
\end{pgfscope}%
\begin{pgfscope}%
\pgfsetbuttcap%
\pgfsetroundjoin%
\definecolor{currentfill}{rgb}{0.000000,0.000000,0.000000}%
\pgfsetfillcolor{currentfill}%
\pgfsetlinewidth{0.602250pt}%
\definecolor{currentstroke}{rgb}{0.000000,0.000000,0.000000}%
\pgfsetstrokecolor{currentstroke}%
\pgfsetdash{}{0pt}%
\pgfsys@defobject{currentmarker}{\pgfqpoint{0.000000in}{-0.027778in}}{\pgfqpoint{0.000000in}{0.000000in}}{%
\pgfpathmoveto{\pgfqpoint{0.000000in}{0.000000in}}%
\pgfpathlineto{\pgfqpoint{0.000000in}{-0.027778in}}%
\pgfusepath{stroke,fill}%
}%
\begin{pgfscope}%
\pgfsys@transformshift{5.738938in}{3.093143in}%
\pgfsys@useobject{currentmarker}{}%
\end{pgfscope}%
\end{pgfscope}%
\begin{pgfscope}%
\pgfsetbuttcap%
\pgfsetroundjoin%
\definecolor{currentfill}{rgb}{0.000000,0.000000,0.000000}%
\pgfsetfillcolor{currentfill}%
\pgfsetlinewidth{0.602250pt}%
\definecolor{currentstroke}{rgb}{0.000000,0.000000,0.000000}%
\pgfsetstrokecolor{currentstroke}%
\pgfsetdash{}{0pt}%
\pgfsys@defobject{currentmarker}{\pgfqpoint{0.000000in}{-0.027778in}}{\pgfqpoint{0.000000in}{0.000000in}}{%
\pgfpathmoveto{\pgfqpoint{0.000000in}{0.000000in}}%
\pgfpathlineto{\pgfqpoint{0.000000in}{-0.027778in}}%
\pgfusepath{stroke,fill}%
}%
\begin{pgfscope}%
\pgfsys@transformshift{5.895858in}{3.093143in}%
\pgfsys@useobject{currentmarker}{}%
\end{pgfscope}%
\end{pgfscope}%
\begin{pgfscope}%
\pgfsetbuttcap%
\pgfsetroundjoin%
\definecolor{currentfill}{rgb}{0.000000,0.000000,0.000000}%
\pgfsetfillcolor{currentfill}%
\pgfsetlinewidth{0.602250pt}%
\definecolor{currentstroke}{rgb}{0.000000,0.000000,0.000000}%
\pgfsetstrokecolor{currentstroke}%
\pgfsetdash{}{0pt}%
\pgfsys@defobject{currentmarker}{\pgfqpoint{0.000000in}{-0.027778in}}{\pgfqpoint{0.000000in}{0.000000in}}{%
\pgfpathmoveto{\pgfqpoint{0.000000in}{0.000000in}}%
\pgfpathlineto{\pgfqpoint{0.000000in}{-0.027778in}}%
\pgfusepath{stroke,fill}%
}%
\begin{pgfscope}%
\pgfsys@transformshift{6.028533in}{3.093143in}%
\pgfsys@useobject{currentmarker}{}%
\end{pgfscope}%
\end{pgfscope}%
\begin{pgfscope}%
\pgfsetbuttcap%
\pgfsetroundjoin%
\definecolor{currentfill}{rgb}{0.000000,0.000000,0.000000}%
\pgfsetfillcolor{currentfill}%
\pgfsetlinewidth{0.602250pt}%
\definecolor{currentstroke}{rgb}{0.000000,0.000000,0.000000}%
\pgfsetstrokecolor{currentstroke}%
\pgfsetdash{}{0pt}%
\pgfsys@defobject{currentmarker}{\pgfqpoint{0.000000in}{-0.027778in}}{\pgfqpoint{0.000000in}{0.000000in}}{%
\pgfpathmoveto{\pgfqpoint{0.000000in}{0.000000in}}%
\pgfpathlineto{\pgfqpoint{0.000000in}{-0.027778in}}%
\pgfusepath{stroke,fill}%
}%
\begin{pgfscope}%
\pgfsys@transformshift{6.143461in}{3.093143in}%
\pgfsys@useobject{currentmarker}{}%
\end{pgfscope}%
\end{pgfscope}%
\begin{pgfscope}%
\pgfsetbuttcap%
\pgfsetroundjoin%
\definecolor{currentfill}{rgb}{0.000000,0.000000,0.000000}%
\pgfsetfillcolor{currentfill}%
\pgfsetlinewidth{0.602250pt}%
\definecolor{currentstroke}{rgb}{0.000000,0.000000,0.000000}%
\pgfsetstrokecolor{currentstroke}%
\pgfsetdash{}{0pt}%
\pgfsys@defobject{currentmarker}{\pgfqpoint{0.000000in}{-0.027778in}}{\pgfqpoint{0.000000in}{0.000000in}}{%
\pgfpathmoveto{\pgfqpoint{0.000000in}{0.000000in}}%
\pgfpathlineto{\pgfqpoint{0.000000in}{-0.027778in}}%
\pgfusepath{stroke,fill}%
}%
\begin{pgfscope}%
\pgfsys@transformshift{6.244835in}{3.093143in}%
\pgfsys@useobject{currentmarker}{}%
\end{pgfscope}%
\end{pgfscope}%
\begin{pgfscope}%
\pgfsetbuttcap%
\pgfsetroundjoin%
\definecolor{currentfill}{rgb}{0.000000,0.000000,0.000000}%
\pgfsetfillcolor{currentfill}%
\pgfsetlinewidth{0.602250pt}%
\definecolor{currentstroke}{rgb}{0.000000,0.000000,0.000000}%
\pgfsetstrokecolor{currentstroke}%
\pgfsetdash{}{0pt}%
\pgfsys@defobject{currentmarker}{\pgfqpoint{0.000000in}{-0.027778in}}{\pgfqpoint{0.000000in}{0.000000in}}{%
\pgfpathmoveto{\pgfqpoint{0.000000in}{0.000000in}}%
\pgfpathlineto{\pgfqpoint{0.000000in}{-0.027778in}}%
\pgfusepath{stroke,fill}%
}%
\begin{pgfscope}%
\pgfsys@transformshift{6.932096in}{3.093143in}%
\pgfsys@useobject{currentmarker}{}%
\end{pgfscope}%
\end{pgfscope}%
\begin{pgfscope}%
\pgfsetbuttcap%
\pgfsetroundjoin%
\definecolor{currentfill}{rgb}{0.000000,0.000000,0.000000}%
\pgfsetfillcolor{currentfill}%
\pgfsetlinewidth{0.602250pt}%
\definecolor{currentstroke}{rgb}{0.000000,0.000000,0.000000}%
\pgfsetstrokecolor{currentstroke}%
\pgfsetdash{}{0pt}%
\pgfsys@defobject{currentmarker}{\pgfqpoint{0.000000in}{-0.027778in}}{\pgfqpoint{0.000000in}{0.000000in}}{%
\pgfpathmoveto{\pgfqpoint{0.000000in}{0.000000in}}%
\pgfpathlineto{\pgfqpoint{0.000000in}{-0.027778in}}%
\pgfusepath{stroke,fill}%
}%
\begin{pgfscope}%
\pgfsys@transformshift{7.281072in}{3.093143in}%
\pgfsys@useobject{currentmarker}{}%
\end{pgfscope}%
\end{pgfscope}%
\begin{pgfscope}%
\pgfsetbuttcap%
\pgfsetroundjoin%
\definecolor{currentfill}{rgb}{0.000000,0.000000,0.000000}%
\pgfsetfillcolor{currentfill}%
\pgfsetlinewidth{0.602250pt}%
\definecolor{currentstroke}{rgb}{0.000000,0.000000,0.000000}%
\pgfsetstrokecolor{currentstroke}%
\pgfsetdash{}{0pt}%
\pgfsys@defobject{currentmarker}{\pgfqpoint{0.000000in}{-0.027778in}}{\pgfqpoint{0.000000in}{0.000000in}}{%
\pgfpathmoveto{\pgfqpoint{0.000000in}{0.000000in}}%
\pgfpathlineto{\pgfqpoint{0.000000in}{-0.027778in}}%
\pgfusepath{stroke,fill}%
}%
\begin{pgfscope}%
\pgfsys@transformshift{7.528675in}{3.093143in}%
\pgfsys@useobject{currentmarker}{}%
\end{pgfscope}%
\end{pgfscope}%
\begin{pgfscope}%
\pgfsetbuttcap%
\pgfsetroundjoin%
\definecolor{currentfill}{rgb}{0.000000,0.000000,0.000000}%
\pgfsetfillcolor{currentfill}%
\pgfsetlinewidth{0.602250pt}%
\definecolor{currentstroke}{rgb}{0.000000,0.000000,0.000000}%
\pgfsetstrokecolor{currentstroke}%
\pgfsetdash{}{0pt}%
\pgfsys@defobject{currentmarker}{\pgfqpoint{0.000000in}{-0.027778in}}{\pgfqpoint{0.000000in}{0.000000in}}{%
\pgfpathmoveto{\pgfqpoint{0.000000in}{0.000000in}}%
\pgfpathlineto{\pgfqpoint{0.000000in}{-0.027778in}}%
\pgfusepath{stroke,fill}%
}%
\begin{pgfscope}%
\pgfsys@transformshift{7.720731in}{3.093143in}%
\pgfsys@useobject{currentmarker}{}%
\end{pgfscope}%
\end{pgfscope}%
\begin{pgfscope}%
\pgfsetbuttcap%
\pgfsetroundjoin%
\definecolor{currentfill}{rgb}{0.000000,0.000000,0.000000}%
\pgfsetfillcolor{currentfill}%
\pgfsetlinewidth{0.602250pt}%
\definecolor{currentstroke}{rgb}{0.000000,0.000000,0.000000}%
\pgfsetstrokecolor{currentstroke}%
\pgfsetdash{}{0pt}%
\pgfsys@defobject{currentmarker}{\pgfqpoint{0.000000in}{-0.027778in}}{\pgfqpoint{0.000000in}{0.000000in}}{%
\pgfpathmoveto{\pgfqpoint{0.000000in}{0.000000in}}%
\pgfpathlineto{\pgfqpoint{0.000000in}{-0.027778in}}%
\pgfusepath{stroke,fill}%
}%
\begin{pgfscope}%
\pgfsys@transformshift{7.877651in}{3.093143in}%
\pgfsys@useobject{currentmarker}{}%
\end{pgfscope}%
\end{pgfscope}%
\begin{pgfscope}%
\pgfsetbuttcap%
\pgfsetroundjoin%
\definecolor{currentfill}{rgb}{0.000000,0.000000,0.000000}%
\pgfsetfillcolor{currentfill}%
\pgfsetlinewidth{0.602250pt}%
\definecolor{currentstroke}{rgb}{0.000000,0.000000,0.000000}%
\pgfsetstrokecolor{currentstroke}%
\pgfsetdash{}{0pt}%
\pgfsys@defobject{currentmarker}{\pgfqpoint{0.000000in}{-0.027778in}}{\pgfqpoint{0.000000in}{0.000000in}}{%
\pgfpathmoveto{\pgfqpoint{0.000000in}{0.000000in}}%
\pgfpathlineto{\pgfqpoint{0.000000in}{-0.027778in}}%
\pgfusepath{stroke,fill}%
}%
\begin{pgfscope}%
\pgfsys@transformshift{8.010326in}{3.093143in}%
\pgfsys@useobject{currentmarker}{}%
\end{pgfscope}%
\end{pgfscope}%
\begin{pgfscope}%
\pgfsetbuttcap%
\pgfsetroundjoin%
\definecolor{currentfill}{rgb}{0.000000,0.000000,0.000000}%
\pgfsetfillcolor{currentfill}%
\pgfsetlinewidth{0.602250pt}%
\definecolor{currentstroke}{rgb}{0.000000,0.000000,0.000000}%
\pgfsetstrokecolor{currentstroke}%
\pgfsetdash{}{0pt}%
\pgfsys@defobject{currentmarker}{\pgfqpoint{0.000000in}{-0.027778in}}{\pgfqpoint{0.000000in}{0.000000in}}{%
\pgfpathmoveto{\pgfqpoint{0.000000in}{0.000000in}}%
\pgfpathlineto{\pgfqpoint{0.000000in}{-0.027778in}}%
\pgfusepath{stroke,fill}%
}%
\begin{pgfscope}%
\pgfsys@transformshift{8.125254in}{3.093143in}%
\pgfsys@useobject{currentmarker}{}%
\end{pgfscope}%
\end{pgfscope}%
\begin{pgfscope}%
\pgfsetbuttcap%
\pgfsetroundjoin%
\definecolor{currentfill}{rgb}{0.000000,0.000000,0.000000}%
\pgfsetfillcolor{currentfill}%
\pgfsetlinewidth{0.602250pt}%
\definecolor{currentstroke}{rgb}{0.000000,0.000000,0.000000}%
\pgfsetstrokecolor{currentstroke}%
\pgfsetdash{}{0pt}%
\pgfsys@defobject{currentmarker}{\pgfqpoint{0.000000in}{-0.027778in}}{\pgfqpoint{0.000000in}{0.000000in}}{%
\pgfpathmoveto{\pgfqpoint{0.000000in}{0.000000in}}%
\pgfpathlineto{\pgfqpoint{0.000000in}{-0.027778in}}%
\pgfusepath{stroke,fill}%
}%
\begin{pgfscope}%
\pgfsys@transformshift{8.226628in}{3.093143in}%
\pgfsys@useobject{currentmarker}{}%
\end{pgfscope}%
\end{pgfscope}%
\begin{pgfscope}%
\pgfsetbuttcap%
\pgfsetroundjoin%
\definecolor{currentfill}{rgb}{0.000000,0.000000,0.000000}%
\pgfsetfillcolor{currentfill}%
\pgfsetlinewidth{0.602250pt}%
\definecolor{currentstroke}{rgb}{0.000000,0.000000,0.000000}%
\pgfsetstrokecolor{currentstroke}%
\pgfsetdash{}{0pt}%
\pgfsys@defobject{currentmarker}{\pgfqpoint{0.000000in}{-0.027778in}}{\pgfqpoint{0.000000in}{0.000000in}}{%
\pgfpathmoveto{\pgfqpoint{0.000000in}{0.000000in}}%
\pgfpathlineto{\pgfqpoint{0.000000in}{-0.027778in}}%
\pgfusepath{stroke,fill}%
}%
\begin{pgfscope}%
\pgfsys@transformshift{8.913889in}{3.093143in}%
\pgfsys@useobject{currentmarker}{}%
\end{pgfscope}%
\end{pgfscope}%
\begin{pgfscope}%
\pgfsetbuttcap%
\pgfsetroundjoin%
\definecolor{currentfill}{rgb}{0.000000,0.000000,0.000000}%
\pgfsetfillcolor{currentfill}%
\pgfsetlinewidth{0.602250pt}%
\definecolor{currentstroke}{rgb}{0.000000,0.000000,0.000000}%
\pgfsetstrokecolor{currentstroke}%
\pgfsetdash{}{0pt}%
\pgfsys@defobject{currentmarker}{\pgfqpoint{0.000000in}{-0.027778in}}{\pgfqpoint{0.000000in}{0.000000in}}{%
\pgfpathmoveto{\pgfqpoint{0.000000in}{0.000000in}}%
\pgfpathlineto{\pgfqpoint{0.000000in}{-0.027778in}}%
\pgfusepath{stroke,fill}%
}%
\begin{pgfscope}%
\pgfsys@transformshift{9.262865in}{3.093143in}%
\pgfsys@useobject{currentmarker}{}%
\end{pgfscope}%
\end{pgfscope}%
\begin{pgfscope}%
\pgfsetbuttcap%
\pgfsetroundjoin%
\definecolor{currentfill}{rgb}{0.000000,0.000000,0.000000}%
\pgfsetfillcolor{currentfill}%
\pgfsetlinewidth{0.602250pt}%
\definecolor{currentstroke}{rgb}{0.000000,0.000000,0.000000}%
\pgfsetstrokecolor{currentstroke}%
\pgfsetdash{}{0pt}%
\pgfsys@defobject{currentmarker}{\pgfqpoint{0.000000in}{-0.027778in}}{\pgfqpoint{0.000000in}{0.000000in}}{%
\pgfpathmoveto{\pgfqpoint{0.000000in}{0.000000in}}%
\pgfpathlineto{\pgfqpoint{0.000000in}{-0.027778in}}%
\pgfusepath{stroke,fill}%
}%
\begin{pgfscope}%
\pgfsys@transformshift{9.510468in}{3.093143in}%
\pgfsys@useobject{currentmarker}{}%
\end{pgfscope}%
\end{pgfscope}%
\begin{pgfscope}%
\pgfsetbuttcap%
\pgfsetroundjoin%
\definecolor{currentfill}{rgb}{0.000000,0.000000,0.000000}%
\pgfsetfillcolor{currentfill}%
\pgfsetlinewidth{0.602250pt}%
\definecolor{currentstroke}{rgb}{0.000000,0.000000,0.000000}%
\pgfsetstrokecolor{currentstroke}%
\pgfsetdash{}{0pt}%
\pgfsys@defobject{currentmarker}{\pgfqpoint{0.000000in}{-0.027778in}}{\pgfqpoint{0.000000in}{0.000000in}}{%
\pgfpathmoveto{\pgfqpoint{0.000000in}{0.000000in}}%
\pgfpathlineto{\pgfqpoint{0.000000in}{-0.027778in}}%
\pgfusepath{stroke,fill}%
}%
\begin{pgfscope}%
\pgfsys@transformshift{9.702524in}{3.093143in}%
\pgfsys@useobject{currentmarker}{}%
\end{pgfscope}%
\end{pgfscope}%
\begin{pgfscope}%
\pgfsetbuttcap%
\pgfsetroundjoin%
\definecolor{currentfill}{rgb}{0.000000,0.000000,0.000000}%
\pgfsetfillcolor{currentfill}%
\pgfsetlinewidth{0.602250pt}%
\definecolor{currentstroke}{rgb}{0.000000,0.000000,0.000000}%
\pgfsetstrokecolor{currentstroke}%
\pgfsetdash{}{0pt}%
\pgfsys@defobject{currentmarker}{\pgfqpoint{0.000000in}{-0.027778in}}{\pgfqpoint{0.000000in}{0.000000in}}{%
\pgfpathmoveto{\pgfqpoint{0.000000in}{0.000000in}}%
\pgfpathlineto{\pgfqpoint{0.000000in}{-0.027778in}}%
\pgfusepath{stroke,fill}%
}%
\begin{pgfscope}%
\pgfsys@transformshift{9.859444in}{3.093143in}%
\pgfsys@useobject{currentmarker}{}%
\end{pgfscope}%
\end{pgfscope}%
\begin{pgfscope}%
\pgfsetbuttcap%
\pgfsetroundjoin%
\definecolor{currentfill}{rgb}{0.000000,0.000000,0.000000}%
\pgfsetfillcolor{currentfill}%
\pgfsetlinewidth{0.602250pt}%
\definecolor{currentstroke}{rgb}{0.000000,0.000000,0.000000}%
\pgfsetstrokecolor{currentstroke}%
\pgfsetdash{}{0pt}%
\pgfsys@defobject{currentmarker}{\pgfqpoint{0.000000in}{-0.027778in}}{\pgfqpoint{0.000000in}{0.000000in}}{%
\pgfpathmoveto{\pgfqpoint{0.000000in}{0.000000in}}%
\pgfpathlineto{\pgfqpoint{0.000000in}{-0.027778in}}%
\pgfusepath{stroke,fill}%
}%
\begin{pgfscope}%
\pgfsys@transformshift{9.992119in}{3.093143in}%
\pgfsys@useobject{currentmarker}{}%
\end{pgfscope}%
\end{pgfscope}%
\begin{pgfscope}%
\pgfsetbuttcap%
\pgfsetroundjoin%
\definecolor{currentfill}{rgb}{0.000000,0.000000,0.000000}%
\pgfsetfillcolor{currentfill}%
\pgfsetlinewidth{0.602250pt}%
\definecolor{currentstroke}{rgb}{0.000000,0.000000,0.000000}%
\pgfsetstrokecolor{currentstroke}%
\pgfsetdash{}{0pt}%
\pgfsys@defobject{currentmarker}{\pgfqpoint{0.000000in}{-0.027778in}}{\pgfqpoint{0.000000in}{0.000000in}}{%
\pgfpathmoveto{\pgfqpoint{0.000000in}{0.000000in}}%
\pgfpathlineto{\pgfqpoint{0.000000in}{-0.027778in}}%
\pgfusepath{stroke,fill}%
}%
\begin{pgfscope}%
\pgfsys@transformshift{10.107047in}{3.093143in}%
\pgfsys@useobject{currentmarker}{}%
\end{pgfscope}%
\end{pgfscope}%
\begin{pgfscope}%
\pgfsetbuttcap%
\pgfsetroundjoin%
\definecolor{currentfill}{rgb}{0.000000,0.000000,0.000000}%
\pgfsetfillcolor{currentfill}%
\pgfsetlinewidth{0.602250pt}%
\definecolor{currentstroke}{rgb}{0.000000,0.000000,0.000000}%
\pgfsetstrokecolor{currentstroke}%
\pgfsetdash{}{0pt}%
\pgfsys@defobject{currentmarker}{\pgfqpoint{0.000000in}{-0.027778in}}{\pgfqpoint{0.000000in}{0.000000in}}{%
\pgfpathmoveto{\pgfqpoint{0.000000in}{0.000000in}}%
\pgfpathlineto{\pgfqpoint{0.000000in}{-0.027778in}}%
\pgfusepath{stroke,fill}%
}%
\begin{pgfscope}%
\pgfsys@transformshift{10.208421in}{3.093143in}%
\pgfsys@useobject{currentmarker}{}%
\end{pgfscope}%
\end{pgfscope}%
\begin{pgfscope}%
\pgfsetbuttcap%
\pgfsetroundjoin%
\definecolor{currentfill}{rgb}{0.000000,0.000000,0.000000}%
\pgfsetfillcolor{currentfill}%
\pgfsetlinewidth{0.803000pt}%
\definecolor{currentstroke}{rgb}{0.000000,0.000000,0.000000}%
\pgfsetstrokecolor{currentstroke}%
\pgfsetdash{}{0pt}%
\pgfsys@defobject{currentmarker}{\pgfqpoint{-0.048611in}{0.000000in}}{\pgfqpoint{-0.000000in}{0.000000in}}{%
\pgfpathmoveto{\pgfqpoint{-0.000000in}{0.000000in}}%
\pgfpathlineto{\pgfqpoint{-0.048611in}{0.000000in}}%
\pgfusepath{stroke,fill}%
}%
\begin{pgfscope}%
\pgfsys@transformshift{0.390138in}{3.093143in}%
\pgfsys@useobject{currentmarker}{}%
\end{pgfscope}%
\end{pgfscope}%
\begin{pgfscope}%
\definecolor{textcolor}{rgb}{0.000000,0.000000,0.000000}%
\pgfsetstrokecolor{textcolor}%
\pgfsetfillcolor{textcolor}%
\pgftext[x=0.195000in, y=3.023699in, left, base]{\color{textcolor}{\rmfamily\fontsize{15.000000}{18.000000}\selectfont\catcode`\^=\active\def^{\ifmmode\sp\else\^{}\fi}\catcode`\%=\active\def%{\%}0}}%
\end{pgfscope}%
\begin{pgfscope}%
\pgfsetbuttcap%
\pgfsetroundjoin%
\definecolor{currentfill}{rgb}{0.000000,0.000000,0.000000}%
\pgfsetfillcolor{currentfill}%
\pgfsetlinewidth{0.803000pt}%
\definecolor{currentstroke}{rgb}{0.000000,0.000000,0.000000}%
\pgfsetstrokecolor{currentstroke}%
\pgfsetdash{}{0pt}%
\pgfsys@defobject{currentmarker}{\pgfqpoint{-0.048611in}{0.000000in}}{\pgfqpoint{-0.000000in}{0.000000in}}{%
\pgfpathmoveto{\pgfqpoint{-0.000000in}{0.000000in}}%
\pgfpathlineto{\pgfqpoint{-0.048611in}{0.000000in}}%
\pgfusepath{stroke,fill}%
}%
\begin{pgfscope}%
\pgfsys@transformshift{0.390138in}{4.977961in}%
\pgfsys@useobject{currentmarker}{}%
\end{pgfscope}%
\end{pgfscope}%
\begin{pgfscope}%
\definecolor{textcolor}{rgb}{0.000000,0.000000,0.000000}%
\pgfsetstrokecolor{textcolor}%
\pgfsetfillcolor{textcolor}%
\pgftext[x=0.195000in, y=4.908516in, left, base]{\color{textcolor}{\rmfamily\fontsize{15.000000}{18.000000}\selectfont\catcode`\^=\active\def^{\ifmmode\sp\else\^{}\fi}\catcode`\%=\active\def%{\%}1}}%
\end{pgfscope}%
\begin{pgfscope}%
\pgfpathrectangle{\pgfqpoint{0.390138in}{3.093143in}}{\pgfqpoint{10.414862in}{1.884818in}}%
\pgfusepath{clip}%
\pgfsetrectcap%
\pgfsetroundjoin%
\pgfsetlinewidth{1.606000pt}%
\definecolor{currentstroke}{rgb}{0.000000,0.000000,0.000000}%
\pgfsetstrokecolor{currentstroke}%
\pgfsetdash{}{0pt}%
\pgfpathmoveto{\pgfqpoint{0.390138in}{4.977960in}}%
\pgfpathlineto{\pgfqpoint{4.176488in}{4.976848in}}%
\pgfpathlineto{\pgfqpoint{4.669040in}{4.974468in}}%
\pgfpathlineto{\pgfqpoint{4.972508in}{4.970896in}}%
\pgfpathlineto{\pgfqpoint{5.193106in}{4.966175in}}%
\pgfpathlineto{\pgfqpoint{5.368184in}{4.960276in}}%
\pgfpathlineto{\pgfqpoint{5.512915in}{4.953238in}}%
\pgfpathlineto{\pgfqpoint{5.636637in}{4.945051in}}%
\pgfpathlineto{\pgfqpoint{5.745185in}{4.935680in}}%
\pgfpathlineto{\pgfqpoint{5.842061in}{4.925107in}}%
\pgfpathlineto{\pgfqpoint{5.928433in}{4.913495in}}%
\pgfpathlineto{\pgfqpoint{6.007802in}{4.900619in}}%
\pgfpathlineto{\pgfqpoint{6.081335in}{4.886445in}}%
\pgfpathlineto{\pgfqpoint{6.149031in}{4.871156in}}%
\pgfpathlineto{\pgfqpoint{6.212059in}{4.854686in}}%
\pgfpathlineto{\pgfqpoint{6.271586in}{4.836868in}}%
\pgfpathlineto{\pgfqpoint{6.327611in}{4.817825in}}%
\pgfpathlineto{\pgfqpoint{6.381301in}{4.797252in}}%
\pgfpathlineto{\pgfqpoint{6.432657in}{4.775204in}}%
\pgfpathlineto{\pgfqpoint{6.481679in}{4.751766in}}%
\pgfpathlineto{\pgfqpoint{6.529534in}{4.726410in}}%
\pgfpathlineto{\pgfqpoint{6.575054in}{4.699808in}}%
\pgfpathlineto{\pgfqpoint{6.619407in}{4.671360in}}%
\pgfpathlineto{\pgfqpoint{6.662593in}{4.641070in}}%
\pgfpathlineto{\pgfqpoint{6.705779in}{4.608030in}}%
\pgfpathlineto{\pgfqpoint{6.747797in}{4.573058in}}%
\pgfpathlineto{\pgfqpoint{6.788649in}{4.536214in}}%
\pgfpathlineto{\pgfqpoint{6.829500in}{4.496400in}}%
\pgfpathlineto{\pgfqpoint{6.870352in}{4.453458in}}%
\pgfpathlineto{\pgfqpoint{6.911203in}{4.407239in}}%
\pgfpathlineto{\pgfqpoint{6.952055in}{4.357611in}}%
\pgfpathlineto{\pgfqpoint{6.992906in}{4.304465in}}%
\pgfpathlineto{\pgfqpoint{7.034925in}{4.246044in}}%
\pgfpathlineto{\pgfqpoint{7.076944in}{4.183774in}}%
\pgfpathlineto{\pgfqpoint{7.120130in}{4.115781in}}%
\pgfpathlineto{\pgfqpoint{7.165650in}{4.039841in}}%
\pgfpathlineto{\pgfqpoint{7.213505in}{3.955544in}}%
\pgfpathlineto{\pgfqpoint{7.266028in}{3.858318in}}%
\pgfpathlineto{\pgfqpoint{7.326722in}{3.741085in}}%
\pgfpathlineto{\pgfqpoint{7.518140in}{3.367316in}}%
\pgfpathlineto{\pgfqpoint{7.555490in}{3.301572in}}%
\pgfpathlineto{\pgfqpoint{7.587004in}{3.250435in}}%
\pgfpathlineto{\pgfqpoint{7.613849in}{3.210849in}}%
\pgfpathlineto{\pgfqpoint{7.638360in}{3.178538in}}%
\pgfpathlineto{\pgfqpoint{7.660537in}{3.152954in}}%
\pgfpathlineto{\pgfqpoint{7.680379in}{3.133382in}}%
\pgfpathlineto{\pgfqpoint{7.699054in}{3.118131in}}%
\pgfpathlineto{\pgfqpoint{7.715394in}{3.107538in}}%
\pgfpathlineto{\pgfqpoint{7.730568in}{3.100175in}}%
\pgfpathlineto{\pgfqpoint{7.744574in}{3.095633in}}%
\pgfpathlineto{\pgfqpoint{7.758580in}{3.093378in}}%
\pgfpathlineto{\pgfqpoint{7.771419in}{3.093426in}}%
\pgfpathlineto{\pgfqpoint{7.784258in}{3.095592in}}%
\pgfpathlineto{\pgfqpoint{7.797097in}{3.099966in}}%
\pgfpathlineto{\pgfqpoint{7.809936in}{3.106638in}}%
\pgfpathlineto{\pgfqpoint{7.822775in}{3.115692in}}%
\pgfpathlineto{\pgfqpoint{7.836782in}{3.128383in}}%
\pgfpathlineto{\pgfqpoint{7.850788in}{3.144108in}}%
\pgfpathlineto{\pgfqpoint{7.865961in}{3.164674in}}%
\pgfpathlineto{\pgfqpoint{7.881135in}{3.189014in}}%
\pgfpathlineto{\pgfqpoint{7.897475in}{3.219550in}}%
\pgfpathlineto{\pgfqpoint{7.914983in}{3.257344in}}%
\pgfpathlineto{\pgfqpoint{7.932491in}{3.300464in}}%
\pgfpathlineto{\pgfqpoint{7.951166in}{3.352367in}}%
\pgfpathlineto{\pgfqpoint{7.971008in}{3.414176in}}%
\pgfpathlineto{\pgfqpoint{7.992017in}{3.486985in}}%
\pgfpathlineto{\pgfqpoint{8.014194in}{3.571781in}}%
\pgfpathlineto{\pgfqpoint{8.038705in}{3.674431in}}%
\pgfpathlineto{\pgfqpoint{8.065550in}{3.796551in}}%
\pgfpathlineto{\pgfqpoint{8.097064in}{3.950637in}}%
\pgfpathlineto{\pgfqpoint{8.140250in}{4.174160in}}%
\pgfpathlineto{\pgfqpoint{8.206779in}{4.518612in}}%
\pgfpathlineto{\pgfqpoint{8.234792in}{4.651194in}}%
\pgfpathlineto{\pgfqpoint{8.256968in}{4.745717in}}%
\pgfpathlineto{\pgfqpoint{8.275643in}{4.815878in}}%
\pgfpathlineto{\pgfqpoint{8.291984in}{4.868740in}}%
\pgfpathlineto{\pgfqpoint{8.305990in}{4.906795in}}%
\pgfpathlineto{\pgfqpoint{8.317662in}{4.932848in}}%
\pgfpathlineto{\pgfqpoint{8.328167in}{4.951545in}}%
\pgfpathlineto{\pgfqpoint{8.337504in}{4.964155in}}%
\pgfpathlineto{\pgfqpoint{8.345675in}{4.971942in}}%
\pgfpathlineto{\pgfqpoint{8.353845in}{4.976575in}}%
\pgfpathlineto{\pgfqpoint{8.360848in}{4.977955in}}%
\pgfpathlineto{\pgfqpoint{8.367851in}{4.976877in}}%
\pgfpathlineto{\pgfqpoint{8.374854in}{4.973282in}}%
\pgfpathlineto{\pgfqpoint{8.381857in}{4.967118in}}%
\pgfpathlineto{\pgfqpoint{8.390028in}{4.956621in}}%
\pgfpathlineto{\pgfqpoint{8.398198in}{4.942504in}}%
\pgfpathlineto{\pgfqpoint{8.407535in}{4.921881in}}%
\pgfpathlineto{\pgfqpoint{8.418040in}{4.892902in}}%
\pgfpathlineto{\pgfqpoint{8.429712in}{4.853503in}}%
\pgfpathlineto{\pgfqpoint{8.442551in}{4.801458in}}%
\pgfpathlineto{\pgfqpoint{8.456557in}{4.734473in}}%
\pgfpathlineto{\pgfqpoint{8.471731in}{4.650325in}}%
\pgfpathlineto{\pgfqpoint{8.488071in}{4.547081in}}%
\pgfpathlineto{\pgfqpoint{8.506746in}{4.414713in}}%
\pgfpathlineto{\pgfqpoint{8.528923in}{4.241097in}}%
\pgfpathlineto{\pgfqpoint{8.560437in}{3.974497in}}%
\pgfpathlineto{\pgfqpoint{8.608291in}{3.569257in}}%
\pgfpathlineto{\pgfqpoint{8.629301in}{3.410988in}}%
\pgfpathlineto{\pgfqpoint{8.645641in}{3.303896in}}%
\pgfpathlineto{\pgfqpoint{8.659647in}{3.226501in}}%
\pgfpathlineto{\pgfqpoint{8.671319in}{3.173955in}}%
\pgfpathlineto{\pgfqpoint{8.680657in}{3.140655in}}%
\pgfpathlineto{\pgfqpoint{8.688827in}{3.118399in}}%
\pgfpathlineto{\pgfqpoint{8.695830in}{3.104724in}}%
\pgfpathlineto{\pgfqpoint{8.701666in}{3.097293in}}%
\pgfpathlineto{\pgfqpoint{8.706335in}{3.094019in}}%
\pgfpathlineto{\pgfqpoint{8.711004in}{3.093172in}}%
\pgfpathlineto{\pgfqpoint{8.715672in}{3.094794in}}%
\pgfpathlineto{\pgfqpoint{8.720341in}{3.098921in}}%
\pgfpathlineto{\pgfqpoint{8.726177in}{3.107647in}}%
\pgfpathlineto{\pgfqpoint{8.732013in}{3.120377in}}%
\pgfpathlineto{\pgfqpoint{8.739016in}{3.140972in}}%
\pgfpathlineto{\pgfqpoint{8.747186in}{3.172343in}}%
\pgfpathlineto{\pgfqpoint{8.756524in}{3.217811in}}%
\pgfpathlineto{\pgfqpoint{8.767029in}{3.280967in}}%
\pgfpathlineto{\pgfqpoint{8.778700in}{3.365440in}}%
\pgfpathlineto{\pgfqpoint{8.791539in}{3.474533in}}%
\pgfpathlineto{\pgfqpoint{8.806713in}{3.622738in}}%
\pgfpathlineto{\pgfqpoint{8.825388in}{3.827889in}}%
\pgfpathlineto{\pgfqpoint{8.856902in}{4.203904in}}%
\pgfpathlineto{\pgfqpoint{8.884914in}{4.527578in}}%
\pgfpathlineto{\pgfqpoint{8.901255in}{4.691992in}}%
\pgfpathlineto{\pgfqpoint{8.914094in}{4.800705in}}%
\pgfpathlineto{\pgfqpoint{8.924599in}{4.872887in}}%
\pgfpathlineto{\pgfqpoint{8.933936in}{4.922479in}}%
\pgfpathlineto{\pgfqpoint{8.940939in}{4.949819in}}%
\pgfpathlineto{\pgfqpoint{8.946775in}{4.965748in}}%
\pgfpathlineto{\pgfqpoint{8.951444in}{4.973827in}}%
\pgfpathlineto{\pgfqpoint{8.956113in}{4.977646in}}%
\pgfpathlineto{\pgfqpoint{8.959614in}{4.977660in}}%
\pgfpathlineto{\pgfqpoint{8.963116in}{4.975197in}}%
\pgfpathlineto{\pgfqpoint{8.967784in}{4.968020in}}%
\pgfpathlineto{\pgfqpoint{8.972453in}{4.956360in}}%
\pgfpathlineto{\pgfqpoint{8.978289in}{4.935459in}}%
\pgfpathlineto{\pgfqpoint{8.985292in}{4.901127in}}%
\pgfpathlineto{\pgfqpoint{8.993463in}{4.848508in}}%
\pgfpathlineto{\pgfqpoint{9.002800in}{4.772358in}}%
\pgfpathlineto{\pgfqpoint{9.013305in}{4.667533in}}%
\pgfpathlineto{\pgfqpoint{9.026144in}{4.514894in}}%
\pgfpathlineto{\pgfqpoint{9.041317in}{4.306523in}}%
\pgfpathlineto{\pgfqpoint{9.065828in}{3.933670in}}%
\pgfpathlineto{\pgfqpoint{9.091506in}{3.551242in}}%
\pgfpathlineto{\pgfqpoint{9.105512in}{3.372421in}}%
\pgfpathlineto{\pgfqpoint{9.117184in}{3.250682in}}%
\pgfpathlineto{\pgfqpoint{9.126522in}{3.175682in}}%
\pgfpathlineto{\pgfqpoint{9.133525in}{3.134336in}}%
\pgfpathlineto{\pgfqpoint{9.139361in}{3.110443in}}%
\pgfpathlineto{\pgfqpoint{9.144029in}{3.098584in}}%
\pgfpathlineto{\pgfqpoint{9.147531in}{3.094051in}}%
\pgfpathlineto{\pgfqpoint{9.151033in}{3.093330in}}%
\pgfpathlineto{\pgfqpoint{9.153367in}{3.094992in}}%
\pgfpathlineto{\pgfqpoint{9.156869in}{3.100727in}}%
\pgfpathlineto{\pgfqpoint{9.161537in}{3.114457in}}%
\pgfpathlineto{\pgfqpoint{9.167373in}{3.141400in}}%
\pgfpathlineto{\pgfqpoint{9.174376in}{3.187932in}}%
\pgfpathlineto{\pgfqpoint{9.182547in}{3.261222in}}%
\pgfpathlineto{\pgfqpoint{9.191884in}{3.368541in}}%
\pgfpathlineto{\pgfqpoint{9.202389in}{3.516085in}}%
\pgfpathlineto{\pgfqpoint{9.216395in}{3.748143in}}%
\pgfpathlineto{\pgfqpoint{9.239739in}{4.182552in}}%
\pgfpathlineto{\pgfqpoint{9.260748in}{4.557881in}}%
\pgfpathlineto{\pgfqpoint{9.272420in}{4.731873in}}%
\pgfpathlineto{\pgfqpoint{9.281757in}{4.842965in}}%
\pgfpathlineto{\pgfqpoint{9.289928in}{4.915125in}}%
\pgfpathlineto{\pgfqpoint{9.295764in}{4.950615in}}%
\pgfpathlineto{\pgfqpoint{9.300432in}{4.968704in}}%
\pgfpathlineto{\pgfqpoint{9.303934in}{4.976031in}}%
\pgfpathlineto{\pgfqpoint{9.306268in}{4.977881in}}%
\pgfpathlineto{\pgfqpoint{9.308603in}{4.977274in}}%
\pgfpathlineto{\pgfqpoint{9.310937in}{4.974191in}}%
\pgfpathlineto{\pgfqpoint{9.314439in}{4.964900in}}%
\pgfpathlineto{\pgfqpoint{9.319107in}{4.943789in}}%
\pgfpathlineto{\pgfqpoint{9.324943in}{4.903495in}}%
\pgfpathlineto{\pgfqpoint{9.331946in}{4.835279in}}%
\pgfpathlineto{\pgfqpoint{9.340117in}{4.729856in}}%
\pgfpathlineto{\pgfqpoint{9.349454in}{4.578909in}}%
\pgfpathlineto{\pgfqpoint{9.361126in}{4.353469in}}%
\pgfpathlineto{\pgfqpoint{9.380968in}{3.918881in}}%
\pgfpathlineto{\pgfqpoint{9.400810in}{3.500900in}}%
\pgfpathlineto{\pgfqpoint{9.411315in}{3.320482in}}%
\pgfpathlineto{\pgfqpoint{9.419485in}{3.211415in}}%
\pgfpathlineto{\pgfqpoint{9.426488in}{3.144257in}}%
\pgfpathlineto{\pgfqpoint{9.432324in}{3.108964in}}%
\pgfpathlineto{\pgfqpoint{9.435826in}{3.097356in}}%
\pgfpathlineto{\pgfqpoint{9.438160in}{3.093724in}}%
\pgfpathlineto{\pgfqpoint{9.440495in}{3.093426in}}%
\pgfpathlineto{\pgfqpoint{9.442829in}{3.096490in}}%
\pgfpathlineto{\pgfqpoint{9.446331in}{3.107421in}}%
\pgfpathlineto{\pgfqpoint{9.450999in}{3.133814in}}%
\pgfpathlineto{\pgfqpoint{9.456835in}{3.185526in}}%
\pgfpathlineto{\pgfqpoint{9.463838in}{3.273946in}}%
\pgfpathlineto{\pgfqpoint{9.472009in}{3.410437in}}%
\pgfpathlineto{\pgfqpoint{9.482513in}{3.630046in}}%
\pgfpathlineto{\pgfqpoint{9.497687in}{4.003843in}}%
\pgfpathlineto{\pgfqpoint{9.521030in}{4.577913in}}%
\pgfpathlineto{\pgfqpoint{9.531535in}{4.779639in}}%
\pgfpathlineto{\pgfqpoint{9.538538in}{4.879676in}}%
\pgfpathlineto{\pgfqpoint{9.544374in}{4.937820in}}%
\pgfpathlineto{\pgfqpoint{9.549043in}{4.966251in}}%
\pgfpathlineto{\pgfqpoint{9.552544in}{4.976484in}}%
\pgfpathlineto{\pgfqpoint{9.554879in}{4.977891in}}%
\pgfpathlineto{\pgfqpoint{9.557213in}{4.974916in}}%
\pgfpathlineto{\pgfqpoint{9.559548in}{4.967535in}}%
\pgfpathlineto{\pgfqpoint{9.563049in}{4.948208in}}%
\pgfpathlineto{\pgfqpoint{9.567718in}{4.907187in}}%
\pgfpathlineto{\pgfqpoint{9.573554in}{4.832171in}}%
\pgfpathlineto{\pgfqpoint{9.580557in}{4.709748in}}%
\pgfpathlineto{\pgfqpoint{9.589894in}{4.499397in}}%
\pgfpathlineto{\pgfqpoint{9.602733in}{4.149314in}}%
\pgfpathlineto{\pgfqpoint{9.627244in}{3.467545in}}%
\pgfpathlineto{\pgfqpoint{9.636582in}{3.271599in}}%
\pgfpathlineto{\pgfqpoint{9.643585in}{3.167413in}}%
\pgfpathlineto{\pgfqpoint{9.648254in}{3.121987in}}%
\pgfpathlineto{\pgfqpoint{9.651755in}{3.101566in}}%
\pgfpathlineto{\pgfqpoint{9.654090in}{3.094693in}}%
\pgfpathlineto{\pgfqpoint{9.656424in}{3.093318in}}%
\pgfpathlineto{\pgfqpoint{9.658758in}{3.097492in}}%
\pgfpathlineto{\pgfqpoint{9.661093in}{3.107236in}}%
\pgfpathlineto{\pgfqpoint{9.664594in}{3.132260in}}%
\pgfpathlineto{\pgfqpoint{9.669263in}{3.184718in}}%
\pgfpathlineto{\pgfqpoint{9.675099in}{3.279585in}}%
\pgfpathlineto{\pgfqpoint{9.682102in}{3.432275in}}%
\pgfpathlineto{\pgfqpoint{9.691439in}{3.688872in}}%
\pgfpathlineto{\pgfqpoint{9.710114in}{4.286629in}}%
\pgfpathlineto{\pgfqpoint{9.722954in}{4.658573in}}%
\pgfpathlineto{\pgfqpoint{9.731124in}{4.835527in}}%
\pgfpathlineto{\pgfqpoint{9.736960in}{4.921786in}}%
\pgfpathlineto{\pgfqpoint{9.741628in}{4.963066in}}%
\pgfpathlineto{\pgfqpoint{9.745130in}{4.976751in}}%
\pgfpathlineto{\pgfqpoint{9.746297in}{4.977930in}}%
\pgfpathlineto{\pgfqpoint{9.747464in}{4.977400in}}%
\pgfpathlineto{\pgfqpoint{9.749799in}{4.971192in}}%
\pgfpathlineto{\pgfqpoint{9.753300in}{4.949016in}}%
\pgfpathlineto{\pgfqpoint{9.757969in}{4.895781in}}%
\pgfpathlineto{\pgfqpoint{9.763805in}{4.792961in}}%
\pgfpathlineto{\pgfqpoint{9.770808in}{4.621961in}}%
\pgfpathlineto{\pgfqpoint{9.780146in}{4.331209in}}%
\pgfpathlineto{\pgfqpoint{9.810492in}{3.325513in}}%
\pgfpathlineto{\pgfqpoint{9.817496in}{3.183334in}}%
\pgfpathlineto{\pgfqpoint{9.822164in}{3.123736in}}%
\pgfpathlineto{\pgfqpoint{9.825666in}{3.099440in}}%
\pgfpathlineto{\pgfqpoint{9.828000in}{3.093374in}}%
\pgfpathlineto{\pgfqpoint{9.829167in}{3.093431in}}%
\pgfpathlineto{\pgfqpoint{9.831502in}{3.099771in}}%
\pgfpathlineto{\pgfqpoint{9.833836in}{3.114410in}}%
\pgfpathlineto{\pgfqpoint{9.837338in}{3.151772in}}%
\pgfpathlineto{\pgfqpoint{9.842006in}{3.229421in}}%
\pgfpathlineto{\pgfqpoint{9.847842in}{3.367776in}}%
\pgfpathlineto{\pgfqpoint{9.856013in}{3.625291in}}%
\pgfpathlineto{\pgfqpoint{9.870019in}{4.162338in}}%
\pgfpathlineto{\pgfqpoint{9.882858in}{4.628094in}}%
\pgfpathlineto{\pgfqpoint{9.889861in}{4.818570in}}%
\pgfpathlineto{\pgfqpoint{9.895697in}{4.924747in}}%
\pgfpathlineto{\pgfqpoint{9.900366in}{4.969457in}}%
\pgfpathlineto{\pgfqpoint{9.902700in}{4.977466in}}%
\pgfpathlineto{\pgfqpoint{9.903867in}{4.977796in}}%
\pgfpathlineto{\pgfqpoint{9.905034in}{4.975660in}}%
\pgfpathlineto{\pgfqpoint{9.907369in}{4.963980in}}%
\pgfpathlineto{\pgfqpoint{9.910870in}{4.928067in}}%
\pgfpathlineto{\pgfqpoint{9.915539in}{4.846908in}}%
\pgfpathlineto{\pgfqpoint{9.921375in}{4.696310in}}%
\pgfpathlineto{\pgfqpoint{9.929545in}{4.411071in}}%
\pgfpathlineto{\pgfqpoint{9.959892in}{3.250391in}}%
\pgfpathlineto{\pgfqpoint{9.965728in}{3.138685in}}%
\pgfpathlineto{\pgfqpoint{9.969230in}{3.103298in}}%
\pgfpathlineto{\pgfqpoint{9.971564in}{3.093768in}}%
\pgfpathlineto{\pgfqpoint{9.972731in}{3.093312in}}%
\pgfpathlineto{\pgfqpoint{9.973898in}{3.095751in}}%
\pgfpathlineto{\pgfqpoint{9.976233in}{3.109316in}}%
\pgfpathlineto{\pgfqpoint{9.979734in}{3.151181in}}%
\pgfpathlineto{\pgfqpoint{9.984403in}{3.245634in}}%
\pgfpathlineto{\pgfqpoint{9.990239in}{3.419711in}}%
\pgfpathlineto{\pgfqpoint{9.998409in}{3.744083in}}%
\pgfpathlineto{\pgfqpoint{10.021753in}{4.739168in}}%
\pgfpathlineto{\pgfqpoint{10.027589in}{4.890769in}}%
\pgfpathlineto{\pgfqpoint{10.032258in}{4.959792in}}%
\pgfpathlineto{\pgfqpoint{10.034592in}{4.975113in}}%
\pgfpathlineto{\pgfqpoint{10.035759in}{4.977805in}}%
\pgfpathlineto{\pgfqpoint{10.036926in}{4.977150in}}%
\pgfpathlineto{\pgfqpoint{10.039261in}{4.965766in}}%
\pgfpathlineto{\pgfqpoint{10.042762in}{4.923697in}}%
\pgfpathlineto{\pgfqpoint{10.047431in}{4.822763in}}%
\pgfpathlineto{\pgfqpoint{10.053267in}{4.632162in}}%
\pgfpathlineto{\pgfqpoint{10.061437in}{4.275607in}}%
\pgfpathlineto{\pgfqpoint{10.081279in}{3.360291in}}%
\pgfpathlineto{\pgfqpoint{10.087115in}{3.189697in}}%
\pgfpathlineto{\pgfqpoint{10.091784in}{3.112455in}}%
\pgfpathlineto{\pgfqpoint{10.094119in}{3.095818in}}%
\pgfpathlineto{\pgfqpoint{10.095286in}{3.093205in}}%
\pgfpathlineto{\pgfqpoint{10.096453in}{3.094437in}}%
\pgfpathlineto{\pgfqpoint{10.098787in}{3.108461in}}%
\pgfpathlineto{\pgfqpoint{10.102289in}{3.158079in}}%
\pgfpathlineto{\pgfqpoint{10.106958in}{3.275026in}}%
\pgfpathlineto{\pgfqpoint{10.112793in}{3.492478in}}%
\pgfpathlineto{\pgfqpoint{10.122131in}{3.951146in}}%
\pgfpathlineto{\pgfqpoint{10.136137in}{4.648240in}}%
\pgfpathlineto{\pgfqpoint{10.141973in}{4.850130in}}%
\pgfpathlineto{\pgfqpoint{10.146642in}{4.947156in}}%
\pgfpathlineto{\pgfqpoint{10.150143in}{4.976661in}}%
\pgfpathlineto{\pgfqpoint{10.151311in}{4.977846in}}%
\pgfpathlineto{\pgfqpoint{10.152478in}{4.974649in}}%
\pgfpathlineto{\pgfqpoint{10.154812in}{4.955122in}}%
\pgfpathlineto{\pgfqpoint{10.158314in}{4.893572in}}%
\pgfpathlineto{\pgfqpoint{10.162982in}{4.755045in}}%
\pgfpathlineto{\pgfqpoint{10.168818in}{4.505363in}}%
\pgfpathlineto{\pgfqpoint{10.179323in}{3.933744in}}%
\pgfpathlineto{\pgfqpoint{10.189828in}{3.395856in}}%
\pgfpathlineto{\pgfqpoint{10.195664in}{3.194026in}}%
\pgfpathlineto{\pgfqpoint{10.199165in}{3.122701in}}%
\pgfpathlineto{\pgfqpoint{10.201500in}{3.098507in}}%
\pgfpathlineto{\pgfqpoint{10.202667in}{3.093701in}}%
\pgfpathlineto{\pgfqpoint{10.203834in}{3.093824in}}%
\pgfpathlineto{\pgfqpoint{10.206168in}{3.108907in}}%
\pgfpathlineto{\pgfqpoint{10.208503in}{3.143600in}}%
\pgfpathlineto{\pgfqpoint{10.212004in}{3.231029in}}%
\pgfpathlineto{\pgfqpoint{10.216673in}{3.407401in}}%
\pgfpathlineto{\pgfqpoint{10.223676in}{3.769365in}}%
\pgfpathlineto{\pgfqpoint{10.241184in}{4.735775in}}%
\pgfpathlineto{\pgfqpoint{10.245853in}{4.892310in}}%
\pgfpathlineto{\pgfqpoint{10.249354in}{4.959013in}}%
\pgfpathlineto{\pgfqpoint{10.251689in}{4.976883in}}%
\pgfpathlineto{\pgfqpoint{10.252856in}{4.977569in}}%
\pgfpathlineto{\pgfqpoint{10.254023in}{4.972709in}}%
\pgfpathlineto{\pgfqpoint{10.256357in}{4.946409in}}%
\pgfpathlineto{\pgfqpoint{10.259859in}{4.866592in}}%
\pgfpathlineto{\pgfqpoint{10.264528in}{4.691236in}}%
\pgfpathlineto{\pgfqpoint{10.271531in}{4.314885in}}%
\pgfpathlineto{\pgfqpoint{10.289039in}{3.304866in}}%
\pgfpathlineto{\pgfqpoint{10.293707in}{3.154419in}}%
\pgfpathlineto{\pgfqpoint{10.297209in}{3.100004in}}%
\pgfpathlineto{\pgfqpoint{10.298376in}{3.093893in}}%
\pgfpathlineto{\pgfqpoint{10.299543in}{3.093937in}}%
\pgfpathlineto{\pgfqpoint{10.300710in}{3.100160in}}%
\pgfpathlineto{\pgfqpoint{10.303045in}{3.131039in}}%
\pgfpathlineto{\pgfqpoint{10.306546in}{3.221981in}}%
\pgfpathlineto{\pgfqpoint{10.311215in}{3.418258in}}%
\pgfpathlineto{\pgfqpoint{10.318218in}{3.830310in}}%
\pgfpathlineto{\pgfqpoint{10.332224in}{4.699625in}}%
\pgfpathlineto{\pgfqpoint{10.336893in}{4.883569in}}%
\pgfpathlineto{\pgfqpoint{10.340395in}{4.959716in}}%
\pgfpathlineto{\pgfqpoint{10.342729in}{4.977618in}}%
\pgfpathlineto{\pgfqpoint{10.343896in}{4.976362in}}%
\pgfpathlineto{\pgfqpoint{10.346231in}{4.953347in}}%
\pgfpathlineto{\pgfqpoint{10.349732in}{4.868781in}}%
\pgfpathlineto{\pgfqpoint{10.354401in}{4.671193in}}%
\pgfpathlineto{\pgfqpoint{10.361404in}{4.241354in}}%
\pgfpathlineto{\pgfqpoint{10.375410in}{3.338303in}}%
\pgfpathlineto{\pgfqpoint{10.380079in}{3.161460in}}%
\pgfpathlineto{\pgfqpoint{10.383581in}{3.099446in}}%
\pgfpathlineto{\pgfqpoint{10.384748in}{3.093470in}}%
\pgfpathlineto{\pgfqpoint{10.385915in}{3.095021in}}%
\pgfpathlineto{\pgfqpoint{10.388249in}{3.120716in}}%
\pgfpathlineto{\pgfqpoint{10.391751in}{3.214192in}}%
\pgfpathlineto{\pgfqpoint{10.396420in}{3.430794in}}%
\pgfpathlineto{\pgfqpoint{10.403423in}{3.894166in}}%
\pgfpathlineto{\pgfqpoint{10.415095in}{4.691303in}}%
\pgfpathlineto{\pgfqpoint{10.419763in}{4.892304in}}%
\pgfpathlineto{\pgfqpoint{10.423265in}{4.967428in}}%
\pgfpathlineto{\pgfqpoint{10.425599in}{4.977293in}}%
\pgfpathlineto{\pgfqpoint{10.426766in}{4.969825in}}%
\pgfpathlineto{\pgfqpoint{10.429101in}{4.930249in}}%
\pgfpathlineto{\pgfqpoint{10.432602in}{4.811956in}}%
\pgfpathlineto{\pgfqpoint{10.437271in}{4.559059in}}%
\pgfpathlineto{\pgfqpoint{10.445441in}{3.960594in}}%
\pgfpathlineto{\pgfqpoint{10.454779in}{3.326586in}}%
\pgfpathlineto{\pgfqpoint{10.459448in}{3.144041in}}%
\pgfpathlineto{\pgfqpoint{10.461782in}{3.101694in}}%
\pgfpathlineto{\pgfqpoint{10.462949in}{3.093786in}}%
\pgfpathlineto{\pgfqpoint{10.464116in}{3.094900in}}%
\pgfpathlineto{\pgfqpoint{10.466451in}{3.124208in}}%
\pgfpathlineto{\pgfqpoint{10.469952in}{3.233650in}}%
\pgfpathlineto{\pgfqpoint{10.474621in}{3.486691in}}%
\pgfpathlineto{\pgfqpoint{10.482791in}{4.107630in}}%
\pgfpathlineto{\pgfqpoint{10.490962in}{4.699303in}}%
\pgfpathlineto{\pgfqpoint{10.495630in}{4.909495in}}%
\pgfpathlineto{\pgfqpoint{10.499132in}{4.975116in}}%
\pgfpathlineto{\pgfqpoint{10.500299in}{4.977678in}}%
\pgfpathlineto{\pgfqpoint{10.501466in}{4.970389in}}%
\pgfpathlineto{\pgfqpoint{10.503801in}{4.926509in}}%
\pgfpathlineto{\pgfqpoint{10.507302in}{4.791066in}}%
\pgfpathlineto{\pgfqpoint{10.511971in}{4.500964in}}%
\pgfpathlineto{\pgfqpoint{10.530646in}{3.171372in}}%
\pgfpathlineto{\pgfqpoint{10.534147in}{3.097113in}}%
\pgfpathlineto{\pgfqpoint{10.535315in}{3.093244in}}%
\pgfpathlineto{\pgfqpoint{10.536482in}{3.100060in}}%
\pgfpathlineto{\pgfqpoint{10.538816in}{3.145492in}}%
\pgfpathlineto{\pgfqpoint{10.542318in}{3.288990in}}%
\pgfpathlineto{\pgfqpoint{10.546987in}{3.597348in}}%
\pgfpathlineto{\pgfqpoint{10.563327in}{4.849526in}}%
\pgfpathlineto{\pgfqpoint{10.566829in}{4.959665in}}%
\pgfpathlineto{\pgfqpoint{10.569163in}{4.977691in}}%
\pgfpathlineto{\pgfqpoint{10.570330in}{4.969410in}}%
\pgfpathlineto{\pgfqpoint{10.572665in}{4.918534in}}%
\pgfpathlineto{\pgfqpoint{10.576166in}{4.761503in}}%
\pgfpathlineto{\pgfqpoint{10.580835in}{4.429521in}}%
\pgfpathlineto{\pgfqpoint{10.596008in}{3.221982in}}%
\pgfpathlineto{\pgfqpoint{10.599510in}{3.109295in}}%
\pgfpathlineto{\pgfqpoint{10.601844in}{3.094074in}}%
\pgfpathlineto{\pgfqpoint{10.603011in}{3.105124in}}%
\pgfpathlineto{\pgfqpoint{10.605346in}{3.164064in}}%
\pgfpathlineto{\pgfqpoint{10.608847in}{3.338226in}}%
\pgfpathlineto{\pgfqpoint{10.614683in}{3.801613in}}%
\pgfpathlineto{\pgfqpoint{10.626355in}{4.788349in}}%
\pgfpathlineto{\pgfqpoint{10.629857in}{4.938736in}}%
\pgfpathlineto{\pgfqpoint{10.632191in}{4.976716in}}%
\pgfpathlineto{\pgfqpoint{10.633358in}{4.975790in}}%
\pgfpathlineto{\pgfqpoint{10.635693in}{4.933904in}}%
\pgfpathlineto{\pgfqpoint{10.639194in}{4.776083in}}%
\pgfpathlineto{\pgfqpoint{10.643863in}{4.421446in}}%
\pgfpathlineto{\pgfqpoint{10.657869in}{3.222736in}}%
\pgfpathlineto{\pgfqpoint{10.661371in}{3.105389in}}%
\pgfpathlineto{\pgfqpoint{10.662538in}{3.093822in}}%
\pgfpathlineto{\pgfqpoint{10.663705in}{3.096594in}}%
\pgfpathlineto{\pgfqpoint{10.666039in}{3.144999in}}%
\pgfpathlineto{\pgfqpoint{10.669541in}{3.318441in}}%
\pgfpathlineto{\pgfqpoint{10.674210in}{3.698838in}}%
\pgfpathlineto{\pgfqpoint{10.687049in}{4.839569in}}%
\pgfpathlineto{\pgfqpoint{10.690550in}{4.964877in}}%
\pgfpathlineto{\pgfqpoint{10.691718in}{4.977238in}}%
\pgfpathlineto{\pgfqpoint{10.692885in}{4.974256in}}%
\pgfpathlineto{\pgfqpoint{10.695219in}{4.922461in}}%
\pgfpathlineto{\pgfqpoint{10.698721in}{4.737482in}}%
\pgfpathlineto{\pgfqpoint{10.703389in}{4.335280in}}%
\pgfpathlineto{\pgfqpoint{10.715061in}{3.252790in}}%
\pgfpathlineto{\pgfqpoint{10.718563in}{3.111188in}}%
\pgfpathlineto{\pgfqpoint{10.719730in}{3.095079in}}%
\pgfpathlineto{\pgfqpoint{10.720897in}{3.095319in}}%
\pgfpathlineto{\pgfqpoint{10.722064in}{3.111969in}}%
\pgfpathlineto{\pgfqpoint{10.724399in}{3.193317in}}%
\pgfpathlineto{\pgfqpoint{10.727900in}{3.423680in}}%
\pgfpathlineto{\pgfqpoint{10.733736in}{3.999622in}}%
\pgfpathlineto{\pgfqpoint{10.741907in}{4.776910in}}%
\pgfpathlineto{\pgfqpoint{10.745408in}{4.946698in}}%
\pgfpathlineto{\pgfqpoint{10.747742in}{4.977868in}}%
\pgfpathlineto{\pgfqpoint{10.748910in}{4.967311in}}%
\pgfpathlineto{\pgfqpoint{10.751244in}{4.894541in}}%
\pgfpathlineto{\pgfqpoint{10.754746in}{4.667741in}}%
\pgfpathlineto{\pgfqpoint{10.760581in}{4.078565in}}%
\pgfpathlineto{\pgfqpoint{10.768752in}{3.281364in}}%
\pgfpathlineto{\pgfqpoint{10.772253in}{3.116484in}}%
\pgfpathlineto{\pgfqpoint{10.774588in}{3.094724in}}%
\pgfpathlineto{\pgfqpoint{10.775755in}{3.111686in}}%
\pgfpathlineto{\pgfqpoint{10.778089in}{3.199991in}}%
\pgfpathlineto{\pgfqpoint{10.781591in}{3.453737in}}%
\pgfpathlineto{\pgfqpoint{10.788594in}{4.212605in}}%
\pgfpathlineto{\pgfqpoint{10.794430in}{4.777453in}}%
\pgfpathlineto{\pgfqpoint{10.797931in}{4.952768in}}%
\pgfpathlineto{\pgfqpoint{10.800266in}{4.976391in}}%
\pgfpathlineto{\pgfqpoint{10.801433in}{4.958656in}}%
\pgfpathlineto{\pgfqpoint{10.803767in}{4.865556in}}%
\pgfpathlineto{\pgfqpoint{10.807269in}{4.598308in}}%
\pgfpathlineto{\pgfqpoint{10.815000in}{3.728906in}}%
\pgfpathlineto{\pgfqpoint{10.815000in}{3.728906in}}%
\pgfusepath{stroke}%
\end{pgfscope}%
\begin{pgfscope}%
\pgfpathrectangle{\pgfqpoint{0.390138in}{3.093143in}}{\pgfqpoint{10.414862in}{1.884818in}}%
\pgfusepath{clip}%
\pgfsetrectcap%
\pgfsetroundjoin%
\pgfsetlinewidth{1.606000pt}%
\definecolor{currentstroke}{rgb}{0.000000,0.000000,1.000000}%
\pgfsetstrokecolor{currentstroke}%
\pgfsetdash{}{0pt}%
\pgfpathmoveto{\pgfqpoint{0.390138in}{3.093143in}}%
\pgfpathlineto{\pgfqpoint{4.176488in}{3.094255in}}%
\pgfpathlineto{\pgfqpoint{4.669040in}{3.096635in}}%
\pgfpathlineto{\pgfqpoint{4.972508in}{3.100208in}}%
\pgfpathlineto{\pgfqpoint{5.193106in}{3.104929in}}%
\pgfpathlineto{\pgfqpoint{5.368184in}{3.110827in}}%
\pgfpathlineto{\pgfqpoint{5.512915in}{3.117866in}}%
\pgfpathlineto{\pgfqpoint{5.636637in}{3.126053in}}%
\pgfpathlineto{\pgfqpoint{5.745185in}{3.135423in}}%
\pgfpathlineto{\pgfqpoint{5.842061in}{3.145997in}}%
\pgfpathlineto{\pgfqpoint{5.928433in}{3.157609in}}%
\pgfpathlineto{\pgfqpoint{6.007802in}{3.170484in}}%
\pgfpathlineto{\pgfqpoint{6.081335in}{3.184658in}}%
\pgfpathlineto{\pgfqpoint{6.149031in}{3.199948in}}%
\pgfpathlineto{\pgfqpoint{6.212059in}{3.216418in}}%
\pgfpathlineto{\pgfqpoint{6.271586in}{3.234235in}}%
\pgfpathlineto{\pgfqpoint{6.327611in}{3.253279in}}%
\pgfpathlineto{\pgfqpoint{6.381301in}{3.273852in}}%
\pgfpathlineto{\pgfqpoint{6.432657in}{3.295900in}}%
\pgfpathlineto{\pgfqpoint{6.481679in}{3.319337in}}%
\pgfpathlineto{\pgfqpoint{6.529534in}{3.344694in}}%
\pgfpathlineto{\pgfqpoint{6.575054in}{3.371295in}}%
\pgfpathlineto{\pgfqpoint{6.619407in}{3.399744in}}%
\pgfpathlineto{\pgfqpoint{6.662593in}{3.430034in}}%
\pgfpathlineto{\pgfqpoint{6.705779in}{3.463074in}}%
\pgfpathlineto{\pgfqpoint{6.747797in}{3.498046in}}%
\pgfpathlineto{\pgfqpoint{6.788649in}{3.534890in}}%
\pgfpathlineto{\pgfqpoint{6.829500in}{3.574704in}}%
\pgfpathlineto{\pgfqpoint{6.870352in}{3.617646in}}%
\pgfpathlineto{\pgfqpoint{6.911203in}{3.663865in}}%
\pgfpathlineto{\pgfqpoint{6.952055in}{3.713492in}}%
\pgfpathlineto{\pgfqpoint{6.992906in}{3.766639in}}%
\pgfpathlineto{\pgfqpoint{7.034925in}{3.825059in}}%
\pgfpathlineto{\pgfqpoint{7.076944in}{3.887329in}}%
\pgfpathlineto{\pgfqpoint{7.120130in}{3.955322in}}%
\pgfpathlineto{\pgfqpoint{7.165650in}{4.031262in}}%
\pgfpathlineto{\pgfqpoint{7.213505in}{4.115560in}}%
\pgfpathlineto{\pgfqpoint{7.266028in}{4.212786in}}%
\pgfpathlineto{\pgfqpoint{7.326722in}{4.330019in}}%
\pgfpathlineto{\pgfqpoint{7.518140in}{4.703788in}}%
\pgfpathlineto{\pgfqpoint{7.555490in}{4.769532in}}%
\pgfpathlineto{\pgfqpoint{7.587004in}{4.820669in}}%
\pgfpathlineto{\pgfqpoint{7.613849in}{4.860255in}}%
\pgfpathlineto{\pgfqpoint{7.638360in}{4.892565in}}%
\pgfpathlineto{\pgfqpoint{7.660537in}{4.918149in}}%
\pgfpathlineto{\pgfqpoint{7.680379in}{4.937722in}}%
\pgfpathlineto{\pgfqpoint{7.699054in}{4.952972in}}%
\pgfpathlineto{\pgfqpoint{7.715394in}{4.963566in}}%
\pgfpathlineto{\pgfqpoint{7.730568in}{4.970928in}}%
\pgfpathlineto{\pgfqpoint{7.744574in}{4.975471in}}%
\pgfpathlineto{\pgfqpoint{7.758580in}{4.977725in}}%
\pgfpathlineto{\pgfqpoint{7.771419in}{4.977677in}}%
\pgfpathlineto{\pgfqpoint{7.784258in}{4.975512in}}%
\pgfpathlineto{\pgfqpoint{7.797097in}{4.971137in}}%
\pgfpathlineto{\pgfqpoint{7.809936in}{4.964466in}}%
\pgfpathlineto{\pgfqpoint{7.822775in}{4.955412in}}%
\pgfpathlineto{\pgfqpoint{7.836782in}{4.942720in}}%
\pgfpathlineto{\pgfqpoint{7.850788in}{4.926995in}}%
\pgfpathlineto{\pgfqpoint{7.865961in}{4.906430in}}%
\pgfpathlineto{\pgfqpoint{7.881135in}{4.882090in}}%
\pgfpathlineto{\pgfqpoint{7.897475in}{4.851553in}}%
\pgfpathlineto{\pgfqpoint{7.914983in}{4.813759in}}%
\pgfpathlineto{\pgfqpoint{7.932491in}{4.770640in}}%
\pgfpathlineto{\pgfqpoint{7.951166in}{4.718737in}}%
\pgfpathlineto{\pgfqpoint{7.971008in}{4.656927in}}%
\pgfpathlineto{\pgfqpoint{7.992017in}{4.584118in}}%
\pgfpathlineto{\pgfqpoint{8.014194in}{4.499323in}}%
\pgfpathlineto{\pgfqpoint{8.038705in}{4.396673in}}%
\pgfpathlineto{\pgfqpoint{8.065550in}{4.274553in}}%
\pgfpathlineto{\pgfqpoint{8.097064in}{4.120467in}}%
\pgfpathlineto{\pgfqpoint{8.140250in}{3.896944in}}%
\pgfpathlineto{\pgfqpoint{8.206779in}{3.552492in}}%
\pgfpathlineto{\pgfqpoint{8.234792in}{3.419910in}}%
\pgfpathlineto{\pgfqpoint{8.256968in}{3.325386in}}%
\pgfpathlineto{\pgfqpoint{8.275643in}{3.255225in}}%
\pgfpathlineto{\pgfqpoint{8.291984in}{3.202364in}}%
\pgfpathlineto{\pgfqpoint{8.305990in}{3.164309in}}%
\pgfpathlineto{\pgfqpoint{8.317662in}{3.138256in}}%
\pgfpathlineto{\pgfqpoint{8.328167in}{3.119558in}}%
\pgfpathlineto{\pgfqpoint{8.337504in}{3.106948in}}%
\pgfpathlineto{\pgfqpoint{8.345675in}{3.099162in}}%
\pgfpathlineto{\pgfqpoint{8.353845in}{3.094528in}}%
\pgfpathlineto{\pgfqpoint{8.360848in}{3.093148in}}%
\pgfpathlineto{\pgfqpoint{8.367851in}{3.094227in}}%
\pgfpathlineto{\pgfqpoint{8.374854in}{3.097822in}}%
\pgfpathlineto{\pgfqpoint{8.381857in}{3.103985in}}%
\pgfpathlineto{\pgfqpoint{8.390028in}{3.114483in}}%
\pgfpathlineto{\pgfqpoint{8.398198in}{3.128599in}}%
\pgfpathlineto{\pgfqpoint{8.407535in}{3.149222in}}%
\pgfpathlineto{\pgfqpoint{8.418040in}{3.178201in}}%
\pgfpathlineto{\pgfqpoint{8.429712in}{3.217601in}}%
\pgfpathlineto{\pgfqpoint{8.442551in}{3.269645in}}%
\pgfpathlineto{\pgfqpoint{8.456557in}{3.336631in}}%
\pgfpathlineto{\pgfqpoint{8.471731in}{3.420779in}}%
\pgfpathlineto{\pgfqpoint{8.488071in}{3.524022in}}%
\pgfpathlineto{\pgfqpoint{8.506746in}{3.656390in}}%
\pgfpathlineto{\pgfqpoint{8.528923in}{3.830007in}}%
\pgfpathlineto{\pgfqpoint{8.560437in}{4.096606in}}%
\pgfpathlineto{\pgfqpoint{8.608291in}{4.501847in}}%
\pgfpathlineto{\pgfqpoint{8.629301in}{4.660115in}}%
\pgfpathlineto{\pgfqpoint{8.645641in}{4.767208in}}%
\pgfpathlineto{\pgfqpoint{8.659647in}{4.844603in}}%
\pgfpathlineto{\pgfqpoint{8.671319in}{4.897149in}}%
\pgfpathlineto{\pgfqpoint{8.680657in}{4.930449in}}%
\pgfpathlineto{\pgfqpoint{8.688827in}{4.952704in}}%
\pgfpathlineto{\pgfqpoint{8.695830in}{4.966380in}}%
\pgfpathlineto{\pgfqpoint{8.701666in}{4.973811in}}%
\pgfpathlineto{\pgfqpoint{8.706335in}{4.977084in}}%
\pgfpathlineto{\pgfqpoint{8.711004in}{4.977931in}}%
\pgfpathlineto{\pgfqpoint{8.715672in}{4.976309in}}%
\pgfpathlineto{\pgfqpoint{8.720341in}{4.972182in}}%
\pgfpathlineto{\pgfqpoint{8.726177in}{4.963456in}}%
\pgfpathlineto{\pgfqpoint{8.732013in}{4.950727in}}%
\pgfpathlineto{\pgfqpoint{8.739016in}{4.930132in}}%
\pgfpathlineto{\pgfqpoint{8.747186in}{4.898760in}}%
\pgfpathlineto{\pgfqpoint{8.756524in}{4.853292in}}%
\pgfpathlineto{\pgfqpoint{8.767029in}{4.790137in}}%
\pgfpathlineto{\pgfqpoint{8.778700in}{4.705664in}}%
\pgfpathlineto{\pgfqpoint{8.791539in}{4.596571in}}%
\pgfpathlineto{\pgfqpoint{8.806713in}{4.448365in}}%
\pgfpathlineto{\pgfqpoint{8.825388in}{4.243214in}}%
\pgfpathlineto{\pgfqpoint{8.856902in}{3.867200in}}%
\pgfpathlineto{\pgfqpoint{8.884914in}{3.543526in}}%
\pgfpathlineto{\pgfqpoint{8.901255in}{3.379112in}}%
\pgfpathlineto{\pgfqpoint{8.914094in}{3.270399in}}%
\pgfpathlineto{\pgfqpoint{8.924599in}{3.198216in}}%
\pgfpathlineto{\pgfqpoint{8.933936in}{3.148624in}}%
\pgfpathlineto{\pgfqpoint{8.940939in}{3.121285in}}%
\pgfpathlineto{\pgfqpoint{8.946775in}{3.105356in}}%
\pgfpathlineto{\pgfqpoint{8.951444in}{3.097276in}}%
\pgfpathlineto{\pgfqpoint{8.956113in}{3.093458in}}%
\pgfpathlineto{\pgfqpoint{8.959614in}{3.093444in}}%
\pgfpathlineto{\pgfqpoint{8.963116in}{3.095907in}}%
\pgfpathlineto{\pgfqpoint{8.967784in}{3.103084in}}%
\pgfpathlineto{\pgfqpoint{8.972453in}{3.114743in}}%
\pgfpathlineto{\pgfqpoint{8.978289in}{3.135645in}}%
\pgfpathlineto{\pgfqpoint{8.985292in}{3.169977in}}%
\pgfpathlineto{\pgfqpoint{8.993463in}{3.222595in}}%
\pgfpathlineto{\pgfqpoint{9.002800in}{3.298745in}}%
\pgfpathlineto{\pgfqpoint{9.013305in}{3.403570in}}%
\pgfpathlineto{\pgfqpoint{9.026144in}{3.556210in}}%
\pgfpathlineto{\pgfqpoint{9.041317in}{3.764581in}}%
\pgfpathlineto{\pgfqpoint{9.065828in}{4.137434in}}%
\pgfpathlineto{\pgfqpoint{9.091506in}{4.519861in}}%
\pgfpathlineto{\pgfqpoint{9.105512in}{4.698683in}}%
\pgfpathlineto{\pgfqpoint{9.117184in}{4.820422in}}%
\pgfpathlineto{\pgfqpoint{9.126522in}{4.895422in}}%
\pgfpathlineto{\pgfqpoint{9.133525in}{4.936768in}}%
\pgfpathlineto{\pgfqpoint{9.139361in}{4.960661in}}%
\pgfpathlineto{\pgfqpoint{9.144029in}{4.972520in}}%
\pgfpathlineto{\pgfqpoint{9.147531in}{4.977052in}}%
\pgfpathlineto{\pgfqpoint{9.151033in}{4.977773in}}%
\pgfpathlineto{\pgfqpoint{9.153367in}{4.976111in}}%
\pgfpathlineto{\pgfqpoint{9.156869in}{4.970377in}}%
\pgfpathlineto{\pgfqpoint{9.161537in}{4.956647in}}%
\pgfpathlineto{\pgfqpoint{9.167373in}{4.929703in}}%
\pgfpathlineto{\pgfqpoint{9.174376in}{4.883171in}}%
\pgfpathlineto{\pgfqpoint{9.182547in}{4.809881in}}%
\pgfpathlineto{\pgfqpoint{9.191884in}{4.702563in}}%
\pgfpathlineto{\pgfqpoint{9.202389in}{4.555019in}}%
\pgfpathlineto{\pgfqpoint{9.216395in}{4.322961in}}%
\pgfpathlineto{\pgfqpoint{9.239739in}{3.888551in}}%
\pgfpathlineto{\pgfqpoint{9.260748in}{3.513223in}}%
\pgfpathlineto{\pgfqpoint{9.272420in}{3.339231in}}%
\pgfpathlineto{\pgfqpoint{9.281757in}{3.228139in}}%
\pgfpathlineto{\pgfqpoint{9.289928in}{3.155979in}}%
\pgfpathlineto{\pgfqpoint{9.295764in}{3.120488in}}%
\pgfpathlineto{\pgfqpoint{9.300432in}{3.102400in}}%
\pgfpathlineto{\pgfqpoint{9.303934in}{3.095073in}}%
\pgfpathlineto{\pgfqpoint{9.306268in}{3.093222in}}%
\pgfpathlineto{\pgfqpoint{9.308603in}{3.093830in}}%
\pgfpathlineto{\pgfqpoint{9.310937in}{3.096913in}}%
\pgfpathlineto{\pgfqpoint{9.314439in}{3.106204in}}%
\pgfpathlineto{\pgfqpoint{9.319107in}{3.127314in}}%
\pgfpathlineto{\pgfqpoint{9.324943in}{3.167608in}}%
\pgfpathlineto{\pgfqpoint{9.331946in}{3.235825in}}%
\pgfpathlineto{\pgfqpoint{9.340117in}{3.341248in}}%
\pgfpathlineto{\pgfqpoint{9.349454in}{3.492194in}}%
\pgfpathlineto{\pgfqpoint{9.361126in}{3.717635in}}%
\pgfpathlineto{\pgfqpoint{9.380968in}{4.152223in}}%
\pgfpathlineto{\pgfqpoint{9.400810in}{4.570203in}}%
\pgfpathlineto{\pgfqpoint{9.411315in}{4.750621in}}%
\pgfpathlineto{\pgfqpoint{9.419485in}{4.859689in}}%
\pgfpathlineto{\pgfqpoint{9.426488in}{4.926847in}}%
\pgfpathlineto{\pgfqpoint{9.432324in}{4.962140in}}%
\pgfpathlineto{\pgfqpoint{9.435826in}{4.973748in}}%
\pgfpathlineto{\pgfqpoint{9.438160in}{4.977379in}}%
\pgfpathlineto{\pgfqpoint{9.440495in}{4.977678in}}%
\pgfpathlineto{\pgfqpoint{9.442829in}{4.974614in}}%
\pgfpathlineto{\pgfqpoint{9.446331in}{4.963682in}}%
\pgfpathlineto{\pgfqpoint{9.450999in}{4.937290in}}%
\pgfpathlineto{\pgfqpoint{9.456835in}{4.885578in}}%
\pgfpathlineto{\pgfqpoint{9.463838in}{4.797158in}}%
\pgfpathlineto{\pgfqpoint{9.472009in}{4.660667in}}%
\pgfpathlineto{\pgfqpoint{9.482513in}{4.441058in}}%
\pgfpathlineto{\pgfqpoint{9.497687in}{4.067261in}}%
\pgfpathlineto{\pgfqpoint{9.521030in}{3.493190in}}%
\pgfpathlineto{\pgfqpoint{9.531535in}{3.291465in}}%
\pgfpathlineto{\pgfqpoint{9.538538in}{3.191428in}}%
\pgfpathlineto{\pgfqpoint{9.544374in}{3.133283in}}%
\pgfpathlineto{\pgfqpoint{9.549043in}{3.104853in}}%
\pgfpathlineto{\pgfqpoint{9.552544in}{3.094620in}}%
\pgfpathlineto{\pgfqpoint{9.554879in}{3.093212in}}%
\pgfpathlineto{\pgfqpoint{9.557213in}{3.096188in}}%
\pgfpathlineto{\pgfqpoint{9.559548in}{3.103569in}}%
\pgfpathlineto{\pgfqpoint{9.563049in}{3.122896in}}%
\pgfpathlineto{\pgfqpoint{9.567718in}{3.163917in}}%
\pgfpathlineto{\pgfqpoint{9.573554in}{3.238932in}}%
\pgfpathlineto{\pgfqpoint{9.580557in}{3.361356in}}%
\pgfpathlineto{\pgfqpoint{9.589894in}{3.571706in}}%
\pgfpathlineto{\pgfqpoint{9.602733in}{3.921790in}}%
\pgfpathlineto{\pgfqpoint{9.627244in}{4.603559in}}%
\pgfpathlineto{\pgfqpoint{9.636582in}{4.799505in}}%
\pgfpathlineto{\pgfqpoint{9.643585in}{4.903690in}}%
\pgfpathlineto{\pgfqpoint{9.648254in}{4.949116in}}%
\pgfpathlineto{\pgfqpoint{9.651755in}{4.969538in}}%
\pgfpathlineto{\pgfqpoint{9.654090in}{4.976410in}}%
\pgfpathlineto{\pgfqpoint{9.656424in}{4.977786in}}%
\pgfpathlineto{\pgfqpoint{9.658758in}{4.973612in}}%
\pgfpathlineto{\pgfqpoint{9.661093in}{4.963867in}}%
\pgfpathlineto{\pgfqpoint{9.664594in}{4.938843in}}%
\pgfpathlineto{\pgfqpoint{9.669263in}{4.886386in}}%
\pgfpathlineto{\pgfqpoint{9.675099in}{4.791519in}}%
\pgfpathlineto{\pgfqpoint{9.682102in}{4.638829in}}%
\pgfpathlineto{\pgfqpoint{9.691439in}{4.382231in}}%
\pgfpathlineto{\pgfqpoint{9.710114in}{3.784475in}}%
\pgfpathlineto{\pgfqpoint{9.722954in}{3.412531in}}%
\pgfpathlineto{\pgfqpoint{9.731124in}{3.235577in}}%
\pgfpathlineto{\pgfqpoint{9.736960in}{3.149318in}}%
\pgfpathlineto{\pgfqpoint{9.741628in}{3.108037in}}%
\pgfpathlineto{\pgfqpoint{9.745130in}{3.094353in}}%
\pgfpathlineto{\pgfqpoint{9.746297in}{3.093174in}}%
\pgfpathlineto{\pgfqpoint{9.747464in}{3.093704in}}%
\pgfpathlineto{\pgfqpoint{9.749799in}{3.099911in}}%
\pgfpathlineto{\pgfqpoint{9.753300in}{3.122088in}}%
\pgfpathlineto{\pgfqpoint{9.757969in}{3.175323in}}%
\pgfpathlineto{\pgfqpoint{9.763805in}{3.278143in}}%
\pgfpathlineto{\pgfqpoint{9.770808in}{3.449143in}}%
\pgfpathlineto{\pgfqpoint{9.780146in}{3.739894in}}%
\pgfpathlineto{\pgfqpoint{9.810492in}{4.745590in}}%
\pgfpathlineto{\pgfqpoint{9.817496in}{4.887770in}}%
\pgfpathlineto{\pgfqpoint{9.822164in}{4.947368in}}%
\pgfpathlineto{\pgfqpoint{9.825666in}{4.971664in}}%
\pgfpathlineto{\pgfqpoint{9.828000in}{4.977730in}}%
\pgfpathlineto{\pgfqpoint{9.829167in}{4.977672in}}%
\pgfpathlineto{\pgfqpoint{9.831502in}{4.971333in}}%
\pgfpathlineto{\pgfqpoint{9.833836in}{4.956694in}}%
\pgfpathlineto{\pgfqpoint{9.837338in}{4.919332in}}%
\pgfpathlineto{\pgfqpoint{9.842006in}{4.841682in}}%
\pgfpathlineto{\pgfqpoint{9.847842in}{4.703327in}}%
\pgfpathlineto{\pgfqpoint{9.856013in}{4.445813in}}%
\pgfpathlineto{\pgfqpoint{9.870019in}{3.908765in}}%
\pgfpathlineto{\pgfqpoint{9.882858in}{3.443010in}}%
\pgfpathlineto{\pgfqpoint{9.889861in}{3.252533in}}%
\pgfpathlineto{\pgfqpoint{9.895697in}{3.146356in}}%
\pgfpathlineto{\pgfqpoint{9.900366in}{3.101646in}}%
\pgfpathlineto{\pgfqpoint{9.902700in}{3.093638in}}%
\pgfpathlineto{\pgfqpoint{9.903867in}{3.093308in}}%
\pgfpathlineto{\pgfqpoint{9.905034in}{3.095443in}}%
\pgfpathlineto{\pgfqpoint{9.907369in}{3.107124in}}%
\pgfpathlineto{\pgfqpoint{9.910870in}{3.143036in}}%
\pgfpathlineto{\pgfqpoint{9.915539in}{3.224195in}}%
\pgfpathlineto{\pgfqpoint{9.921375in}{3.374794in}}%
\pgfpathlineto{\pgfqpoint{9.929545in}{3.660033in}}%
\pgfpathlineto{\pgfqpoint{9.959892in}{4.820713in}}%
\pgfpathlineto{\pgfqpoint{9.965728in}{4.932418in}}%
\pgfpathlineto{\pgfqpoint{9.969230in}{4.967806in}}%
\pgfpathlineto{\pgfqpoint{9.971564in}{4.977336in}}%
\pgfpathlineto{\pgfqpoint{9.972731in}{4.977791in}}%
\pgfpathlineto{\pgfqpoint{9.973898in}{4.975353in}}%
\pgfpathlineto{\pgfqpoint{9.976233in}{4.961787in}}%
\pgfpathlineto{\pgfqpoint{9.979734in}{4.919923in}}%
\pgfpathlineto{\pgfqpoint{9.984403in}{4.825469in}}%
\pgfpathlineto{\pgfqpoint{9.990239in}{4.651392in}}%
\pgfpathlineto{\pgfqpoint{9.998409in}{4.327021in}}%
\pgfpathlineto{\pgfqpoint{10.021753in}{3.331935in}}%
\pgfpathlineto{\pgfqpoint{10.027589in}{3.180334in}}%
\pgfpathlineto{\pgfqpoint{10.032258in}{3.111312in}}%
\pgfpathlineto{\pgfqpoint{10.034592in}{3.095991in}}%
\pgfpathlineto{\pgfqpoint{10.035759in}{3.093299in}}%
\pgfpathlineto{\pgfqpoint{10.036926in}{3.093954in}}%
\pgfpathlineto{\pgfqpoint{10.039261in}{3.105337in}}%
\pgfpathlineto{\pgfqpoint{10.042762in}{3.147407in}}%
\pgfpathlineto{\pgfqpoint{10.047431in}{3.248341in}}%
\pgfpathlineto{\pgfqpoint{10.053267in}{3.438942in}}%
\pgfpathlineto{\pgfqpoint{10.061437in}{3.795496in}}%
\pgfpathlineto{\pgfqpoint{10.081279in}{4.710812in}}%
\pgfpathlineto{\pgfqpoint{10.087115in}{4.881406in}}%
\pgfpathlineto{\pgfqpoint{10.091784in}{4.958649in}}%
\pgfpathlineto{\pgfqpoint{10.094119in}{4.975286in}}%
\pgfpathlineto{\pgfqpoint{10.095286in}{4.977898in}}%
\pgfpathlineto{\pgfqpoint{10.096453in}{4.976666in}}%
\pgfpathlineto{\pgfqpoint{10.098787in}{4.962642in}}%
\pgfpathlineto{\pgfqpoint{10.102289in}{4.913025in}}%
\pgfpathlineto{\pgfqpoint{10.106958in}{4.796078in}}%
\pgfpathlineto{\pgfqpoint{10.112793in}{4.578626in}}%
\pgfpathlineto{\pgfqpoint{10.122131in}{4.119957in}}%
\pgfpathlineto{\pgfqpoint{10.136137in}{3.422863in}}%
\pgfpathlineto{\pgfqpoint{10.141973in}{3.220973in}}%
\pgfpathlineto{\pgfqpoint{10.146642in}{3.123948in}}%
\pgfpathlineto{\pgfqpoint{10.150143in}{3.094442in}}%
\pgfpathlineto{\pgfqpoint{10.151311in}{3.093258in}}%
\pgfpathlineto{\pgfqpoint{10.152478in}{3.096455in}}%
\pgfpathlineto{\pgfqpoint{10.154812in}{3.115981in}}%
\pgfpathlineto{\pgfqpoint{10.158314in}{3.177532in}}%
\pgfpathlineto{\pgfqpoint{10.162982in}{3.316058in}}%
\pgfpathlineto{\pgfqpoint{10.168818in}{3.565741in}}%
\pgfpathlineto{\pgfqpoint{10.179323in}{4.137359in}}%
\pgfpathlineto{\pgfqpoint{10.189828in}{4.675248in}}%
\pgfpathlineto{\pgfqpoint{10.195664in}{4.877078in}}%
\pgfpathlineto{\pgfqpoint{10.199165in}{4.948402in}}%
\pgfpathlineto{\pgfqpoint{10.201500in}{4.972596in}}%
\pgfpathlineto{\pgfqpoint{10.202667in}{4.977402in}}%
\pgfpathlineto{\pgfqpoint{10.203834in}{4.977280in}}%
\pgfpathlineto{\pgfqpoint{10.206168in}{4.962197in}}%
\pgfpathlineto{\pgfqpoint{10.208503in}{4.927504in}}%
\pgfpathlineto{\pgfqpoint{10.212004in}{4.840075in}}%
\pgfpathlineto{\pgfqpoint{10.216673in}{4.663703in}}%
\pgfpathlineto{\pgfqpoint{10.223676in}{4.301739in}}%
\pgfpathlineto{\pgfqpoint{10.241184in}{3.335329in}}%
\pgfpathlineto{\pgfqpoint{10.245853in}{3.178794in}}%
\pgfpathlineto{\pgfqpoint{10.249354in}{3.112091in}}%
\pgfpathlineto{\pgfqpoint{10.251689in}{3.094221in}}%
\pgfpathlineto{\pgfqpoint{10.252856in}{3.093535in}}%
\pgfpathlineto{\pgfqpoint{10.254023in}{3.098395in}}%
\pgfpathlineto{\pgfqpoint{10.256357in}{3.124694in}}%
\pgfpathlineto{\pgfqpoint{10.259859in}{3.204512in}}%
\pgfpathlineto{\pgfqpoint{10.264528in}{3.379868in}}%
\pgfpathlineto{\pgfqpoint{10.271531in}{3.756219in}}%
\pgfpathlineto{\pgfqpoint{10.289039in}{4.766238in}}%
\pgfpathlineto{\pgfqpoint{10.293707in}{4.916685in}}%
\pgfpathlineto{\pgfqpoint{10.297209in}{4.971100in}}%
\pgfpathlineto{\pgfqpoint{10.298376in}{4.977211in}}%
\pgfpathlineto{\pgfqpoint{10.299543in}{4.977167in}}%
\pgfpathlineto{\pgfqpoint{10.300710in}{4.970944in}}%
\pgfpathlineto{\pgfqpoint{10.303045in}{4.940064in}}%
\pgfpathlineto{\pgfqpoint{10.306546in}{4.849123in}}%
\pgfpathlineto{\pgfqpoint{10.311215in}{4.652845in}}%
\pgfpathlineto{\pgfqpoint{10.318218in}{4.240793in}}%
\pgfpathlineto{\pgfqpoint{10.332224in}{3.371479in}}%
\pgfpathlineto{\pgfqpoint{10.336893in}{3.187535in}}%
\pgfpathlineto{\pgfqpoint{10.340395in}{3.111388in}}%
\pgfpathlineto{\pgfqpoint{10.342729in}{3.093486in}}%
\pgfpathlineto{\pgfqpoint{10.343896in}{3.094741in}}%
\pgfpathlineto{\pgfqpoint{10.346231in}{3.117757in}}%
\pgfpathlineto{\pgfqpoint{10.349732in}{3.202323in}}%
\pgfpathlineto{\pgfqpoint{10.354401in}{3.399910in}}%
\pgfpathlineto{\pgfqpoint{10.361404in}{3.829749in}}%
\pgfpathlineto{\pgfqpoint{10.375410in}{4.732801in}}%
\pgfpathlineto{\pgfqpoint{10.380079in}{4.909643in}}%
\pgfpathlineto{\pgfqpoint{10.383581in}{4.971658in}}%
\pgfpathlineto{\pgfqpoint{10.384748in}{4.977634in}}%
\pgfpathlineto{\pgfqpoint{10.385915in}{4.976082in}}%
\pgfpathlineto{\pgfqpoint{10.388249in}{4.950388in}}%
\pgfpathlineto{\pgfqpoint{10.391751in}{4.856912in}}%
\pgfpathlineto{\pgfqpoint{10.396420in}{4.640310in}}%
\pgfpathlineto{\pgfqpoint{10.403423in}{4.176938in}}%
\pgfpathlineto{\pgfqpoint{10.415095in}{3.379801in}}%
\pgfpathlineto{\pgfqpoint{10.419763in}{3.178799in}}%
\pgfpathlineto{\pgfqpoint{10.423265in}{3.103676in}}%
\pgfpathlineto{\pgfqpoint{10.425599in}{3.093811in}}%
\pgfpathlineto{\pgfqpoint{10.426766in}{3.101279in}}%
\pgfpathlineto{\pgfqpoint{10.429101in}{3.140855in}}%
\pgfpathlineto{\pgfqpoint{10.432602in}{3.259147in}}%
\pgfpathlineto{\pgfqpoint{10.437271in}{3.512044in}}%
\pgfpathlineto{\pgfqpoint{10.445441in}{4.110509in}}%
\pgfpathlineto{\pgfqpoint{10.454779in}{4.744517in}}%
\pgfpathlineto{\pgfqpoint{10.459448in}{4.927062in}}%
\pgfpathlineto{\pgfqpoint{10.461782in}{4.969410in}}%
\pgfpathlineto{\pgfqpoint{10.462949in}{4.977318in}}%
\pgfpathlineto{\pgfqpoint{10.464116in}{4.976204in}}%
\pgfpathlineto{\pgfqpoint{10.466451in}{4.946895in}}%
\pgfpathlineto{\pgfqpoint{10.469952in}{4.837453in}}%
\pgfpathlineto{\pgfqpoint{10.474621in}{4.584413in}}%
\pgfpathlineto{\pgfqpoint{10.482791in}{3.963474in}}%
\pgfpathlineto{\pgfqpoint{10.490962in}{3.371801in}}%
\pgfpathlineto{\pgfqpoint{10.495630in}{3.161609in}}%
\pgfpathlineto{\pgfqpoint{10.499132in}{3.095988in}}%
\pgfpathlineto{\pgfqpoint{10.500299in}{3.093426in}}%
\pgfpathlineto{\pgfqpoint{10.501466in}{3.100714in}}%
\pgfpathlineto{\pgfqpoint{10.503801in}{3.144595in}}%
\pgfpathlineto{\pgfqpoint{10.507302in}{3.280037in}}%
\pgfpathlineto{\pgfqpoint{10.511971in}{3.570140in}}%
\pgfpathlineto{\pgfqpoint{10.530646in}{4.899731in}}%
\pgfpathlineto{\pgfqpoint{10.534147in}{4.973991in}}%
\pgfpathlineto{\pgfqpoint{10.535315in}{4.977860in}}%
\pgfpathlineto{\pgfqpoint{10.536482in}{4.971043in}}%
\pgfpathlineto{\pgfqpoint{10.538816in}{4.925612in}}%
\pgfpathlineto{\pgfqpoint{10.542318in}{4.782113in}}%
\pgfpathlineto{\pgfqpoint{10.546987in}{4.473755in}}%
\pgfpathlineto{\pgfqpoint{10.563327in}{3.221578in}}%
\pgfpathlineto{\pgfqpoint{10.566829in}{3.111439in}}%
\pgfpathlineto{\pgfqpoint{10.569163in}{3.093413in}}%
\pgfpathlineto{\pgfqpoint{10.570330in}{3.101694in}}%
\pgfpathlineto{\pgfqpoint{10.572665in}{3.152569in}}%
\pgfpathlineto{\pgfqpoint{10.576166in}{3.309600in}}%
\pgfpathlineto{\pgfqpoint{10.580835in}{3.641583in}}%
\pgfpathlineto{\pgfqpoint{10.596008in}{4.849122in}}%
\pgfpathlineto{\pgfqpoint{10.599510in}{4.961808in}}%
\pgfpathlineto{\pgfqpoint{10.601844in}{4.977030in}}%
\pgfpathlineto{\pgfqpoint{10.603011in}{4.965980in}}%
\pgfpathlineto{\pgfqpoint{10.605346in}{4.907040in}}%
\pgfpathlineto{\pgfqpoint{10.608847in}{4.732878in}}%
\pgfpathlineto{\pgfqpoint{10.614683in}{4.269490in}}%
\pgfpathlineto{\pgfqpoint{10.626355in}{3.282755in}}%
\pgfpathlineto{\pgfqpoint{10.629857in}{3.132367in}}%
\pgfpathlineto{\pgfqpoint{10.632191in}{3.094388in}}%
\pgfpathlineto{\pgfqpoint{10.633358in}{3.095314in}}%
\pgfpathlineto{\pgfqpoint{10.635693in}{3.137200in}}%
\pgfpathlineto{\pgfqpoint{10.639194in}{3.295021in}}%
\pgfpathlineto{\pgfqpoint{10.643863in}{3.649657in}}%
\pgfpathlineto{\pgfqpoint{10.657869in}{4.848368in}}%
\pgfpathlineto{\pgfqpoint{10.661371in}{4.965714in}}%
\pgfpathlineto{\pgfqpoint{10.662538in}{4.977281in}}%
\pgfpathlineto{\pgfqpoint{10.663705in}{4.974509in}}%
\pgfpathlineto{\pgfqpoint{10.666039in}{4.926104in}}%
\pgfpathlineto{\pgfqpoint{10.669541in}{4.752662in}}%
\pgfpathlineto{\pgfqpoint{10.674210in}{4.372265in}}%
\pgfpathlineto{\pgfqpoint{10.687049in}{3.231535in}}%
\pgfpathlineto{\pgfqpoint{10.690550in}{3.106226in}}%
\pgfpathlineto{\pgfqpoint{10.691718in}{3.093865in}}%
\pgfpathlineto{\pgfqpoint{10.692885in}{3.096847in}}%
\pgfpathlineto{\pgfqpoint{10.695219in}{3.148643in}}%
\pgfpathlineto{\pgfqpoint{10.698721in}{3.333622in}}%
\pgfpathlineto{\pgfqpoint{10.703389in}{3.735824in}}%
\pgfpathlineto{\pgfqpoint{10.715061in}{4.818313in}}%
\pgfpathlineto{\pgfqpoint{10.718563in}{4.959916in}}%
\pgfpathlineto{\pgfqpoint{10.719730in}{4.976025in}}%
\pgfpathlineto{\pgfqpoint{10.720897in}{4.975785in}}%
\pgfpathlineto{\pgfqpoint{10.722064in}{4.959134in}}%
\pgfpathlineto{\pgfqpoint{10.724399in}{4.877786in}}%
\pgfpathlineto{\pgfqpoint{10.727900in}{4.647424in}}%
\pgfpathlineto{\pgfqpoint{10.733736in}{4.071482in}}%
\pgfpathlineto{\pgfqpoint{10.741907in}{3.294194in}}%
\pgfpathlineto{\pgfqpoint{10.745408in}{3.124406in}}%
\pgfpathlineto{\pgfqpoint{10.747742in}{3.093236in}}%
\pgfpathlineto{\pgfqpoint{10.748910in}{3.103793in}}%
\pgfpathlineto{\pgfqpoint{10.751244in}{3.176563in}}%
\pgfpathlineto{\pgfqpoint{10.754746in}{3.403363in}}%
\pgfpathlineto{\pgfqpoint{10.760581in}{3.992539in}}%
\pgfpathlineto{\pgfqpoint{10.768752in}{4.789739in}}%
\pgfpathlineto{\pgfqpoint{10.772253in}{4.954620in}}%
\pgfpathlineto{\pgfqpoint{10.774588in}{4.976380in}}%
\pgfpathlineto{\pgfqpoint{10.775755in}{4.959418in}}%
\pgfpathlineto{\pgfqpoint{10.778089in}{4.871113in}}%
\pgfpathlineto{\pgfqpoint{10.781591in}{4.617366in}}%
\pgfpathlineto{\pgfqpoint{10.788594in}{3.858498in}}%
\pgfpathlineto{\pgfqpoint{10.794430in}{3.293651in}}%
\pgfpathlineto{\pgfqpoint{10.797931in}{3.118336in}}%
\pgfpathlineto{\pgfqpoint{10.800266in}{3.094713in}}%
\pgfpathlineto{\pgfqpoint{10.801433in}{3.112448in}}%
\pgfpathlineto{\pgfqpoint{10.803767in}{3.205548in}}%
\pgfpathlineto{\pgfqpoint{10.807269in}{3.472796in}}%
\pgfpathlineto{\pgfqpoint{10.815000in}{4.342197in}}%
\pgfpathlineto{\pgfqpoint{10.815000in}{4.342197in}}%
\pgfusepath{stroke}%
\end{pgfscope}%
\begin{pgfscope}%
\pgfsetrectcap%
\pgfsetmiterjoin%
\pgfsetlinewidth{0.803000pt}%
\definecolor{currentstroke}{rgb}{0.000000,0.000000,0.000000}%
\pgfsetstrokecolor{currentstroke}%
\pgfsetdash{}{0pt}%
\pgfpathmoveto{\pgfqpoint{0.390138in}{3.093143in}}%
\pgfpathlineto{\pgfqpoint{0.390138in}{4.977961in}}%
\pgfusepath{stroke}%
\end{pgfscope}%
\begin{pgfscope}%
\pgfsetrectcap%
\pgfsetmiterjoin%
\pgfsetlinewidth{0.803000pt}%
\definecolor{currentstroke}{rgb}{0.000000,0.000000,0.000000}%
\pgfsetstrokecolor{currentstroke}%
\pgfsetdash{}{0pt}%
\pgfpathmoveto{\pgfqpoint{10.805000in}{3.093143in}}%
\pgfpathlineto{\pgfqpoint{10.805000in}{4.977961in}}%
\pgfusepath{stroke}%
\end{pgfscope}%
\begin{pgfscope}%
\pgfsetrectcap%
\pgfsetmiterjoin%
\pgfsetlinewidth{0.803000pt}%
\definecolor{currentstroke}{rgb}{0.000000,0.000000,0.000000}%
\pgfsetstrokecolor{currentstroke}%
\pgfsetdash{}{0pt}%
\pgfpathmoveto{\pgfqpoint{0.390138in}{3.093143in}}%
\pgfpathlineto{\pgfqpoint{10.805000in}{3.093143in}}%
\pgfusepath{stroke}%
\end{pgfscope}%
\begin{pgfscope}%
\pgfsetrectcap%
\pgfsetmiterjoin%
\pgfsetlinewidth{0.803000pt}%
\definecolor{currentstroke}{rgb}{0.000000,0.000000,0.000000}%
\pgfsetstrokecolor{currentstroke}%
\pgfsetdash{}{0pt}%
\pgfpathmoveto{\pgfqpoint{0.390138in}{4.977961in}}%
\pgfpathlineto{\pgfqpoint{10.805000in}{4.977961in}}%
\pgfusepath{stroke}%
\end{pgfscope}%
\begin{pgfscope}%
\definecolor{textcolor}{rgb}{0.000000,0.000000,0.000000}%
\pgfsetstrokecolor{textcolor}%
\pgfsetfillcolor{textcolor}%
\pgftext[x=6.143461in,y=4.506756in,left,base]{\color{textcolor}{\rmfamily\fontsize{16.000000}{19.200000}\selectfont\catcode`\^=\active\def^{\ifmmode\sp\else\^{}\fi}\catcode`\%=\active\def%{\%}$P_{0v}$}}%
\end{pgfscope}%
\begin{pgfscope}%
\definecolor{textcolor}{rgb}{0.000000,0.000000,1.000000}%
\pgfsetstrokecolor{textcolor}%
\pgfsetfillcolor{textcolor}%
\pgftext[x=6.143461in,y=3.470106in,left,base]{\color{textcolor}{\rmfamily\fontsize{16.000000}{19.200000}\selectfont\catcode`\^=\active\def^{\ifmmode\sp\else\^{}\fi}\catcode`\%=\active\def%{\%}$P_{2c}$}}%
\end{pgfscope}%
\begin{pgfscope}%
\definecolor{textcolor}{rgb}{0.000000,0.000000,0.000000}%
\pgfsetstrokecolor{textcolor}%
\pgfsetfillcolor{textcolor}%
\pgftext[x=0.494286in,y=4.883720in,left,top]{\color{textcolor}{\rmfamily\fontsize{16.000000}{19.200000}\selectfont\catcode`\^=\active\def^{\ifmmode\sp\else\^{}\fi}\catcode`\%=\active\def%{\%}$(b)$}}%
\end{pgfscope}%
\begin{pgfscope}%
\pgfsetbuttcap%
\pgfsetmiterjoin%
\definecolor{currentfill}{rgb}{1.000000,1.000000,1.000000}%
\pgfsetfillcolor{currentfill}%
\pgfsetlinewidth{0.000000pt}%
\definecolor{currentstroke}{rgb}{0.000000,0.000000,0.000000}%
\pgfsetstrokecolor{currentstroke}%
\pgfsetstrokeopacity{0.000000}%
\pgfsetdash{}{0pt}%
\pgfpathmoveto{\pgfqpoint{0.390138in}{0.835548in}}%
\pgfpathlineto{\pgfqpoint{10.805000in}{0.835548in}}%
\pgfpathlineto{\pgfqpoint{10.805000in}{2.720366in}}%
\pgfpathlineto{\pgfqpoint{0.390138in}{2.720366in}}%
\pgfpathlineto{\pgfqpoint{0.390138in}{0.835548in}}%
\pgfpathclose%
\pgfusepath{fill}%
\end{pgfscope}%
\begin{pgfscope}%
\pgfsetbuttcap%
\pgfsetroundjoin%
\definecolor{currentfill}{rgb}{0.000000,0.000000,0.000000}%
\pgfsetfillcolor{currentfill}%
\pgfsetlinewidth{0.803000pt}%
\definecolor{currentstroke}{rgb}{0.000000,0.000000,0.000000}%
\pgfsetstrokecolor{currentstroke}%
\pgfsetdash{}{0pt}%
\pgfsys@defobject{currentmarker}{\pgfqpoint{0.000000in}{-0.048611in}}{\pgfqpoint{0.000000in}{0.000000in}}{%
\pgfpathmoveto{\pgfqpoint{0.000000in}{0.000000in}}%
\pgfpathlineto{\pgfqpoint{0.000000in}{-0.048611in}}%
\pgfusepath{stroke,fill}%
}%
\begin{pgfscope}%
\pgfsys@transformshift{2.371931in}{0.835548in}%
\pgfsys@useobject{currentmarker}{}%
\end{pgfscope}%
\end{pgfscope}%
\begin{pgfscope}%
\definecolor{textcolor}{rgb}{0.000000,0.000000,0.000000}%
\pgfsetstrokecolor{textcolor}%
\pgfsetfillcolor{textcolor}%
\pgftext[x=2.371931in,y=0.738326in,,top]{\color{textcolor}{\rmfamily\fontsize{15.000000}{18.000000}\selectfont\catcode`\^=\active\def^{\ifmmode\sp\else\^{}\fi}\catcode`\%=\active\def%{\%}$\mathdefault{10^{2}}$}}%
\end{pgfscope}%
\begin{pgfscope}%
\pgfsetbuttcap%
\pgfsetroundjoin%
\definecolor{currentfill}{rgb}{0.000000,0.000000,0.000000}%
\pgfsetfillcolor{currentfill}%
\pgfsetlinewidth{0.803000pt}%
\definecolor{currentstroke}{rgb}{0.000000,0.000000,0.000000}%
\pgfsetstrokecolor{currentstroke}%
\pgfsetdash{}{0pt}%
\pgfsys@defobject{currentmarker}{\pgfqpoint{0.000000in}{-0.048611in}}{\pgfqpoint{0.000000in}{0.000000in}}{%
\pgfpathmoveto{\pgfqpoint{0.000000in}{0.000000in}}%
\pgfpathlineto{\pgfqpoint{0.000000in}{-0.048611in}}%
\pgfusepath{stroke,fill}%
}%
\begin{pgfscope}%
\pgfsys@transformshift{6.335517in}{0.835548in}%
\pgfsys@useobject{currentmarker}{}%
\end{pgfscope}%
\end{pgfscope}%
\begin{pgfscope}%
\definecolor{textcolor}{rgb}{0.000000,0.000000,0.000000}%
\pgfsetstrokecolor{textcolor}%
\pgfsetfillcolor{textcolor}%
\pgftext[x=6.335517in,y=0.738326in,,top]{\color{textcolor}{\rmfamily\fontsize{15.000000}{18.000000}\selectfont\catcode`\^=\active\def^{\ifmmode\sp\else\^{}\fi}\catcode`\%=\active\def%{\%}$\mathdefault{10^{4}}$}}%
\end{pgfscope}%
\begin{pgfscope}%
\pgfsetbuttcap%
\pgfsetroundjoin%
\definecolor{currentfill}{rgb}{0.000000,0.000000,0.000000}%
\pgfsetfillcolor{currentfill}%
\pgfsetlinewidth{0.803000pt}%
\definecolor{currentstroke}{rgb}{0.000000,0.000000,0.000000}%
\pgfsetstrokecolor{currentstroke}%
\pgfsetdash{}{0pt}%
\pgfsys@defobject{currentmarker}{\pgfqpoint{0.000000in}{-0.048611in}}{\pgfqpoint{0.000000in}{0.000000in}}{%
\pgfpathmoveto{\pgfqpoint{0.000000in}{0.000000in}}%
\pgfpathlineto{\pgfqpoint{0.000000in}{-0.048611in}}%
\pgfusepath{stroke,fill}%
}%
\begin{pgfscope}%
\pgfsys@transformshift{10.299103in}{0.835548in}%
\pgfsys@useobject{currentmarker}{}%
\end{pgfscope}%
\end{pgfscope}%
\begin{pgfscope}%
\definecolor{textcolor}{rgb}{0.000000,0.000000,0.000000}%
\pgfsetstrokecolor{textcolor}%
\pgfsetfillcolor{textcolor}%
\pgftext[x=10.299103in,y=0.738326in,,top]{\color{textcolor}{\rmfamily\fontsize{15.000000}{18.000000}\selectfont\catcode`\^=\active\def^{\ifmmode\sp\else\^{}\fi}\catcode`\%=\active\def%{\%}$\mathdefault{10^{6}}$}}%
\end{pgfscope}%
\begin{pgfscope}%
\pgfsetbuttcap%
\pgfsetroundjoin%
\definecolor{currentfill}{rgb}{0.000000,0.000000,0.000000}%
\pgfsetfillcolor{currentfill}%
\pgfsetlinewidth{0.602250pt}%
\definecolor{currentstroke}{rgb}{0.000000,0.000000,0.000000}%
\pgfsetstrokecolor{currentstroke}%
\pgfsetdash{}{0pt}%
\pgfsys@defobject{currentmarker}{\pgfqpoint{0.000000in}{-0.027778in}}{\pgfqpoint{0.000000in}{0.000000in}}{%
\pgfpathmoveto{\pgfqpoint{0.000000in}{0.000000in}}%
\pgfpathlineto{\pgfqpoint{0.000000in}{-0.027778in}}%
\pgfusepath{stroke,fill}%
}%
\begin{pgfscope}%
\pgfsys@transformshift{0.986717in}{0.835548in}%
\pgfsys@useobject{currentmarker}{}%
\end{pgfscope}%
\end{pgfscope}%
\begin{pgfscope}%
\pgfsetbuttcap%
\pgfsetroundjoin%
\definecolor{currentfill}{rgb}{0.000000,0.000000,0.000000}%
\pgfsetfillcolor{currentfill}%
\pgfsetlinewidth{0.602250pt}%
\definecolor{currentstroke}{rgb}{0.000000,0.000000,0.000000}%
\pgfsetstrokecolor{currentstroke}%
\pgfsetdash{}{0pt}%
\pgfsys@defobject{currentmarker}{\pgfqpoint{0.000000in}{-0.027778in}}{\pgfqpoint{0.000000in}{0.000000in}}{%
\pgfpathmoveto{\pgfqpoint{0.000000in}{0.000000in}}%
\pgfpathlineto{\pgfqpoint{0.000000in}{-0.027778in}}%
\pgfusepath{stroke,fill}%
}%
\begin{pgfscope}%
\pgfsys@transformshift{1.335693in}{0.835548in}%
\pgfsys@useobject{currentmarker}{}%
\end{pgfscope}%
\end{pgfscope}%
\begin{pgfscope}%
\pgfsetbuttcap%
\pgfsetroundjoin%
\definecolor{currentfill}{rgb}{0.000000,0.000000,0.000000}%
\pgfsetfillcolor{currentfill}%
\pgfsetlinewidth{0.602250pt}%
\definecolor{currentstroke}{rgb}{0.000000,0.000000,0.000000}%
\pgfsetstrokecolor{currentstroke}%
\pgfsetdash{}{0pt}%
\pgfsys@defobject{currentmarker}{\pgfqpoint{0.000000in}{-0.027778in}}{\pgfqpoint{0.000000in}{0.000000in}}{%
\pgfpathmoveto{\pgfqpoint{0.000000in}{0.000000in}}%
\pgfpathlineto{\pgfqpoint{0.000000in}{-0.027778in}}%
\pgfusepath{stroke,fill}%
}%
\begin{pgfscope}%
\pgfsys@transformshift{1.583296in}{0.835548in}%
\pgfsys@useobject{currentmarker}{}%
\end{pgfscope}%
\end{pgfscope}%
\begin{pgfscope}%
\pgfsetbuttcap%
\pgfsetroundjoin%
\definecolor{currentfill}{rgb}{0.000000,0.000000,0.000000}%
\pgfsetfillcolor{currentfill}%
\pgfsetlinewidth{0.602250pt}%
\definecolor{currentstroke}{rgb}{0.000000,0.000000,0.000000}%
\pgfsetstrokecolor{currentstroke}%
\pgfsetdash{}{0pt}%
\pgfsys@defobject{currentmarker}{\pgfqpoint{0.000000in}{-0.027778in}}{\pgfqpoint{0.000000in}{0.000000in}}{%
\pgfpathmoveto{\pgfqpoint{0.000000in}{0.000000in}}%
\pgfpathlineto{\pgfqpoint{0.000000in}{-0.027778in}}%
\pgfusepath{stroke,fill}%
}%
\begin{pgfscope}%
\pgfsys@transformshift{1.775352in}{0.835548in}%
\pgfsys@useobject{currentmarker}{}%
\end{pgfscope}%
\end{pgfscope}%
\begin{pgfscope}%
\pgfsetbuttcap%
\pgfsetroundjoin%
\definecolor{currentfill}{rgb}{0.000000,0.000000,0.000000}%
\pgfsetfillcolor{currentfill}%
\pgfsetlinewidth{0.602250pt}%
\definecolor{currentstroke}{rgb}{0.000000,0.000000,0.000000}%
\pgfsetstrokecolor{currentstroke}%
\pgfsetdash{}{0pt}%
\pgfsys@defobject{currentmarker}{\pgfqpoint{0.000000in}{-0.027778in}}{\pgfqpoint{0.000000in}{0.000000in}}{%
\pgfpathmoveto{\pgfqpoint{0.000000in}{0.000000in}}%
\pgfpathlineto{\pgfqpoint{0.000000in}{-0.027778in}}%
\pgfusepath{stroke,fill}%
}%
\begin{pgfscope}%
\pgfsys@transformshift{1.932272in}{0.835548in}%
\pgfsys@useobject{currentmarker}{}%
\end{pgfscope}%
\end{pgfscope}%
\begin{pgfscope}%
\pgfsetbuttcap%
\pgfsetroundjoin%
\definecolor{currentfill}{rgb}{0.000000,0.000000,0.000000}%
\pgfsetfillcolor{currentfill}%
\pgfsetlinewidth{0.602250pt}%
\definecolor{currentstroke}{rgb}{0.000000,0.000000,0.000000}%
\pgfsetstrokecolor{currentstroke}%
\pgfsetdash{}{0pt}%
\pgfsys@defobject{currentmarker}{\pgfqpoint{0.000000in}{-0.027778in}}{\pgfqpoint{0.000000in}{0.000000in}}{%
\pgfpathmoveto{\pgfqpoint{0.000000in}{0.000000in}}%
\pgfpathlineto{\pgfqpoint{0.000000in}{-0.027778in}}%
\pgfusepath{stroke,fill}%
}%
\begin{pgfscope}%
\pgfsys@transformshift{2.064947in}{0.835548in}%
\pgfsys@useobject{currentmarker}{}%
\end{pgfscope}%
\end{pgfscope}%
\begin{pgfscope}%
\pgfsetbuttcap%
\pgfsetroundjoin%
\definecolor{currentfill}{rgb}{0.000000,0.000000,0.000000}%
\pgfsetfillcolor{currentfill}%
\pgfsetlinewidth{0.602250pt}%
\definecolor{currentstroke}{rgb}{0.000000,0.000000,0.000000}%
\pgfsetstrokecolor{currentstroke}%
\pgfsetdash{}{0pt}%
\pgfsys@defobject{currentmarker}{\pgfqpoint{0.000000in}{-0.027778in}}{\pgfqpoint{0.000000in}{0.000000in}}{%
\pgfpathmoveto{\pgfqpoint{0.000000in}{0.000000in}}%
\pgfpathlineto{\pgfqpoint{0.000000in}{-0.027778in}}%
\pgfusepath{stroke,fill}%
}%
\begin{pgfscope}%
\pgfsys@transformshift{2.179875in}{0.835548in}%
\pgfsys@useobject{currentmarker}{}%
\end{pgfscope}%
\end{pgfscope}%
\begin{pgfscope}%
\pgfsetbuttcap%
\pgfsetroundjoin%
\definecolor{currentfill}{rgb}{0.000000,0.000000,0.000000}%
\pgfsetfillcolor{currentfill}%
\pgfsetlinewidth{0.602250pt}%
\definecolor{currentstroke}{rgb}{0.000000,0.000000,0.000000}%
\pgfsetstrokecolor{currentstroke}%
\pgfsetdash{}{0pt}%
\pgfsys@defobject{currentmarker}{\pgfqpoint{0.000000in}{-0.027778in}}{\pgfqpoint{0.000000in}{0.000000in}}{%
\pgfpathmoveto{\pgfqpoint{0.000000in}{0.000000in}}%
\pgfpathlineto{\pgfqpoint{0.000000in}{-0.027778in}}%
\pgfusepath{stroke,fill}%
}%
\begin{pgfscope}%
\pgfsys@transformshift{2.281249in}{0.835548in}%
\pgfsys@useobject{currentmarker}{}%
\end{pgfscope}%
\end{pgfscope}%
\begin{pgfscope}%
\pgfsetbuttcap%
\pgfsetroundjoin%
\definecolor{currentfill}{rgb}{0.000000,0.000000,0.000000}%
\pgfsetfillcolor{currentfill}%
\pgfsetlinewidth{0.602250pt}%
\definecolor{currentstroke}{rgb}{0.000000,0.000000,0.000000}%
\pgfsetstrokecolor{currentstroke}%
\pgfsetdash{}{0pt}%
\pgfsys@defobject{currentmarker}{\pgfqpoint{0.000000in}{-0.027778in}}{\pgfqpoint{0.000000in}{0.000000in}}{%
\pgfpathmoveto{\pgfqpoint{0.000000in}{0.000000in}}%
\pgfpathlineto{\pgfqpoint{0.000000in}{-0.027778in}}%
\pgfusepath{stroke,fill}%
}%
\begin{pgfscope}%
\pgfsys@transformshift{2.968510in}{0.835548in}%
\pgfsys@useobject{currentmarker}{}%
\end{pgfscope}%
\end{pgfscope}%
\begin{pgfscope}%
\pgfsetbuttcap%
\pgfsetroundjoin%
\definecolor{currentfill}{rgb}{0.000000,0.000000,0.000000}%
\pgfsetfillcolor{currentfill}%
\pgfsetlinewidth{0.602250pt}%
\definecolor{currentstroke}{rgb}{0.000000,0.000000,0.000000}%
\pgfsetstrokecolor{currentstroke}%
\pgfsetdash{}{0pt}%
\pgfsys@defobject{currentmarker}{\pgfqpoint{0.000000in}{-0.027778in}}{\pgfqpoint{0.000000in}{0.000000in}}{%
\pgfpathmoveto{\pgfqpoint{0.000000in}{0.000000in}}%
\pgfpathlineto{\pgfqpoint{0.000000in}{-0.027778in}}%
\pgfusepath{stroke,fill}%
}%
\begin{pgfscope}%
\pgfsys@transformshift{3.317486in}{0.835548in}%
\pgfsys@useobject{currentmarker}{}%
\end{pgfscope}%
\end{pgfscope}%
\begin{pgfscope}%
\pgfsetbuttcap%
\pgfsetroundjoin%
\definecolor{currentfill}{rgb}{0.000000,0.000000,0.000000}%
\pgfsetfillcolor{currentfill}%
\pgfsetlinewidth{0.602250pt}%
\definecolor{currentstroke}{rgb}{0.000000,0.000000,0.000000}%
\pgfsetstrokecolor{currentstroke}%
\pgfsetdash{}{0pt}%
\pgfsys@defobject{currentmarker}{\pgfqpoint{0.000000in}{-0.027778in}}{\pgfqpoint{0.000000in}{0.000000in}}{%
\pgfpathmoveto{\pgfqpoint{0.000000in}{0.000000in}}%
\pgfpathlineto{\pgfqpoint{0.000000in}{-0.027778in}}%
\pgfusepath{stroke,fill}%
}%
\begin{pgfscope}%
\pgfsys@transformshift{3.565089in}{0.835548in}%
\pgfsys@useobject{currentmarker}{}%
\end{pgfscope}%
\end{pgfscope}%
\begin{pgfscope}%
\pgfsetbuttcap%
\pgfsetroundjoin%
\definecolor{currentfill}{rgb}{0.000000,0.000000,0.000000}%
\pgfsetfillcolor{currentfill}%
\pgfsetlinewidth{0.602250pt}%
\definecolor{currentstroke}{rgb}{0.000000,0.000000,0.000000}%
\pgfsetstrokecolor{currentstroke}%
\pgfsetdash{}{0pt}%
\pgfsys@defobject{currentmarker}{\pgfqpoint{0.000000in}{-0.027778in}}{\pgfqpoint{0.000000in}{0.000000in}}{%
\pgfpathmoveto{\pgfqpoint{0.000000in}{0.000000in}}%
\pgfpathlineto{\pgfqpoint{0.000000in}{-0.027778in}}%
\pgfusepath{stroke,fill}%
}%
\begin{pgfscope}%
\pgfsys@transformshift{3.757145in}{0.835548in}%
\pgfsys@useobject{currentmarker}{}%
\end{pgfscope}%
\end{pgfscope}%
\begin{pgfscope}%
\pgfsetbuttcap%
\pgfsetroundjoin%
\definecolor{currentfill}{rgb}{0.000000,0.000000,0.000000}%
\pgfsetfillcolor{currentfill}%
\pgfsetlinewidth{0.602250pt}%
\definecolor{currentstroke}{rgb}{0.000000,0.000000,0.000000}%
\pgfsetstrokecolor{currentstroke}%
\pgfsetdash{}{0pt}%
\pgfsys@defobject{currentmarker}{\pgfqpoint{0.000000in}{-0.027778in}}{\pgfqpoint{0.000000in}{0.000000in}}{%
\pgfpathmoveto{\pgfqpoint{0.000000in}{0.000000in}}%
\pgfpathlineto{\pgfqpoint{0.000000in}{-0.027778in}}%
\pgfusepath{stroke,fill}%
}%
\begin{pgfscope}%
\pgfsys@transformshift{3.914065in}{0.835548in}%
\pgfsys@useobject{currentmarker}{}%
\end{pgfscope}%
\end{pgfscope}%
\begin{pgfscope}%
\pgfsetbuttcap%
\pgfsetroundjoin%
\definecolor{currentfill}{rgb}{0.000000,0.000000,0.000000}%
\pgfsetfillcolor{currentfill}%
\pgfsetlinewidth{0.602250pt}%
\definecolor{currentstroke}{rgb}{0.000000,0.000000,0.000000}%
\pgfsetstrokecolor{currentstroke}%
\pgfsetdash{}{0pt}%
\pgfsys@defobject{currentmarker}{\pgfqpoint{0.000000in}{-0.027778in}}{\pgfqpoint{0.000000in}{0.000000in}}{%
\pgfpathmoveto{\pgfqpoint{0.000000in}{0.000000in}}%
\pgfpathlineto{\pgfqpoint{0.000000in}{-0.027778in}}%
\pgfusepath{stroke,fill}%
}%
\begin{pgfscope}%
\pgfsys@transformshift{4.046740in}{0.835548in}%
\pgfsys@useobject{currentmarker}{}%
\end{pgfscope}%
\end{pgfscope}%
\begin{pgfscope}%
\pgfsetbuttcap%
\pgfsetroundjoin%
\definecolor{currentfill}{rgb}{0.000000,0.000000,0.000000}%
\pgfsetfillcolor{currentfill}%
\pgfsetlinewidth{0.602250pt}%
\definecolor{currentstroke}{rgb}{0.000000,0.000000,0.000000}%
\pgfsetstrokecolor{currentstroke}%
\pgfsetdash{}{0pt}%
\pgfsys@defobject{currentmarker}{\pgfqpoint{0.000000in}{-0.027778in}}{\pgfqpoint{0.000000in}{0.000000in}}{%
\pgfpathmoveto{\pgfqpoint{0.000000in}{0.000000in}}%
\pgfpathlineto{\pgfqpoint{0.000000in}{-0.027778in}}%
\pgfusepath{stroke,fill}%
}%
\begin{pgfscope}%
\pgfsys@transformshift{4.161668in}{0.835548in}%
\pgfsys@useobject{currentmarker}{}%
\end{pgfscope}%
\end{pgfscope}%
\begin{pgfscope}%
\pgfsetbuttcap%
\pgfsetroundjoin%
\definecolor{currentfill}{rgb}{0.000000,0.000000,0.000000}%
\pgfsetfillcolor{currentfill}%
\pgfsetlinewidth{0.602250pt}%
\definecolor{currentstroke}{rgb}{0.000000,0.000000,0.000000}%
\pgfsetstrokecolor{currentstroke}%
\pgfsetdash{}{0pt}%
\pgfsys@defobject{currentmarker}{\pgfqpoint{0.000000in}{-0.027778in}}{\pgfqpoint{0.000000in}{0.000000in}}{%
\pgfpathmoveto{\pgfqpoint{0.000000in}{0.000000in}}%
\pgfpathlineto{\pgfqpoint{0.000000in}{-0.027778in}}%
\pgfusepath{stroke,fill}%
}%
\begin{pgfscope}%
\pgfsys@transformshift{4.263042in}{0.835548in}%
\pgfsys@useobject{currentmarker}{}%
\end{pgfscope}%
\end{pgfscope}%
\begin{pgfscope}%
\pgfsetbuttcap%
\pgfsetroundjoin%
\definecolor{currentfill}{rgb}{0.000000,0.000000,0.000000}%
\pgfsetfillcolor{currentfill}%
\pgfsetlinewidth{0.602250pt}%
\definecolor{currentstroke}{rgb}{0.000000,0.000000,0.000000}%
\pgfsetstrokecolor{currentstroke}%
\pgfsetdash{}{0pt}%
\pgfsys@defobject{currentmarker}{\pgfqpoint{0.000000in}{-0.027778in}}{\pgfqpoint{0.000000in}{0.000000in}}{%
\pgfpathmoveto{\pgfqpoint{0.000000in}{0.000000in}}%
\pgfpathlineto{\pgfqpoint{0.000000in}{-0.027778in}}%
\pgfusepath{stroke,fill}%
}%
\begin{pgfscope}%
\pgfsys@transformshift{4.950303in}{0.835548in}%
\pgfsys@useobject{currentmarker}{}%
\end{pgfscope}%
\end{pgfscope}%
\begin{pgfscope}%
\pgfsetbuttcap%
\pgfsetroundjoin%
\definecolor{currentfill}{rgb}{0.000000,0.000000,0.000000}%
\pgfsetfillcolor{currentfill}%
\pgfsetlinewidth{0.602250pt}%
\definecolor{currentstroke}{rgb}{0.000000,0.000000,0.000000}%
\pgfsetstrokecolor{currentstroke}%
\pgfsetdash{}{0pt}%
\pgfsys@defobject{currentmarker}{\pgfqpoint{0.000000in}{-0.027778in}}{\pgfqpoint{0.000000in}{0.000000in}}{%
\pgfpathmoveto{\pgfqpoint{0.000000in}{0.000000in}}%
\pgfpathlineto{\pgfqpoint{0.000000in}{-0.027778in}}%
\pgfusepath{stroke,fill}%
}%
\begin{pgfscope}%
\pgfsys@transformshift{5.299279in}{0.835548in}%
\pgfsys@useobject{currentmarker}{}%
\end{pgfscope}%
\end{pgfscope}%
\begin{pgfscope}%
\pgfsetbuttcap%
\pgfsetroundjoin%
\definecolor{currentfill}{rgb}{0.000000,0.000000,0.000000}%
\pgfsetfillcolor{currentfill}%
\pgfsetlinewidth{0.602250pt}%
\definecolor{currentstroke}{rgb}{0.000000,0.000000,0.000000}%
\pgfsetstrokecolor{currentstroke}%
\pgfsetdash{}{0pt}%
\pgfsys@defobject{currentmarker}{\pgfqpoint{0.000000in}{-0.027778in}}{\pgfqpoint{0.000000in}{0.000000in}}{%
\pgfpathmoveto{\pgfqpoint{0.000000in}{0.000000in}}%
\pgfpathlineto{\pgfqpoint{0.000000in}{-0.027778in}}%
\pgfusepath{stroke,fill}%
}%
\begin{pgfscope}%
\pgfsys@transformshift{5.546882in}{0.835548in}%
\pgfsys@useobject{currentmarker}{}%
\end{pgfscope}%
\end{pgfscope}%
\begin{pgfscope}%
\pgfsetbuttcap%
\pgfsetroundjoin%
\definecolor{currentfill}{rgb}{0.000000,0.000000,0.000000}%
\pgfsetfillcolor{currentfill}%
\pgfsetlinewidth{0.602250pt}%
\definecolor{currentstroke}{rgb}{0.000000,0.000000,0.000000}%
\pgfsetstrokecolor{currentstroke}%
\pgfsetdash{}{0pt}%
\pgfsys@defobject{currentmarker}{\pgfqpoint{0.000000in}{-0.027778in}}{\pgfqpoint{0.000000in}{0.000000in}}{%
\pgfpathmoveto{\pgfqpoint{0.000000in}{0.000000in}}%
\pgfpathlineto{\pgfqpoint{0.000000in}{-0.027778in}}%
\pgfusepath{stroke,fill}%
}%
\begin{pgfscope}%
\pgfsys@transformshift{5.738938in}{0.835548in}%
\pgfsys@useobject{currentmarker}{}%
\end{pgfscope}%
\end{pgfscope}%
\begin{pgfscope}%
\pgfsetbuttcap%
\pgfsetroundjoin%
\definecolor{currentfill}{rgb}{0.000000,0.000000,0.000000}%
\pgfsetfillcolor{currentfill}%
\pgfsetlinewidth{0.602250pt}%
\definecolor{currentstroke}{rgb}{0.000000,0.000000,0.000000}%
\pgfsetstrokecolor{currentstroke}%
\pgfsetdash{}{0pt}%
\pgfsys@defobject{currentmarker}{\pgfqpoint{0.000000in}{-0.027778in}}{\pgfqpoint{0.000000in}{0.000000in}}{%
\pgfpathmoveto{\pgfqpoint{0.000000in}{0.000000in}}%
\pgfpathlineto{\pgfqpoint{0.000000in}{-0.027778in}}%
\pgfusepath{stroke,fill}%
}%
\begin{pgfscope}%
\pgfsys@transformshift{5.895858in}{0.835548in}%
\pgfsys@useobject{currentmarker}{}%
\end{pgfscope}%
\end{pgfscope}%
\begin{pgfscope}%
\pgfsetbuttcap%
\pgfsetroundjoin%
\definecolor{currentfill}{rgb}{0.000000,0.000000,0.000000}%
\pgfsetfillcolor{currentfill}%
\pgfsetlinewidth{0.602250pt}%
\definecolor{currentstroke}{rgb}{0.000000,0.000000,0.000000}%
\pgfsetstrokecolor{currentstroke}%
\pgfsetdash{}{0pt}%
\pgfsys@defobject{currentmarker}{\pgfqpoint{0.000000in}{-0.027778in}}{\pgfqpoint{0.000000in}{0.000000in}}{%
\pgfpathmoveto{\pgfqpoint{0.000000in}{0.000000in}}%
\pgfpathlineto{\pgfqpoint{0.000000in}{-0.027778in}}%
\pgfusepath{stroke,fill}%
}%
\begin{pgfscope}%
\pgfsys@transformshift{6.028533in}{0.835548in}%
\pgfsys@useobject{currentmarker}{}%
\end{pgfscope}%
\end{pgfscope}%
\begin{pgfscope}%
\pgfsetbuttcap%
\pgfsetroundjoin%
\definecolor{currentfill}{rgb}{0.000000,0.000000,0.000000}%
\pgfsetfillcolor{currentfill}%
\pgfsetlinewidth{0.602250pt}%
\definecolor{currentstroke}{rgb}{0.000000,0.000000,0.000000}%
\pgfsetstrokecolor{currentstroke}%
\pgfsetdash{}{0pt}%
\pgfsys@defobject{currentmarker}{\pgfqpoint{0.000000in}{-0.027778in}}{\pgfqpoint{0.000000in}{0.000000in}}{%
\pgfpathmoveto{\pgfqpoint{0.000000in}{0.000000in}}%
\pgfpathlineto{\pgfqpoint{0.000000in}{-0.027778in}}%
\pgfusepath{stroke,fill}%
}%
\begin{pgfscope}%
\pgfsys@transformshift{6.143461in}{0.835548in}%
\pgfsys@useobject{currentmarker}{}%
\end{pgfscope}%
\end{pgfscope}%
\begin{pgfscope}%
\pgfsetbuttcap%
\pgfsetroundjoin%
\definecolor{currentfill}{rgb}{0.000000,0.000000,0.000000}%
\pgfsetfillcolor{currentfill}%
\pgfsetlinewidth{0.602250pt}%
\definecolor{currentstroke}{rgb}{0.000000,0.000000,0.000000}%
\pgfsetstrokecolor{currentstroke}%
\pgfsetdash{}{0pt}%
\pgfsys@defobject{currentmarker}{\pgfqpoint{0.000000in}{-0.027778in}}{\pgfqpoint{0.000000in}{0.000000in}}{%
\pgfpathmoveto{\pgfqpoint{0.000000in}{0.000000in}}%
\pgfpathlineto{\pgfqpoint{0.000000in}{-0.027778in}}%
\pgfusepath{stroke,fill}%
}%
\begin{pgfscope}%
\pgfsys@transformshift{6.244835in}{0.835548in}%
\pgfsys@useobject{currentmarker}{}%
\end{pgfscope}%
\end{pgfscope}%
\begin{pgfscope}%
\pgfsetbuttcap%
\pgfsetroundjoin%
\definecolor{currentfill}{rgb}{0.000000,0.000000,0.000000}%
\pgfsetfillcolor{currentfill}%
\pgfsetlinewidth{0.602250pt}%
\definecolor{currentstroke}{rgb}{0.000000,0.000000,0.000000}%
\pgfsetstrokecolor{currentstroke}%
\pgfsetdash{}{0pt}%
\pgfsys@defobject{currentmarker}{\pgfqpoint{0.000000in}{-0.027778in}}{\pgfqpoint{0.000000in}{0.000000in}}{%
\pgfpathmoveto{\pgfqpoint{0.000000in}{0.000000in}}%
\pgfpathlineto{\pgfqpoint{0.000000in}{-0.027778in}}%
\pgfusepath{stroke,fill}%
}%
\begin{pgfscope}%
\pgfsys@transformshift{6.932096in}{0.835548in}%
\pgfsys@useobject{currentmarker}{}%
\end{pgfscope}%
\end{pgfscope}%
\begin{pgfscope}%
\pgfsetbuttcap%
\pgfsetroundjoin%
\definecolor{currentfill}{rgb}{0.000000,0.000000,0.000000}%
\pgfsetfillcolor{currentfill}%
\pgfsetlinewidth{0.602250pt}%
\definecolor{currentstroke}{rgb}{0.000000,0.000000,0.000000}%
\pgfsetstrokecolor{currentstroke}%
\pgfsetdash{}{0pt}%
\pgfsys@defobject{currentmarker}{\pgfqpoint{0.000000in}{-0.027778in}}{\pgfqpoint{0.000000in}{0.000000in}}{%
\pgfpathmoveto{\pgfqpoint{0.000000in}{0.000000in}}%
\pgfpathlineto{\pgfqpoint{0.000000in}{-0.027778in}}%
\pgfusepath{stroke,fill}%
}%
\begin{pgfscope}%
\pgfsys@transformshift{7.281072in}{0.835548in}%
\pgfsys@useobject{currentmarker}{}%
\end{pgfscope}%
\end{pgfscope}%
\begin{pgfscope}%
\pgfsetbuttcap%
\pgfsetroundjoin%
\definecolor{currentfill}{rgb}{0.000000,0.000000,0.000000}%
\pgfsetfillcolor{currentfill}%
\pgfsetlinewidth{0.602250pt}%
\definecolor{currentstroke}{rgb}{0.000000,0.000000,0.000000}%
\pgfsetstrokecolor{currentstroke}%
\pgfsetdash{}{0pt}%
\pgfsys@defobject{currentmarker}{\pgfqpoint{0.000000in}{-0.027778in}}{\pgfqpoint{0.000000in}{0.000000in}}{%
\pgfpathmoveto{\pgfqpoint{0.000000in}{0.000000in}}%
\pgfpathlineto{\pgfqpoint{0.000000in}{-0.027778in}}%
\pgfusepath{stroke,fill}%
}%
\begin{pgfscope}%
\pgfsys@transformshift{7.528675in}{0.835548in}%
\pgfsys@useobject{currentmarker}{}%
\end{pgfscope}%
\end{pgfscope}%
\begin{pgfscope}%
\pgfsetbuttcap%
\pgfsetroundjoin%
\definecolor{currentfill}{rgb}{0.000000,0.000000,0.000000}%
\pgfsetfillcolor{currentfill}%
\pgfsetlinewidth{0.602250pt}%
\definecolor{currentstroke}{rgb}{0.000000,0.000000,0.000000}%
\pgfsetstrokecolor{currentstroke}%
\pgfsetdash{}{0pt}%
\pgfsys@defobject{currentmarker}{\pgfqpoint{0.000000in}{-0.027778in}}{\pgfqpoint{0.000000in}{0.000000in}}{%
\pgfpathmoveto{\pgfqpoint{0.000000in}{0.000000in}}%
\pgfpathlineto{\pgfqpoint{0.000000in}{-0.027778in}}%
\pgfusepath{stroke,fill}%
}%
\begin{pgfscope}%
\pgfsys@transformshift{7.720731in}{0.835548in}%
\pgfsys@useobject{currentmarker}{}%
\end{pgfscope}%
\end{pgfscope}%
\begin{pgfscope}%
\pgfsetbuttcap%
\pgfsetroundjoin%
\definecolor{currentfill}{rgb}{0.000000,0.000000,0.000000}%
\pgfsetfillcolor{currentfill}%
\pgfsetlinewidth{0.602250pt}%
\definecolor{currentstroke}{rgb}{0.000000,0.000000,0.000000}%
\pgfsetstrokecolor{currentstroke}%
\pgfsetdash{}{0pt}%
\pgfsys@defobject{currentmarker}{\pgfqpoint{0.000000in}{-0.027778in}}{\pgfqpoint{0.000000in}{0.000000in}}{%
\pgfpathmoveto{\pgfqpoint{0.000000in}{0.000000in}}%
\pgfpathlineto{\pgfqpoint{0.000000in}{-0.027778in}}%
\pgfusepath{stroke,fill}%
}%
\begin{pgfscope}%
\pgfsys@transformshift{7.877651in}{0.835548in}%
\pgfsys@useobject{currentmarker}{}%
\end{pgfscope}%
\end{pgfscope}%
\begin{pgfscope}%
\pgfsetbuttcap%
\pgfsetroundjoin%
\definecolor{currentfill}{rgb}{0.000000,0.000000,0.000000}%
\pgfsetfillcolor{currentfill}%
\pgfsetlinewidth{0.602250pt}%
\definecolor{currentstroke}{rgb}{0.000000,0.000000,0.000000}%
\pgfsetstrokecolor{currentstroke}%
\pgfsetdash{}{0pt}%
\pgfsys@defobject{currentmarker}{\pgfqpoint{0.000000in}{-0.027778in}}{\pgfqpoint{0.000000in}{0.000000in}}{%
\pgfpathmoveto{\pgfqpoint{0.000000in}{0.000000in}}%
\pgfpathlineto{\pgfqpoint{0.000000in}{-0.027778in}}%
\pgfusepath{stroke,fill}%
}%
\begin{pgfscope}%
\pgfsys@transformshift{8.010326in}{0.835548in}%
\pgfsys@useobject{currentmarker}{}%
\end{pgfscope}%
\end{pgfscope}%
\begin{pgfscope}%
\pgfsetbuttcap%
\pgfsetroundjoin%
\definecolor{currentfill}{rgb}{0.000000,0.000000,0.000000}%
\pgfsetfillcolor{currentfill}%
\pgfsetlinewidth{0.602250pt}%
\definecolor{currentstroke}{rgb}{0.000000,0.000000,0.000000}%
\pgfsetstrokecolor{currentstroke}%
\pgfsetdash{}{0pt}%
\pgfsys@defobject{currentmarker}{\pgfqpoint{0.000000in}{-0.027778in}}{\pgfqpoint{0.000000in}{0.000000in}}{%
\pgfpathmoveto{\pgfqpoint{0.000000in}{0.000000in}}%
\pgfpathlineto{\pgfqpoint{0.000000in}{-0.027778in}}%
\pgfusepath{stroke,fill}%
}%
\begin{pgfscope}%
\pgfsys@transformshift{8.125254in}{0.835548in}%
\pgfsys@useobject{currentmarker}{}%
\end{pgfscope}%
\end{pgfscope}%
\begin{pgfscope}%
\pgfsetbuttcap%
\pgfsetroundjoin%
\definecolor{currentfill}{rgb}{0.000000,0.000000,0.000000}%
\pgfsetfillcolor{currentfill}%
\pgfsetlinewidth{0.602250pt}%
\definecolor{currentstroke}{rgb}{0.000000,0.000000,0.000000}%
\pgfsetstrokecolor{currentstroke}%
\pgfsetdash{}{0pt}%
\pgfsys@defobject{currentmarker}{\pgfqpoint{0.000000in}{-0.027778in}}{\pgfqpoint{0.000000in}{0.000000in}}{%
\pgfpathmoveto{\pgfqpoint{0.000000in}{0.000000in}}%
\pgfpathlineto{\pgfqpoint{0.000000in}{-0.027778in}}%
\pgfusepath{stroke,fill}%
}%
\begin{pgfscope}%
\pgfsys@transformshift{8.226628in}{0.835548in}%
\pgfsys@useobject{currentmarker}{}%
\end{pgfscope}%
\end{pgfscope}%
\begin{pgfscope}%
\pgfsetbuttcap%
\pgfsetroundjoin%
\definecolor{currentfill}{rgb}{0.000000,0.000000,0.000000}%
\pgfsetfillcolor{currentfill}%
\pgfsetlinewidth{0.602250pt}%
\definecolor{currentstroke}{rgb}{0.000000,0.000000,0.000000}%
\pgfsetstrokecolor{currentstroke}%
\pgfsetdash{}{0pt}%
\pgfsys@defobject{currentmarker}{\pgfqpoint{0.000000in}{-0.027778in}}{\pgfqpoint{0.000000in}{0.000000in}}{%
\pgfpathmoveto{\pgfqpoint{0.000000in}{0.000000in}}%
\pgfpathlineto{\pgfqpoint{0.000000in}{-0.027778in}}%
\pgfusepath{stroke,fill}%
}%
\begin{pgfscope}%
\pgfsys@transformshift{8.913889in}{0.835548in}%
\pgfsys@useobject{currentmarker}{}%
\end{pgfscope}%
\end{pgfscope}%
\begin{pgfscope}%
\pgfsetbuttcap%
\pgfsetroundjoin%
\definecolor{currentfill}{rgb}{0.000000,0.000000,0.000000}%
\pgfsetfillcolor{currentfill}%
\pgfsetlinewidth{0.602250pt}%
\definecolor{currentstroke}{rgb}{0.000000,0.000000,0.000000}%
\pgfsetstrokecolor{currentstroke}%
\pgfsetdash{}{0pt}%
\pgfsys@defobject{currentmarker}{\pgfqpoint{0.000000in}{-0.027778in}}{\pgfqpoint{0.000000in}{0.000000in}}{%
\pgfpathmoveto{\pgfqpoint{0.000000in}{0.000000in}}%
\pgfpathlineto{\pgfqpoint{0.000000in}{-0.027778in}}%
\pgfusepath{stroke,fill}%
}%
\begin{pgfscope}%
\pgfsys@transformshift{9.262865in}{0.835548in}%
\pgfsys@useobject{currentmarker}{}%
\end{pgfscope}%
\end{pgfscope}%
\begin{pgfscope}%
\pgfsetbuttcap%
\pgfsetroundjoin%
\definecolor{currentfill}{rgb}{0.000000,0.000000,0.000000}%
\pgfsetfillcolor{currentfill}%
\pgfsetlinewidth{0.602250pt}%
\definecolor{currentstroke}{rgb}{0.000000,0.000000,0.000000}%
\pgfsetstrokecolor{currentstroke}%
\pgfsetdash{}{0pt}%
\pgfsys@defobject{currentmarker}{\pgfqpoint{0.000000in}{-0.027778in}}{\pgfqpoint{0.000000in}{0.000000in}}{%
\pgfpathmoveto{\pgfqpoint{0.000000in}{0.000000in}}%
\pgfpathlineto{\pgfqpoint{0.000000in}{-0.027778in}}%
\pgfusepath{stroke,fill}%
}%
\begin{pgfscope}%
\pgfsys@transformshift{9.510468in}{0.835548in}%
\pgfsys@useobject{currentmarker}{}%
\end{pgfscope}%
\end{pgfscope}%
\begin{pgfscope}%
\pgfsetbuttcap%
\pgfsetroundjoin%
\definecolor{currentfill}{rgb}{0.000000,0.000000,0.000000}%
\pgfsetfillcolor{currentfill}%
\pgfsetlinewidth{0.602250pt}%
\definecolor{currentstroke}{rgb}{0.000000,0.000000,0.000000}%
\pgfsetstrokecolor{currentstroke}%
\pgfsetdash{}{0pt}%
\pgfsys@defobject{currentmarker}{\pgfqpoint{0.000000in}{-0.027778in}}{\pgfqpoint{0.000000in}{0.000000in}}{%
\pgfpathmoveto{\pgfqpoint{0.000000in}{0.000000in}}%
\pgfpathlineto{\pgfqpoint{0.000000in}{-0.027778in}}%
\pgfusepath{stroke,fill}%
}%
\begin{pgfscope}%
\pgfsys@transformshift{9.702524in}{0.835548in}%
\pgfsys@useobject{currentmarker}{}%
\end{pgfscope}%
\end{pgfscope}%
\begin{pgfscope}%
\pgfsetbuttcap%
\pgfsetroundjoin%
\definecolor{currentfill}{rgb}{0.000000,0.000000,0.000000}%
\pgfsetfillcolor{currentfill}%
\pgfsetlinewidth{0.602250pt}%
\definecolor{currentstroke}{rgb}{0.000000,0.000000,0.000000}%
\pgfsetstrokecolor{currentstroke}%
\pgfsetdash{}{0pt}%
\pgfsys@defobject{currentmarker}{\pgfqpoint{0.000000in}{-0.027778in}}{\pgfqpoint{0.000000in}{0.000000in}}{%
\pgfpathmoveto{\pgfqpoint{0.000000in}{0.000000in}}%
\pgfpathlineto{\pgfqpoint{0.000000in}{-0.027778in}}%
\pgfusepath{stroke,fill}%
}%
\begin{pgfscope}%
\pgfsys@transformshift{9.859444in}{0.835548in}%
\pgfsys@useobject{currentmarker}{}%
\end{pgfscope}%
\end{pgfscope}%
\begin{pgfscope}%
\pgfsetbuttcap%
\pgfsetroundjoin%
\definecolor{currentfill}{rgb}{0.000000,0.000000,0.000000}%
\pgfsetfillcolor{currentfill}%
\pgfsetlinewidth{0.602250pt}%
\definecolor{currentstroke}{rgb}{0.000000,0.000000,0.000000}%
\pgfsetstrokecolor{currentstroke}%
\pgfsetdash{}{0pt}%
\pgfsys@defobject{currentmarker}{\pgfqpoint{0.000000in}{-0.027778in}}{\pgfqpoint{0.000000in}{0.000000in}}{%
\pgfpathmoveto{\pgfqpoint{0.000000in}{0.000000in}}%
\pgfpathlineto{\pgfqpoint{0.000000in}{-0.027778in}}%
\pgfusepath{stroke,fill}%
}%
\begin{pgfscope}%
\pgfsys@transformshift{9.992119in}{0.835548in}%
\pgfsys@useobject{currentmarker}{}%
\end{pgfscope}%
\end{pgfscope}%
\begin{pgfscope}%
\pgfsetbuttcap%
\pgfsetroundjoin%
\definecolor{currentfill}{rgb}{0.000000,0.000000,0.000000}%
\pgfsetfillcolor{currentfill}%
\pgfsetlinewidth{0.602250pt}%
\definecolor{currentstroke}{rgb}{0.000000,0.000000,0.000000}%
\pgfsetstrokecolor{currentstroke}%
\pgfsetdash{}{0pt}%
\pgfsys@defobject{currentmarker}{\pgfqpoint{0.000000in}{-0.027778in}}{\pgfqpoint{0.000000in}{0.000000in}}{%
\pgfpathmoveto{\pgfqpoint{0.000000in}{0.000000in}}%
\pgfpathlineto{\pgfqpoint{0.000000in}{-0.027778in}}%
\pgfusepath{stroke,fill}%
}%
\begin{pgfscope}%
\pgfsys@transformshift{10.107047in}{0.835548in}%
\pgfsys@useobject{currentmarker}{}%
\end{pgfscope}%
\end{pgfscope}%
\begin{pgfscope}%
\pgfsetbuttcap%
\pgfsetroundjoin%
\definecolor{currentfill}{rgb}{0.000000,0.000000,0.000000}%
\pgfsetfillcolor{currentfill}%
\pgfsetlinewidth{0.602250pt}%
\definecolor{currentstroke}{rgb}{0.000000,0.000000,0.000000}%
\pgfsetstrokecolor{currentstroke}%
\pgfsetdash{}{0pt}%
\pgfsys@defobject{currentmarker}{\pgfqpoint{0.000000in}{-0.027778in}}{\pgfqpoint{0.000000in}{0.000000in}}{%
\pgfpathmoveto{\pgfqpoint{0.000000in}{0.000000in}}%
\pgfpathlineto{\pgfqpoint{0.000000in}{-0.027778in}}%
\pgfusepath{stroke,fill}%
}%
\begin{pgfscope}%
\pgfsys@transformshift{10.208421in}{0.835548in}%
\pgfsys@useobject{currentmarker}{}%
\end{pgfscope}%
\end{pgfscope}%
\begin{pgfscope}%
\definecolor{textcolor}{rgb}{0.000000,0.000000,0.000000}%
\pgfsetstrokecolor{textcolor}%
\pgfsetfillcolor{textcolor}%
\pgftext[x=5.597569in,y=0.502183in,,top]{\color{textcolor}{\rmfamily\fontsize{24.000000}{28.800000}\selectfont\catcode`\^=\active\def^{\ifmmode\sp\else\^{}\fi}\catcode`\%=\active\def%{\%}$\omega_b\,t$}}%
\end{pgfscope}%
\begin{pgfscope}%
\pgfsetbuttcap%
\pgfsetroundjoin%
\definecolor{currentfill}{rgb}{0.000000,0.000000,0.000000}%
\pgfsetfillcolor{currentfill}%
\pgfsetlinewidth{0.803000pt}%
\definecolor{currentstroke}{rgb}{0.000000,0.000000,0.000000}%
\pgfsetstrokecolor{currentstroke}%
\pgfsetdash{}{0pt}%
\pgfsys@defobject{currentmarker}{\pgfqpoint{-0.048611in}{0.000000in}}{\pgfqpoint{-0.000000in}{0.000000in}}{%
\pgfpathmoveto{\pgfqpoint{-0.000000in}{0.000000in}}%
\pgfpathlineto{\pgfqpoint{-0.048611in}{0.000000in}}%
\pgfusepath{stroke,fill}%
}%
\begin{pgfscope}%
\pgfsys@transformshift{0.390138in}{0.835548in}%
\pgfsys@useobject{currentmarker}{}%
\end{pgfscope}%
\end{pgfscope}%
\begin{pgfscope}%
\definecolor{textcolor}{rgb}{0.000000,0.000000,0.000000}%
\pgfsetstrokecolor{textcolor}%
\pgfsetfillcolor{textcolor}%
\pgftext[x=0.195000in, y=0.766104in, left, base]{\color{textcolor}{\rmfamily\fontsize{15.000000}{18.000000}\selectfont\catcode`\^=\active\def^{\ifmmode\sp\else\^{}\fi}\catcode`\%=\active\def%{\%}0}}%
\end{pgfscope}%
\begin{pgfscope}%
\pgfsetbuttcap%
\pgfsetroundjoin%
\definecolor{currentfill}{rgb}{0.000000,0.000000,0.000000}%
\pgfsetfillcolor{currentfill}%
\pgfsetlinewidth{0.803000pt}%
\definecolor{currentstroke}{rgb}{0.000000,0.000000,0.000000}%
\pgfsetstrokecolor{currentstroke}%
\pgfsetdash{}{0pt}%
\pgfsys@defobject{currentmarker}{\pgfqpoint{-0.048611in}{0.000000in}}{\pgfqpoint{-0.000000in}{0.000000in}}{%
\pgfpathmoveto{\pgfqpoint{-0.000000in}{0.000000in}}%
\pgfpathlineto{\pgfqpoint{-0.048611in}{0.000000in}}%
\pgfusepath{stroke,fill}%
}%
\begin{pgfscope}%
\pgfsys@transformshift{0.390138in}{2.720366in}%
\pgfsys@useobject{currentmarker}{}%
\end{pgfscope}%
\end{pgfscope}%
\begin{pgfscope}%
\definecolor{textcolor}{rgb}{0.000000,0.000000,0.000000}%
\pgfsetstrokecolor{textcolor}%
\pgfsetfillcolor{textcolor}%
\pgftext[x=0.195000in, y=2.650921in, left, base]{\color{textcolor}{\rmfamily\fontsize{15.000000}{18.000000}\selectfont\catcode`\^=\active\def^{\ifmmode\sp\else\^{}\fi}\catcode`\%=\active\def%{\%}1}}%
\end{pgfscope}%
\begin{pgfscope}%
\pgfpathrectangle{\pgfqpoint{0.390138in}{0.835548in}}{\pgfqpoint{10.414862in}{1.884818in}}%
\pgfusepath{clip}%
\pgfsetrectcap%
\pgfsetroundjoin%
\pgfsetlinewidth{1.606000pt}%
\definecolor{currentstroke}{rgb}{0.000000,0.000000,0.000000}%
\pgfsetstrokecolor{currentstroke}%
\pgfsetdash{}{0pt}%
\pgfpathmoveto{\pgfqpoint{0.390138in}{2.720341in}}%
\pgfpathlineto{\pgfqpoint{2.072052in}{2.719136in}}%
\pgfpathlineto{\pgfqpoint{2.544762in}{2.716679in}}%
\pgfpathlineto{\pgfqpoint{2.841227in}{2.713028in}}%
\pgfpathlineto{\pgfqpoint{3.058324in}{2.708223in}}%
\pgfpathlineto{\pgfqpoint{3.231067in}{2.702245in}}%
\pgfpathlineto{\pgfqpoint{3.374631in}{2.695102in}}%
\pgfpathlineto{\pgfqpoint{3.497186in}{2.686828in}}%
\pgfpathlineto{\pgfqpoint{3.604567in}{2.677395in}}%
\pgfpathlineto{\pgfqpoint{3.700276in}{2.666795in}}%
\pgfpathlineto{\pgfqpoint{3.786648in}{2.655028in}}%
\pgfpathlineto{\pgfqpoint{3.866016in}{2.641981in}}%
\pgfpathlineto{\pgfqpoint{3.938382in}{2.627866in}}%
\pgfpathlineto{\pgfqpoint{4.006079in}{2.612415in}}%
\pgfpathlineto{\pgfqpoint{4.069107in}{2.595773in}}%
\pgfpathlineto{\pgfqpoint{4.128633in}{2.577770in}}%
\pgfpathlineto{\pgfqpoint{4.184658in}{2.558529in}}%
\pgfpathlineto{\pgfqpoint{4.238348in}{2.537746in}}%
\pgfpathlineto{\pgfqpoint{4.289705in}{2.515474in}}%
\pgfpathlineto{\pgfqpoint{4.338726in}{2.491801in}}%
\pgfpathlineto{\pgfqpoint{4.386581in}{2.466193in}}%
\pgfpathlineto{\pgfqpoint{4.432101in}{2.439331in}}%
\pgfpathlineto{\pgfqpoint{4.476454in}{2.410607in}}%
\pgfpathlineto{\pgfqpoint{4.519640in}{2.380028in}}%
\pgfpathlineto{\pgfqpoint{4.562826in}{2.346678in}}%
\pgfpathlineto{\pgfqpoint{4.604845in}{2.311385in}}%
\pgfpathlineto{\pgfqpoint{4.645696in}{2.274208in}}%
\pgfpathlineto{\pgfqpoint{4.686548in}{2.234043in}}%
\pgfpathlineto{\pgfqpoint{4.727399in}{2.190733in}}%
\pgfpathlineto{\pgfqpoint{4.768251in}{2.144130in}}%
\pgfpathlineto{\pgfqpoint{4.809102in}{2.094104in}}%
\pgfpathlineto{\pgfqpoint{4.849954in}{2.040548in}}%
\pgfpathlineto{\pgfqpoint{4.891972in}{1.981700in}}%
\pgfpathlineto{\pgfqpoint{4.933991in}{1.919001in}}%
\pgfpathlineto{\pgfqpoint{4.977177in}{1.850573in}}%
\pgfpathlineto{\pgfqpoint{5.022697in}{1.774193in}}%
\pgfpathlineto{\pgfqpoint{5.070552in}{1.689469in}}%
\pgfpathlineto{\pgfqpoint{5.123075in}{1.591842in}}%
\pgfpathlineto{\pgfqpoint{5.184936in}{1.471983in}}%
\pgfpathlineto{\pgfqpoint{5.364683in}{1.120372in}}%
\pgfpathlineto{\pgfqpoint{5.403200in}{1.051895in}}%
\pgfpathlineto{\pgfqpoint{5.435881in}{0.998267in}}%
\pgfpathlineto{\pgfqpoint{5.463893in}{0.956511in}}%
\pgfpathlineto{\pgfqpoint{5.488404in}{0.923828in}}%
\pgfpathlineto{\pgfqpoint{5.510581in}{0.897859in}}%
\pgfpathlineto{\pgfqpoint{5.530423in}{0.877902in}}%
\pgfpathlineto{\pgfqpoint{5.549098in}{0.862254in}}%
\pgfpathlineto{\pgfqpoint{5.565438in}{0.851282in}}%
\pgfpathlineto{\pgfqpoint{5.580612in}{0.843545in}}%
\pgfpathlineto{\pgfqpoint{5.595785in}{0.838327in}}%
\pgfpathlineto{\pgfqpoint{5.609792in}{0.835880in}}%
\pgfpathlineto{\pgfqpoint{5.622631in}{0.835742in}}%
\pgfpathlineto{\pgfqpoint{5.635470in}{0.837714in}}%
\pgfpathlineto{\pgfqpoint{5.648309in}{0.841886in}}%
\pgfpathlineto{\pgfqpoint{5.661148in}{0.848348in}}%
\pgfpathlineto{\pgfqpoint{5.673987in}{0.857185in}}%
\pgfpathlineto{\pgfqpoint{5.687993in}{0.869631in}}%
\pgfpathlineto{\pgfqpoint{5.701999in}{0.885102in}}%
\pgfpathlineto{\pgfqpoint{5.717173in}{0.905385in}}%
\pgfpathlineto{\pgfqpoint{5.732346in}{0.929435in}}%
\pgfpathlineto{\pgfqpoint{5.748687in}{0.959652in}}%
\pgfpathlineto{\pgfqpoint{5.766194in}{0.997098in}}%
\pgfpathlineto{\pgfqpoint{5.783702in}{1.039865in}}%
\pgfpathlineto{\pgfqpoint{5.802377in}{1.091391in}}%
\pgfpathlineto{\pgfqpoint{5.822219in}{1.152804in}}%
\pgfpathlineto{\pgfqpoint{5.843229in}{1.225203in}}%
\pgfpathlineto{\pgfqpoint{5.865405in}{1.309584in}}%
\pgfpathlineto{\pgfqpoint{5.889916in}{1.411810in}}%
\pgfpathlineto{\pgfqpoint{5.916761in}{1.533524in}}%
\pgfpathlineto{\pgfqpoint{5.948275in}{1.687238in}}%
\pgfpathlineto{\pgfqpoint{5.990294in}{1.904359in}}%
\pgfpathlineto{\pgfqpoint{6.060325in}{2.266841in}}%
\pgfpathlineto{\pgfqpoint{6.087170in}{2.393664in}}%
\pgfpathlineto{\pgfqpoint{6.109347in}{2.488180in}}%
\pgfpathlineto{\pgfqpoint{6.128022in}{2.558334in}}%
\pgfpathlineto{\pgfqpoint{6.144363in}{2.611188in}}%
\pgfpathlineto{\pgfqpoint{6.158369in}{2.649236in}}%
\pgfpathlineto{\pgfqpoint{6.170041in}{2.675282in}}%
\pgfpathlineto{\pgfqpoint{6.180545in}{2.693973in}}%
\pgfpathlineto{\pgfqpoint{6.189883in}{2.706577in}}%
\pgfpathlineto{\pgfqpoint{6.198053in}{2.714358in}}%
\pgfpathlineto{\pgfqpoint{6.206223in}{2.718986in}}%
\pgfpathlineto{\pgfqpoint{6.213226in}{2.720361in}}%
\pgfpathlineto{\pgfqpoint{6.220230in}{2.719277in}}%
\pgfpathlineto{\pgfqpoint{6.227233in}{2.715677in}}%
\pgfpathlineto{\pgfqpoint{6.234236in}{2.709508in}}%
\pgfpathlineto{\pgfqpoint{6.242406in}{2.699004in}}%
\pgfpathlineto{\pgfqpoint{6.250576in}{2.684881in}}%
\pgfpathlineto{\pgfqpoint{6.259914in}{2.664250in}}%
\pgfpathlineto{\pgfqpoint{6.270419in}{2.635263in}}%
\pgfpathlineto{\pgfqpoint{6.282090in}{2.595854in}}%
\pgfpathlineto{\pgfqpoint{6.294929in}{2.543799in}}%
\pgfpathlineto{\pgfqpoint{6.308936in}{2.476803in}}%
\pgfpathlineto{\pgfqpoint{6.324109in}{2.392644in}}%
\pgfpathlineto{\pgfqpoint{6.340450in}{2.289389in}}%
\pgfpathlineto{\pgfqpoint{6.359125in}{2.157010in}}%
\pgfpathlineto{\pgfqpoint{6.381301in}{1.983383in}}%
\pgfpathlineto{\pgfqpoint{6.412815in}{1.716776in}}%
\pgfpathlineto{\pgfqpoint{6.460670in}{1.311546in}}%
\pgfpathlineto{\pgfqpoint{6.481679in}{1.153291in}}%
\pgfpathlineto{\pgfqpoint{6.498020in}{1.046213in}}%
\pgfpathlineto{\pgfqpoint{6.512026in}{0.968833in}}%
\pgfpathlineto{\pgfqpoint{6.523698in}{0.916301in}}%
\pgfpathlineto{\pgfqpoint{6.533035in}{0.883014in}}%
\pgfpathlineto{\pgfqpoint{6.541206in}{0.860771in}}%
\pgfpathlineto{\pgfqpoint{6.548209in}{0.847106in}}%
\pgfpathlineto{\pgfqpoint{6.554045in}{0.839684in}}%
\pgfpathlineto{\pgfqpoint{6.558713in}{0.836418in}}%
\pgfpathlineto{\pgfqpoint{6.563382in}{0.835579in}}%
\pgfpathlineto{\pgfqpoint{6.568051in}{0.837208in}}%
\pgfpathlineto{\pgfqpoint{6.572720in}{0.841343in}}%
\pgfpathlineto{\pgfqpoint{6.578556in}{0.850079in}}%
\pgfpathlineto{\pgfqpoint{6.584391in}{0.862818in}}%
\pgfpathlineto{\pgfqpoint{6.591395in}{0.883425in}}%
\pgfpathlineto{\pgfqpoint{6.599565in}{0.914811in}}%
\pgfpathlineto{\pgfqpoint{6.608902in}{0.960295in}}%
\pgfpathlineto{\pgfqpoint{6.619407in}{1.023468in}}%
\pgfpathlineto{\pgfqpoint{6.631079in}{1.107959in}}%
\pgfpathlineto{\pgfqpoint{6.643918in}{1.217071in}}%
\pgfpathlineto{\pgfqpoint{6.659091in}{1.365294in}}%
\pgfpathlineto{\pgfqpoint{6.677766in}{1.570462in}}%
\pgfpathlineto{\pgfqpoint{6.709280in}{1.946484in}}%
\pgfpathlineto{\pgfqpoint{6.737293in}{2.270140in}}%
\pgfpathlineto{\pgfqpoint{6.753633in}{2.434531in}}%
\pgfpathlineto{\pgfqpoint{6.766472in}{2.543221in}}%
\pgfpathlineto{\pgfqpoint{6.776977in}{2.615381in}}%
\pgfpathlineto{\pgfqpoint{6.786315in}{2.664950in}}%
\pgfpathlineto{\pgfqpoint{6.793318in}{2.692271in}}%
\pgfpathlineto{\pgfqpoint{6.799154in}{2.708185in}}%
\pgfpathlineto{\pgfqpoint{6.803822in}{2.716251in}}%
\pgfpathlineto{\pgfqpoint{6.808491in}{2.720056in}}%
\pgfpathlineto{\pgfqpoint{6.811993in}{2.720060in}}%
\pgfpathlineto{\pgfqpoint{6.815494in}{2.717586in}}%
\pgfpathlineto{\pgfqpoint{6.820163in}{2.710396in}}%
\pgfpathlineto{\pgfqpoint{6.824832in}{2.698722in}}%
\pgfpathlineto{\pgfqpoint{6.830668in}{2.677803in}}%
\pgfpathlineto{\pgfqpoint{6.837671in}{2.643450in}}%
\pgfpathlineto{\pgfqpoint{6.845841in}{2.590808in}}%
\pgfpathlineto{\pgfqpoint{6.855179in}{2.514632in}}%
\pgfpathlineto{\pgfqpoint{6.865683in}{2.409780in}}%
\pgfpathlineto{\pgfqpoint{6.878522in}{2.257112in}}%
\pgfpathlineto{\pgfqpoint{6.893696in}{2.048717in}}%
\pgfpathlineto{\pgfqpoint{6.918207in}{1.675849in}}%
\pgfpathlineto{\pgfqpoint{6.943885in}{1.293447in}}%
\pgfpathlineto{\pgfqpoint{6.957891in}{1.114657in}}%
\pgfpathlineto{\pgfqpoint{6.969563in}{0.992953in}}%
\pgfpathlineto{\pgfqpoint{6.978900in}{0.917987in}}%
\pgfpathlineto{\pgfqpoint{6.985903in}{0.876669in}}%
\pgfpathlineto{\pgfqpoint{6.991739in}{0.852801in}}%
\pgfpathlineto{\pgfqpoint{6.996408in}{0.840962in}}%
\pgfpathlineto{\pgfqpoint{6.999910in}{0.836445in}}%
\pgfpathlineto{\pgfqpoint{7.002244in}{0.835548in}}%
\pgfpathlineto{\pgfqpoint{7.004578in}{0.836361in}}%
\pgfpathlineto{\pgfqpoint{7.008080in}{0.840813in}}%
\pgfpathlineto{\pgfqpoint{7.011581in}{0.849169in}}%
\pgfpathlineto{\pgfqpoint{7.016250in}{0.866402in}}%
\pgfpathlineto{\pgfqpoint{7.022086in}{0.897700in}}%
\pgfpathlineto{\pgfqpoint{7.029089in}{0.949350in}}%
\pgfpathlineto{\pgfqpoint{7.037259in}{1.028339in}}%
\pgfpathlineto{\pgfqpoint{7.046597in}{1.141620in}}%
\pgfpathlineto{\pgfqpoint{7.058269in}{1.313400in}}%
\pgfpathlineto{\pgfqpoint{7.073442in}{1.574757in}}%
\pgfpathlineto{\pgfqpoint{7.120130in}{2.409036in}}%
\pgfpathlineto{\pgfqpoint{7.130634in}{2.547215in}}%
\pgfpathlineto{\pgfqpoint{7.138805in}{2.629813in}}%
\pgfpathlineto{\pgfqpoint{7.145808in}{2.680589in}}%
\pgfpathlineto{\pgfqpoint{7.151644in}{2.707518in}}%
\pgfpathlineto{\pgfqpoint{7.156312in}{2.718455in}}%
\pgfpathlineto{\pgfqpoint{7.158647in}{2.720290in}}%
\pgfpathlineto{\pgfqpoint{7.160981in}{2.719667in}}%
\pgfpathlineto{\pgfqpoint{7.163316in}{2.716569in}}%
\pgfpathlineto{\pgfqpoint{7.166817in}{2.707254in}}%
\pgfpathlineto{\pgfqpoint{7.171486in}{2.686113in}}%
\pgfpathlineto{\pgfqpoint{7.177322in}{2.645781in}}%
\pgfpathlineto{\pgfqpoint{7.184325in}{2.577520in}}%
\pgfpathlineto{\pgfqpoint{7.192495in}{2.472049in}}%
\pgfpathlineto{\pgfqpoint{7.201833in}{2.321056in}}%
\pgfpathlineto{\pgfqpoint{7.213505in}{2.095573in}}%
\pgfpathlineto{\pgfqpoint{7.233347in}{1.660961in}}%
\pgfpathlineto{\pgfqpoint{7.253189in}{1.243029in}}%
\pgfpathlineto{\pgfqpoint{7.263693in}{1.062666in}}%
\pgfpathlineto{\pgfqpoint{7.271864in}{0.953653in}}%
\pgfpathlineto{\pgfqpoint{7.278867in}{0.886550in}}%
\pgfpathlineto{\pgfqpoint{7.284703in}{0.851306in}}%
\pgfpathlineto{\pgfqpoint{7.288204in}{0.839728in}}%
\pgfpathlineto{\pgfqpoint{7.290539in}{0.836117in}}%
\pgfpathlineto{\pgfqpoint{7.292873in}{0.835840in}}%
\pgfpathlineto{\pgfqpoint{7.295208in}{0.838924in}}%
\pgfpathlineto{\pgfqpoint{7.298709in}{0.849888in}}%
\pgfpathlineto{\pgfqpoint{7.303378in}{0.876322in}}%
\pgfpathlineto{\pgfqpoint{7.309214in}{0.928085in}}%
\pgfpathlineto{\pgfqpoint{7.316217in}{1.016563in}}%
\pgfpathlineto{\pgfqpoint{7.324387in}{1.153114in}}%
\pgfpathlineto{\pgfqpoint{7.334892in}{1.372783in}}%
\pgfpathlineto{\pgfqpoint{7.350065in}{1.746623in}}%
\pgfpathlineto{\pgfqpoint{7.373409in}{2.320634in}}%
\pgfpathlineto{\pgfqpoint{7.383914in}{2.522283in}}%
\pgfpathlineto{\pgfqpoint{7.390917in}{2.622256in}}%
\pgfpathlineto{\pgfqpoint{7.396753in}{2.680339in}}%
\pgfpathlineto{\pgfqpoint{7.401421in}{2.708718in}}%
\pgfpathlineto{\pgfqpoint{7.404923in}{2.718911in}}%
\pgfpathlineto{\pgfqpoint{7.407257in}{2.720291in}}%
\pgfpathlineto{\pgfqpoint{7.409592in}{2.717288in}}%
\pgfpathlineto{\pgfqpoint{7.411926in}{2.709880in}}%
\pgfpathlineto{\pgfqpoint{7.415428in}{2.690512in}}%
\pgfpathlineto{\pgfqpoint{7.420096in}{2.649437in}}%
\pgfpathlineto{\pgfqpoint{7.425932in}{2.574357in}}%
\pgfpathlineto{\pgfqpoint{7.432935in}{2.451864in}}%
\pgfpathlineto{\pgfqpoint{7.442273in}{2.241439in}}%
\pgfpathlineto{\pgfqpoint{7.455112in}{1.891298in}}%
\pgfpathlineto{\pgfqpoint{7.479623in}{1.209602in}}%
\pgfpathlineto{\pgfqpoint{7.488960in}{1.013746in}}%
\pgfpathlineto{\pgfqpoint{7.495963in}{0.909646in}}%
\pgfpathlineto{\pgfqpoint{7.500632in}{0.864283in}}%
\pgfpathlineto{\pgfqpoint{7.504134in}{0.843911in}}%
\pgfpathlineto{\pgfqpoint{7.506468in}{0.837073in}}%
\pgfpathlineto{\pgfqpoint{7.508802in}{0.835732in}}%
\pgfpathlineto{\pgfqpoint{7.511137in}{0.839940in}}%
\pgfpathlineto{\pgfqpoint{7.513471in}{0.849720in}}%
\pgfpathlineto{\pgfqpoint{7.516973in}{0.874795in}}%
\pgfpathlineto{\pgfqpoint{7.521642in}{0.927320in}}%
\pgfpathlineto{\pgfqpoint{7.527477in}{1.022265in}}%
\pgfpathlineto{\pgfqpoint{7.534481in}{1.175037in}}%
\pgfpathlineto{\pgfqpoint{7.543818in}{1.431714in}}%
\pgfpathlineto{\pgfqpoint{7.562493in}{2.029496in}}%
\pgfpathlineto{\pgfqpoint{7.575332in}{2.401343in}}%
\pgfpathlineto{\pgfqpoint{7.583502in}{2.578192in}}%
\pgfpathlineto{\pgfqpoint{7.589338in}{2.664359in}}%
\pgfpathlineto{\pgfqpoint{7.594007in}{2.705559in}}%
\pgfpathlineto{\pgfqpoint{7.597509in}{2.719181in}}%
\pgfpathlineto{\pgfqpoint{7.598676in}{2.720339in}}%
\pgfpathlineto{\pgfqpoint{7.599843in}{2.719787in}}%
\pgfpathlineto{\pgfqpoint{7.602177in}{2.713537in}}%
\pgfpathlineto{\pgfqpoint{7.605679in}{2.691297in}}%
\pgfpathlineto{\pgfqpoint{7.610348in}{2.637978in}}%
\pgfpathlineto{\pgfqpoint{7.616184in}{2.535062in}}%
\pgfpathlineto{\pgfqpoint{7.623187in}{2.363963in}}%
\pgfpathlineto{\pgfqpoint{7.632524in}{2.073120in}}%
\pgfpathlineto{\pgfqpoint{7.662871in}{1.067564in}}%
\pgfpathlineto{\pgfqpoint{7.669874in}{0.925507in}}%
\pgfpathlineto{\pgfqpoint{7.674543in}{0.866003in}}%
\pgfpathlineto{\pgfqpoint{7.678044in}{0.841781in}}%
\pgfpathlineto{\pgfqpoint{7.680379in}{0.835766in}}%
\pgfpathlineto{\pgfqpoint{7.681546in}{0.835850in}}%
\pgfpathlineto{\pgfqpoint{7.683880in}{0.842241in}}%
\pgfpathlineto{\pgfqpoint{7.686215in}{0.856932in}}%
\pgfpathlineto{\pgfqpoint{7.689716in}{0.894370in}}%
\pgfpathlineto{\pgfqpoint{7.694385in}{0.972116in}}%
\pgfpathlineto{\pgfqpoint{7.700221in}{1.110579in}}%
\pgfpathlineto{\pgfqpoint{7.708391in}{1.368208in}}%
\pgfpathlineto{\pgfqpoint{7.722397in}{1.905316in}}%
\pgfpathlineto{\pgfqpoint{7.735236in}{2.370955in}}%
\pgfpathlineto{\pgfqpoint{7.742240in}{2.561304in}}%
\pgfpathlineto{\pgfqpoint{7.748076in}{2.667349in}}%
\pgfpathlineto{\pgfqpoint{7.751577in}{2.704335in}}%
\pgfpathlineto{\pgfqpoint{7.755079in}{2.719890in}}%
\pgfpathlineto{\pgfqpoint{7.756246in}{2.720189in}}%
\pgfpathlineto{\pgfqpoint{7.757413in}{2.718023in}}%
\pgfpathlineto{\pgfqpoint{7.759747in}{2.706281in}}%
\pgfpathlineto{\pgfqpoint{7.763249in}{2.670277in}}%
\pgfpathlineto{\pgfqpoint{7.767918in}{2.589003in}}%
\pgfpathlineto{\pgfqpoint{7.773754in}{2.438277in}}%
\pgfpathlineto{\pgfqpoint{7.781924in}{2.152907in}}%
\pgfpathlineto{\pgfqpoint{7.812271in}{0.992441in}}%
\pgfpathlineto{\pgfqpoint{7.818107in}{0.880892in}}%
\pgfpathlineto{\pgfqpoint{7.821608in}{0.845608in}}%
\pgfpathlineto{\pgfqpoint{7.823943in}{0.836149in}}%
\pgfpathlineto{\pgfqpoint{7.825110in}{0.835730in}}%
\pgfpathlineto{\pgfqpoint{7.826277in}{0.838204in}}%
\pgfpathlineto{\pgfqpoint{7.828611in}{0.851842in}}%
\pgfpathlineto{\pgfqpoint{7.832113in}{0.893813in}}%
\pgfpathlineto{\pgfqpoint{7.836782in}{0.988400in}}%
\pgfpathlineto{\pgfqpoint{7.842618in}{1.162620in}}%
\pgfpathlineto{\pgfqpoint{7.850788in}{1.487126in}}%
\pgfpathlineto{\pgfqpoint{7.874132in}{2.482032in}}%
\pgfpathlineto{\pgfqpoint{7.879967in}{2.633466in}}%
\pgfpathlineto{\pgfqpoint{7.884636in}{2.702333in}}%
\pgfpathlineto{\pgfqpoint{7.886971in}{2.717572in}}%
\pgfpathlineto{\pgfqpoint{7.888138in}{2.720222in}}%
\pgfpathlineto{\pgfqpoint{7.889305in}{2.719526in}}%
\pgfpathlineto{\pgfqpoint{7.891639in}{2.708058in}}%
\pgfpathlineto{\pgfqpoint{7.895141in}{2.665865in}}%
\pgfpathlineto{\pgfqpoint{7.899810in}{2.564777in}}%
\pgfpathlineto{\pgfqpoint{7.905646in}{2.374013in}}%
\pgfpathlineto{\pgfqpoint{7.913816in}{2.017314in}}%
\pgfpathlineto{\pgfqpoint{7.933658in}{1.102181in}}%
\pgfpathlineto{\pgfqpoint{7.939494in}{0.931774in}}%
\pgfpathlineto{\pgfqpoint{7.944163in}{0.854710in}}%
\pgfpathlineto{\pgfqpoint{7.946497in}{0.838166in}}%
\pgfpathlineto{\pgfqpoint{7.947664in}{0.835602in}}%
\pgfpathlineto{\pgfqpoint{7.948831in}{0.836882in}}%
\pgfpathlineto{\pgfqpoint{7.951166in}{0.851002in}}%
\pgfpathlineto{\pgfqpoint{7.954667in}{0.900760in}}%
\pgfpathlineto{\pgfqpoint{7.959336in}{1.017881in}}%
\pgfpathlineto{\pgfqpoint{7.965172in}{1.235510in}}%
\pgfpathlineto{\pgfqpoint{7.974510in}{1.694323in}}%
\pgfpathlineto{\pgfqpoint{7.988516in}{2.391244in}}%
\pgfpathlineto{\pgfqpoint{7.994352in}{2.592934in}}%
\pgfpathlineto{\pgfqpoint{7.999020in}{2.689763in}}%
\pgfpathlineto{\pgfqpoint{8.002522in}{2.719108in}}%
\pgfpathlineto{\pgfqpoint{8.003689in}{2.720238in}}%
\pgfpathlineto{\pgfqpoint{8.004856in}{2.716986in}}%
\pgfpathlineto{\pgfqpoint{8.007191in}{2.697351in}}%
\pgfpathlineto{\pgfqpoint{8.010692in}{2.635642in}}%
\pgfpathlineto{\pgfqpoint{8.015361in}{2.496925in}}%
\pgfpathlineto{\pgfqpoint{8.021197in}{2.247058in}}%
\pgfpathlineto{\pgfqpoint{8.031702in}{1.675326in}}%
\pgfpathlineto{\pgfqpoint{8.042206in}{1.137646in}}%
\pgfpathlineto{\pgfqpoint{8.048042in}{0.936051in}}%
\pgfpathlineto{\pgfqpoint{8.051544in}{0.864896in}}%
\pgfpathlineto{\pgfqpoint{8.053878in}{0.840822in}}%
\pgfpathlineto{\pgfqpoint{8.055045in}{0.836077in}}%
\pgfpathlineto{\pgfqpoint{8.056213in}{0.836261in}}%
\pgfpathlineto{\pgfqpoint{8.058547in}{0.851468in}}%
\pgfpathlineto{\pgfqpoint{8.060881in}{0.886281in}}%
\pgfpathlineto{\pgfqpoint{8.064383in}{0.973882in}}%
\pgfpathlineto{\pgfqpoint{8.069052in}{1.150451in}}%
\pgfpathlineto{\pgfqpoint{8.076055in}{1.512606in}}%
\pgfpathlineto{\pgfqpoint{8.093562in}{2.478775in}}%
\pgfpathlineto{\pgfqpoint{8.098231in}{2.635087in}}%
\pgfpathlineto{\pgfqpoint{8.101733in}{2.701597in}}%
\pgfpathlineto{\pgfqpoint{8.104067in}{2.719331in}}%
\pgfpathlineto{\pgfqpoint{8.105234in}{2.719947in}}%
\pgfpathlineto{\pgfqpoint{8.106401in}{2.715018in}}%
\pgfpathlineto{\pgfqpoint{8.108736in}{2.688582in}}%
\pgfpathlineto{\pgfqpoint{8.112237in}{2.608567in}}%
\pgfpathlineto{\pgfqpoint{8.116906in}{2.432984in}}%
\pgfpathlineto{\pgfqpoint{8.123909in}{2.056409in}}%
\pgfpathlineto{\pgfqpoint{8.141417in}{1.046677in}}%
\pgfpathlineto{\pgfqpoint{8.146086in}{0.896489in}}%
\pgfpathlineto{\pgfqpoint{8.149587in}{0.842295in}}%
\pgfpathlineto{\pgfqpoint{8.150755in}{0.836261in}}%
\pgfpathlineto{\pgfqpoint{8.151922in}{0.836381in}}%
\pgfpathlineto{\pgfqpoint{8.153089in}{0.842681in}}%
\pgfpathlineto{\pgfqpoint{8.155423in}{0.873713in}}%
\pgfpathlineto{\pgfqpoint{8.158925in}{0.964871in}}%
\pgfpathlineto{\pgfqpoint{8.163594in}{1.161393in}}%
\pgfpathlineto{\pgfqpoint{8.170597in}{1.573665in}}%
\pgfpathlineto{\pgfqpoint{8.184603in}{2.442731in}}%
\pgfpathlineto{\pgfqpoint{8.189272in}{2.626407in}}%
\pgfpathlineto{\pgfqpoint{8.192773in}{2.702316in}}%
\pgfpathlineto{\pgfqpoint{8.195108in}{2.720049in}}%
\pgfpathlineto{\pgfqpoint{8.196275in}{2.718708in}}%
\pgfpathlineto{\pgfqpoint{8.198609in}{2.695523in}}%
\pgfpathlineto{\pgfqpoint{8.202111in}{2.610714in}}%
\pgfpathlineto{\pgfqpoint{8.206779in}{2.412849in}}%
\pgfpathlineto{\pgfqpoint{8.213783in}{1.982760in}}%
\pgfpathlineto{\pgfqpoint{8.227789in}{1.080008in}}%
\pgfpathlineto{\pgfqpoint{8.232458in}{0.903475in}}%
\pgfpathlineto{\pgfqpoint{8.235959in}{0.841730in}}%
\pgfpathlineto{\pgfqpoint{8.237126in}{0.835848in}}%
\pgfpathlineto{\pgfqpoint{8.238293in}{0.837493in}}%
\pgfpathlineto{\pgfqpoint{8.240628in}{0.863375in}}%
\pgfpathlineto{\pgfqpoint{8.244129in}{0.957117in}}%
\pgfpathlineto{\pgfqpoint{8.248798in}{1.174017in}}%
\pgfpathlineto{\pgfqpoint{8.255801in}{1.637633in}}%
\pgfpathlineto{\pgfqpoint{8.267473in}{2.434490in}}%
\pgfpathlineto{\pgfqpoint{8.272142in}{2.635165in}}%
\pgfpathlineto{\pgfqpoint{8.275643in}{2.709996in}}%
\pgfpathlineto{\pgfqpoint{8.277978in}{2.719655in}}%
\pgfpathlineto{\pgfqpoint{8.279145in}{2.712084in}}%
\pgfpathlineto{\pgfqpoint{8.281479in}{2.672305in}}%
\pgfpathlineto{\pgfqpoint{8.284981in}{2.553731in}}%
\pgfpathlineto{\pgfqpoint{8.289650in}{2.300535in}}%
\pgfpathlineto{\pgfqpoint{8.297820in}{1.701875in}}%
\pgfpathlineto{\pgfqpoint{8.307157in}{1.068240in}}%
\pgfpathlineto{\pgfqpoint{8.311826in}{0.886075in}}%
\pgfpathlineto{\pgfqpoint{8.314161in}{0.843945in}}%
\pgfpathlineto{\pgfqpoint{8.315328in}{0.836149in}}%
\pgfpathlineto{\pgfqpoint{8.316495in}{0.837376in}}%
\pgfpathlineto{\pgfqpoint{8.318829in}{0.866908in}}%
\pgfpathlineto{\pgfqpoint{8.322331in}{0.976666in}}%
\pgfpathlineto{\pgfqpoint{8.327000in}{1.230045in}}%
\pgfpathlineto{\pgfqpoint{8.335170in}{1.851209in}}%
\pgfpathlineto{\pgfqpoint{8.343340in}{2.442551in}}%
\pgfpathlineto{\pgfqpoint{8.348009in}{2.652347in}}%
\pgfpathlineto{\pgfqpoint{8.351510in}{2.717613in}}%
\pgfpathlineto{\pgfqpoint{8.352678in}{2.720053in}}%
\pgfpathlineto{\pgfqpoint{8.353845in}{2.712641in}}%
\pgfpathlineto{\pgfqpoint{8.356179in}{2.668520in}}%
\pgfpathlineto{\pgfqpoint{8.359681in}{2.532746in}}%
\pgfpathlineto{\pgfqpoint{8.364349in}{2.242309in}}%
\pgfpathlineto{\pgfqpoint{8.383024in}{0.913281in}}%
\pgfpathlineto{\pgfqpoint{8.386526in}{0.839404in}}%
\pgfpathlineto{\pgfqpoint{8.387693in}{0.835668in}}%
\pgfpathlineto{\pgfqpoint{8.388860in}{0.842618in}}%
\pgfpathlineto{\pgfqpoint{8.391195in}{0.888311in}}%
\pgfpathlineto{\pgfqpoint{8.394696in}{1.032167in}}%
\pgfpathlineto{\pgfqpoint{8.399365in}{1.340877in}}%
\pgfpathlineto{\pgfqpoint{8.415706in}{2.592582in}}%
\pgfpathlineto{\pgfqpoint{8.419207in}{2.702324in}}%
\pgfpathlineto{\pgfqpoint{8.421542in}{2.720064in}}%
\pgfpathlineto{\pgfqpoint{8.422709in}{2.711638in}}%
\pgfpathlineto{\pgfqpoint{8.425043in}{2.660482in}}%
\pgfpathlineto{\pgfqpoint{8.428545in}{2.503070in}}%
\pgfpathlineto{\pgfqpoint{8.433213in}{2.170726in}}%
\pgfpathlineto{\pgfqpoint{8.448387in}{0.963709in}}%
\pgfpathlineto{\pgfqpoint{8.451888in}{0.851453in}}%
\pgfpathlineto{\pgfqpoint{8.454223in}{0.836540in}}%
\pgfpathlineto{\pgfqpoint{8.455390in}{0.847745in}}%
\pgfpathlineto{\pgfqpoint{8.457724in}{0.906987in}}%
\pgfpathlineto{\pgfqpoint{8.461226in}{1.081549in}}%
\pgfpathlineto{\pgfqpoint{8.467062in}{1.545349in}}%
\pgfpathlineto{\pgfqpoint{8.478734in}{2.531590in}}%
\pgfpathlineto{\pgfqpoint{8.482235in}{2.681539in}}%
\pgfpathlineto{\pgfqpoint{8.484570in}{2.719191in}}%
\pgfpathlineto{\pgfqpoint{8.485737in}{2.718098in}}%
\pgfpathlineto{\pgfqpoint{8.488071in}{2.675882in}}%
\pgfpathlineto{\pgfqpoint{8.491573in}{2.517613in}}%
\pgfpathlineto{\pgfqpoint{8.496241in}{2.162555in}}%
\pgfpathlineto{\pgfqpoint{8.510248in}{0.964411in}}%
\pgfpathlineto{\pgfqpoint{8.513749in}{0.847562in}}%
\pgfpathlineto{\pgfqpoint{8.514916in}{0.836173in}}%
\pgfpathlineto{\pgfqpoint{8.516084in}{0.839125in}}%
\pgfpathlineto{\pgfqpoint{8.518418in}{0.887882in}}%
\pgfpathlineto{\pgfqpoint{8.521920in}{1.061797in}}%
\pgfpathlineto{\pgfqpoint{8.526588in}{1.442618in}}%
\pgfpathlineto{\pgfqpoint{8.539427in}{2.582752in}}%
\pgfpathlineto{\pgfqpoint{8.542929in}{2.707530in}}%
\pgfpathlineto{\pgfqpoint{8.544096in}{2.719701in}}%
\pgfpathlineto{\pgfqpoint{8.545263in}{2.716527in}}%
\pgfpathlineto{\pgfqpoint{8.547598in}{2.664354in}}%
\pgfpathlineto{\pgfqpoint{8.551099in}{2.478876in}}%
\pgfpathlineto{\pgfqpoint{8.555768in}{2.076241in}}%
\pgfpathlineto{\pgfqpoint{8.567440in}{0.994337in}}%
\pgfpathlineto{\pgfqpoint{8.570941in}{0.853292in}}%
\pgfpathlineto{\pgfqpoint{8.572109in}{0.837386in}}%
\pgfpathlineto{\pgfqpoint{8.573276in}{0.837830in}}%
\pgfpathlineto{\pgfqpoint{8.574443in}{0.854685in}}%
\pgfpathlineto{\pgfqpoint{8.576777in}{0.936423in}}%
\pgfpathlineto{\pgfqpoint{8.580279in}{1.167277in}}%
\pgfpathlineto{\pgfqpoint{8.586115in}{1.743602in}}%
\pgfpathlineto{\pgfqpoint{8.594285in}{2.520297in}}%
\pgfpathlineto{\pgfqpoint{8.597787in}{2.689510in}}%
\pgfpathlineto{\pgfqpoint{8.600121in}{2.720249in}}%
\pgfpathlineto{\pgfqpoint{8.601288in}{2.709474in}}%
\pgfpathlineto{\pgfqpoint{8.603623in}{2.636283in}}%
\pgfpathlineto{\pgfqpoint{8.607124in}{2.408946in}}%
\pgfpathlineto{\pgfqpoint{8.612960in}{1.819344in}}%
\pgfpathlineto{\pgfqpoint{8.621130in}{1.022785in}}%
\pgfpathlineto{\pgfqpoint{8.624632in}{0.858526in}}%
\pgfpathlineto{\pgfqpoint{8.626966in}{0.837226in}}%
\pgfpathlineto{\pgfqpoint{8.628133in}{0.854419in}}%
\pgfpathlineto{\pgfqpoint{8.630468in}{0.943165in}}%
\pgfpathlineto{\pgfqpoint{8.633969in}{1.197455in}}%
\pgfpathlineto{\pgfqpoint{8.640972in}{1.956661in}}%
\pgfpathlineto{\pgfqpoint{8.646808in}{2.520900in}}%
\pgfpathlineto{\pgfqpoint{8.650310in}{2.695561in}}%
\pgfpathlineto{\pgfqpoint{8.652644in}{2.718696in}}%
\pgfpathlineto{\pgfqpoint{8.653812in}{2.700716in}}%
\pgfpathlineto{\pgfqpoint{8.656146in}{2.607149in}}%
\pgfpathlineto{\pgfqpoint{8.659647in}{2.339334in}}%
\pgfpathlineto{\pgfqpoint{8.667818in}{1.420034in}}%
\pgfpathlineto{\pgfqpoint{8.672486in}{0.994647in}}%
\pgfpathlineto{\pgfqpoint{8.675988in}{0.845265in}}%
\pgfpathlineto{\pgfqpoint{8.677155in}{0.835561in}}%
\pgfpathlineto{\pgfqpoint{8.678322in}{0.846775in}}%
\pgfpathlineto{\pgfqpoint{8.680657in}{0.930834in}}%
\pgfpathlineto{\pgfqpoint{8.684158in}{1.194854in}}%
\pgfpathlineto{\pgfqpoint{8.691161in}{2.000666in}}%
\pgfpathlineto{\pgfqpoint{8.696997in}{2.570449in}}%
\pgfpathlineto{\pgfqpoint{8.700499in}{2.714346in}}%
\pgfpathlineto{\pgfqpoint{8.701666in}{2.719603in}}%
\pgfpathlineto{\pgfqpoint{8.702833in}{2.702729in}}%
\pgfpathlineto{\pgfqpoint{8.705168in}{2.604338in}}%
\pgfpathlineto{\pgfqpoint{8.708669in}{2.314921in}}%
\pgfpathlineto{\pgfqpoint{8.722675in}{0.879233in}}%
\pgfpathlineto{\pgfqpoint{8.725010in}{0.835588in}}%
\pgfpathlineto{\pgfqpoint{8.726177in}{0.848651in}}%
\pgfpathlineto{\pgfqpoint{8.728511in}{0.943396in}}%
\pgfpathlineto{\pgfqpoint{8.732013in}{1.236640in}}%
\pgfpathlineto{\pgfqpoint{8.746019in}{2.688727in}}%
\pgfpathlineto{\pgfqpoint{8.748354in}{2.718456in}}%
\pgfpathlineto{\pgfqpoint{8.749521in}{2.696451in}}%
\pgfpathlineto{\pgfqpoint{8.751855in}{2.581087in}}%
\pgfpathlineto{\pgfqpoint{8.755357in}{2.255416in}}%
\pgfpathlineto{\pgfqpoint{8.767029in}{0.932639in}}%
\pgfpathlineto{\pgfqpoint{8.769363in}{0.843066in}}%
\pgfpathlineto{\pgfqpoint{8.770530in}{0.836290in}}%
\pgfpathlineto{\pgfqpoint{8.771697in}{0.855477in}}%
\pgfpathlineto{\pgfqpoint{8.774032in}{0.969323in}}%
\pgfpathlineto{\pgfqpoint{8.777533in}{1.301715in}}%
\pgfpathlineto{\pgfqpoint{8.789205in}{2.640938in}}%
\pgfpathlineto{\pgfqpoint{8.791539in}{2.717833in}}%
\pgfpathlineto{\pgfqpoint{8.792707in}{2.715927in}}%
\pgfpathlineto{\pgfqpoint{8.795041in}{2.631143in}}%
\pgfpathlineto{\pgfqpoint{8.798543in}{2.324049in}}%
\pgfpathlineto{\pgfqpoint{8.811382in}{0.873811in}}%
\pgfpathlineto{\pgfqpoint{8.813716in}{0.837424in}}%
\pgfpathlineto{\pgfqpoint{8.814883in}{0.862111in}}%
\pgfpathlineto{\pgfqpoint{8.817217in}{0.994066in}}%
\pgfpathlineto{\pgfqpoint{8.820719in}{1.363938in}}%
\pgfpathlineto{\pgfqpoint{8.831224in}{2.634485in}}%
\pgfpathlineto{\pgfqpoint{8.833558in}{2.717853in}}%
\pgfpathlineto{\pgfqpoint{8.834725in}{2.715084in}}%
\pgfpathlineto{\pgfqpoint{8.837060in}{2.620486in}}%
\pgfpathlineto{\pgfqpoint{8.840561in}{2.283681in}}%
\pgfpathlineto{\pgfqpoint{8.852233in}{0.883977in}}%
\pgfpathlineto{\pgfqpoint{8.854567in}{0.836462in}}%
\pgfpathlineto{\pgfqpoint{8.855735in}{0.859813in}}%
\pgfpathlineto{\pgfqpoint{8.858069in}{0.997462in}}%
\pgfpathlineto{\pgfqpoint{8.861571in}{1.391322in}}%
\pgfpathlineto{\pgfqpoint{8.870908in}{2.601601in}}%
\pgfpathlineto{\pgfqpoint{8.873242in}{2.711989in}}%
\pgfpathlineto{\pgfqpoint{8.874410in}{2.718993in}}%
\pgfpathlineto{\pgfqpoint{8.875577in}{2.692987in}}%
\pgfpathlineto{\pgfqpoint{8.877911in}{2.546215in}}%
\pgfpathlineto{\pgfqpoint{8.881413in}{2.133680in}}%
\pgfpathlineto{\pgfqpoint{8.890750in}{0.924160in}}%
\pgfpathlineto{\pgfqpoint{8.893085in}{0.836940in}}%
\pgfpathlineto{\pgfqpoint{8.894252in}{0.844432in}}%
\pgfpathlineto{\pgfqpoint{8.896586in}{0.960934in}}%
\pgfpathlineto{\pgfqpoint{8.900088in}{1.350644in}}%
\pgfpathlineto{\pgfqpoint{8.909425in}{2.610662in}}%
\pgfpathlineto{\pgfqpoint{8.911760in}{2.716012in}}%
\pgfpathlineto{\pgfqpoint{8.912927in}{2.715674in}}%
\pgfpathlineto{\pgfqpoint{8.914094in}{2.679362in}}%
\pgfpathlineto{\pgfqpoint{8.916428in}{2.505170in}}%
\pgfpathlineto{\pgfqpoint{8.921097in}{1.864207in}}%
\pgfpathlineto{\pgfqpoint{8.926933in}{1.038790in}}%
\pgfpathlineto{\pgfqpoint{8.929267in}{0.869682in}}%
\pgfpathlineto{\pgfqpoint{8.930434in}{0.837853in}}%
\pgfpathlineto{\pgfqpoint{8.931602in}{0.843539in}}%
\pgfpathlineto{\pgfqpoint{8.933936in}{0.965646in}}%
\pgfpathlineto{\pgfqpoint{8.937438in}{1.380813in}}%
\pgfpathlineto{\pgfqpoint{8.946775in}{2.652160in}}%
\pgfpathlineto{\pgfqpoint{8.949109in}{2.720054in}}%
\pgfpathlineto{\pgfqpoint{8.950277in}{2.695464in}}%
\pgfpathlineto{\pgfqpoint{8.952611in}{2.533539in}}%
\pgfpathlineto{\pgfqpoint{8.956113in}{2.068025in}}%
\pgfpathlineto{\pgfqpoint{8.963116in}{1.022845in}}%
\pgfpathlineto{\pgfqpoint{8.965450in}{0.859315in}}%
\pgfpathlineto{\pgfqpoint{8.966617in}{0.835676in}}%
\pgfpathlineto{\pgfqpoint{8.967784in}{0.852939in}}%
\pgfpathlineto{\pgfqpoint{8.970119in}{1.006062in}}%
\pgfpathlineto{\pgfqpoint{8.973620in}{1.472808in}}%
\pgfpathlineto{\pgfqpoint{8.980623in}{2.537925in}}%
\pgfpathlineto{\pgfqpoint{8.982958in}{2.700083in}}%
\pgfpathlineto{\pgfqpoint{8.984125in}{2.720354in}}%
\pgfpathlineto{\pgfqpoint{8.985292in}{2.698018in}}%
\pgfpathlineto{\pgfqpoint{8.987627in}{2.530800in}}%
\pgfpathlineto{\pgfqpoint{8.991128in}{2.040562in}}%
\pgfpathlineto{\pgfqpoint{8.998131in}{0.979165in}}%
\pgfpathlineto{\pgfqpoint{9.000466in}{0.842984in}}%
\pgfpathlineto{\pgfqpoint{9.001633in}{0.839494in}}%
\pgfpathlineto{\pgfqpoint{9.002800in}{0.880210in}}%
\pgfpathlineto{\pgfqpoint{9.005134in}{1.085257in}}%
\pgfpathlineto{\pgfqpoint{9.009803in}{1.824885in}}%
\pgfpathlineto{\pgfqpoint{9.014472in}{2.533471in}}%
\pgfpathlineto{\pgfqpoint{9.016806in}{2.701814in}}%
\pgfpathlineto{\pgfqpoint{9.017973in}{2.720123in}}%
\pgfpathlineto{\pgfqpoint{9.019141in}{2.692362in}}%
\pgfpathlineto{\pgfqpoint{9.021475in}{2.505544in}}%
\pgfpathlineto{\pgfqpoint{9.026144in}{1.768298in}}%
\pgfpathlineto{\pgfqpoint{9.030812in}{1.035420in}}%
\pgfpathlineto{\pgfqpoint{9.033147in}{0.856840in}}%
\pgfpathlineto{\pgfqpoint{9.034314in}{0.835624in}}%
\pgfpathlineto{\pgfqpoint{9.035481in}{0.862260in}}%
\pgfpathlineto{\pgfqpoint{9.037816in}{1.052041in}}%
\pgfpathlineto{\pgfqpoint{9.042484in}{1.806355in}}%
\pgfpathlineto{\pgfqpoint{9.047153in}{2.541499in}}%
\pgfpathlineto{\pgfqpoint{9.049487in}{2.707215in}}%
\pgfpathlineto{\pgfqpoint{9.050655in}{2.718512in}}%
\pgfpathlineto{\pgfqpoint{9.051822in}{2.680192in}}%
\pgfpathlineto{\pgfqpoint{9.054156in}{2.464560in}}%
\pgfpathlineto{\pgfqpoint{9.058825in}{1.673916in}}%
\pgfpathlineto{\pgfqpoint{9.063494in}{0.964025in}}%
\pgfpathlineto{\pgfqpoint{9.065828in}{0.837269in}}%
\pgfpathlineto{\pgfqpoint{9.066995in}{0.849770in}}%
\pgfpathlineto{\pgfqpoint{9.069330in}{1.024143in}}%
\pgfpathlineto{\pgfqpoint{9.072831in}{1.572224in}}%
\pgfpathlineto{\pgfqpoint{9.078667in}{2.552499in}}%
\pgfpathlineto{\pgfqpoint{9.081001in}{2.712421in}}%
\pgfpathlineto{\pgfqpoint{9.082169in}{2.714811in}}%
\pgfpathlineto{\pgfqpoint{9.083336in}{2.664030in}}%
\pgfpathlineto{\pgfqpoint{9.085670in}{2.416555in}}%
\pgfpathlineto{\pgfqpoint{9.096175in}{0.845410in}}%
\pgfpathlineto{\pgfqpoint{9.097342in}{0.840071in}}%
\pgfpathlineto{\pgfqpoint{9.098509in}{0.889860in}}%
\pgfpathlineto{\pgfqpoint{9.100844in}{1.140851in}}%
\pgfpathlineto{\pgfqpoint{9.111348in}{2.715270in}}%
\pgfpathlineto{\pgfqpoint{9.112515in}{2.710734in}}%
\pgfpathlineto{\pgfqpoint{9.114850in}{2.534785in}}%
\pgfpathlineto{\pgfqpoint{9.118351in}{1.953186in}}%
\pgfpathlineto{\pgfqpoint{9.124187in}{0.954879in}}%
\pgfpathlineto{\pgfqpoint{9.126522in}{0.835569in}}%
\pgfpathlineto{\pgfqpoint{9.127689in}{0.863655in}}%
\pgfpathlineto{\pgfqpoint{9.130023in}{1.087710in}}%
\pgfpathlineto{\pgfqpoint{9.134692in}{1.956283in}}%
\pgfpathlineto{\pgfqpoint{9.139361in}{2.659066in}}%
\pgfpathlineto{\pgfqpoint{9.140528in}{2.714939in}}%
\pgfpathlineto{\pgfqpoint{9.141695in}{2.710023in}}%
\pgfpathlineto{\pgfqpoint{9.144029in}{2.522090in}}%
\pgfpathlineto{\pgfqpoint{9.147531in}{1.908574in}}%
\pgfpathlineto{\pgfqpoint{9.153367in}{0.916553in}}%
\pgfpathlineto{\pgfqpoint{9.155701in}{0.840416in}}%
\pgfpathlineto{\pgfqpoint{9.156869in}{0.896586in}}%
\pgfpathlineto{\pgfqpoint{9.159203in}{1.179927in}}%
\pgfpathlineto{\pgfqpoint{9.168540in}{2.708591in}}%
\pgfpathlineto{\pgfqpoint{9.169708in}{2.715089in}}%
\pgfpathlineto{\pgfqpoint{9.170875in}{2.656477in}}%
\pgfpathlineto{\pgfqpoint{9.173209in}{2.363516in}}%
\pgfpathlineto{\pgfqpoint{9.182547in}{0.840916in}}%
\pgfpathlineto{\pgfqpoint{9.183714in}{0.847835in}}%
\pgfpathlineto{\pgfqpoint{9.186048in}{1.056938in}}%
\pgfpathlineto{\pgfqpoint{9.190717in}{1.973646in}}%
\pgfpathlineto{\pgfqpoint{9.194218in}{2.582602in}}%
\pgfpathlineto{\pgfqpoint{9.196553in}{2.720362in}}%
\pgfpathlineto{\pgfqpoint{9.197720in}{2.686247in}}%
\pgfpathlineto{\pgfqpoint{9.200054in}{2.423155in}}%
\pgfpathlineto{\pgfqpoint{9.209392in}{0.842644in}}%
\pgfpathlineto{\pgfqpoint{9.210559in}{0.846723in}}%
\pgfpathlineto{\pgfqpoint{9.212893in}{1.062759in}}%
\pgfpathlineto{\pgfqpoint{9.217562in}{2.013791in}}%
\pgfpathlineto{\pgfqpoint{9.221064in}{2.614753in}}%
\pgfpathlineto{\pgfqpoint{9.223398in}{2.717053in}}%
\pgfpathlineto{\pgfqpoint{9.224565in}{2.658162in}}%
\pgfpathlineto{\pgfqpoint{9.226900in}{2.341750in}}%
\pgfpathlineto{\pgfqpoint{9.235070in}{0.859424in}}%
\pgfpathlineto{\pgfqpoint{9.236237in}{0.837230in}}%
\pgfpathlineto{\pgfqpoint{9.237404in}{0.891229in}}%
\pgfpathlineto{\pgfqpoint{9.239739in}{1.205583in}}%
\pgfpathlineto{\pgfqpoint{9.247909in}{2.700221in}}%
\pgfpathlineto{\pgfqpoint{9.249076in}{2.717161in}}%
\pgfpathlineto{\pgfqpoint{9.250243in}{2.655736in}}%
\pgfpathlineto{\pgfqpoint{9.252578in}{2.323295in}}%
\pgfpathlineto{\pgfqpoint{9.260748in}{0.844927in}}%
\pgfpathlineto{\pgfqpoint{9.261915in}{0.846525in}}%
\pgfpathlineto{\pgfqpoint{9.264250in}{1.083238in}}%
\pgfpathlineto{\pgfqpoint{9.273587in}{2.720021in}}%
\pgfpathlineto{\pgfqpoint{9.274754in}{2.685886in}}%
\pgfpathlineto{\pgfqpoint{9.277089in}{2.386126in}}%
\pgfpathlineto{\pgfqpoint{9.285259in}{0.848572in}}%
\pgfpathlineto{\pgfqpoint{9.286426in}{0.844252in}}%
\pgfpathlineto{\pgfqpoint{9.287593in}{0.924868in}}%
\pgfpathlineto{\pgfqpoint{9.289928in}{1.305679in}}%
\pgfpathlineto{\pgfqpoint{9.296931in}{2.687507in}}%
\pgfpathlineto{\pgfqpoint{9.298098in}{2.719544in}}%
\pgfpathlineto{\pgfqpoint{9.299265in}{2.663630in}}%
\pgfpathlineto{\pgfqpoint{9.301600in}{2.315328in}}%
\pgfpathlineto{\pgfqpoint{9.308603in}{0.883893in}}%
\pgfpathlineto{\pgfqpoint{9.309770in}{0.835604in}}%
\pgfpathlineto{\pgfqpoint{9.310937in}{0.877720in}}%
\pgfpathlineto{\pgfqpoint{9.313271in}{1.210002in}}%
\pgfpathlineto{\pgfqpoint{9.320274in}{2.667534in}}%
\pgfpathlineto{\pgfqpoint{9.321442in}{2.720150in}}%
\pgfpathlineto{\pgfqpoint{9.322609in}{2.679916in}}%
\pgfpathlineto{\pgfqpoint{9.324943in}{2.344198in}}%
\pgfpathlineto{\pgfqpoint{9.331946in}{0.880029in}}%
\pgfpathlineto{\pgfqpoint{9.333114in}{0.835603in}}%
\pgfpathlineto{\pgfqpoint{9.334281in}{0.886587in}}%
\pgfpathlineto{\pgfqpoint{9.336615in}{1.246333in}}%
\pgfpathlineto{\pgfqpoint{9.343618in}{2.693971in}}%
\pgfpathlineto{\pgfqpoint{9.344785in}{2.716846in}}%
\pgfpathlineto{\pgfqpoint{9.345953in}{2.642041in}}%
\pgfpathlineto{\pgfqpoint{9.348287in}{2.238550in}}%
\pgfpathlineto{\pgfqpoint{9.354123in}{0.929281in}}%
\pgfpathlineto{\pgfqpoint{9.355290in}{0.842567in}}%
\pgfpathlineto{\pgfqpoint{9.356457in}{0.855469in}}%
\pgfpathlineto{\pgfqpoint{9.358792in}{1.165615in}}%
\pgfpathlineto{\pgfqpoint{9.365795in}{2.679740in}}%
\pgfpathlineto{\pgfqpoint{9.366962in}{2.719705in}}%
\pgfpathlineto{\pgfqpoint{9.368129in}{2.656581in}}%
\pgfpathlineto{\pgfqpoint{9.370463in}{2.257824in}}%
\pgfpathlineto{\pgfqpoint{9.376299in}{0.921654in}}%
\pgfpathlineto{\pgfqpoint{9.377467in}{0.839761in}}%
\pgfpathlineto{\pgfqpoint{9.378634in}{0.863028in}}%
\pgfpathlineto{\pgfqpoint{9.380968in}{1.204623in}}%
\pgfpathlineto{\pgfqpoint{9.387971in}{2.703467in}}%
\pgfpathlineto{\pgfqpoint{9.389138in}{2.709622in}}%
\pgfpathlineto{\pgfqpoint{9.391473in}{2.411082in}}%
\pgfpathlineto{\pgfqpoint{9.398476in}{0.864223in}}%
\pgfpathlineto{\pgfqpoint{9.399643in}{0.840039in}}%
\pgfpathlineto{\pgfqpoint{9.400810in}{0.926570in}}%
\pgfpathlineto{\pgfqpoint{9.403145in}{1.380516in}}%
\pgfpathlineto{\pgfqpoint{9.408981in}{2.686946in}}%
\pgfpathlineto{\pgfqpoint{9.410148in}{2.717190in}}%
\pgfpathlineto{\pgfqpoint{9.411315in}{2.633860in}}%
\pgfpathlineto{\pgfqpoint{9.413649in}{2.177797in}}%
\pgfpathlineto{\pgfqpoint{9.419485in}{0.864511in}}%
\pgfpathlineto{\pgfqpoint{9.420652in}{0.840733in}}%
\pgfpathlineto{\pgfqpoint{9.421820in}{0.933067in}}%
\pgfpathlineto{\pgfqpoint{9.424154in}{1.408860in}}%
\pgfpathlineto{\pgfqpoint{9.428823in}{2.582081in}}%
\pgfpathlineto{\pgfqpoint{9.429990in}{2.703190in}}%
\pgfpathlineto{\pgfqpoint{9.431157in}{2.707311in}}%
\pgfpathlineto{\pgfqpoint{9.433491in}{2.375644in}}%
\pgfpathlineto{\pgfqpoint{9.440495in}{0.839760in}}%
\pgfpathlineto{\pgfqpoint{9.441662in}{0.869107in}}%
\pgfpathlineto{\pgfqpoint{9.443996in}{1.263739in}}%
\pgfpathlineto{\pgfqpoint{9.449832in}{2.671217in}}%
\pgfpathlineto{\pgfqpoint{9.450999in}{2.719542in}}%
\pgfpathlineto{\pgfqpoint{9.452166in}{2.642816in}}%
\pgfpathlineto{\pgfqpoint{9.454501in}{2.168586in}}%
\pgfpathlineto{\pgfqpoint{9.459170in}{0.961253in}}%
\pgfpathlineto{\pgfqpoint{9.460337in}{0.846446in}}%
\pgfpathlineto{\pgfqpoint{9.461504in}{0.857949in}}%
\pgfpathlineto{\pgfqpoint{9.463838in}{1.238527in}}%
\pgfpathlineto{\pgfqpoint{9.469674in}{2.674252in}}%
\pgfpathlineto{\pgfqpoint{9.470841in}{2.718635in}}%
\pgfpathlineto{\pgfqpoint{9.472009in}{2.632254in}}%
\pgfpathlineto{\pgfqpoint{9.474343in}{2.130053in}}%
\pgfpathlineto{\pgfqpoint{9.479012in}{0.926726in}}%
\pgfpathlineto{\pgfqpoint{9.480179in}{0.837519in}}%
\pgfpathlineto{\pgfqpoint{9.481346in}{0.881818in}}%
\pgfpathlineto{\pgfqpoint{9.483680in}{1.329530in}}%
\pgfpathlineto{\pgfqpoint{9.489516in}{2.708762in}}%
\pgfpathlineto{\pgfqpoint{9.490684in}{2.696317in}}%
\pgfpathlineto{\pgfqpoint{9.493018in}{2.291167in}}%
\pgfpathlineto{\pgfqpoint{9.498854in}{0.857493in}}%
\pgfpathlineto{\pgfqpoint{9.500021in}{0.849363in}}%
\pgfpathlineto{\pgfqpoint{9.502355in}{1.228503in}}%
\pgfpathlineto{\pgfqpoint{9.508191in}{2.692962in}}%
\pgfpathlineto{\pgfqpoint{9.509359in}{2.709763in}}%
\pgfpathlineto{\pgfqpoint{9.510526in}{2.585020in}}%
\pgfpathlineto{\pgfqpoint{9.514027in}{1.635034in}}%
\pgfpathlineto{\pgfqpoint{9.517529in}{0.861073in}}%
\pgfpathlineto{\pgfqpoint{9.518696in}{0.848024in}}%
\pgfpathlineto{\pgfqpoint{9.521030in}{1.235031in}}%
\pgfpathlineto{\pgfqpoint{9.526866in}{2.703328in}}%
\pgfpathlineto{\pgfqpoint{9.528034in}{2.699652in}}%
\pgfpathlineto{\pgfqpoint{9.530368in}{2.277065in}}%
\pgfpathlineto{\pgfqpoint{9.536204in}{0.841544in}}%
\pgfpathlineto{\pgfqpoint{9.537371in}{0.875437in}}%
\pgfpathlineto{\pgfqpoint{9.539705in}{1.353299in}}%
\pgfpathlineto{\pgfqpoint{9.544374in}{2.642671in}}%
\pgfpathlineto{\pgfqpoint{9.545541in}{2.720366in}}%
\pgfpathlineto{\pgfqpoint{9.546708in}{2.642602in}}%
\pgfpathlineto{\pgfqpoint{9.549043in}{2.093567in}}%
\pgfpathlineto{\pgfqpoint{9.553712in}{0.868489in}}%
\pgfpathlineto{\pgfqpoint{9.554879in}{0.845850in}}%
\pgfpathlineto{\pgfqpoint{9.556046in}{0.980365in}}%
\pgfpathlineto{\pgfqpoint{9.559548in}{1.997866in}}%
\pgfpathlineto{\pgfqpoint{9.561882in}{2.603758in}}%
\pgfpathlineto{\pgfqpoint{9.563049in}{2.716913in}}%
\pgfpathlineto{\pgfqpoint{9.564216in}{2.668884in}}%
\pgfpathlineto{\pgfqpoint{9.566551in}{2.146394in}}%
\pgfpathlineto{\pgfqpoint{9.571219in}{0.875305in}}%
\pgfpathlineto{\pgfqpoint{9.572387in}{0.843520in}}%
\pgfpathlineto{\pgfqpoint{9.573554in}{0.975717in}}%
\pgfpathlineto{\pgfqpoint{9.577055in}{2.012726in}}%
\pgfpathlineto{\pgfqpoint{9.579390in}{2.618443in}}%
\pgfpathlineto{\pgfqpoint{9.580557in}{2.719466in}}%
\pgfpathlineto{\pgfqpoint{9.581724in}{2.652235in}}%
\pgfpathlineto{\pgfqpoint{9.584058in}{2.086859in}}%
\pgfpathlineto{\pgfqpoint{9.588727in}{0.852373in}}%
\pgfpathlineto{\pgfqpoint{9.589894in}{0.862804in}}%
\pgfpathlineto{\pgfqpoint{9.592229in}{1.353432in}}%
\pgfpathlineto{\pgfqpoint{9.596897in}{2.676525in}}%
\pgfpathlineto{\pgfqpoint{9.598065in}{2.712630in}}%
\pgfpathlineto{\pgfqpoint{9.599232in}{2.574789in}}%
\pgfpathlineto{\pgfqpoint{9.602733in}{1.498490in}}%
\pgfpathlineto{\pgfqpoint{9.605068in}{0.907879in}}%
\pgfpathlineto{\pgfqpoint{9.606235in}{0.836470in}}%
\pgfpathlineto{\pgfqpoint{9.607402in}{0.943535in}}%
\pgfpathlineto{\pgfqpoint{9.609737in}{1.584043in}}%
\pgfpathlineto{\pgfqpoint{9.613238in}{2.625170in}}%
\pgfpathlineto{\pgfqpoint{9.614405in}{2.720313in}}%
\pgfpathlineto{\pgfqpoint{9.615572in}{2.633478in}}%
\pgfpathlineto{\pgfqpoint{9.617907in}{2.010637in}}%
\pgfpathlineto{\pgfqpoint{9.621408in}{0.943331in}}%
\pgfpathlineto{\pgfqpoint{9.622576in}{0.836157in}}%
\pgfpathlineto{\pgfqpoint{9.623743in}{0.914269in}}%
\pgfpathlineto{\pgfqpoint{9.626077in}{1.533721in}}%
\pgfpathlineto{\pgfqpoint{9.629579in}{2.612618in}}%
\pgfpathlineto{\pgfqpoint{9.630746in}{2.719895in}}%
\pgfpathlineto{\pgfqpoint{9.631913in}{2.638365in}}%
\pgfpathlineto{\pgfqpoint{9.634247in}{2.006529in}}%
\pgfpathlineto{\pgfqpoint{9.637749in}{0.930526in}}%
\pgfpathlineto{\pgfqpoint{9.638916in}{0.835551in}}%
\pgfpathlineto{\pgfqpoint{9.640083in}{0.933053in}}%
\pgfpathlineto{\pgfqpoint{9.642418in}{1.592868in}}%
\pgfpathlineto{\pgfqpoint{9.645919in}{2.648694in}}%
\pgfpathlineto{\pgfqpoint{9.647086in}{2.718227in}}%
\pgfpathlineto{\pgfqpoint{9.648254in}{2.592071in}}%
\pgfpathlineto{\pgfqpoint{9.651755in}{1.461517in}}%
\pgfpathlineto{\pgfqpoint{9.654090in}{0.878109in}}%
\pgfpathlineto{\pgfqpoint{9.655257in}{0.847979in}}%
\pgfpathlineto{\pgfqpoint{9.656424in}{1.015090in}}%
\pgfpathlineto{\pgfqpoint{9.662260in}{2.705138in}}%
\pgfpathlineto{\pgfqpoint{9.663427in}{2.681363in}}%
\pgfpathlineto{\pgfqpoint{9.665761in}{2.094686in}}%
\pgfpathlineto{\pgfqpoint{9.669263in}{0.950808in}}%
\pgfpathlineto{\pgfqpoint{9.670430in}{0.835897in}}%
\pgfpathlineto{\pgfqpoint{9.671597in}{0.927721in}}%
\pgfpathlineto{\pgfqpoint{9.673932in}{1.612453in}}%
\pgfpathlineto{\pgfqpoint{9.677433in}{2.672509in}}%
\pgfpathlineto{\pgfqpoint{9.678600in}{2.708775in}}%
\pgfpathlineto{\pgfqpoint{9.679768in}{2.536833in}}%
\pgfpathlineto{\pgfqpoint{9.685604in}{0.841106in}}%
\pgfpathlineto{\pgfqpoint{9.686771in}{0.900327in}}%
\pgfpathlineto{\pgfqpoint{9.689105in}{1.560634in}}%
\pgfpathlineto{\pgfqpoint{9.692607in}{2.662961in}}%
\pgfpathlineto{\pgfqpoint{9.693774in}{2.711829in}}%
\pgfpathlineto{\pgfqpoint{9.694941in}{2.544471in}}%
\pgfpathlineto{\pgfqpoint{9.700777in}{0.838548in}}%
\pgfpathlineto{\pgfqpoint{9.701944in}{0.913788in}}%
\pgfpathlineto{\pgfqpoint{9.704279in}{1.610991in}}%
\pgfpathlineto{\pgfqpoint{9.707780in}{2.686121in}}%
\pgfpathlineto{\pgfqpoint{9.708947in}{2.696871in}}%
\pgfpathlineto{\pgfqpoint{9.711282in}{2.107011in}}%
\pgfpathlineto{\pgfqpoint{9.714783in}{0.924069in}}%
\pgfpathlineto{\pgfqpoint{9.715950in}{0.837319in}}%
\pgfpathlineto{\pgfqpoint{9.717118in}{0.979595in}}%
\pgfpathlineto{\pgfqpoint{9.722954in}{2.717588in}}%
\pgfpathlineto{\pgfqpoint{9.724121in}{2.636076in}}%
\pgfpathlineto{\pgfqpoint{9.726455in}{1.906352in}}%
\pgfpathlineto{\pgfqpoint{9.729957in}{0.852574in}}%
\pgfpathlineto{\pgfqpoint{9.731124in}{0.882414in}}%
\pgfpathlineto{\pgfqpoint{9.733458in}{1.555869in}}%
\pgfpathlineto{\pgfqpoint{9.736960in}{2.683869in}}%
\pgfpathlineto{\pgfqpoint{9.738127in}{2.695117in}}%
\pgfpathlineto{\pgfqpoint{9.740461in}{2.069292in}}%
\pgfpathlineto{\pgfqpoint{9.743963in}{0.890476in}}%
\pgfpathlineto{\pgfqpoint{9.745130in}{0.849500in}}%
\pgfpathlineto{\pgfqpoint{9.747464in}{1.441005in}}%
\pgfpathlineto{\pgfqpoint{9.750966in}{2.652489in}}%
\pgfpathlineto{\pgfqpoint{9.752133in}{2.711439in}}%
\pgfpathlineto{\pgfqpoint{9.753300in}{2.523578in}}%
\pgfpathlineto{\pgfqpoint{9.759136in}{0.843329in}}%
\pgfpathlineto{\pgfqpoint{9.760303in}{1.029517in}}%
\pgfpathlineto{\pgfqpoint{9.766139in}{2.710472in}}%
\pgfpathlineto{\pgfqpoint{9.767307in}{2.514369in}}%
\pgfpathlineto{\pgfqpoint{9.773142in}{0.852023in}}%
\pgfpathlineto{\pgfqpoint{9.775477in}{1.484650in}}%
\pgfpathlineto{\pgfqpoint{9.778978in}{2.683330in}}%
\pgfpathlineto{\pgfqpoint{9.780146in}{2.689770in}}%
\pgfpathlineto{\pgfqpoint{9.782480in}{2.001532in}}%
\pgfpathlineto{\pgfqpoint{9.785982in}{0.852720in}}%
\pgfpathlineto{\pgfqpoint{9.787149in}{0.892678in}}%
\pgfpathlineto{\pgfqpoint{9.789483in}{1.649941in}}%
\pgfpathlineto{\pgfqpoint{9.792985in}{2.717737in}}%
\pgfpathlineto{\pgfqpoint{9.794152in}{2.617797in}}%
\pgfpathlineto{\pgfqpoint{9.799988in}{0.837752in}}%
\pgfpathlineto{\pgfqpoint{9.801155in}{1.010108in}}%
\pgfpathlineto{\pgfqpoint{9.806991in}{2.693723in}}%
\pgfpathlineto{\pgfqpoint{9.809325in}{1.986009in}}%
\pgfpathlineto{\pgfqpoint{9.812827in}{0.842665in}}%
\pgfpathlineto{\pgfqpoint{9.813994in}{0.923453in}}%
\pgfpathlineto{\pgfqpoint{9.817496in}{2.265762in}}%
\pgfpathlineto{\pgfqpoint{9.819830in}{2.715959in}}%
\pgfpathlineto{\pgfqpoint{9.820997in}{2.522670in}}%
\pgfpathlineto{\pgfqpoint{9.825666in}{0.858841in}}%
\pgfpathlineto{\pgfqpoint{9.826833in}{0.890734in}}%
\pgfpathlineto{\pgfqpoint{9.829167in}{1.692175in}}%
\pgfpathlineto{\pgfqpoint{9.831502in}{2.594327in}}%
\pgfpathlineto{\pgfqpoint{9.832669in}{2.719306in}}%
\pgfpathlineto{\pgfqpoint{9.833836in}{2.545633in}}%
\pgfpathlineto{\pgfqpoint{9.838505in}{0.860298in}}%
\pgfpathlineto{\pgfqpoint{9.839672in}{0.891251in}}%
\pgfpathlineto{\pgfqpoint{9.842006in}{1.709732in}}%
\pgfpathlineto{\pgfqpoint{9.844341in}{2.610659in}}%
\pgfpathlineto{\pgfqpoint{9.845508in}{2.716396in}}%
\pgfpathlineto{\pgfqpoint{9.846675in}{2.515614in}}%
\pgfpathlineto{\pgfqpoint{9.851344in}{0.845067in}}%
\pgfpathlineto{\pgfqpoint{9.852511in}{0.925962in}}%
\pgfpathlineto{\pgfqpoint{9.858347in}{2.695269in}}%
\pgfpathlineto{\pgfqpoint{9.860681in}{1.929412in}}%
\pgfpathlineto{\pgfqpoint{9.863016in}{0.974984in}}%
\pgfpathlineto{\pgfqpoint{9.864183in}{0.836337in}}%
\pgfpathlineto{\pgfqpoint{9.865350in}{1.018453in}}%
\pgfpathlineto{\pgfqpoint{9.870019in}{2.712490in}}%
\pgfpathlineto{\pgfqpoint{9.871186in}{2.619325in}}%
\pgfpathlineto{\pgfqpoint{9.875855in}{0.872108in}}%
\pgfpathlineto{\pgfqpoint{9.877022in}{0.883771in}}%
\pgfpathlineto{\pgfqpoint{9.879356in}{1.735014in}}%
\pgfpathlineto{\pgfqpoint{9.881691in}{2.642566in}}%
\pgfpathlineto{\pgfqpoint{9.882858in}{2.702387in}}%
\pgfpathlineto{\pgfqpoint{9.885192in}{1.930921in}}%
\pgfpathlineto{\pgfqpoint{9.887527in}{0.959377in}}%
\pgfpathlineto{\pgfqpoint{9.888694in}{0.839407in}}%
\pgfpathlineto{\pgfqpoint{9.889861in}{1.057340in}}%
\pgfpathlineto{\pgfqpoint{9.894530in}{2.720348in}}%
\pgfpathlineto{\pgfqpoint{9.895697in}{2.544975in}}%
\pgfpathlineto{\pgfqpoint{9.900366in}{0.837111in}}%
\pgfpathlineto{\pgfqpoint{9.901533in}{0.979743in}}%
\pgfpathlineto{\pgfqpoint{9.906202in}{2.715627in}}%
\pgfpathlineto{\pgfqpoint{9.907369in}{2.594870in}}%
\pgfpathlineto{\pgfqpoint{9.912038in}{0.842548in}}%
\pgfpathlineto{\pgfqpoint{9.913205in}{0.952990in}}%
\pgfpathlineto{\pgfqpoint{9.917873in}{2.713318in}}%
\pgfpathlineto{\pgfqpoint{9.919041in}{2.601163in}}%
\pgfpathlineto{\pgfqpoint{9.923709in}{0.840390in}}%
\pgfpathlineto{\pgfqpoint{9.924877in}{0.966596in}}%
\pgfpathlineto{\pgfqpoint{9.929545in}{2.718729in}}%
\pgfpathlineto{\pgfqpoint{9.930713in}{2.566002in}}%
\pgfpathlineto{\pgfqpoint{9.935381in}{0.835561in}}%
\pgfpathlineto{\pgfqpoint{9.936548in}{1.027130in}}%
\pgfpathlineto{\pgfqpoint{9.941217in}{2.716453in}}%
\pgfpathlineto{\pgfqpoint{9.942384in}{2.474344in}}%
\pgfpathlineto{\pgfqpoint{9.947053in}{0.854159in}}%
\pgfpathlineto{\pgfqpoint{9.949388in}{1.716040in}}%
\pgfpathlineto{\pgfqpoint{9.951722in}{2.670423in}}%
\pgfpathlineto{\pgfqpoint{9.952889in}{2.669795in}}%
\pgfpathlineto{\pgfqpoint{9.955223in}{1.709075in}}%
\pgfpathlineto{\pgfqpoint{9.957558in}{0.850639in}}%
\pgfpathlineto{\pgfqpoint{9.958725in}{0.942638in}}%
\pgfpathlineto{\pgfqpoint{9.963394in}{2.720327in}}%
\pgfpathlineto{\pgfqpoint{9.964561in}{2.524730in}}%
\pgfpathlineto{\pgfqpoint{9.969230in}{0.851407in}}%
\pgfpathlineto{\pgfqpoint{9.971564in}{1.733672in}}%
\pgfpathlineto{\pgfqpoint{9.973898in}{2.685186in}}%
\pgfpathlineto{\pgfqpoint{9.975066in}{2.646292in}}%
\pgfpathlineto{\pgfqpoint{9.979734in}{0.836318in}}%
\pgfpathlineto{\pgfqpoint{9.980902in}{1.020582in}}%
\pgfpathlineto{\pgfqpoint{9.985570in}{2.698424in}}%
\pgfpathlineto{\pgfqpoint{9.987905in}{1.767431in}}%
\pgfpathlineto{\pgfqpoint{9.990239in}{0.852536in}}%
\pgfpathlineto{\pgfqpoint{9.991406in}{0.949099in}}%
\pgfpathlineto{\pgfqpoint{9.996075in}{2.714854in}}%
\pgfpathlineto{\pgfqpoint{9.997242in}{2.435217in}}%
\pgfpathlineto{\pgfqpoint{10.000744in}{0.868342in}}%
\pgfpathlineto{\pgfqpoint{10.001911in}{0.921328in}}%
\pgfpathlineto{\pgfqpoint{10.006580in}{2.717643in}}%
\pgfpathlineto{\pgfqpoint{10.007747in}{2.451034in}}%
\pgfpathlineto{\pgfqpoint{10.011248in}{0.867894in}}%
\pgfpathlineto{\pgfqpoint{10.012416in}{0.925440in}}%
\pgfpathlineto{\pgfqpoint{10.017084in}{2.713520in}}%
\pgfpathlineto{\pgfqpoint{10.018251in}{2.415027in}}%
\pgfpathlineto{\pgfqpoint{10.021753in}{0.851414in}}%
\pgfpathlineto{\pgfqpoint{10.022920in}{0.963868in}}%
\pgfpathlineto{\pgfqpoint{10.027589in}{2.692702in}}%
\pgfpathlineto{\pgfqpoint{10.029923in}{1.678074in}}%
\pgfpathlineto{\pgfqpoint{10.032258in}{0.835882in}}%
\pgfpathlineto{\pgfqpoint{10.033425in}{1.053428in}}%
\pgfpathlineto{\pgfqpoint{10.036926in}{2.683525in}}%
\pgfpathlineto{\pgfqpoint{10.038094in}{2.629055in}}%
\pgfpathlineto{\pgfqpoint{10.042762in}{0.856170in}}%
\pgfpathlineto{\pgfqpoint{10.045097in}{1.870523in}}%
\pgfpathlineto{\pgfqpoint{10.047431in}{2.720329in}}%
\pgfpathlineto{\pgfqpoint{10.048598in}{2.483717in}}%
\pgfpathlineto{\pgfqpoint{10.052100in}{0.856528in}}%
\pgfpathlineto{\pgfqpoint{10.053267in}{0.962873in}}%
\pgfpathlineto{\pgfqpoint{10.057936in}{2.665372in}}%
\pgfpathlineto{\pgfqpoint{10.062605in}{0.850248in}}%
\pgfpathlineto{\pgfqpoint{10.064939in}{1.871755in}}%
\pgfpathlineto{\pgfqpoint{10.067273in}{2.720033in}}%
\pgfpathlineto{\pgfqpoint{10.068440in}{2.449941in}}%
\pgfpathlineto{\pgfqpoint{10.071942in}{0.840979in}}%
\pgfpathlineto{\pgfqpoint{10.073109in}{1.026036in}}%
\pgfpathlineto{\pgfqpoint{10.076611in}{2.696601in}}%
\pgfpathlineto{\pgfqpoint{10.077778in}{2.589399in}}%
\pgfpathlineto{\pgfqpoint{10.081279in}{0.885257in}}%
\pgfpathlineto{\pgfqpoint{10.082447in}{0.923886in}}%
\pgfpathlineto{\pgfqpoint{10.087115in}{2.661298in}}%
\pgfpathlineto{\pgfqpoint{10.091784in}{0.875460in}}%
\pgfpathlineto{\pgfqpoint{10.096453in}{2.692232in}}%
\pgfpathlineto{\pgfqpoint{10.098787in}{1.565213in}}%
\pgfpathlineto{\pgfqpoint{10.101122in}{0.857180in}}%
\pgfpathlineto{\pgfqpoint{10.103456in}{1.970605in}}%
\pgfpathlineto{\pgfqpoint{10.105790in}{2.701334in}}%
\pgfpathlineto{\pgfqpoint{10.108125in}{1.589799in}}%
\pgfpathlineto{\pgfqpoint{10.110459in}{0.855275in}}%
\pgfpathlineto{\pgfqpoint{10.112793in}{1.977362in}}%
\pgfpathlineto{\pgfqpoint{10.115128in}{2.696458in}}%
\pgfpathlineto{\pgfqpoint{10.117462in}{1.551607in}}%
\pgfpathlineto{\pgfqpoint{10.119797in}{0.868122in}}%
\pgfpathlineto{\pgfqpoint{10.124465in}{2.672837in}}%
\pgfpathlineto{\pgfqpoint{10.129134in}{0.906885in}}%
\pgfpathlineto{\pgfqpoint{10.132636in}{2.670168in}}%
\pgfpathlineto{\pgfqpoint{10.133803in}{2.613120in}}%
\pgfpathlineto{\pgfqpoint{10.137304in}{0.859208in}}%
\pgfpathlineto{\pgfqpoint{10.138472in}{0.994569in}}%
\pgfpathlineto{\pgfqpoint{10.141973in}{2.715312in}}%
\pgfpathlineto{\pgfqpoint{10.143140in}{2.489685in}}%
\pgfpathlineto{\pgfqpoint{10.146642in}{0.836128in}}%
\pgfpathlineto{\pgfqpoint{10.147809in}{1.161632in}}%
\pgfpathlineto{\pgfqpoint{10.151311in}{2.703114in}}%
\pgfpathlineto{\pgfqpoint{10.153645in}{1.520575in}}%
\pgfpathlineto{\pgfqpoint{10.155979in}{0.898005in}}%
\pgfpathlineto{\pgfqpoint{10.159481in}{2.684546in}}%
\pgfpathlineto{\pgfqpoint{10.160648in}{2.577103in}}%
\pgfpathlineto{\pgfqpoint{10.164150in}{0.837888in}}%
\pgfpathlineto{\pgfqpoint{10.165317in}{1.100919in}}%
\pgfpathlineto{\pgfqpoint{10.168818in}{2.709237in}}%
\pgfpathlineto{\pgfqpoint{10.171153in}{1.524002in}}%
\pgfpathlineto{\pgfqpoint{10.172320in}{0.929910in}}%
\pgfpathlineto{\pgfqpoint{10.173487in}{0.908645in}}%
\pgfpathlineto{\pgfqpoint{10.176989in}{2.702523in}}%
\pgfpathlineto{\pgfqpoint{10.178156in}{2.523903in}}%
\pgfpathlineto{\pgfqpoint{10.181657in}{0.838562in}}%
\pgfpathlineto{\pgfqpoint{10.182825in}{1.219873in}}%
\pgfpathlineto{\pgfqpoint{10.185159in}{2.610789in}}%
\pgfpathlineto{\pgfqpoint{10.186326in}{2.655397in}}%
\pgfpathlineto{\pgfqpoint{10.189828in}{0.850696in}}%
\pgfpathlineto{\pgfqpoint{10.190995in}{1.048529in}}%
\pgfpathlineto{\pgfqpoint{10.194496in}{2.710242in}}%
\pgfpathlineto{\pgfqpoint{10.196831in}{1.484280in}}%
\pgfpathlineto{\pgfqpoint{10.197998in}{0.903346in}}%
\pgfpathlineto{\pgfqpoint{10.199165in}{0.947248in}}%
\pgfpathlineto{\pgfqpoint{10.202667in}{2.720131in}}%
\pgfpathlineto{\pgfqpoint{10.203834in}{2.395744in}}%
\pgfpathlineto{\pgfqpoint{10.206168in}{0.962806in}}%
\pgfpathlineto{\pgfqpoint{10.207336in}{0.894660in}}%
\pgfpathlineto{\pgfqpoint{10.210837in}{2.712043in}}%
\pgfpathlineto{\pgfqpoint{10.212004in}{2.465203in}}%
\pgfpathlineto{\pgfqpoint{10.215506in}{0.871975in}}%
\pgfpathlineto{\pgfqpoint{10.219007in}{2.704316in}}%
\pgfpathlineto{\pgfqpoint{10.220175in}{2.492512in}}%
\pgfpathlineto{\pgfqpoint{10.223676in}{0.867469in}}%
\pgfpathlineto{\pgfqpoint{10.227178in}{2.705608in}}%
\pgfpathlineto{\pgfqpoint{10.228345in}{2.482495in}}%
\pgfpathlineto{\pgfqpoint{10.231846in}{0.878436in}}%
\pgfpathlineto{\pgfqpoint{10.235348in}{2.714712in}}%
\pgfpathlineto{\pgfqpoint{10.236515in}{2.432780in}}%
\pgfpathlineto{\pgfqpoint{10.238850in}{0.955889in}}%
\pgfpathlineto{\pgfqpoint{10.240017in}{0.911789in}}%
\pgfpathlineto{\pgfqpoint{10.243518in}{2.720203in}}%
\pgfpathlineto{\pgfqpoint{10.244685in}{2.334167in}}%
\pgfpathlineto{\pgfqpoint{10.247020in}{0.894827in}}%
\pgfpathlineto{\pgfqpoint{10.248187in}{0.983561in}}%
\pgfpathlineto{\pgfqpoint{10.251689in}{2.700064in}}%
\pgfpathlineto{\pgfqpoint{10.255190in}{0.844880in}}%
\pgfpathlineto{\pgfqpoint{10.256357in}{1.116400in}}%
\pgfpathlineto{\pgfqpoint{10.258692in}{2.614629in}}%
\pgfpathlineto{\pgfqpoint{10.259859in}{2.622815in}}%
\pgfpathlineto{\pgfqpoint{10.263360in}{0.844361in}}%
\pgfpathlineto{\pgfqpoint{10.266862in}{2.706170in}}%
\pgfpathlineto{\pgfqpoint{10.268029in}{2.452934in}}%
\pgfpathlineto{\pgfqpoint{10.270364in}{0.938979in}}%
\pgfpathlineto{\pgfqpoint{10.271531in}{0.940812in}}%
\pgfpathlineto{\pgfqpoint{10.275032in}{2.702777in}}%
\pgfpathlineto{\pgfqpoint{10.278534in}{0.838450in}}%
\pgfpathlineto{\pgfqpoint{10.279701in}{1.177955in}}%
\pgfpathlineto{\pgfqpoint{10.282035in}{2.666792in}}%
\pgfpathlineto{\pgfqpoint{10.283203in}{2.542640in}}%
\pgfpathlineto{\pgfqpoint{10.286704in}{0.902634in}}%
\pgfpathlineto{\pgfqpoint{10.290206in}{2.710233in}}%
\pgfpathlineto{\pgfqpoint{10.293707in}{0.838141in}}%
\pgfpathlineto{\pgfqpoint{10.294874in}{1.193617in}}%
\pgfpathlineto{\pgfqpoint{10.297209in}{2.682812in}}%
\pgfpathlineto{\pgfqpoint{10.298376in}{2.500623in}}%
\pgfpathlineto{\pgfqpoint{10.300710in}{0.942201in}}%
\pgfpathlineto{\pgfqpoint{10.301878in}{0.952639in}}%
\pgfpathlineto{\pgfqpoint{10.305379in}{2.671717in}}%
\pgfpathlineto{\pgfqpoint{10.308881in}{0.846656in}}%
\pgfpathlineto{\pgfqpoint{10.312382in}{2.720361in}}%
\pgfpathlineto{\pgfqpoint{10.313549in}{2.291585in}}%
\pgfpathlineto{\pgfqpoint{10.315884in}{0.845843in}}%
\pgfpathlineto{\pgfqpoint{10.317051in}{1.158620in}}%
\pgfpathlineto{\pgfqpoint{10.319385in}{2.683543in}}%
\pgfpathlineto{\pgfqpoint{10.320553in}{2.483171in}}%
\pgfpathlineto{\pgfqpoint{10.322887in}{0.910203in}}%
\pgfpathlineto{\pgfqpoint{10.324054in}{1.005046in}}%
\pgfpathlineto{\pgfqpoint{10.326388in}{2.601060in}}%
\pgfpathlineto{\pgfqpoint{10.327556in}{2.602585in}}%
\pgfpathlineto{\pgfqpoint{10.331057in}{0.915119in}}%
\pgfpathlineto{\pgfqpoint{10.334559in}{2.668013in}}%
\pgfpathlineto{\pgfqpoint{10.338060in}{0.869282in}}%
\pgfpathlineto{\pgfqpoint{10.341562in}{2.698790in}}%
\pgfpathlineto{\pgfqpoint{10.345063in}{0.849616in}}%
\pgfpathlineto{\pgfqpoint{10.348565in}{2.710578in}}%
\pgfpathlineto{\pgfqpoint{10.352067in}{0.843280in}}%
\pgfpathlineto{\pgfqpoint{10.355568in}{2.713028in}}%
\pgfpathlineto{\pgfqpoint{10.359070in}{0.844039in}}%
\pgfpathlineto{\pgfqpoint{10.362571in}{2.708825in}}%
\pgfpathlineto{\pgfqpoint{10.366073in}{0.852843in}}%
\pgfpathlineto{\pgfqpoint{10.369574in}{2.693345in}}%
\pgfpathlineto{\pgfqpoint{10.373076in}{0.877978in}}%
\pgfpathlineto{\pgfqpoint{10.376577in}{2.654723in}}%
\pgfpathlineto{\pgfqpoint{10.380079in}{0.934675in}}%
\pgfpathlineto{\pgfqpoint{10.382413in}{2.596319in}}%
\pgfpathlineto{\pgfqpoint{10.383581in}{2.574772in}}%
\pgfpathlineto{\pgfqpoint{10.385915in}{0.913496in}}%
\pgfpathlineto{\pgfqpoint{10.387082in}{1.043388in}}%
\pgfpathlineto{\pgfqpoint{10.389416in}{2.681813in}}%
\pgfpathlineto{\pgfqpoint{10.390584in}{2.431841in}}%
\pgfpathlineto{\pgfqpoint{10.392918in}{0.846294in}}%
\pgfpathlineto{\pgfqpoint{10.394085in}{1.225410in}}%
\pgfpathlineto{\pgfqpoint{10.396420in}{2.720364in}}%
\pgfpathlineto{\pgfqpoint{10.399921in}{0.847692in}}%
\pgfpathlineto{\pgfqpoint{10.403423in}{2.667411in}}%
\pgfpathlineto{\pgfqpoint{10.406924in}{0.963159in}}%
\pgfpathlineto{\pgfqpoint{10.409259in}{2.650752in}}%
\pgfpathlineto{\pgfqpoint{10.410426in}{2.480418in}}%
\pgfpathlineto{\pgfqpoint{10.412760in}{0.849855in}}%
\pgfpathlineto{\pgfqpoint{10.413927in}{1.227117in}}%
\pgfpathlineto{\pgfqpoint{10.416262in}{2.719116in}}%
\pgfpathlineto{\pgfqpoint{10.419763in}{0.875225in}}%
\pgfpathlineto{\pgfqpoint{10.423265in}{2.583986in}}%
\pgfpathlineto{\pgfqpoint{10.425599in}{0.885132in}}%
\pgfpathlineto{\pgfqpoint{10.426766in}{1.129548in}}%
\pgfpathlineto{\pgfqpoint{10.429101in}{2.719223in}}%
\pgfpathlineto{\pgfqpoint{10.430268in}{2.211094in}}%
\pgfpathlineto{\pgfqpoint{10.432602in}{0.862846in}}%
\pgfpathlineto{\pgfqpoint{10.436104in}{2.583450in}}%
\pgfpathlineto{\pgfqpoint{10.438438in}{0.875061in}}%
\pgfpathlineto{\pgfqpoint{10.439605in}{1.166207in}}%
\pgfpathlineto{\pgfqpoint{10.441940in}{2.719708in}}%
\pgfpathlineto{\pgfqpoint{10.445441in}{0.902587in}}%
\pgfpathlineto{\pgfqpoint{10.447776in}{2.633872in}}%
\pgfpathlineto{\pgfqpoint{10.448943in}{2.476581in}}%
\pgfpathlineto{\pgfqpoint{10.451277in}{0.837598in}}%
\pgfpathlineto{\pgfqpoint{10.452444in}{1.355807in}}%
\pgfpathlineto{\pgfqpoint{10.454779in}{2.674251in}}%
\pgfpathlineto{\pgfqpoint{10.457113in}{0.938462in}}%
\pgfpathlineto{\pgfqpoint{10.458280in}{1.062399in}}%
\pgfpathlineto{\pgfqpoint{10.460615in}{2.718211in}}%
\pgfpathlineto{\pgfqpoint{10.461782in}{2.190705in}}%
\pgfpathlineto{\pgfqpoint{10.464116in}{0.892968in}}%
\pgfpathlineto{\pgfqpoint{10.466451in}{2.642541in}}%
\pgfpathlineto{\pgfqpoint{10.467618in}{2.445333in}}%
\pgfpathlineto{\pgfqpoint{10.469952in}{0.836076in}}%
\pgfpathlineto{\pgfqpoint{10.473454in}{2.609103in}}%
\pgfpathlineto{\pgfqpoint{10.475788in}{0.863306in}}%
\pgfpathlineto{\pgfqpoint{10.476955in}{1.239677in}}%
\pgfpathlineto{\pgfqpoint{10.479290in}{2.693684in}}%
\pgfpathlineto{\pgfqpoint{10.481624in}{0.941717in}}%
\pgfpathlineto{\pgfqpoint{10.482791in}{1.078933in}}%
\pgfpathlineto{\pgfqpoint{10.485126in}{2.720016in}}%
\pgfpathlineto{\pgfqpoint{10.488627in}{0.973843in}}%
\pgfpathlineto{\pgfqpoint{10.490962in}{2.710783in}}%
\pgfpathlineto{\pgfqpoint{10.492129in}{2.218066in}}%
\pgfpathlineto{\pgfqpoint{10.494463in}{0.911250in}}%
\pgfpathlineto{\pgfqpoint{10.496798in}{2.685944in}}%
\pgfpathlineto{\pgfqpoint{10.497965in}{2.314681in}}%
\pgfpathlineto{\pgfqpoint{10.500299in}{0.877713in}}%
\pgfpathlineto{\pgfqpoint{10.502633in}{2.660745in}}%
\pgfpathlineto{\pgfqpoint{10.503801in}{2.373479in}}%
\pgfpathlineto{\pgfqpoint{10.506135in}{0.862341in}}%
\pgfpathlineto{\pgfqpoint{10.508469in}{2.645253in}}%
\pgfpathlineto{\pgfqpoint{10.509637in}{2.399296in}}%
\pgfpathlineto{\pgfqpoint{10.511971in}{0.858285in}}%
\pgfpathlineto{\pgfqpoint{10.514305in}{2.644552in}}%
\pgfpathlineto{\pgfqpoint{10.515473in}{2.394587in}}%
\pgfpathlineto{\pgfqpoint{10.517807in}{0.863422in}}%
\pgfpathlineto{\pgfqpoint{10.520141in}{2.658949in}}%
\pgfpathlineto{\pgfqpoint{10.521308in}{2.358628in}}%
\pgfpathlineto{\pgfqpoint{10.523643in}{0.880596in}}%
\pgfpathlineto{\pgfqpoint{10.525977in}{2.683885in}}%
\pgfpathlineto{\pgfqpoint{10.527144in}{2.287581in}}%
\pgfpathlineto{\pgfqpoint{10.529479in}{0.917554in}}%
\pgfpathlineto{\pgfqpoint{10.531813in}{2.709504in}}%
\pgfpathlineto{\pgfqpoint{10.532980in}{2.175496in}}%
\pgfpathlineto{\pgfqpoint{10.535315in}{0.986366in}}%
\pgfpathlineto{\pgfqpoint{10.537649in}{2.720199in}}%
\pgfpathlineto{\pgfqpoint{10.539983in}{0.964313in}}%
\pgfpathlineto{\pgfqpoint{10.541151in}{1.101840in}}%
\pgfpathlineto{\pgfqpoint{10.543485in}{2.694783in}}%
\pgfpathlineto{\pgfqpoint{10.545819in}{0.876680in}}%
\pgfpathlineto{\pgfqpoint{10.546987in}{1.278228in}}%
\pgfpathlineto{\pgfqpoint{10.549321in}{2.608384in}}%
\pgfpathlineto{\pgfqpoint{10.551655in}{0.835734in}}%
\pgfpathlineto{\pgfqpoint{10.553990in}{2.593437in}}%
\pgfpathlineto{\pgfqpoint{10.555157in}{2.437319in}}%
\pgfpathlineto{\pgfqpoint{10.557491in}{0.878048in}}%
\pgfpathlineto{\pgfqpoint{10.559826in}{2.704152in}}%
\pgfpathlineto{\pgfqpoint{10.560993in}{2.167970in}}%
\pgfpathlineto{\pgfqpoint{10.562160in}{1.036815in}}%
\pgfpathlineto{\pgfqpoint{10.563327in}{1.037940in}}%
\pgfpathlineto{\pgfqpoint{10.565661in}{2.701852in}}%
\pgfpathlineto{\pgfqpoint{10.567996in}{0.868043in}}%
\pgfpathlineto{\pgfqpoint{10.569163in}{1.333111in}}%
\pgfpathlineto{\pgfqpoint{10.570330in}{2.487708in}}%
\pgfpathlineto{\pgfqpoint{10.571497in}{2.540438in}}%
\pgfpathlineto{\pgfqpoint{10.573832in}{0.851549in}}%
\pgfpathlineto{\pgfqpoint{10.576166in}{2.690644in}}%
\pgfpathlineto{\pgfqpoint{10.577333in}{2.202436in}}%
\pgfpathlineto{\pgfqpoint{10.578501in}{1.045650in}}%
\pgfpathlineto{\pgfqpoint{10.579668in}{1.044123in}}%
\pgfpathlineto{\pgfqpoint{10.582002in}{2.688620in}}%
\pgfpathlineto{\pgfqpoint{10.584336in}{0.846166in}}%
\pgfpathlineto{\pgfqpoint{10.586671in}{2.579710in}}%
\pgfpathlineto{\pgfqpoint{10.587838in}{2.421831in}}%
\pgfpathlineto{\pgfqpoint{10.590172in}{0.920033in}}%
\pgfpathlineto{\pgfqpoint{10.592507in}{2.719528in}}%
\pgfpathlineto{\pgfqpoint{10.594841in}{0.887890in}}%
\pgfpathlineto{\pgfqpoint{10.596008in}{1.309250in}}%
\pgfpathlineto{\pgfqpoint{10.597176in}{2.496101in}}%
\pgfpathlineto{\pgfqpoint{10.598343in}{2.508066in}}%
\pgfpathlineto{\pgfqpoint{10.600677in}{0.885277in}}%
\pgfpathlineto{\pgfqpoint{10.603011in}{2.720087in}}%
\pgfpathlineto{\pgfqpoint{10.605346in}{0.898439in}}%
\pgfpathlineto{\pgfqpoint{10.606513in}{1.296604in}}%
\pgfpathlineto{\pgfqpoint{10.607680in}{2.497012in}}%
\pgfpathlineto{\pgfqpoint{10.608847in}{2.496888in}}%
\pgfpathlineto{\pgfqpoint{10.611182in}{0.902440in}}%
\pgfpathlineto{\pgfqpoint{10.613516in}{2.718516in}}%
\pgfpathlineto{\pgfqpoint{10.615850in}{0.863927in}}%
\pgfpathlineto{\pgfqpoint{10.617018in}{1.413141in}}%
\pgfpathlineto{\pgfqpoint{10.618185in}{2.582841in}}%
\pgfpathlineto{\pgfqpoint{10.619352in}{2.381204in}}%
\pgfpathlineto{\pgfqpoint{10.621686in}{0.993160in}}%
\pgfpathlineto{\pgfqpoint{10.624021in}{2.676856in}}%
\pgfpathlineto{\pgfqpoint{10.626355in}{0.836005in}}%
\pgfpathlineto{\pgfqpoint{10.628690in}{2.694467in}}%
\pgfpathlineto{\pgfqpoint{10.629857in}{2.114451in}}%
\pgfpathlineto{\pgfqpoint{10.631024in}{0.947904in}}%
\pgfpathlineto{\pgfqpoint{10.632191in}{1.230756in}}%
\pgfpathlineto{\pgfqpoint{10.633358in}{2.471492in}}%
\pgfpathlineto{\pgfqpoint{10.634525in}{2.495876in}}%
\pgfpathlineto{\pgfqpoint{10.636860in}{0.936046in}}%
\pgfpathlineto{\pgfqpoint{10.639194in}{2.694416in}}%
\pgfpathlineto{\pgfqpoint{10.641529in}{0.835619in}}%
\pgfpathlineto{\pgfqpoint{10.643863in}{2.701077in}}%
\pgfpathlineto{\pgfqpoint{10.646197in}{0.913542in}}%
\pgfpathlineto{\pgfqpoint{10.647364in}{1.317706in}}%
\pgfpathlineto{\pgfqpoint{10.648532in}{2.551068in}}%
\pgfpathlineto{\pgfqpoint{10.649699in}{2.388883in}}%
\pgfpathlineto{\pgfqpoint{10.652033in}{1.044238in}}%
\pgfpathlineto{\pgfqpoint{10.654368in}{2.605272in}}%
\pgfpathlineto{\pgfqpoint{10.656702in}{0.885910in}}%
\pgfpathlineto{\pgfqpoint{10.659036in}{2.707493in}}%
\pgfpathlineto{\pgfqpoint{10.661371in}{0.835601in}}%
\pgfpathlineto{\pgfqpoint{10.663705in}{2.711652in}}%
\pgfpathlineto{\pgfqpoint{10.666039in}{0.870878in}}%
\pgfpathlineto{\pgfqpoint{10.667207in}{1.462857in}}%
\pgfpathlineto{\pgfqpoint{10.668374in}{2.644083in}}%
\pgfpathlineto{\pgfqpoint{10.669541in}{2.202467in}}%
\pgfpathlineto{\pgfqpoint{10.670708in}{0.963593in}}%
\pgfpathlineto{\pgfqpoint{10.671875in}{1.257220in}}%
\pgfpathlineto{\pgfqpoint{10.673043in}{2.533051in}}%
\pgfpathlineto{\pgfqpoint{10.674210in}{2.381737in}}%
\pgfpathlineto{\pgfqpoint{10.675377in}{1.086655in}}%
\pgfpathlineto{\pgfqpoint{10.676544in}{1.103824in}}%
\pgfpathlineto{\pgfqpoint{10.678878in}{2.510605in}}%
\pgfpathlineto{\pgfqpoint{10.681213in}{0.997534in}}%
\pgfpathlineto{\pgfqpoint{10.683547in}{2.596648in}}%
\pgfpathlineto{\pgfqpoint{10.685882in}{0.929213in}}%
\pgfpathlineto{\pgfqpoint{10.688216in}{2.649809in}}%
\pgfpathlineto{\pgfqpoint{10.690550in}{0.888736in}}%
\pgfpathlineto{\pgfqpoint{10.692885in}{2.679896in}}%
\pgfpathlineto{\pgfqpoint{10.695219in}{0.867001in}}%
\pgfpathlineto{\pgfqpoint{10.697553in}{2.695011in}}%
\pgfpathlineto{\pgfqpoint{10.699888in}{0.857115in}}%
\pgfpathlineto{\pgfqpoint{10.702222in}{2.700691in}}%
\pgfpathlineto{\pgfqpoint{10.704557in}{0.855003in}}%
\pgfpathlineto{\pgfqpoint{10.706891in}{2.699480in}}%
\pgfpathlineto{\pgfqpoint{10.709225in}{0.859694in}}%
\pgfpathlineto{\pgfqpoint{10.711560in}{2.690749in}}%
\pgfpathlineto{\pgfqpoint{10.713894in}{0.873423in}}%
\pgfpathlineto{\pgfqpoint{10.716228in}{2.670669in}}%
\pgfpathlineto{\pgfqpoint{10.718563in}{0.901583in}}%
\pgfpathlineto{\pgfqpoint{10.720897in}{2.632353in}}%
\pgfpathlineto{\pgfqpoint{10.723232in}{0.952436in}}%
\pgfpathlineto{\pgfqpoint{10.725566in}{2.566346in}}%
\pgfpathlineto{\pgfqpoint{10.727900in}{1.036360in}}%
\pgfpathlineto{\pgfqpoint{10.730235in}{2.461725in}}%
\pgfpathlineto{\pgfqpoint{10.731402in}{1.095883in}}%
\pgfpathlineto{\pgfqpoint{10.732569in}{1.164309in}}%
\pgfpathlineto{\pgfqpoint{10.733736in}{2.525664in}}%
\pgfpathlineto{\pgfqpoint{10.734903in}{2.308166in}}%
\pgfpathlineto{\pgfqpoint{10.736071in}{0.968895in}}%
\pgfpathlineto{\pgfqpoint{10.737238in}{1.345156in}}%
\pgfpathlineto{\pgfqpoint{10.738405in}{2.640889in}}%
\pgfpathlineto{\pgfqpoint{10.739572in}{2.099247in}}%
\pgfpathlineto{\pgfqpoint{10.740739in}{0.872236in}}%
\pgfpathlineto{\pgfqpoint{10.743074in}{2.711489in}}%
\pgfpathlineto{\pgfqpoint{10.745408in}{0.835642in}}%
\pgfpathlineto{\pgfqpoint{10.747742in}{2.706039in}}%
\pgfpathlineto{\pgfqpoint{10.750077in}{0.890762in}}%
\pgfpathlineto{\pgfqpoint{10.752411in}{2.594666in}}%
\pgfpathlineto{\pgfqpoint{10.754746in}{1.063168in}}%
\pgfpathlineto{\pgfqpoint{10.755913in}{2.458537in}}%
\pgfpathlineto{\pgfqpoint{10.757080in}{2.359105in}}%
\pgfpathlineto{\pgfqpoint{10.758247in}{0.980934in}}%
\pgfpathlineto{\pgfqpoint{10.759414in}{1.360451in}}%
\pgfpathlineto{\pgfqpoint{10.760581in}{2.662324in}}%
\pgfpathlineto{\pgfqpoint{10.762916in}{0.843948in}}%
\pgfpathlineto{\pgfqpoint{10.765250in}{2.716144in}}%
\pgfpathlineto{\pgfqpoint{10.767585in}{0.887055in}}%
\pgfpathlineto{\pgfqpoint{10.769919in}{2.566877in}}%
\pgfpathlineto{\pgfqpoint{10.771086in}{1.166853in}}%
\pgfpathlineto{\pgfqpoint{10.772253in}{1.145164in}}%
\pgfpathlineto{\pgfqpoint{10.773421in}{2.553195in}}%
\pgfpathlineto{\pgfqpoint{10.774588in}{2.205700in}}%
\pgfpathlineto{\pgfqpoint{10.775755in}{0.887973in}}%
\pgfpathlineto{\pgfqpoint{10.778089in}{2.718879in}}%
\pgfpathlineto{\pgfqpoint{10.780424in}{0.860812in}}%
\pgfpathlineto{\pgfqpoint{10.782758in}{2.591251in}}%
\pgfpathlineto{\pgfqpoint{10.783925in}{1.183980in}}%
\pgfpathlineto{\pgfqpoint{10.785092in}{1.146498in}}%
\pgfpathlineto{\pgfqpoint{10.786260in}{2.567631in}}%
\pgfpathlineto{\pgfqpoint{10.787427in}{2.160521in}}%
\pgfpathlineto{\pgfqpoint{10.788594in}{0.866346in}}%
\pgfpathlineto{\pgfqpoint{10.790928in}{2.717787in}}%
\pgfpathlineto{\pgfqpoint{10.793263in}{0.915083in}}%
\pgfpathlineto{\pgfqpoint{10.795597in}{2.459078in}}%
\pgfpathlineto{\pgfqpoint{10.796764in}{1.016238in}}%
\pgfpathlineto{\pgfqpoint{10.797931in}{1.368854in}}%
\pgfpathlineto{\pgfqpoint{10.799099in}{2.687192in}}%
\pgfpathlineto{\pgfqpoint{10.801433in}{0.840925in}}%
\pgfpathlineto{\pgfqpoint{10.803767in}{2.609734in}}%
\pgfpathlineto{\pgfqpoint{10.804935in}{1.182252in}}%
\pgfpathlineto{\pgfqpoint{10.806102in}{1.178608in}}%
\pgfpathlineto{\pgfqpoint{10.807269in}{2.609644in}}%
\pgfpathlineto{\pgfqpoint{10.808436in}{2.045240in}}%
\pgfpathlineto{\pgfqpoint{10.809603in}{0.838800in}}%
\pgfpathlineto{\pgfqpoint{10.811938in}{2.668580in}}%
\pgfpathlineto{\pgfqpoint{10.814272in}{1.094127in}}%
\pgfpathlineto{\pgfqpoint{10.815000in}{2.007281in}}%
\pgfpathlineto{\pgfqpoint{10.815000in}{2.007281in}}%
\pgfusepath{stroke}%
\end{pgfscope}%
\begin{pgfscope}%
\pgfpathrectangle{\pgfqpoint{0.390138in}{0.835548in}}{\pgfqpoint{10.414862in}{1.884818in}}%
\pgfusepath{clip}%
\pgfsetrectcap%
\pgfsetroundjoin%
\pgfsetlinewidth{1.606000pt}%
\definecolor{currentstroke}{rgb}{0.000000,0.000000,1.000000}%
\pgfsetstrokecolor{currentstroke}%
\pgfsetdash{}{0pt}%
\pgfpathmoveto{\pgfqpoint{0.390138in}{0.835573in}}%
\pgfpathlineto{\pgfqpoint{2.072052in}{0.836778in}}%
\pgfpathlineto{\pgfqpoint{2.544762in}{0.839235in}}%
\pgfpathlineto{\pgfqpoint{2.841227in}{0.842886in}}%
\pgfpathlineto{\pgfqpoint{3.058324in}{0.847690in}}%
\pgfpathlineto{\pgfqpoint{3.231067in}{0.853668in}}%
\pgfpathlineto{\pgfqpoint{3.374631in}{0.860812in}}%
\pgfpathlineto{\pgfqpoint{3.497186in}{0.869086in}}%
\pgfpathlineto{\pgfqpoint{3.604567in}{0.878518in}}%
\pgfpathlineto{\pgfqpoint{3.700276in}{0.889119in}}%
\pgfpathlineto{\pgfqpoint{3.786648in}{0.900886in}}%
\pgfpathlineto{\pgfqpoint{3.866016in}{0.913933in}}%
\pgfpathlineto{\pgfqpoint{3.938382in}{0.928048in}}%
\pgfpathlineto{\pgfqpoint{4.006079in}{0.943499in}}%
\pgfpathlineto{\pgfqpoint{4.069107in}{0.960141in}}%
\pgfpathlineto{\pgfqpoint{4.128633in}{0.978144in}}%
\pgfpathlineto{\pgfqpoint{4.184658in}{0.997384in}}%
\pgfpathlineto{\pgfqpoint{4.238348in}{1.018168in}}%
\pgfpathlineto{\pgfqpoint{4.289705in}{1.040439in}}%
\pgfpathlineto{\pgfqpoint{4.338726in}{1.064112in}}%
\pgfpathlineto{\pgfqpoint{4.386581in}{1.089721in}}%
\pgfpathlineto{\pgfqpoint{4.432101in}{1.116583in}}%
\pgfpathlineto{\pgfqpoint{4.476454in}{1.145306in}}%
\pgfpathlineto{\pgfqpoint{4.519640in}{1.175886in}}%
\pgfpathlineto{\pgfqpoint{4.562826in}{1.209235in}}%
\pgfpathlineto{\pgfqpoint{4.604845in}{1.244529in}}%
\pgfpathlineto{\pgfqpoint{4.645696in}{1.281705in}}%
\pgfpathlineto{\pgfqpoint{4.686548in}{1.321870in}}%
\pgfpathlineto{\pgfqpoint{4.727399in}{1.365181in}}%
\pgfpathlineto{\pgfqpoint{4.768251in}{1.411784in}}%
\pgfpathlineto{\pgfqpoint{4.809102in}{1.461810in}}%
\pgfpathlineto{\pgfqpoint{4.849954in}{1.515366in}}%
\pgfpathlineto{\pgfqpoint{4.891972in}{1.574214in}}%
\pgfpathlineto{\pgfqpoint{4.933991in}{1.636913in}}%
\pgfpathlineto{\pgfqpoint{4.977177in}{1.705341in}}%
\pgfpathlineto{\pgfqpoint{5.022697in}{1.781721in}}%
\pgfpathlineto{\pgfqpoint{5.070552in}{1.866445in}}%
\pgfpathlineto{\pgfqpoint{5.123075in}{1.964071in}}%
\pgfpathlineto{\pgfqpoint{5.184936in}{2.083930in}}%
\pgfpathlineto{\pgfqpoint{5.364683in}{2.435542in}}%
\pgfpathlineto{\pgfqpoint{5.403200in}{2.504019in}}%
\pgfpathlineto{\pgfqpoint{5.435881in}{2.557646in}}%
\pgfpathlineto{\pgfqpoint{5.463893in}{2.599403in}}%
\pgfpathlineto{\pgfqpoint{5.488404in}{2.632085in}}%
\pgfpathlineto{\pgfqpoint{5.510581in}{2.658054in}}%
\pgfpathlineto{\pgfqpoint{5.530423in}{2.678012in}}%
\pgfpathlineto{\pgfqpoint{5.549098in}{2.693660in}}%
\pgfpathlineto{\pgfqpoint{5.565438in}{2.704631in}}%
\pgfpathlineto{\pgfqpoint{5.580612in}{2.712368in}}%
\pgfpathlineto{\pgfqpoint{5.595785in}{2.717587in}}%
\pgfpathlineto{\pgfqpoint{5.609792in}{2.720034in}}%
\pgfpathlineto{\pgfqpoint{5.622631in}{2.720172in}}%
\pgfpathlineto{\pgfqpoint{5.635470in}{2.718200in}}%
\pgfpathlineto{\pgfqpoint{5.648309in}{2.714028in}}%
\pgfpathlineto{\pgfqpoint{5.661148in}{2.707566in}}%
\pgfpathlineto{\pgfqpoint{5.673987in}{2.698729in}}%
\pgfpathlineto{\pgfqpoint{5.687993in}{2.686283in}}%
\pgfpathlineto{\pgfqpoint{5.701999in}{2.670811in}}%
\pgfpathlineto{\pgfqpoint{5.717173in}{2.650528in}}%
\pgfpathlineto{\pgfqpoint{5.732346in}{2.626478in}}%
\pgfpathlineto{\pgfqpoint{5.748687in}{2.596261in}}%
\pgfpathlineto{\pgfqpoint{5.766194in}{2.558816in}}%
\pgfpathlineto{\pgfqpoint{5.783702in}{2.516048in}}%
\pgfpathlineto{\pgfqpoint{5.802377in}{2.464522in}}%
\pgfpathlineto{\pgfqpoint{5.822219in}{2.403109in}}%
\pgfpathlineto{\pgfqpoint{5.843229in}{2.330711in}}%
\pgfpathlineto{\pgfqpoint{5.865405in}{2.246329in}}%
\pgfpathlineto{\pgfqpoint{5.889916in}{2.144103in}}%
\pgfpathlineto{\pgfqpoint{5.916761in}{2.022390in}}%
\pgfpathlineto{\pgfqpoint{5.948275in}{1.868676in}}%
\pgfpathlineto{\pgfqpoint{5.990294in}{1.651555in}}%
\pgfpathlineto{\pgfqpoint{6.060325in}{1.289072in}}%
\pgfpathlineto{\pgfqpoint{6.087170in}{1.162249in}}%
\pgfpathlineto{\pgfqpoint{6.109347in}{1.067733in}}%
\pgfpathlineto{\pgfqpoint{6.128022in}{0.997580in}}%
\pgfpathlineto{\pgfqpoint{6.144363in}{0.944725in}}%
\pgfpathlineto{\pgfqpoint{6.158369in}{0.906678in}}%
\pgfpathlineto{\pgfqpoint{6.170041in}{0.880632in}}%
\pgfpathlineto{\pgfqpoint{6.180545in}{0.861940in}}%
\pgfpathlineto{\pgfqpoint{6.189883in}{0.849337in}}%
\pgfpathlineto{\pgfqpoint{6.198053in}{0.841556in}}%
\pgfpathlineto{\pgfqpoint{6.206223in}{0.836928in}}%
\pgfpathlineto{\pgfqpoint{6.213226in}{0.835553in}}%
\pgfpathlineto{\pgfqpoint{6.220230in}{0.836637in}}%
\pgfpathlineto{\pgfqpoint{6.227233in}{0.840237in}}%
\pgfpathlineto{\pgfqpoint{6.234236in}{0.846406in}}%
\pgfpathlineto{\pgfqpoint{6.242406in}{0.856910in}}%
\pgfpathlineto{\pgfqpoint{6.250576in}{0.871033in}}%
\pgfpathlineto{\pgfqpoint{6.259914in}{0.891663in}}%
\pgfpathlineto{\pgfqpoint{6.270419in}{0.920651in}}%
\pgfpathlineto{\pgfqpoint{6.282090in}{0.960059in}}%
\pgfpathlineto{\pgfqpoint{6.294929in}{1.012114in}}%
\pgfpathlineto{\pgfqpoint{6.308936in}{1.079111in}}%
\pgfpathlineto{\pgfqpoint{6.324109in}{1.163270in}}%
\pgfpathlineto{\pgfqpoint{6.340450in}{1.266525in}}%
\pgfpathlineto{\pgfqpoint{6.359125in}{1.398904in}}%
\pgfpathlineto{\pgfqpoint{6.381301in}{1.572530in}}%
\pgfpathlineto{\pgfqpoint{6.412815in}{1.839137in}}%
\pgfpathlineto{\pgfqpoint{6.460670in}{2.244368in}}%
\pgfpathlineto{\pgfqpoint{6.481679in}{2.402623in}}%
\pgfpathlineto{\pgfqpoint{6.498020in}{2.509701in}}%
\pgfpathlineto{\pgfqpoint{6.512026in}{2.587080in}}%
\pgfpathlineto{\pgfqpoint{6.523698in}{2.639612in}}%
\pgfpathlineto{\pgfqpoint{6.533035in}{2.672899in}}%
\pgfpathlineto{\pgfqpoint{6.541206in}{2.695143in}}%
\pgfpathlineto{\pgfqpoint{6.548209in}{2.708808in}}%
\pgfpathlineto{\pgfqpoint{6.554045in}{2.716230in}}%
\pgfpathlineto{\pgfqpoint{6.558713in}{2.719496in}}%
\pgfpathlineto{\pgfqpoint{6.563382in}{2.720335in}}%
\pgfpathlineto{\pgfqpoint{6.568051in}{2.718705in}}%
\pgfpathlineto{\pgfqpoint{6.572720in}{2.714570in}}%
\pgfpathlineto{\pgfqpoint{6.578556in}{2.705835in}}%
\pgfpathlineto{\pgfqpoint{6.584391in}{2.693095in}}%
\pgfpathlineto{\pgfqpoint{6.591395in}{2.672488in}}%
\pgfpathlineto{\pgfqpoint{6.599565in}{2.641102in}}%
\pgfpathlineto{\pgfqpoint{6.608902in}{2.595619in}}%
\pgfpathlineto{\pgfqpoint{6.619407in}{2.532446in}}%
\pgfpathlineto{\pgfqpoint{6.631079in}{2.447954in}}%
\pgfpathlineto{\pgfqpoint{6.643918in}{2.338843in}}%
\pgfpathlineto{\pgfqpoint{6.659091in}{2.190619in}}%
\pgfpathlineto{\pgfqpoint{6.677766in}{1.985452in}}%
\pgfpathlineto{\pgfqpoint{6.709280in}{1.609429in}}%
\pgfpathlineto{\pgfqpoint{6.737293in}{1.285774in}}%
\pgfpathlineto{\pgfqpoint{6.753633in}{1.121382in}}%
\pgfpathlineto{\pgfqpoint{6.766472in}{1.012692in}}%
\pgfpathlineto{\pgfqpoint{6.776977in}{0.940533in}}%
\pgfpathlineto{\pgfqpoint{6.786315in}{0.890963in}}%
\pgfpathlineto{\pgfqpoint{6.793318in}{0.863642in}}%
\pgfpathlineto{\pgfqpoint{6.799154in}{0.847729in}}%
\pgfpathlineto{\pgfqpoint{6.803822in}{0.839663in}}%
\pgfpathlineto{\pgfqpoint{6.808491in}{0.835858in}}%
\pgfpathlineto{\pgfqpoint{6.811993in}{0.835854in}}%
\pgfpathlineto{\pgfqpoint{6.815494in}{0.838327in}}%
\pgfpathlineto{\pgfqpoint{6.820163in}{0.845518in}}%
\pgfpathlineto{\pgfqpoint{6.824832in}{0.857192in}}%
\pgfpathlineto{\pgfqpoint{6.830668in}{0.878111in}}%
\pgfpathlineto{\pgfqpoint{6.837671in}{0.912464in}}%
\pgfpathlineto{\pgfqpoint{6.845841in}{0.965106in}}%
\pgfpathlineto{\pgfqpoint{6.855179in}{1.041282in}}%
\pgfpathlineto{\pgfqpoint{6.865683in}{1.146134in}}%
\pgfpathlineto{\pgfqpoint{6.878522in}{1.298802in}}%
\pgfpathlineto{\pgfqpoint{6.893696in}{1.507197in}}%
\pgfpathlineto{\pgfqpoint{6.918207in}{1.880065in}}%
\pgfpathlineto{\pgfqpoint{6.943885in}{2.262467in}}%
\pgfpathlineto{\pgfqpoint{6.957891in}{2.441257in}}%
\pgfpathlineto{\pgfqpoint{6.969563in}{2.562960in}}%
\pgfpathlineto{\pgfqpoint{6.978900in}{2.637927in}}%
\pgfpathlineto{\pgfqpoint{6.985903in}{2.679245in}}%
\pgfpathlineto{\pgfqpoint{6.991739in}{2.703113in}}%
\pgfpathlineto{\pgfqpoint{6.996408in}{2.714951in}}%
\pgfpathlineto{\pgfqpoint{6.999910in}{2.719468in}}%
\pgfpathlineto{\pgfqpoint{7.002244in}{2.720365in}}%
\pgfpathlineto{\pgfqpoint{7.004578in}{2.719552in}}%
\pgfpathlineto{\pgfqpoint{7.008080in}{2.715100in}}%
\pgfpathlineto{\pgfqpoint{7.011581in}{2.706744in}}%
\pgfpathlineto{\pgfqpoint{7.016250in}{2.689512in}}%
\pgfpathlineto{\pgfqpoint{7.022086in}{2.658213in}}%
\pgfpathlineto{\pgfqpoint{7.029089in}{2.606564in}}%
\pgfpathlineto{\pgfqpoint{7.037259in}{2.527575in}}%
\pgfpathlineto{\pgfqpoint{7.046597in}{2.414294in}}%
\pgfpathlineto{\pgfqpoint{7.058269in}{2.242514in}}%
\pgfpathlineto{\pgfqpoint{7.073442in}{1.981156in}}%
\pgfpathlineto{\pgfqpoint{7.120130in}{1.146878in}}%
\pgfpathlineto{\pgfqpoint{7.130634in}{1.008699in}}%
\pgfpathlineto{\pgfqpoint{7.138805in}{0.926100in}}%
\pgfpathlineto{\pgfqpoint{7.145808in}{0.875325in}}%
\pgfpathlineto{\pgfqpoint{7.151644in}{0.848396in}}%
\pgfpathlineto{\pgfqpoint{7.156312in}{0.837458in}}%
\pgfpathlineto{\pgfqpoint{7.158647in}{0.835623in}}%
\pgfpathlineto{\pgfqpoint{7.160981in}{0.836246in}}%
\pgfpathlineto{\pgfqpoint{7.163316in}{0.839345in}}%
\pgfpathlineto{\pgfqpoint{7.166817in}{0.848659in}}%
\pgfpathlineto{\pgfqpoint{7.171486in}{0.869801in}}%
\pgfpathlineto{\pgfqpoint{7.177322in}{0.910133in}}%
\pgfpathlineto{\pgfqpoint{7.184325in}{0.978394in}}%
\pgfpathlineto{\pgfqpoint{7.192495in}{1.083864in}}%
\pgfpathlineto{\pgfqpoint{7.201833in}{1.234857in}}%
\pgfpathlineto{\pgfqpoint{7.213505in}{1.460341in}}%
\pgfpathlineto{\pgfqpoint{7.233347in}{1.894953in}}%
\pgfpathlineto{\pgfqpoint{7.253189in}{2.312884in}}%
\pgfpathlineto{\pgfqpoint{7.263693in}{2.493247in}}%
\pgfpathlineto{\pgfqpoint{7.271864in}{2.602260in}}%
\pgfpathlineto{\pgfqpoint{7.278867in}{2.669364in}}%
\pgfpathlineto{\pgfqpoint{7.284703in}{2.704608in}}%
\pgfpathlineto{\pgfqpoint{7.288204in}{2.716185in}}%
\pgfpathlineto{\pgfqpoint{7.290539in}{2.719797in}}%
\pgfpathlineto{\pgfqpoint{7.292873in}{2.720074in}}%
\pgfpathlineto{\pgfqpoint{7.295208in}{2.716989in}}%
\pgfpathlineto{\pgfqpoint{7.298709in}{2.706026in}}%
\pgfpathlineto{\pgfqpoint{7.303378in}{2.679592in}}%
\pgfpathlineto{\pgfqpoint{7.309214in}{2.627828in}}%
\pgfpathlineto{\pgfqpoint{7.316217in}{2.539350in}}%
\pgfpathlineto{\pgfqpoint{7.324387in}{2.402800in}}%
\pgfpathlineto{\pgfqpoint{7.334892in}{2.183130in}}%
\pgfpathlineto{\pgfqpoint{7.350065in}{1.809291in}}%
\pgfpathlineto{\pgfqpoint{7.373409in}{1.235280in}}%
\pgfpathlineto{\pgfqpoint{7.383914in}{1.033630in}}%
\pgfpathlineto{\pgfqpoint{7.390917in}{0.933658in}}%
\pgfpathlineto{\pgfqpoint{7.396753in}{0.875574in}}%
\pgfpathlineto{\pgfqpoint{7.401421in}{0.847195in}}%
\pgfpathlineto{\pgfqpoint{7.404923in}{0.837002in}}%
\pgfpathlineto{\pgfqpoint{7.407257in}{0.835622in}}%
\pgfpathlineto{\pgfqpoint{7.409592in}{0.838625in}}%
\pgfpathlineto{\pgfqpoint{7.411926in}{0.846033in}}%
\pgfpathlineto{\pgfqpoint{7.415428in}{0.865402in}}%
\pgfpathlineto{\pgfqpoint{7.420096in}{0.906476in}}%
\pgfpathlineto{\pgfqpoint{7.425932in}{0.981556in}}%
\pgfpathlineto{\pgfqpoint{7.432935in}{1.104049in}}%
\pgfpathlineto{\pgfqpoint{7.442273in}{1.314475in}}%
\pgfpathlineto{\pgfqpoint{7.455112in}{1.664616in}}%
\pgfpathlineto{\pgfqpoint{7.479623in}{2.346312in}}%
\pgfpathlineto{\pgfqpoint{7.488960in}{2.542168in}}%
\pgfpathlineto{\pgfqpoint{7.495963in}{2.646268in}}%
\pgfpathlineto{\pgfqpoint{7.500632in}{2.691631in}}%
\pgfpathlineto{\pgfqpoint{7.504134in}{2.712003in}}%
\pgfpathlineto{\pgfqpoint{7.506468in}{2.718841in}}%
\pgfpathlineto{\pgfqpoint{7.508802in}{2.720182in}}%
\pgfpathlineto{\pgfqpoint{7.511137in}{2.715973in}}%
\pgfpathlineto{\pgfqpoint{7.513471in}{2.706194in}}%
\pgfpathlineto{\pgfqpoint{7.516973in}{2.681118in}}%
\pgfpathlineto{\pgfqpoint{7.521642in}{2.628594in}}%
\pgfpathlineto{\pgfqpoint{7.527477in}{2.533648in}}%
\pgfpathlineto{\pgfqpoint{7.534481in}{2.380877in}}%
\pgfpathlineto{\pgfqpoint{7.543818in}{2.124199in}}%
\pgfpathlineto{\pgfqpoint{7.562493in}{1.526417in}}%
\pgfpathlineto{\pgfqpoint{7.575332in}{1.154570in}}%
\pgfpathlineto{\pgfqpoint{7.583502in}{0.977722in}}%
\pgfpathlineto{\pgfqpoint{7.589338in}{0.891555in}}%
\pgfpathlineto{\pgfqpoint{7.594007in}{0.850354in}}%
\pgfpathlineto{\pgfqpoint{7.597509in}{0.836733in}}%
\pgfpathlineto{\pgfqpoint{7.598676in}{0.835575in}}%
\pgfpathlineto{\pgfqpoint{7.599843in}{0.836126in}}%
\pgfpathlineto{\pgfqpoint{7.602177in}{0.842377in}}%
\pgfpathlineto{\pgfqpoint{7.605679in}{0.864617in}}%
\pgfpathlineto{\pgfqpoint{7.610348in}{0.917935in}}%
\pgfpathlineto{\pgfqpoint{7.616184in}{1.020852in}}%
\pgfpathlineto{\pgfqpoint{7.623187in}{1.191951in}}%
\pgfpathlineto{\pgfqpoint{7.632524in}{1.482794in}}%
\pgfpathlineto{\pgfqpoint{7.662871in}{2.488350in}}%
\pgfpathlineto{\pgfqpoint{7.669874in}{2.630407in}}%
\pgfpathlineto{\pgfqpoint{7.674543in}{2.689911in}}%
\pgfpathlineto{\pgfqpoint{7.678044in}{2.714132in}}%
\pgfpathlineto{\pgfqpoint{7.680379in}{2.720147in}}%
\pgfpathlineto{\pgfqpoint{7.681546in}{2.720063in}}%
\pgfpathlineto{\pgfqpoint{7.683880in}{2.713672in}}%
\pgfpathlineto{\pgfqpoint{7.686215in}{2.698982in}}%
\pgfpathlineto{\pgfqpoint{7.689716in}{2.661543in}}%
\pgfpathlineto{\pgfqpoint{7.694385in}{2.583797in}}%
\pgfpathlineto{\pgfqpoint{7.700221in}{2.445335in}}%
\pgfpathlineto{\pgfqpoint{7.708391in}{2.187706in}}%
\pgfpathlineto{\pgfqpoint{7.722397in}{1.650598in}}%
\pgfpathlineto{\pgfqpoint{7.735236in}{1.184959in}}%
\pgfpathlineto{\pgfqpoint{7.742240in}{0.994609in}}%
\pgfpathlineto{\pgfqpoint{7.748076in}{0.888564in}}%
\pgfpathlineto{\pgfqpoint{7.751577in}{0.851579in}}%
\pgfpathlineto{\pgfqpoint{7.755079in}{0.836024in}}%
\pgfpathlineto{\pgfqpoint{7.756246in}{0.835724in}}%
\pgfpathlineto{\pgfqpoint{7.757413in}{0.837891in}}%
\pgfpathlineto{\pgfqpoint{7.759747in}{0.849632in}}%
\pgfpathlineto{\pgfqpoint{7.763249in}{0.885636in}}%
\pgfpathlineto{\pgfqpoint{7.767918in}{0.966910in}}%
\pgfpathlineto{\pgfqpoint{7.773754in}{1.117636in}}%
\pgfpathlineto{\pgfqpoint{7.781924in}{1.403006in}}%
\pgfpathlineto{\pgfqpoint{7.812271in}{2.563473in}}%
\pgfpathlineto{\pgfqpoint{7.818107in}{2.675022in}}%
\pgfpathlineto{\pgfqpoint{7.821608in}{2.710306in}}%
\pgfpathlineto{\pgfqpoint{7.823943in}{2.719764in}}%
\pgfpathlineto{\pgfqpoint{7.825110in}{2.720184in}}%
\pgfpathlineto{\pgfqpoint{7.826277in}{2.717709in}}%
\pgfpathlineto{\pgfqpoint{7.828611in}{2.704071in}}%
\pgfpathlineto{\pgfqpoint{7.832113in}{2.662101in}}%
\pgfpathlineto{\pgfqpoint{7.836782in}{2.567514in}}%
\pgfpathlineto{\pgfqpoint{7.842618in}{2.393294in}}%
\pgfpathlineto{\pgfqpoint{7.850788in}{2.068788in}}%
\pgfpathlineto{\pgfqpoint{7.874132in}{1.073882in}}%
\pgfpathlineto{\pgfqpoint{7.879967in}{0.922448in}}%
\pgfpathlineto{\pgfqpoint{7.884636in}{0.853581in}}%
\pgfpathlineto{\pgfqpoint{7.886971in}{0.838342in}}%
\pgfpathlineto{\pgfqpoint{7.888138in}{0.835691in}}%
\pgfpathlineto{\pgfqpoint{7.889305in}{0.836388in}}%
\pgfpathlineto{\pgfqpoint{7.891639in}{0.847856in}}%
\pgfpathlineto{\pgfqpoint{7.895141in}{0.890048in}}%
\pgfpathlineto{\pgfqpoint{7.899810in}{0.991136in}}%
\pgfpathlineto{\pgfqpoint{7.905646in}{1.181901in}}%
\pgfpathlineto{\pgfqpoint{7.913816in}{1.538600in}}%
\pgfpathlineto{\pgfqpoint{7.933658in}{2.453732in}}%
\pgfpathlineto{\pgfqpoint{7.939494in}{2.624139in}}%
\pgfpathlineto{\pgfqpoint{7.944163in}{2.701204in}}%
\pgfpathlineto{\pgfqpoint{7.946497in}{2.717747in}}%
\pgfpathlineto{\pgfqpoint{7.947664in}{2.720311in}}%
\pgfpathlineto{\pgfqpoint{7.948831in}{2.719031in}}%
\pgfpathlineto{\pgfqpoint{7.951166in}{2.704912in}}%
\pgfpathlineto{\pgfqpoint{7.954667in}{2.655153in}}%
\pgfpathlineto{\pgfqpoint{7.959336in}{2.538033in}}%
\pgfpathlineto{\pgfqpoint{7.965172in}{2.320404in}}%
\pgfpathlineto{\pgfqpoint{7.974510in}{1.861591in}}%
\pgfpathlineto{\pgfqpoint{7.988516in}{1.164670in}}%
\pgfpathlineto{\pgfqpoint{7.994352in}{0.962980in}}%
\pgfpathlineto{\pgfqpoint{7.999020in}{0.866151in}}%
\pgfpathlineto{\pgfqpoint{8.002522in}{0.836806in}}%
\pgfpathlineto{\pgfqpoint{8.003689in}{0.835676in}}%
\pgfpathlineto{\pgfqpoint{8.004856in}{0.838927in}}%
\pgfpathlineto{\pgfqpoint{8.007191in}{0.858563in}}%
\pgfpathlineto{\pgfqpoint{8.010692in}{0.920271in}}%
\pgfpathlineto{\pgfqpoint{8.015361in}{1.058988in}}%
\pgfpathlineto{\pgfqpoint{8.021197in}{1.308855in}}%
\pgfpathlineto{\pgfqpoint{8.031702in}{1.880587in}}%
\pgfpathlineto{\pgfqpoint{8.042206in}{2.418268in}}%
\pgfpathlineto{\pgfqpoint{8.048042in}{2.619863in}}%
\pgfpathlineto{\pgfqpoint{8.051544in}{2.691018in}}%
\pgfpathlineto{\pgfqpoint{8.053878in}{2.715091in}}%
\pgfpathlineto{\pgfqpoint{8.055045in}{2.719836in}}%
\pgfpathlineto{\pgfqpoint{8.056213in}{2.719652in}}%
\pgfpathlineto{\pgfqpoint{8.058547in}{2.704446in}}%
\pgfpathlineto{\pgfqpoint{8.060881in}{2.669632in}}%
\pgfpathlineto{\pgfqpoint{8.064383in}{2.582032in}}%
\pgfpathlineto{\pgfqpoint{8.069052in}{2.405463in}}%
\pgfpathlineto{\pgfqpoint{8.076055in}{2.043308in}}%
\pgfpathlineto{\pgfqpoint{8.093562in}{1.077138in}}%
\pgfpathlineto{\pgfqpoint{8.098231in}{0.920826in}}%
\pgfpathlineto{\pgfqpoint{8.101733in}{0.854317in}}%
\pgfpathlineto{\pgfqpoint{8.104067in}{0.836583in}}%
\pgfpathlineto{\pgfqpoint{8.105234in}{0.835966in}}%
\pgfpathlineto{\pgfqpoint{8.106401in}{0.840896in}}%
\pgfpathlineto{\pgfqpoint{8.108736in}{0.867332in}}%
\pgfpathlineto{\pgfqpoint{8.112237in}{0.947346in}}%
\pgfpathlineto{\pgfqpoint{8.116906in}{1.122930in}}%
\pgfpathlineto{\pgfqpoint{8.123909in}{1.499505in}}%
\pgfpathlineto{\pgfqpoint{8.141417in}{2.509236in}}%
\pgfpathlineto{\pgfqpoint{8.146086in}{2.659425in}}%
\pgfpathlineto{\pgfqpoint{8.149587in}{2.713619in}}%
\pgfpathlineto{\pgfqpoint{8.150755in}{2.719653in}}%
\pgfpathlineto{\pgfqpoint{8.151922in}{2.719532in}}%
\pgfpathlineto{\pgfqpoint{8.153089in}{2.713232in}}%
\pgfpathlineto{\pgfqpoint{8.155423in}{2.682201in}}%
\pgfpathlineto{\pgfqpoint{8.158925in}{2.591043in}}%
\pgfpathlineto{\pgfqpoint{8.163594in}{2.394521in}}%
\pgfpathlineto{\pgfqpoint{8.170597in}{1.982248in}}%
\pgfpathlineto{\pgfqpoint{8.184603in}{1.113182in}}%
\pgfpathlineto{\pgfqpoint{8.189272in}{0.929507in}}%
\pgfpathlineto{\pgfqpoint{8.192773in}{0.853598in}}%
\pgfpathlineto{\pgfqpoint{8.195108in}{0.835864in}}%
\pgfpathlineto{\pgfqpoint{8.196275in}{0.837205in}}%
\pgfpathlineto{\pgfqpoint{8.198609in}{0.860390in}}%
\pgfpathlineto{\pgfqpoint{8.202111in}{0.945200in}}%
\pgfpathlineto{\pgfqpoint{8.206779in}{1.143065in}}%
\pgfpathlineto{\pgfqpoint{8.213783in}{1.573153in}}%
\pgfpathlineto{\pgfqpoint{8.227789in}{2.475905in}}%
\pgfpathlineto{\pgfqpoint{8.232458in}{2.652439in}}%
\pgfpathlineto{\pgfqpoint{8.235959in}{2.714183in}}%
\pgfpathlineto{\pgfqpoint{8.237126in}{2.720066in}}%
\pgfpathlineto{\pgfqpoint{8.238293in}{2.718420in}}%
\pgfpathlineto{\pgfqpoint{8.240628in}{2.692538in}}%
\pgfpathlineto{\pgfqpoint{8.244129in}{2.598797in}}%
\pgfpathlineto{\pgfqpoint{8.248798in}{2.381897in}}%
\pgfpathlineto{\pgfqpoint{8.255801in}{1.918281in}}%
\pgfpathlineto{\pgfqpoint{8.267473in}{1.121424in}}%
\pgfpathlineto{\pgfqpoint{8.272142in}{0.920749in}}%
\pgfpathlineto{\pgfqpoint{8.275643in}{0.845918in}}%
\pgfpathlineto{\pgfqpoint{8.277978in}{0.836258in}}%
\pgfpathlineto{\pgfqpoint{8.279145in}{0.843829in}}%
\pgfpathlineto{\pgfqpoint{8.281479in}{0.883608in}}%
\pgfpathlineto{\pgfqpoint{8.284981in}{1.002183in}}%
\pgfpathlineto{\pgfqpoint{8.289650in}{1.255379in}}%
\pgfpathlineto{\pgfqpoint{8.297820in}{1.854039in}}%
\pgfpathlineto{\pgfqpoint{8.307157in}{2.487673in}}%
\pgfpathlineto{\pgfqpoint{8.311826in}{2.669838in}}%
\pgfpathlineto{\pgfqpoint{8.314161in}{2.711969in}}%
\pgfpathlineto{\pgfqpoint{8.315328in}{2.719765in}}%
\pgfpathlineto{\pgfqpoint{8.316495in}{2.718538in}}%
\pgfpathlineto{\pgfqpoint{8.318829in}{2.689005in}}%
\pgfpathlineto{\pgfqpoint{8.322331in}{2.579248in}}%
\pgfpathlineto{\pgfqpoint{8.327000in}{2.325869in}}%
\pgfpathlineto{\pgfqpoint{8.335170in}{1.704704in}}%
\pgfpathlineto{\pgfqpoint{8.343340in}{1.113362in}}%
\pgfpathlineto{\pgfqpoint{8.348009in}{0.903567in}}%
\pgfpathlineto{\pgfqpoint{8.351510in}{0.838300in}}%
\pgfpathlineto{\pgfqpoint{8.352678in}{0.835861in}}%
\pgfpathlineto{\pgfqpoint{8.353845in}{0.843272in}}%
\pgfpathlineto{\pgfqpoint{8.356179in}{0.887394in}}%
\pgfpathlineto{\pgfqpoint{8.359681in}{1.023168in}}%
\pgfpathlineto{\pgfqpoint{8.364349in}{1.313605in}}%
\pgfpathlineto{\pgfqpoint{8.383024in}{2.642632in}}%
\pgfpathlineto{\pgfqpoint{8.386526in}{2.716509in}}%
\pgfpathlineto{\pgfqpoint{8.387693in}{2.720246in}}%
\pgfpathlineto{\pgfqpoint{8.388860in}{2.713296in}}%
\pgfpathlineto{\pgfqpoint{8.391195in}{2.667603in}}%
\pgfpathlineto{\pgfqpoint{8.394696in}{2.523747in}}%
\pgfpathlineto{\pgfqpoint{8.399365in}{2.215036in}}%
\pgfpathlineto{\pgfqpoint{8.415706in}{0.963332in}}%
\pgfpathlineto{\pgfqpoint{8.419207in}{0.853590in}}%
\pgfpathlineto{\pgfqpoint{8.421542in}{0.835850in}}%
\pgfpathlineto{\pgfqpoint{8.422709in}{0.844275in}}%
\pgfpathlineto{\pgfqpoint{8.425043in}{0.895432in}}%
\pgfpathlineto{\pgfqpoint{8.428545in}{1.052843in}}%
\pgfpathlineto{\pgfqpoint{8.433213in}{1.385188in}}%
\pgfpathlineto{\pgfqpoint{8.448387in}{2.592204in}}%
\pgfpathlineto{\pgfqpoint{8.451888in}{2.704461in}}%
\pgfpathlineto{\pgfqpoint{8.454223in}{2.719373in}}%
\pgfpathlineto{\pgfqpoint{8.455390in}{2.708168in}}%
\pgfpathlineto{\pgfqpoint{8.457724in}{2.648927in}}%
\pgfpathlineto{\pgfqpoint{8.461226in}{2.474364in}}%
\pgfpathlineto{\pgfqpoint{8.467062in}{2.010565in}}%
\pgfpathlineto{\pgfqpoint{8.478734in}{1.024323in}}%
\pgfpathlineto{\pgfqpoint{8.482235in}{0.874374in}}%
\pgfpathlineto{\pgfqpoint{8.484570in}{0.836722in}}%
\pgfpathlineto{\pgfqpoint{8.485737in}{0.837815in}}%
\pgfpathlineto{\pgfqpoint{8.488071in}{0.880031in}}%
\pgfpathlineto{\pgfqpoint{8.491573in}{1.038300in}}%
\pgfpathlineto{\pgfqpoint{8.496241in}{1.393359in}}%
\pgfpathlineto{\pgfqpoint{8.510248in}{2.591502in}}%
\pgfpathlineto{\pgfqpoint{8.513749in}{2.708351in}}%
\pgfpathlineto{\pgfqpoint{8.514916in}{2.719740in}}%
\pgfpathlineto{\pgfqpoint{8.516084in}{2.716789in}}%
\pgfpathlineto{\pgfqpoint{8.518418in}{2.668031in}}%
\pgfpathlineto{\pgfqpoint{8.521920in}{2.494117in}}%
\pgfpathlineto{\pgfqpoint{8.526588in}{2.113295in}}%
\pgfpathlineto{\pgfqpoint{8.539427in}{0.973161in}}%
\pgfpathlineto{\pgfqpoint{8.542929in}{0.848383in}}%
\pgfpathlineto{\pgfqpoint{8.544096in}{0.836213in}}%
\pgfpathlineto{\pgfqpoint{8.545263in}{0.839387in}}%
\pgfpathlineto{\pgfqpoint{8.547598in}{0.891559in}}%
\pgfpathlineto{\pgfqpoint{8.551099in}{1.077038in}}%
\pgfpathlineto{\pgfqpoint{8.555768in}{1.479673in}}%
\pgfpathlineto{\pgfqpoint{8.567440in}{2.561576in}}%
\pgfpathlineto{\pgfqpoint{8.570941in}{2.702621in}}%
\pgfpathlineto{\pgfqpoint{8.572109in}{2.718528in}}%
\pgfpathlineto{\pgfqpoint{8.573276in}{2.718083in}}%
\pgfpathlineto{\pgfqpoint{8.574443in}{2.701229in}}%
\pgfpathlineto{\pgfqpoint{8.576777in}{2.619490in}}%
\pgfpathlineto{\pgfqpoint{8.580279in}{2.388636in}}%
\pgfpathlineto{\pgfqpoint{8.586115in}{1.812311in}}%
\pgfpathlineto{\pgfqpoint{8.594285in}{1.035617in}}%
\pgfpathlineto{\pgfqpoint{8.597787in}{0.866404in}}%
\pgfpathlineto{\pgfqpoint{8.600121in}{0.835665in}}%
\pgfpathlineto{\pgfqpoint{8.601288in}{0.846440in}}%
\pgfpathlineto{\pgfqpoint{8.603623in}{0.919631in}}%
\pgfpathlineto{\pgfqpoint{8.607124in}{1.146967in}}%
\pgfpathlineto{\pgfqpoint{8.612960in}{1.736569in}}%
\pgfpathlineto{\pgfqpoint{8.621130in}{2.533128in}}%
\pgfpathlineto{\pgfqpoint{8.624632in}{2.697388in}}%
\pgfpathlineto{\pgfqpoint{8.626966in}{2.718687in}}%
\pgfpathlineto{\pgfqpoint{8.628133in}{2.701494in}}%
\pgfpathlineto{\pgfqpoint{8.630468in}{2.612748in}}%
\pgfpathlineto{\pgfqpoint{8.633969in}{2.358459in}}%
\pgfpathlineto{\pgfqpoint{8.640972in}{1.599253in}}%
\pgfpathlineto{\pgfqpoint{8.646808in}{1.035013in}}%
\pgfpathlineto{\pgfqpoint{8.650310in}{0.860352in}}%
\pgfpathlineto{\pgfqpoint{8.652644in}{0.837217in}}%
\pgfpathlineto{\pgfqpoint{8.653812in}{0.855198in}}%
\pgfpathlineto{\pgfqpoint{8.656146in}{0.948764in}}%
\pgfpathlineto{\pgfqpoint{8.659647in}{1.216580in}}%
\pgfpathlineto{\pgfqpoint{8.667818in}{2.135880in}}%
\pgfpathlineto{\pgfqpoint{8.672486in}{2.561266in}}%
\pgfpathlineto{\pgfqpoint{8.675988in}{2.710648in}}%
\pgfpathlineto{\pgfqpoint{8.677155in}{2.720352in}}%
\pgfpathlineto{\pgfqpoint{8.678322in}{2.709138in}}%
\pgfpathlineto{\pgfqpoint{8.680657in}{2.625080in}}%
\pgfpathlineto{\pgfqpoint{8.684158in}{2.361060in}}%
\pgfpathlineto{\pgfqpoint{8.691161in}{1.555247in}}%
\pgfpathlineto{\pgfqpoint{8.696997in}{0.985465in}}%
\pgfpathlineto{\pgfqpoint{8.700499in}{0.841567in}}%
\pgfpathlineto{\pgfqpoint{8.701666in}{0.836311in}}%
\pgfpathlineto{\pgfqpoint{8.702833in}{0.853185in}}%
\pgfpathlineto{\pgfqpoint{8.705168in}{0.951575in}}%
\pgfpathlineto{\pgfqpoint{8.708669in}{1.240993in}}%
\pgfpathlineto{\pgfqpoint{8.722675in}{2.676680in}}%
\pgfpathlineto{\pgfqpoint{8.725010in}{2.720326in}}%
\pgfpathlineto{\pgfqpoint{8.726177in}{2.707263in}}%
\pgfpathlineto{\pgfqpoint{8.728511in}{2.612517in}}%
\pgfpathlineto{\pgfqpoint{8.732013in}{2.319273in}}%
\pgfpathlineto{\pgfqpoint{8.746019in}{0.867186in}}%
\pgfpathlineto{\pgfqpoint{8.748354in}{0.837457in}}%
\pgfpathlineto{\pgfqpoint{8.749521in}{0.859462in}}%
\pgfpathlineto{\pgfqpoint{8.751855in}{0.974826in}}%
\pgfpathlineto{\pgfqpoint{8.755357in}{1.300498in}}%
\pgfpathlineto{\pgfqpoint{8.767029in}{2.623275in}}%
\pgfpathlineto{\pgfqpoint{8.769363in}{2.712847in}}%
\pgfpathlineto{\pgfqpoint{8.770530in}{2.719623in}}%
\pgfpathlineto{\pgfqpoint{8.771697in}{2.700437in}}%
\pgfpathlineto{\pgfqpoint{8.774032in}{2.586590in}}%
\pgfpathlineto{\pgfqpoint{8.777533in}{2.254199in}}%
\pgfpathlineto{\pgfqpoint{8.789205in}{0.914976in}}%
\pgfpathlineto{\pgfqpoint{8.791539in}{0.838081in}}%
\pgfpathlineto{\pgfqpoint{8.792707in}{0.839987in}}%
\pgfpathlineto{\pgfqpoint{8.795041in}{0.924770in}}%
\pgfpathlineto{\pgfqpoint{8.798543in}{1.231865in}}%
\pgfpathlineto{\pgfqpoint{8.811382in}{2.682102in}}%
\pgfpathlineto{\pgfqpoint{8.813716in}{2.718489in}}%
\pgfpathlineto{\pgfqpoint{8.814883in}{2.693802in}}%
\pgfpathlineto{\pgfqpoint{8.817217in}{2.561848in}}%
\pgfpathlineto{\pgfqpoint{8.820719in}{2.191976in}}%
\pgfpathlineto{\pgfqpoint{8.831224in}{0.921428in}}%
\pgfpathlineto{\pgfqpoint{8.833558in}{0.838060in}}%
\pgfpathlineto{\pgfqpoint{8.834725in}{0.840829in}}%
\pgfpathlineto{\pgfqpoint{8.837060in}{0.935428in}}%
\pgfpathlineto{\pgfqpoint{8.840561in}{1.272232in}}%
\pgfpathlineto{\pgfqpoint{8.852233in}{2.671936in}}%
\pgfpathlineto{\pgfqpoint{8.854567in}{2.719452in}}%
\pgfpathlineto{\pgfqpoint{8.855735in}{2.696101in}}%
\pgfpathlineto{\pgfqpoint{8.858069in}{2.558452in}}%
\pgfpathlineto{\pgfqpoint{8.861571in}{2.164591in}}%
\pgfpathlineto{\pgfqpoint{8.870908in}{0.954313in}}%
\pgfpathlineto{\pgfqpoint{8.873242in}{0.843924in}}%
\pgfpathlineto{\pgfqpoint{8.874410in}{0.836920in}}%
\pgfpathlineto{\pgfqpoint{8.875577in}{0.862926in}}%
\pgfpathlineto{\pgfqpoint{8.877911in}{1.009699in}}%
\pgfpathlineto{\pgfqpoint{8.881413in}{1.422234in}}%
\pgfpathlineto{\pgfqpoint{8.890750in}{2.631754in}}%
\pgfpathlineto{\pgfqpoint{8.893085in}{2.718974in}}%
\pgfpathlineto{\pgfqpoint{8.894252in}{2.711482in}}%
\pgfpathlineto{\pgfqpoint{8.896586in}{2.594980in}}%
\pgfpathlineto{\pgfqpoint{8.900088in}{2.205269in}}%
\pgfpathlineto{\pgfqpoint{8.909425in}{0.945252in}}%
\pgfpathlineto{\pgfqpoint{8.911760in}{0.839901in}}%
\pgfpathlineto{\pgfqpoint{8.912927in}{0.840239in}}%
\pgfpathlineto{\pgfqpoint{8.914094in}{0.876551in}}%
\pgfpathlineto{\pgfqpoint{8.916428in}{1.050743in}}%
\pgfpathlineto{\pgfqpoint{8.921097in}{1.691706in}}%
\pgfpathlineto{\pgfqpoint{8.926933in}{2.517124in}}%
\pgfpathlineto{\pgfqpoint{8.929267in}{2.686231in}}%
\pgfpathlineto{\pgfqpoint{8.930434in}{2.718060in}}%
\pgfpathlineto{\pgfqpoint{8.931602in}{2.712375in}}%
\pgfpathlineto{\pgfqpoint{8.933936in}{2.590267in}}%
\pgfpathlineto{\pgfqpoint{8.937438in}{2.175100in}}%
\pgfpathlineto{\pgfqpoint{8.946775in}{0.903754in}}%
\pgfpathlineto{\pgfqpoint{8.949109in}{0.835860in}}%
\pgfpathlineto{\pgfqpoint{8.950277in}{0.860449in}}%
\pgfpathlineto{\pgfqpoint{8.952611in}{1.022374in}}%
\pgfpathlineto{\pgfqpoint{8.956113in}{1.487888in}}%
\pgfpathlineto{\pgfqpoint{8.963116in}{2.533069in}}%
\pgfpathlineto{\pgfqpoint{8.965450in}{2.696598in}}%
\pgfpathlineto{\pgfqpoint{8.966617in}{2.720238in}}%
\pgfpathlineto{\pgfqpoint{8.967784in}{2.702975in}}%
\pgfpathlineto{\pgfqpoint{8.970119in}{2.549852in}}%
\pgfpathlineto{\pgfqpoint{8.973620in}{2.083106in}}%
\pgfpathlineto{\pgfqpoint{8.980623in}{1.017988in}}%
\pgfpathlineto{\pgfqpoint{8.982958in}{0.855830in}}%
\pgfpathlineto{\pgfqpoint{8.984125in}{0.835560in}}%
\pgfpathlineto{\pgfqpoint{8.985292in}{0.857895in}}%
\pgfpathlineto{\pgfqpoint{8.987627in}{1.025114in}}%
\pgfpathlineto{\pgfqpoint{8.991128in}{1.515351in}}%
\pgfpathlineto{\pgfqpoint{8.998131in}{2.576749in}}%
\pgfpathlineto{\pgfqpoint{9.000466in}{2.712929in}}%
\pgfpathlineto{\pgfqpoint{9.001633in}{2.716419in}}%
\pgfpathlineto{\pgfqpoint{9.002800in}{2.675704in}}%
\pgfpathlineto{\pgfqpoint{9.005134in}{2.470657in}}%
\pgfpathlineto{\pgfqpoint{9.009803in}{1.731029in}}%
\pgfpathlineto{\pgfqpoint{9.014472in}{1.022442in}}%
\pgfpathlineto{\pgfqpoint{9.016806in}{0.854100in}}%
\pgfpathlineto{\pgfqpoint{9.017973in}{0.835790in}}%
\pgfpathlineto{\pgfqpoint{9.019141in}{0.863552in}}%
\pgfpathlineto{\pgfqpoint{9.021475in}{1.050369in}}%
\pgfpathlineto{\pgfqpoint{9.026144in}{1.787616in}}%
\pgfpathlineto{\pgfqpoint{9.030812in}{2.520494in}}%
\pgfpathlineto{\pgfqpoint{9.033147in}{2.699074in}}%
\pgfpathlineto{\pgfqpoint{9.034314in}{2.720290in}}%
\pgfpathlineto{\pgfqpoint{9.035481in}{2.693653in}}%
\pgfpathlineto{\pgfqpoint{9.037816in}{2.503872in}}%
\pgfpathlineto{\pgfqpoint{9.042484in}{1.749558in}}%
\pgfpathlineto{\pgfqpoint{9.047153in}{1.014414in}}%
\pgfpathlineto{\pgfqpoint{9.049487in}{0.848699in}}%
\pgfpathlineto{\pgfqpoint{9.050655in}{0.837402in}}%
\pgfpathlineto{\pgfqpoint{9.051822in}{0.875722in}}%
\pgfpathlineto{\pgfqpoint{9.054156in}{1.091354in}}%
\pgfpathlineto{\pgfqpoint{9.058825in}{1.881997in}}%
\pgfpathlineto{\pgfqpoint{9.063494in}{2.591889in}}%
\pgfpathlineto{\pgfqpoint{9.065828in}{2.718645in}}%
\pgfpathlineto{\pgfqpoint{9.066995in}{2.706143in}}%
\pgfpathlineto{\pgfqpoint{9.069330in}{2.531770in}}%
\pgfpathlineto{\pgfqpoint{9.072831in}{1.983690in}}%
\pgfpathlineto{\pgfqpoint{9.078667in}{1.003415in}}%
\pgfpathlineto{\pgfqpoint{9.081001in}{0.843493in}}%
\pgfpathlineto{\pgfqpoint{9.082169in}{0.841103in}}%
\pgfpathlineto{\pgfqpoint{9.083336in}{0.891884in}}%
\pgfpathlineto{\pgfqpoint{9.085670in}{1.139358in}}%
\pgfpathlineto{\pgfqpoint{9.096175in}{2.710504in}}%
\pgfpathlineto{\pgfqpoint{9.097342in}{2.715842in}}%
\pgfpathlineto{\pgfqpoint{9.098509in}{2.666054in}}%
\pgfpathlineto{\pgfqpoint{9.100844in}{2.415062in}}%
\pgfpathlineto{\pgfqpoint{9.111348in}{0.840643in}}%
\pgfpathlineto{\pgfqpoint{9.112515in}{0.845179in}}%
\pgfpathlineto{\pgfqpoint{9.114850in}{1.021129in}}%
\pgfpathlineto{\pgfqpoint{9.118351in}{1.602727in}}%
\pgfpathlineto{\pgfqpoint{9.124187in}{2.601035in}}%
\pgfpathlineto{\pgfqpoint{9.126522in}{2.720345in}}%
\pgfpathlineto{\pgfqpoint{9.127689in}{2.692258in}}%
\pgfpathlineto{\pgfqpoint{9.130023in}{2.468203in}}%
\pgfpathlineto{\pgfqpoint{9.134692in}{1.599631in}}%
\pgfpathlineto{\pgfqpoint{9.139361in}{0.896848in}}%
\pgfpathlineto{\pgfqpoint{9.140528in}{0.840974in}}%
\pgfpathlineto{\pgfqpoint{9.141695in}{0.845890in}}%
\pgfpathlineto{\pgfqpoint{9.144029in}{1.033824in}}%
\pgfpathlineto{\pgfqpoint{9.147531in}{1.647339in}}%
\pgfpathlineto{\pgfqpoint{9.153367in}{2.639361in}}%
\pgfpathlineto{\pgfqpoint{9.155701in}{2.715497in}}%
\pgfpathlineto{\pgfqpoint{9.156869in}{2.659327in}}%
\pgfpathlineto{\pgfqpoint{9.159203in}{2.375986in}}%
\pgfpathlineto{\pgfqpoint{9.168540in}{0.847322in}}%
\pgfpathlineto{\pgfqpoint{9.169708in}{0.840824in}}%
\pgfpathlineto{\pgfqpoint{9.170875in}{0.899436in}}%
\pgfpathlineto{\pgfqpoint{9.173209in}{1.192397in}}%
\pgfpathlineto{\pgfqpoint{9.182547in}{2.714998in}}%
\pgfpathlineto{\pgfqpoint{9.183714in}{2.708079in}}%
\pgfpathlineto{\pgfqpoint{9.186048in}{2.498975in}}%
\pgfpathlineto{\pgfqpoint{9.190717in}{1.582268in}}%
\pgfpathlineto{\pgfqpoint{9.194218in}{0.973311in}}%
\pgfpathlineto{\pgfqpoint{9.196553in}{0.835552in}}%
\pgfpathlineto{\pgfqpoint{9.197720in}{0.869666in}}%
\pgfpathlineto{\pgfqpoint{9.200054in}{1.132758in}}%
\pgfpathlineto{\pgfqpoint{9.209392in}{2.713269in}}%
\pgfpathlineto{\pgfqpoint{9.210559in}{2.709191in}}%
\pgfpathlineto{\pgfqpoint{9.212893in}{2.493154in}}%
\pgfpathlineto{\pgfqpoint{9.217562in}{1.542122in}}%
\pgfpathlineto{\pgfqpoint{9.221064in}{0.941160in}}%
\pgfpathlineto{\pgfqpoint{9.223398in}{0.838861in}}%
\pgfpathlineto{\pgfqpoint{9.224565in}{0.897751in}}%
\pgfpathlineto{\pgfqpoint{9.226900in}{1.214163in}}%
\pgfpathlineto{\pgfqpoint{9.235070in}{2.696489in}}%
\pgfpathlineto{\pgfqpoint{9.236237in}{2.718684in}}%
\pgfpathlineto{\pgfqpoint{9.237404in}{2.664684in}}%
\pgfpathlineto{\pgfqpoint{9.239739in}{2.350330in}}%
\pgfpathlineto{\pgfqpoint{9.247909in}{0.855693in}}%
\pgfpathlineto{\pgfqpoint{9.249076in}{0.838752in}}%
\pgfpathlineto{\pgfqpoint{9.250243in}{0.900177in}}%
\pgfpathlineto{\pgfqpoint{9.252578in}{1.232619in}}%
\pgfpathlineto{\pgfqpoint{9.260748in}{2.710987in}}%
\pgfpathlineto{\pgfqpoint{9.261915in}{2.709389in}}%
\pgfpathlineto{\pgfqpoint{9.264250in}{2.472675in}}%
\pgfpathlineto{\pgfqpoint{9.273587in}{0.835893in}}%
\pgfpathlineto{\pgfqpoint{9.274754in}{0.870027in}}%
\pgfpathlineto{\pgfqpoint{9.277089in}{1.169788in}}%
\pgfpathlineto{\pgfqpoint{9.285259in}{2.707341in}}%
\pgfpathlineto{\pgfqpoint{9.286426in}{2.711661in}}%
\pgfpathlineto{\pgfqpoint{9.287593in}{2.631045in}}%
\pgfpathlineto{\pgfqpoint{9.289928in}{2.250235in}}%
\pgfpathlineto{\pgfqpoint{9.296931in}{0.868407in}}%
\pgfpathlineto{\pgfqpoint{9.298098in}{0.836369in}}%
\pgfpathlineto{\pgfqpoint{9.299265in}{0.892284in}}%
\pgfpathlineto{\pgfqpoint{9.301600in}{1.240585in}}%
\pgfpathlineto{\pgfqpoint{9.308603in}{2.672021in}}%
\pgfpathlineto{\pgfqpoint{9.309770in}{2.720309in}}%
\pgfpathlineto{\pgfqpoint{9.310937in}{2.678194in}}%
\pgfpathlineto{\pgfqpoint{9.313271in}{2.345911in}}%
\pgfpathlineto{\pgfqpoint{9.320274in}{0.888379in}}%
\pgfpathlineto{\pgfqpoint{9.321442in}{0.835764in}}%
\pgfpathlineto{\pgfqpoint{9.322609in}{0.875998in}}%
\pgfpathlineto{\pgfqpoint{9.324943in}{1.211715in}}%
\pgfpathlineto{\pgfqpoint{9.331946in}{2.675884in}}%
\pgfpathlineto{\pgfqpoint{9.333114in}{2.720310in}}%
\pgfpathlineto{\pgfqpoint{9.334281in}{2.669326in}}%
\pgfpathlineto{\pgfqpoint{9.336615in}{2.309581in}}%
\pgfpathlineto{\pgfqpoint{9.343618in}{0.861942in}}%
\pgfpathlineto{\pgfqpoint{9.344785in}{0.839068in}}%
\pgfpathlineto{\pgfqpoint{9.345953in}{0.913872in}}%
\pgfpathlineto{\pgfqpoint{9.348287in}{1.317363in}}%
\pgfpathlineto{\pgfqpoint{9.354123in}{2.626633in}}%
\pgfpathlineto{\pgfqpoint{9.355290in}{2.713346in}}%
\pgfpathlineto{\pgfqpoint{9.356457in}{2.700444in}}%
\pgfpathlineto{\pgfqpoint{9.358792in}{2.390299in}}%
\pgfpathlineto{\pgfqpoint{9.365795in}{0.876174in}}%
\pgfpathlineto{\pgfqpoint{9.366962in}{0.836208in}}%
\pgfpathlineto{\pgfqpoint{9.368129in}{0.899332in}}%
\pgfpathlineto{\pgfqpoint{9.370463in}{1.298090in}}%
\pgfpathlineto{\pgfqpoint{9.376299in}{2.634259in}}%
\pgfpathlineto{\pgfqpoint{9.377467in}{2.716153in}}%
\pgfpathlineto{\pgfqpoint{9.378634in}{2.692886in}}%
\pgfpathlineto{\pgfqpoint{9.380968in}{2.351290in}}%
\pgfpathlineto{\pgfqpoint{9.387971in}{0.852447in}}%
\pgfpathlineto{\pgfqpoint{9.389138in}{0.846292in}}%
\pgfpathlineto{\pgfqpoint{9.391473in}{1.144832in}}%
\pgfpathlineto{\pgfqpoint{9.398476in}{2.691690in}}%
\pgfpathlineto{\pgfqpoint{9.399643in}{2.715874in}}%
\pgfpathlineto{\pgfqpoint{9.400810in}{2.629344in}}%
\pgfpathlineto{\pgfqpoint{9.403145in}{2.175397in}}%
\pgfpathlineto{\pgfqpoint{9.408981in}{0.868968in}}%
\pgfpathlineto{\pgfqpoint{9.410148in}{0.838723in}}%
\pgfpathlineto{\pgfqpoint{9.411315in}{0.922053in}}%
\pgfpathlineto{\pgfqpoint{9.413649in}{1.378116in}}%
\pgfpathlineto{\pgfqpoint{9.419485in}{2.691402in}}%
\pgfpathlineto{\pgfqpoint{9.420652in}{2.715180in}}%
\pgfpathlineto{\pgfqpoint{9.421820in}{2.622847in}}%
\pgfpathlineto{\pgfqpoint{9.424154in}{2.147054in}}%
\pgfpathlineto{\pgfqpoint{9.428823in}{0.973833in}}%
\pgfpathlineto{\pgfqpoint{9.429990in}{0.852724in}}%
\pgfpathlineto{\pgfqpoint{9.431157in}{0.848603in}}%
\pgfpathlineto{\pgfqpoint{9.433491in}{1.180269in}}%
\pgfpathlineto{\pgfqpoint{9.440495in}{2.716154in}}%
\pgfpathlineto{\pgfqpoint{9.441662in}{2.686807in}}%
\pgfpathlineto{\pgfqpoint{9.443996in}{2.292175in}}%
\pgfpathlineto{\pgfqpoint{9.449832in}{0.884697in}}%
\pgfpathlineto{\pgfqpoint{9.450999in}{0.836372in}}%
\pgfpathlineto{\pgfqpoint{9.452166in}{0.913097in}}%
\pgfpathlineto{\pgfqpoint{9.454501in}{1.387328in}}%
\pgfpathlineto{\pgfqpoint{9.459170in}{2.594661in}}%
\pgfpathlineto{\pgfqpoint{9.460337in}{2.709467in}}%
\pgfpathlineto{\pgfqpoint{9.461504in}{2.697965in}}%
\pgfpathlineto{\pgfqpoint{9.463838in}{2.317386in}}%
\pgfpathlineto{\pgfqpoint{9.469674in}{0.881662in}}%
\pgfpathlineto{\pgfqpoint{9.470841in}{0.837279in}}%
\pgfpathlineto{\pgfqpoint{9.472009in}{0.923660in}}%
\pgfpathlineto{\pgfqpoint{9.474343in}{1.425860in}}%
\pgfpathlineto{\pgfqpoint{9.479012in}{2.629188in}}%
\pgfpathlineto{\pgfqpoint{9.480179in}{2.718395in}}%
\pgfpathlineto{\pgfqpoint{9.481346in}{2.674095in}}%
\pgfpathlineto{\pgfqpoint{9.483680in}{2.226384in}}%
\pgfpathlineto{\pgfqpoint{9.489516in}{0.847151in}}%
\pgfpathlineto{\pgfqpoint{9.490684in}{0.859596in}}%
\pgfpathlineto{\pgfqpoint{9.493018in}{1.264747in}}%
\pgfpathlineto{\pgfqpoint{9.498854in}{2.698421in}}%
\pgfpathlineto{\pgfqpoint{9.500021in}{2.706551in}}%
\pgfpathlineto{\pgfqpoint{9.502355in}{2.327410in}}%
\pgfpathlineto{\pgfqpoint{9.508191in}{0.862951in}}%
\pgfpathlineto{\pgfqpoint{9.509359in}{0.846151in}}%
\pgfpathlineto{\pgfqpoint{9.510526in}{0.970893in}}%
\pgfpathlineto{\pgfqpoint{9.514027in}{1.920879in}}%
\pgfpathlineto{\pgfqpoint{9.517529in}{2.694841in}}%
\pgfpathlineto{\pgfqpoint{9.518696in}{2.707890in}}%
\pgfpathlineto{\pgfqpoint{9.521030in}{2.320883in}}%
\pgfpathlineto{\pgfqpoint{9.526866in}{0.852585in}}%
\pgfpathlineto{\pgfqpoint{9.528034in}{0.856262in}}%
\pgfpathlineto{\pgfqpoint{9.530368in}{1.278849in}}%
\pgfpathlineto{\pgfqpoint{9.536204in}{2.714369in}}%
\pgfpathlineto{\pgfqpoint{9.537371in}{2.680477in}}%
\pgfpathlineto{\pgfqpoint{9.539705in}{2.202614in}}%
\pgfpathlineto{\pgfqpoint{9.544374in}{0.913242in}}%
\pgfpathlineto{\pgfqpoint{9.545541in}{0.835548in}}%
\pgfpathlineto{\pgfqpoint{9.546708in}{0.913312in}}%
\pgfpathlineto{\pgfqpoint{9.549043in}{1.462346in}}%
\pgfpathlineto{\pgfqpoint{9.553712in}{2.687424in}}%
\pgfpathlineto{\pgfqpoint{9.554879in}{2.710064in}}%
\pgfpathlineto{\pgfqpoint{9.556046in}{2.575548in}}%
\pgfpathlineto{\pgfqpoint{9.559548in}{1.558047in}}%
\pgfpathlineto{\pgfqpoint{9.561882in}{0.952155in}}%
\pgfpathlineto{\pgfqpoint{9.563049in}{0.839000in}}%
\pgfpathlineto{\pgfqpoint{9.564216in}{0.887030in}}%
\pgfpathlineto{\pgfqpoint{9.566551in}{1.409520in}}%
\pgfpathlineto{\pgfqpoint{9.571219in}{2.680609in}}%
\pgfpathlineto{\pgfqpoint{9.572387in}{2.712393in}}%
\pgfpathlineto{\pgfqpoint{9.573554in}{2.580196in}}%
\pgfpathlineto{\pgfqpoint{9.577055in}{1.543187in}}%
\pgfpathlineto{\pgfqpoint{9.579390in}{0.937471in}}%
\pgfpathlineto{\pgfqpoint{9.580557in}{0.836447in}}%
\pgfpathlineto{\pgfqpoint{9.581724in}{0.903678in}}%
\pgfpathlineto{\pgfqpoint{9.584058in}{1.469055in}}%
\pgfpathlineto{\pgfqpoint{9.588727in}{2.703541in}}%
\pgfpathlineto{\pgfqpoint{9.589894in}{2.693110in}}%
\pgfpathlineto{\pgfqpoint{9.592229in}{2.202482in}}%
\pgfpathlineto{\pgfqpoint{9.596897in}{0.879388in}}%
\pgfpathlineto{\pgfqpoint{9.598065in}{0.843284in}}%
\pgfpathlineto{\pgfqpoint{9.599232in}{0.981125in}}%
\pgfpathlineto{\pgfqpoint{9.602733in}{2.057424in}}%
\pgfpathlineto{\pgfqpoint{9.605068in}{2.648035in}}%
\pgfpathlineto{\pgfqpoint{9.606235in}{2.719444in}}%
\pgfpathlineto{\pgfqpoint{9.607402in}{2.612379in}}%
\pgfpathlineto{\pgfqpoint{9.609737in}{1.971871in}}%
\pgfpathlineto{\pgfqpoint{9.613238in}{0.930744in}}%
\pgfpathlineto{\pgfqpoint{9.614405in}{0.835601in}}%
\pgfpathlineto{\pgfqpoint{9.615572in}{0.922436in}}%
\pgfpathlineto{\pgfqpoint{9.617907in}{1.545276in}}%
\pgfpathlineto{\pgfqpoint{9.621408in}{2.612583in}}%
\pgfpathlineto{\pgfqpoint{9.622576in}{2.719756in}}%
\pgfpathlineto{\pgfqpoint{9.623743in}{2.641645in}}%
\pgfpathlineto{\pgfqpoint{9.626077in}{2.022192in}}%
\pgfpathlineto{\pgfqpoint{9.629579in}{0.943295in}}%
\pgfpathlineto{\pgfqpoint{9.630746in}{0.836019in}}%
\pgfpathlineto{\pgfqpoint{9.631913in}{0.917549in}}%
\pgfpathlineto{\pgfqpoint{9.634247in}{1.549385in}}%
\pgfpathlineto{\pgfqpoint{9.637749in}{2.625388in}}%
\pgfpathlineto{\pgfqpoint{9.638916in}{2.720362in}}%
\pgfpathlineto{\pgfqpoint{9.640083in}{2.622861in}}%
\pgfpathlineto{\pgfqpoint{9.642418in}{1.963046in}}%
\pgfpathlineto{\pgfqpoint{9.645919in}{0.907220in}}%
\pgfpathlineto{\pgfqpoint{9.647086in}{0.837686in}}%
\pgfpathlineto{\pgfqpoint{9.648254in}{0.963843in}}%
\pgfpathlineto{\pgfqpoint{9.651755in}{2.094396in}}%
\pgfpathlineto{\pgfqpoint{9.654090in}{2.677805in}}%
\pgfpathlineto{\pgfqpoint{9.655257in}{2.707934in}}%
\pgfpathlineto{\pgfqpoint{9.656424in}{2.540823in}}%
\pgfpathlineto{\pgfqpoint{9.662260in}{0.850776in}}%
\pgfpathlineto{\pgfqpoint{9.663427in}{0.874550in}}%
\pgfpathlineto{\pgfqpoint{9.665761in}{1.461227in}}%
\pgfpathlineto{\pgfqpoint{9.669263in}{2.605106in}}%
\pgfpathlineto{\pgfqpoint{9.670430in}{2.720017in}}%
\pgfpathlineto{\pgfqpoint{9.671597in}{2.628192in}}%
\pgfpathlineto{\pgfqpoint{9.673932in}{1.943460in}}%
\pgfpathlineto{\pgfqpoint{9.677433in}{0.883405in}}%
\pgfpathlineto{\pgfqpoint{9.678600in}{0.847139in}}%
\pgfpathlineto{\pgfqpoint{9.679768in}{1.019080in}}%
\pgfpathlineto{\pgfqpoint{9.685604in}{2.714807in}}%
\pgfpathlineto{\pgfqpoint{9.686771in}{2.655586in}}%
\pgfpathlineto{\pgfqpoint{9.689105in}{1.995279in}}%
\pgfpathlineto{\pgfqpoint{9.692607in}{0.892953in}}%
\pgfpathlineto{\pgfqpoint{9.693774in}{0.844085in}}%
\pgfpathlineto{\pgfqpoint{9.694941in}{1.011442in}}%
\pgfpathlineto{\pgfqpoint{9.700777in}{2.717366in}}%
\pgfpathlineto{\pgfqpoint{9.701944in}{2.642125in}}%
\pgfpathlineto{\pgfqpoint{9.704279in}{1.944923in}}%
\pgfpathlineto{\pgfqpoint{9.707780in}{0.869793in}}%
\pgfpathlineto{\pgfqpoint{9.708947in}{0.859043in}}%
\pgfpathlineto{\pgfqpoint{9.711282in}{1.448903in}}%
\pgfpathlineto{\pgfqpoint{9.714783in}{2.631845in}}%
\pgfpathlineto{\pgfqpoint{9.715950in}{2.718595in}}%
\pgfpathlineto{\pgfqpoint{9.717118in}{2.576319in}}%
\pgfpathlineto{\pgfqpoint{9.722954in}{0.838326in}}%
\pgfpathlineto{\pgfqpoint{9.724121in}{0.919837in}}%
\pgfpathlineto{\pgfqpoint{9.726455in}{1.649561in}}%
\pgfpathlineto{\pgfqpoint{9.729957in}{2.703340in}}%
\pgfpathlineto{\pgfqpoint{9.731124in}{2.673499in}}%
\pgfpathlineto{\pgfqpoint{9.733458in}{2.000044in}}%
\pgfpathlineto{\pgfqpoint{9.736960in}{0.872044in}}%
\pgfpathlineto{\pgfqpoint{9.738127in}{0.860797in}}%
\pgfpathlineto{\pgfqpoint{9.740461in}{1.486621in}}%
\pgfpathlineto{\pgfqpoint{9.743963in}{2.665437in}}%
\pgfpathlineto{\pgfqpoint{9.745130in}{2.706414in}}%
\pgfpathlineto{\pgfqpoint{9.747464in}{2.114909in}}%
\pgfpathlineto{\pgfqpoint{9.750966in}{0.903425in}}%
\pgfpathlineto{\pgfqpoint{9.752133in}{0.844474in}}%
\pgfpathlineto{\pgfqpoint{9.753300in}{1.032335in}}%
\pgfpathlineto{\pgfqpoint{9.759136in}{2.712585in}}%
\pgfpathlineto{\pgfqpoint{9.760303in}{2.526396in}}%
\pgfpathlineto{\pgfqpoint{9.766139in}{0.845441in}}%
\pgfpathlineto{\pgfqpoint{9.767307in}{1.041544in}}%
\pgfpathlineto{\pgfqpoint{9.773142in}{2.703890in}}%
\pgfpathlineto{\pgfqpoint{9.775477in}{2.071263in}}%
\pgfpathlineto{\pgfqpoint{9.778978in}{0.872583in}}%
\pgfpathlineto{\pgfqpoint{9.780146in}{0.866144in}}%
\pgfpathlineto{\pgfqpoint{9.782480in}{1.554382in}}%
\pgfpathlineto{\pgfqpoint{9.785982in}{2.703194in}}%
\pgfpathlineto{\pgfqpoint{9.787149in}{2.663235in}}%
\pgfpathlineto{\pgfqpoint{9.789483in}{1.905973in}}%
\pgfpathlineto{\pgfqpoint{9.792985in}{0.838176in}}%
\pgfpathlineto{\pgfqpoint{9.794152in}{0.938117in}}%
\pgfpathlineto{\pgfqpoint{9.799988in}{2.718161in}}%
\pgfpathlineto{\pgfqpoint{9.801155in}{2.545806in}}%
\pgfpathlineto{\pgfqpoint{9.806991in}{0.862191in}}%
\pgfpathlineto{\pgfqpoint{9.809325in}{1.569905in}}%
\pgfpathlineto{\pgfqpoint{9.812827in}{2.713249in}}%
\pgfpathlineto{\pgfqpoint{9.813994in}{2.632461in}}%
\pgfpathlineto{\pgfqpoint{9.817496in}{1.290151in}}%
\pgfpathlineto{\pgfqpoint{9.819830in}{0.839954in}}%
\pgfpathlineto{\pgfqpoint{9.820997in}{1.033243in}}%
\pgfpathlineto{\pgfqpoint{9.825666in}{2.697073in}}%
\pgfpathlineto{\pgfqpoint{9.826833in}{2.665180in}}%
\pgfpathlineto{\pgfqpoint{9.829167in}{1.863739in}}%
\pgfpathlineto{\pgfqpoint{9.831502in}{0.961587in}}%
\pgfpathlineto{\pgfqpoint{9.832669in}{0.836608in}}%
\pgfpathlineto{\pgfqpoint{9.833836in}{1.010281in}}%
\pgfpathlineto{\pgfqpoint{9.838505in}{2.695615in}}%
\pgfpathlineto{\pgfqpoint{9.839672in}{2.664663in}}%
\pgfpathlineto{\pgfqpoint{9.842006in}{1.846182in}}%
\pgfpathlineto{\pgfqpoint{9.844341in}{0.945254in}}%
\pgfpathlineto{\pgfqpoint{9.845508in}{0.839518in}}%
\pgfpathlineto{\pgfqpoint{9.846675in}{1.040299in}}%
\pgfpathlineto{\pgfqpoint{9.851344in}{2.710846in}}%
\pgfpathlineto{\pgfqpoint{9.852511in}{2.629952in}}%
\pgfpathlineto{\pgfqpoint{9.858347in}{0.860645in}}%
\pgfpathlineto{\pgfqpoint{9.860681in}{1.626501in}}%
\pgfpathlineto{\pgfqpoint{9.863016in}{2.580929in}}%
\pgfpathlineto{\pgfqpoint{9.864183in}{2.719577in}}%
\pgfpathlineto{\pgfqpoint{9.865350in}{2.537461in}}%
\pgfpathlineto{\pgfqpoint{9.870019in}{0.843424in}}%
\pgfpathlineto{\pgfqpoint{9.871186in}{0.936589in}}%
\pgfpathlineto{\pgfqpoint{9.875855in}{2.683806in}}%
\pgfpathlineto{\pgfqpoint{9.877022in}{2.672143in}}%
\pgfpathlineto{\pgfqpoint{9.879356in}{1.820900in}}%
\pgfpathlineto{\pgfqpoint{9.881691in}{0.913347in}}%
\pgfpathlineto{\pgfqpoint{9.882858in}{0.853527in}}%
\pgfpathlineto{\pgfqpoint{9.885192in}{1.624993in}}%
\pgfpathlineto{\pgfqpoint{9.887527in}{2.596537in}}%
\pgfpathlineto{\pgfqpoint{9.888694in}{2.716507in}}%
\pgfpathlineto{\pgfqpoint{9.889861in}{2.498573in}}%
\pgfpathlineto{\pgfqpoint{9.894530in}{0.835565in}}%
\pgfpathlineto{\pgfqpoint{9.895697in}{1.010939in}}%
\pgfpathlineto{\pgfqpoint{9.900366in}{2.718803in}}%
\pgfpathlineto{\pgfqpoint{9.901533in}{2.576171in}}%
\pgfpathlineto{\pgfqpoint{9.906202in}{0.840287in}}%
\pgfpathlineto{\pgfqpoint{9.907369in}{0.961043in}}%
\pgfpathlineto{\pgfqpoint{9.912038in}{2.713365in}}%
\pgfpathlineto{\pgfqpoint{9.913205in}{2.602924in}}%
\pgfpathlineto{\pgfqpoint{9.917873in}{0.842595in}}%
\pgfpathlineto{\pgfqpoint{9.919041in}{0.954750in}}%
\pgfpathlineto{\pgfqpoint{9.923709in}{2.715524in}}%
\pgfpathlineto{\pgfqpoint{9.924877in}{2.589317in}}%
\pgfpathlineto{\pgfqpoint{9.929545in}{0.837184in}}%
\pgfpathlineto{\pgfqpoint{9.930713in}{0.989912in}}%
\pgfpathlineto{\pgfqpoint{9.935381in}{2.720353in}}%
\pgfpathlineto{\pgfqpoint{9.936548in}{2.528784in}}%
\pgfpathlineto{\pgfqpoint{9.941217in}{0.839461in}}%
\pgfpathlineto{\pgfqpoint{9.942384in}{1.081570in}}%
\pgfpathlineto{\pgfqpoint{9.947053in}{2.701755in}}%
\pgfpathlineto{\pgfqpoint{9.949388in}{1.839873in}}%
\pgfpathlineto{\pgfqpoint{9.951722in}{0.885490in}}%
\pgfpathlineto{\pgfqpoint{9.952889in}{0.886119in}}%
\pgfpathlineto{\pgfqpoint{9.955223in}{1.846838in}}%
\pgfpathlineto{\pgfqpoint{9.957558in}{2.705274in}}%
\pgfpathlineto{\pgfqpoint{9.958725in}{2.613275in}}%
\pgfpathlineto{\pgfqpoint{9.963394in}{0.835586in}}%
\pgfpathlineto{\pgfqpoint{9.964561in}{1.031184in}}%
\pgfpathlineto{\pgfqpoint{9.969230in}{2.704506in}}%
\pgfpathlineto{\pgfqpoint{9.971564in}{1.822241in}}%
\pgfpathlineto{\pgfqpoint{9.973898in}{0.870727in}}%
\pgfpathlineto{\pgfqpoint{9.975066in}{0.909621in}}%
\pgfpathlineto{\pgfqpoint{9.979734in}{2.719596in}}%
\pgfpathlineto{\pgfqpoint{9.980902in}{2.535331in}}%
\pgfpathlineto{\pgfqpoint{9.985570in}{0.857490in}}%
\pgfpathlineto{\pgfqpoint{9.987905in}{1.788483in}}%
\pgfpathlineto{\pgfqpoint{9.990239in}{2.703377in}}%
\pgfpathlineto{\pgfqpoint{9.991406in}{2.606815in}}%
\pgfpathlineto{\pgfqpoint{9.996075in}{0.841059in}}%
\pgfpathlineto{\pgfqpoint{9.997242in}{1.120696in}}%
\pgfpathlineto{\pgfqpoint{10.000744in}{2.687572in}}%
\pgfpathlineto{\pgfqpoint{10.001911in}{2.634586in}}%
\pgfpathlineto{\pgfqpoint{10.006580in}{0.838270in}}%
\pgfpathlineto{\pgfqpoint{10.007747in}{1.104879in}}%
\pgfpathlineto{\pgfqpoint{10.011248in}{2.688019in}}%
\pgfpathlineto{\pgfqpoint{10.012416in}{2.630473in}}%
\pgfpathlineto{\pgfqpoint{10.017084in}{0.842394in}}%
\pgfpathlineto{\pgfqpoint{10.018251in}{1.140886in}}%
\pgfpathlineto{\pgfqpoint{10.021753in}{2.704499in}}%
\pgfpathlineto{\pgfqpoint{10.022920in}{2.592045in}}%
\pgfpathlineto{\pgfqpoint{10.027589in}{0.863212in}}%
\pgfpathlineto{\pgfqpoint{10.029923in}{1.877839in}}%
\pgfpathlineto{\pgfqpoint{10.032258in}{2.720031in}}%
\pgfpathlineto{\pgfqpoint{10.033425in}{2.502486in}}%
\pgfpathlineto{\pgfqpoint{10.036926in}{0.872388in}}%
\pgfpathlineto{\pgfqpoint{10.038094in}{0.926859in}}%
\pgfpathlineto{\pgfqpoint{10.042762in}{2.699743in}}%
\pgfpathlineto{\pgfqpoint{10.045097in}{1.685390in}}%
\pgfpathlineto{\pgfqpoint{10.047431in}{0.835585in}}%
\pgfpathlineto{\pgfqpoint{10.048598in}{1.072196in}}%
\pgfpathlineto{\pgfqpoint{10.052100in}{2.699386in}}%
\pgfpathlineto{\pgfqpoint{10.053267in}{2.593041in}}%
\pgfpathlineto{\pgfqpoint{10.057936in}{0.890542in}}%
\pgfpathlineto{\pgfqpoint{10.062605in}{2.705665in}}%
\pgfpathlineto{\pgfqpoint{10.064939in}{1.684159in}}%
\pgfpathlineto{\pgfqpoint{10.067273in}{0.835881in}}%
\pgfpathlineto{\pgfqpoint{10.068440in}{1.105973in}}%
\pgfpathlineto{\pgfqpoint{10.071942in}{2.714935in}}%
\pgfpathlineto{\pgfqpoint{10.073109in}{2.529878in}}%
\pgfpathlineto{\pgfqpoint{10.076611in}{0.859313in}}%
\pgfpathlineto{\pgfqpoint{10.077778in}{0.966514in}}%
\pgfpathlineto{\pgfqpoint{10.081279in}{2.670657in}}%
\pgfpathlineto{\pgfqpoint{10.082447in}{2.632027in}}%
\pgfpathlineto{\pgfqpoint{10.087115in}{0.894616in}}%
\pgfpathlineto{\pgfqpoint{10.091784in}{2.680453in}}%
\pgfpathlineto{\pgfqpoint{10.096453in}{0.863681in}}%
\pgfpathlineto{\pgfqpoint{10.098787in}{1.990700in}}%
\pgfpathlineto{\pgfqpoint{10.101122in}{2.698733in}}%
\pgfpathlineto{\pgfqpoint{10.103456in}{1.585308in}}%
\pgfpathlineto{\pgfqpoint{10.105790in}{0.854580in}}%
\pgfpathlineto{\pgfqpoint{10.108125in}{1.966115in}}%
\pgfpathlineto{\pgfqpoint{10.110459in}{2.700639in}}%
\pgfpathlineto{\pgfqpoint{10.112793in}{1.578551in}}%
\pgfpathlineto{\pgfqpoint{10.115128in}{0.859455in}}%
\pgfpathlineto{\pgfqpoint{10.117462in}{2.004306in}}%
\pgfpathlineto{\pgfqpoint{10.119797in}{2.687791in}}%
\pgfpathlineto{\pgfqpoint{10.124465in}{0.883077in}}%
\pgfpathlineto{\pgfqpoint{10.129134in}{2.649028in}}%
\pgfpathlineto{\pgfqpoint{10.132636in}{0.885745in}}%
\pgfpathlineto{\pgfqpoint{10.133803in}{0.942794in}}%
\pgfpathlineto{\pgfqpoint{10.137304in}{2.696705in}}%
\pgfpathlineto{\pgfqpoint{10.138472in}{2.561345in}}%
\pgfpathlineto{\pgfqpoint{10.141973in}{0.840601in}}%
\pgfpathlineto{\pgfqpoint{10.143140in}{1.066228in}}%
\pgfpathlineto{\pgfqpoint{10.146642in}{2.719785in}}%
\pgfpathlineto{\pgfqpoint{10.147809in}{2.394281in}}%
\pgfpathlineto{\pgfqpoint{10.151311in}{0.852799in}}%
\pgfpathlineto{\pgfqpoint{10.153645in}{2.035339in}}%
\pgfpathlineto{\pgfqpoint{10.155979in}{2.657908in}}%
\pgfpathlineto{\pgfqpoint{10.159481in}{0.871368in}}%
\pgfpathlineto{\pgfqpoint{10.160648in}{0.978810in}}%
\pgfpathlineto{\pgfqpoint{10.164150in}{2.718025in}}%
\pgfpathlineto{\pgfqpoint{10.165317in}{2.454995in}}%
\pgfpathlineto{\pgfqpoint{10.168818in}{0.846676in}}%
\pgfpathlineto{\pgfqpoint{10.171153in}{2.031911in}}%
\pgfpathlineto{\pgfqpoint{10.172320in}{2.626003in}}%
\pgfpathlineto{\pgfqpoint{10.173487in}{2.647269in}}%
\pgfpathlineto{\pgfqpoint{10.176989in}{0.853391in}}%
\pgfpathlineto{\pgfqpoint{10.178156in}{1.032010in}}%
\pgfpathlineto{\pgfqpoint{10.181657in}{2.717351in}}%
\pgfpathlineto{\pgfqpoint{10.182825in}{2.336040in}}%
\pgfpathlineto{\pgfqpoint{10.185159in}{0.945125in}}%
\pgfpathlineto{\pgfqpoint{10.186326in}{0.900517in}}%
\pgfpathlineto{\pgfqpoint{10.189828in}{2.705217in}}%
\pgfpathlineto{\pgfqpoint{10.190995in}{2.507384in}}%
\pgfpathlineto{\pgfqpoint{10.194496in}{0.845672in}}%
\pgfpathlineto{\pgfqpoint{10.196831in}{2.071633in}}%
\pgfpathlineto{\pgfqpoint{10.197998in}{2.652568in}}%
\pgfpathlineto{\pgfqpoint{10.199165in}{2.608666in}}%
\pgfpathlineto{\pgfqpoint{10.202667in}{0.835782in}}%
\pgfpathlineto{\pgfqpoint{10.203834in}{1.160170in}}%
\pgfpathlineto{\pgfqpoint{10.206168in}{2.593108in}}%
\pgfpathlineto{\pgfqpoint{10.207336in}{2.661253in}}%
\pgfpathlineto{\pgfqpoint{10.210837in}{0.843870in}}%
\pgfpathlineto{\pgfqpoint{10.212004in}{1.090711in}}%
\pgfpathlineto{\pgfqpoint{10.215506in}{2.683938in}}%
\pgfpathlineto{\pgfqpoint{10.219007in}{0.851597in}}%
\pgfpathlineto{\pgfqpoint{10.220175in}{1.063402in}}%
\pgfpathlineto{\pgfqpoint{10.223676in}{2.688444in}}%
\pgfpathlineto{\pgfqpoint{10.227178in}{0.850306in}}%
\pgfpathlineto{\pgfqpoint{10.228345in}{1.073418in}}%
\pgfpathlineto{\pgfqpoint{10.231846in}{2.677478in}}%
\pgfpathlineto{\pgfqpoint{10.235348in}{0.841201in}}%
\pgfpathlineto{\pgfqpoint{10.236515in}{1.123134in}}%
\pgfpathlineto{\pgfqpoint{10.238850in}{2.600024in}}%
\pgfpathlineto{\pgfqpoint{10.240017in}{2.644125in}}%
\pgfpathlineto{\pgfqpoint{10.243518in}{0.835711in}}%
\pgfpathlineto{\pgfqpoint{10.244685in}{1.221747in}}%
\pgfpathlineto{\pgfqpoint{10.247020in}{2.661086in}}%
\pgfpathlineto{\pgfqpoint{10.248187in}{2.572352in}}%
\pgfpathlineto{\pgfqpoint{10.251689in}{0.855849in}}%
\pgfpathlineto{\pgfqpoint{10.255190in}{2.711033in}}%
\pgfpathlineto{\pgfqpoint{10.256357in}{2.439514in}}%
\pgfpathlineto{\pgfqpoint{10.258692in}{0.941285in}}%
\pgfpathlineto{\pgfqpoint{10.259859in}{0.933098in}}%
\pgfpathlineto{\pgfqpoint{10.263360in}{2.711552in}}%
\pgfpathlineto{\pgfqpoint{10.266862in}{0.849744in}}%
\pgfpathlineto{\pgfqpoint{10.268029in}{1.102980in}}%
\pgfpathlineto{\pgfqpoint{10.270364in}{2.616935in}}%
\pgfpathlineto{\pgfqpoint{10.271531in}{2.615101in}}%
\pgfpathlineto{\pgfqpoint{10.275032in}{0.853137in}}%
\pgfpathlineto{\pgfqpoint{10.278534in}{2.717463in}}%
\pgfpathlineto{\pgfqpoint{10.279701in}{2.377958in}}%
\pgfpathlineto{\pgfqpoint{10.282035in}{0.889122in}}%
\pgfpathlineto{\pgfqpoint{10.283203in}{1.013274in}}%
\pgfpathlineto{\pgfqpoint{10.286704in}{2.653280in}}%
\pgfpathlineto{\pgfqpoint{10.290206in}{0.845680in}}%
\pgfpathlineto{\pgfqpoint{10.293707in}{2.717773in}}%
\pgfpathlineto{\pgfqpoint{10.294874in}{2.362296in}}%
\pgfpathlineto{\pgfqpoint{10.297209in}{0.873102in}}%
\pgfpathlineto{\pgfqpoint{10.298376in}{1.055291in}}%
\pgfpathlineto{\pgfqpoint{10.300710in}{2.613712in}}%
\pgfpathlineto{\pgfqpoint{10.301878in}{2.603275in}}%
\pgfpathlineto{\pgfqpoint{10.305379in}{0.884196in}}%
\pgfpathlineto{\pgfqpoint{10.308881in}{2.709257in}}%
\pgfpathlineto{\pgfqpoint{10.312382in}{0.835553in}}%
\pgfpathlineto{\pgfqpoint{10.313549in}{1.264328in}}%
\pgfpathlineto{\pgfqpoint{10.315884in}{2.710071in}}%
\pgfpathlineto{\pgfqpoint{10.317051in}{2.397293in}}%
\pgfpathlineto{\pgfqpoint{10.319385in}{0.872371in}}%
\pgfpathlineto{\pgfqpoint{10.320553in}{1.072742in}}%
\pgfpathlineto{\pgfqpoint{10.322887in}{2.645710in}}%
\pgfpathlineto{\pgfqpoint{10.324054in}{2.550868in}}%
\pgfpathlineto{\pgfqpoint{10.326388in}{0.954854in}}%
\pgfpathlineto{\pgfqpoint{10.327556in}{0.953329in}}%
\pgfpathlineto{\pgfqpoint{10.331057in}{2.640794in}}%
\pgfpathlineto{\pgfqpoint{10.334559in}{0.887900in}}%
\pgfpathlineto{\pgfqpoint{10.338060in}{2.686631in}}%
\pgfpathlineto{\pgfqpoint{10.341562in}{0.857124in}}%
\pgfpathlineto{\pgfqpoint{10.345063in}{2.706298in}}%
\pgfpathlineto{\pgfqpoint{10.348565in}{0.845336in}}%
\pgfpathlineto{\pgfqpoint{10.352067in}{2.712633in}}%
\pgfpathlineto{\pgfqpoint{10.355568in}{0.842885in}}%
\pgfpathlineto{\pgfqpoint{10.359070in}{2.711874in}}%
\pgfpathlineto{\pgfqpoint{10.362571in}{0.847088in}}%
\pgfpathlineto{\pgfqpoint{10.366073in}{2.703071in}}%
\pgfpathlineto{\pgfqpoint{10.369574in}{0.862569in}}%
\pgfpathlineto{\pgfqpoint{10.373076in}{2.677936in}}%
\pgfpathlineto{\pgfqpoint{10.376577in}{0.901190in}}%
\pgfpathlineto{\pgfqpoint{10.380079in}{2.621239in}}%
\pgfpathlineto{\pgfqpoint{10.382413in}{0.959595in}}%
\pgfpathlineto{\pgfqpoint{10.383581in}{0.981142in}}%
\pgfpathlineto{\pgfqpoint{10.385915in}{2.642417in}}%
\pgfpathlineto{\pgfqpoint{10.387082in}{2.512525in}}%
\pgfpathlineto{\pgfqpoint{10.389416in}{0.874100in}}%
\pgfpathlineto{\pgfqpoint{10.390584in}{1.124073in}}%
\pgfpathlineto{\pgfqpoint{10.392918in}{2.709620in}}%
\pgfpathlineto{\pgfqpoint{10.394085in}{2.330503in}}%
\pgfpathlineto{\pgfqpoint{10.396420in}{0.835550in}}%
\pgfpathlineto{\pgfqpoint{10.399921in}{2.708221in}}%
\pgfpathlineto{\pgfqpoint{10.403423in}{0.888503in}}%
\pgfpathlineto{\pgfqpoint{10.406924in}{2.592755in}}%
\pgfpathlineto{\pgfqpoint{10.409259in}{0.905162in}}%
\pgfpathlineto{\pgfqpoint{10.410426in}{1.075495in}}%
\pgfpathlineto{\pgfqpoint{10.412760in}{2.706059in}}%
\pgfpathlineto{\pgfqpoint{10.413927in}{2.328796in}}%
\pgfpathlineto{\pgfqpoint{10.416262in}{0.836798in}}%
\pgfpathlineto{\pgfqpoint{10.419763in}{2.680688in}}%
\pgfpathlineto{\pgfqpoint{10.423265in}{0.971927in}}%
\pgfpathlineto{\pgfqpoint{10.425599in}{2.670782in}}%
\pgfpathlineto{\pgfqpoint{10.426766in}{2.426366in}}%
\pgfpathlineto{\pgfqpoint{10.429101in}{0.836690in}}%
\pgfpathlineto{\pgfqpoint{10.430268in}{1.344820in}}%
\pgfpathlineto{\pgfqpoint{10.432602in}{2.693067in}}%
\pgfpathlineto{\pgfqpoint{10.436104in}{0.972463in}}%
\pgfpathlineto{\pgfqpoint{10.438438in}{2.680853in}}%
\pgfpathlineto{\pgfqpoint{10.439605in}{2.389707in}}%
\pgfpathlineto{\pgfqpoint{10.441940in}{0.836206in}}%
\pgfpathlineto{\pgfqpoint{10.445441in}{2.653327in}}%
\pgfpathlineto{\pgfqpoint{10.447776in}{0.922041in}}%
\pgfpathlineto{\pgfqpoint{10.448943in}{1.079332in}}%
\pgfpathlineto{\pgfqpoint{10.451277in}{2.718315in}}%
\pgfpathlineto{\pgfqpoint{10.452444in}{2.200107in}}%
\pgfpathlineto{\pgfqpoint{10.454779in}{0.881662in}}%
\pgfpathlineto{\pgfqpoint{10.457113in}{2.617451in}}%
\pgfpathlineto{\pgfqpoint{10.458280in}{2.493514in}}%
\pgfpathlineto{\pgfqpoint{10.460615in}{0.837703in}}%
\pgfpathlineto{\pgfqpoint{10.461782in}{1.365209in}}%
\pgfpathlineto{\pgfqpoint{10.464116in}{2.662946in}}%
\pgfpathlineto{\pgfqpoint{10.466451in}{0.913372in}}%
\pgfpathlineto{\pgfqpoint{10.467618in}{1.110581in}}%
\pgfpathlineto{\pgfqpoint{10.469952in}{2.719837in}}%
\pgfpathlineto{\pgfqpoint{10.473454in}{0.946811in}}%
\pgfpathlineto{\pgfqpoint{10.475788in}{2.692608in}}%
\pgfpathlineto{\pgfqpoint{10.476955in}{2.316236in}}%
\pgfpathlineto{\pgfqpoint{10.479290in}{0.862230in}}%
\pgfpathlineto{\pgfqpoint{10.481624in}{2.614197in}}%
\pgfpathlineto{\pgfqpoint{10.482791in}{2.476981in}}%
\pgfpathlineto{\pgfqpoint{10.485126in}{0.835898in}}%
\pgfpathlineto{\pgfqpoint{10.488627in}{2.582071in}}%
\pgfpathlineto{\pgfqpoint{10.490962in}{0.845131in}}%
\pgfpathlineto{\pgfqpoint{10.492129in}{1.337848in}}%
\pgfpathlineto{\pgfqpoint{10.494463in}{2.644663in}}%
\pgfpathlineto{\pgfqpoint{10.496798in}{0.869970in}}%
\pgfpathlineto{\pgfqpoint{10.497965in}{1.241232in}}%
\pgfpathlineto{\pgfqpoint{10.500299in}{2.678200in}}%
\pgfpathlineto{\pgfqpoint{10.502633in}{0.895169in}}%
\pgfpathlineto{\pgfqpoint{10.503801in}{1.182434in}}%
\pgfpathlineto{\pgfqpoint{10.506135in}{2.693573in}}%
\pgfpathlineto{\pgfqpoint{10.508469in}{0.910660in}}%
\pgfpathlineto{\pgfqpoint{10.509637in}{1.156618in}}%
\pgfpathlineto{\pgfqpoint{10.511971in}{2.697628in}}%
\pgfpathlineto{\pgfqpoint{10.514305in}{0.911361in}}%
\pgfpathlineto{\pgfqpoint{10.515473in}{1.161327in}}%
\pgfpathlineto{\pgfqpoint{10.517807in}{2.692492in}}%
\pgfpathlineto{\pgfqpoint{10.520141in}{0.896964in}}%
\pgfpathlineto{\pgfqpoint{10.521308in}{1.197285in}}%
\pgfpathlineto{\pgfqpoint{10.523643in}{2.675318in}}%
\pgfpathlineto{\pgfqpoint{10.525977in}{0.872029in}}%
\pgfpathlineto{\pgfqpoint{10.527144in}{1.268333in}}%
\pgfpathlineto{\pgfqpoint{10.529479in}{2.638360in}}%
\pgfpathlineto{\pgfqpoint{10.531813in}{0.846409in}}%
\pgfpathlineto{\pgfqpoint{10.532980in}{1.380417in}}%
\pgfpathlineto{\pgfqpoint{10.535315in}{2.569548in}}%
\pgfpathlineto{\pgfqpoint{10.537649in}{0.835715in}}%
\pgfpathlineto{\pgfqpoint{10.539983in}{2.591601in}}%
\pgfpathlineto{\pgfqpoint{10.541151in}{2.454073in}}%
\pgfpathlineto{\pgfqpoint{10.543485in}{0.861130in}}%
\pgfpathlineto{\pgfqpoint{10.545819in}{2.679233in}}%
\pgfpathlineto{\pgfqpoint{10.546987in}{2.277685in}}%
\pgfpathlineto{\pgfqpoint{10.549321in}{0.947529in}}%
\pgfpathlineto{\pgfqpoint{10.551655in}{2.720180in}}%
\pgfpathlineto{\pgfqpoint{10.553990in}{0.962476in}}%
\pgfpathlineto{\pgfqpoint{10.555157in}{1.118595in}}%
\pgfpathlineto{\pgfqpoint{10.557491in}{2.677866in}}%
\pgfpathlineto{\pgfqpoint{10.559826in}{0.851762in}}%
\pgfpathlineto{\pgfqpoint{10.560993in}{1.387943in}}%
\pgfpathlineto{\pgfqpoint{10.562160in}{2.519098in}}%
\pgfpathlineto{\pgfqpoint{10.563327in}{2.517974in}}%
\pgfpathlineto{\pgfqpoint{10.565661in}{0.854062in}}%
\pgfpathlineto{\pgfqpoint{10.567996in}{2.687871in}}%
\pgfpathlineto{\pgfqpoint{10.569163in}{2.222803in}}%
\pgfpathlineto{\pgfqpoint{10.570330in}{1.068206in}}%
\pgfpathlineto{\pgfqpoint{10.571497in}{1.015476in}}%
\pgfpathlineto{\pgfqpoint{10.573832in}{2.704364in}}%
\pgfpathlineto{\pgfqpoint{10.576166in}{0.865269in}}%
\pgfpathlineto{\pgfqpoint{10.577333in}{1.353478in}}%
\pgfpathlineto{\pgfqpoint{10.578501in}{2.510264in}}%
\pgfpathlineto{\pgfqpoint{10.579668in}{2.511791in}}%
\pgfpathlineto{\pgfqpoint{10.582002in}{0.867293in}}%
\pgfpathlineto{\pgfqpoint{10.584336in}{2.709747in}}%
\pgfpathlineto{\pgfqpoint{10.586671in}{0.976204in}}%
\pgfpathlineto{\pgfqpoint{10.587838in}{1.134082in}}%
\pgfpathlineto{\pgfqpoint{10.590172in}{2.635880in}}%
\pgfpathlineto{\pgfqpoint{10.592507in}{0.836385in}}%
\pgfpathlineto{\pgfqpoint{10.594841in}{2.668024in}}%
\pgfpathlineto{\pgfqpoint{10.596008in}{2.246664in}}%
\pgfpathlineto{\pgfqpoint{10.597176in}{1.059812in}}%
\pgfpathlineto{\pgfqpoint{10.598343in}{1.047847in}}%
\pgfpathlineto{\pgfqpoint{10.600677in}{2.670636in}}%
\pgfpathlineto{\pgfqpoint{10.603011in}{0.835827in}}%
\pgfpathlineto{\pgfqpoint{10.605346in}{2.657475in}}%
\pgfpathlineto{\pgfqpoint{10.606513in}{2.259310in}}%
\pgfpathlineto{\pgfqpoint{10.607680in}{1.058901in}}%
\pgfpathlineto{\pgfqpoint{10.608847in}{1.059025in}}%
\pgfpathlineto{\pgfqpoint{10.611182in}{2.653474in}}%
\pgfpathlineto{\pgfqpoint{10.613516in}{0.837398in}}%
\pgfpathlineto{\pgfqpoint{10.615850in}{2.691987in}}%
\pgfpathlineto{\pgfqpoint{10.617018in}{2.142772in}}%
\pgfpathlineto{\pgfqpoint{10.618185in}{0.973072in}}%
\pgfpathlineto{\pgfqpoint{10.619352in}{1.174710in}}%
\pgfpathlineto{\pgfqpoint{10.621686in}{2.562754in}}%
\pgfpathlineto{\pgfqpoint{10.624021in}{0.879057in}}%
\pgfpathlineto{\pgfqpoint{10.626355in}{2.719909in}}%
\pgfpathlineto{\pgfqpoint{10.628690in}{0.861447in}}%
\pgfpathlineto{\pgfqpoint{10.629857in}{1.441463in}}%
\pgfpathlineto{\pgfqpoint{10.631024in}{2.608010in}}%
\pgfpathlineto{\pgfqpoint{10.632191in}{2.325158in}}%
\pgfpathlineto{\pgfqpoint{10.633358in}{1.084422in}}%
\pgfpathlineto{\pgfqpoint{10.634525in}{1.060038in}}%
\pgfpathlineto{\pgfqpoint{10.636860in}{2.619868in}}%
\pgfpathlineto{\pgfqpoint{10.639194in}{0.861498in}}%
\pgfpathlineto{\pgfqpoint{10.641529in}{2.720294in}}%
\pgfpathlineto{\pgfqpoint{10.643863in}{0.854837in}}%
\pgfpathlineto{\pgfqpoint{10.646197in}{2.642372in}}%
\pgfpathlineto{\pgfqpoint{10.647364in}{2.238208in}}%
\pgfpathlineto{\pgfqpoint{10.648532in}{1.004845in}}%
\pgfpathlineto{\pgfqpoint{10.649699in}{1.167031in}}%
\pgfpathlineto{\pgfqpoint{10.652033in}{2.511675in}}%
\pgfpathlineto{\pgfqpoint{10.654368in}{0.950641in}}%
\pgfpathlineto{\pgfqpoint{10.656702in}{2.670003in}}%
\pgfpathlineto{\pgfqpoint{10.659036in}{0.848420in}}%
\pgfpathlineto{\pgfqpoint{10.661371in}{2.720312in}}%
\pgfpathlineto{\pgfqpoint{10.663705in}{0.844262in}}%
\pgfpathlineto{\pgfqpoint{10.666039in}{2.685036in}}%
\pgfpathlineto{\pgfqpoint{10.667207in}{2.093057in}}%
\pgfpathlineto{\pgfqpoint{10.668374in}{0.911831in}}%
\pgfpathlineto{\pgfqpoint{10.669541in}{1.353447in}}%
\pgfpathlineto{\pgfqpoint{10.670708in}{2.592321in}}%
\pgfpathlineto{\pgfqpoint{10.671875in}{2.298694in}}%
\pgfpathlineto{\pgfqpoint{10.673043in}{1.022863in}}%
\pgfpathlineto{\pgfqpoint{10.674210in}{1.174176in}}%
\pgfpathlineto{\pgfqpoint{10.675377in}{2.469259in}}%
\pgfpathlineto{\pgfqpoint{10.676544in}{2.452090in}}%
\pgfpathlineto{\pgfqpoint{10.678878in}{1.045308in}}%
\pgfpathlineto{\pgfqpoint{10.681213in}{2.558379in}}%
\pgfpathlineto{\pgfqpoint{10.683547in}{0.959266in}}%
\pgfpathlineto{\pgfqpoint{10.685882in}{2.626700in}}%
\pgfpathlineto{\pgfqpoint{10.688216in}{0.906104in}}%
\pgfpathlineto{\pgfqpoint{10.690550in}{2.667177in}}%
\pgfpathlineto{\pgfqpoint{10.692885in}{0.876018in}}%
\pgfpathlineto{\pgfqpoint{10.695219in}{2.688912in}}%
\pgfpathlineto{\pgfqpoint{10.697553in}{0.860902in}}%
\pgfpathlineto{\pgfqpoint{10.699888in}{2.698798in}}%
\pgfpathlineto{\pgfqpoint{10.702222in}{0.855223in}}%
\pgfpathlineto{\pgfqpoint{10.704557in}{2.700911in}}%
\pgfpathlineto{\pgfqpoint{10.706891in}{0.856434in}}%
\pgfpathlineto{\pgfqpoint{10.709225in}{2.696219in}}%
\pgfpathlineto{\pgfqpoint{10.711560in}{0.865164in}}%
\pgfpathlineto{\pgfqpoint{10.713894in}{2.682490in}}%
\pgfpathlineto{\pgfqpoint{10.716228in}{0.885244in}}%
\pgfpathlineto{\pgfqpoint{10.718563in}{2.654330in}}%
\pgfpathlineto{\pgfqpoint{10.720897in}{0.923561in}}%
\pgfpathlineto{\pgfqpoint{10.723232in}{2.603478in}}%
\pgfpathlineto{\pgfqpoint{10.725566in}{0.989568in}}%
\pgfpathlineto{\pgfqpoint{10.727900in}{2.519554in}}%
\pgfpathlineto{\pgfqpoint{10.730235in}{1.094188in}}%
\pgfpathlineto{\pgfqpoint{10.731402in}{2.460030in}}%
\pgfpathlineto{\pgfqpoint{10.732569in}{2.391604in}}%
\pgfpathlineto{\pgfqpoint{10.733736in}{1.030249in}}%
\pgfpathlineto{\pgfqpoint{10.734903in}{1.247747in}}%
\pgfpathlineto{\pgfqpoint{10.736071in}{2.587018in}}%
\pgfpathlineto{\pgfqpoint{10.737238in}{2.210758in}}%
\pgfpathlineto{\pgfqpoint{10.738405in}{0.915025in}}%
\pgfpathlineto{\pgfqpoint{10.739572in}{1.456666in}}%
\pgfpathlineto{\pgfqpoint{10.740739in}{2.683678in}}%
\pgfpathlineto{\pgfqpoint{10.743074in}{0.844425in}}%
\pgfpathlineto{\pgfqpoint{10.745408in}{2.720271in}}%
\pgfpathlineto{\pgfqpoint{10.747742in}{0.849874in}}%
\pgfpathlineto{\pgfqpoint{10.750077in}{2.665152in}}%
\pgfpathlineto{\pgfqpoint{10.752411in}{0.961247in}}%
\pgfpathlineto{\pgfqpoint{10.754746in}{2.492745in}}%
\pgfpathlineto{\pgfqpoint{10.755913in}{1.097377in}}%
\pgfpathlineto{\pgfqpoint{10.757080in}{1.196808in}}%
\pgfpathlineto{\pgfqpoint{10.758247in}{2.574980in}}%
\pgfpathlineto{\pgfqpoint{10.759414in}{2.195463in}}%
\pgfpathlineto{\pgfqpoint{10.760581in}{0.893589in}}%
\pgfpathlineto{\pgfqpoint{10.762916in}{2.711966in}}%
\pgfpathlineto{\pgfqpoint{10.765250in}{0.839770in}}%
\pgfpathlineto{\pgfqpoint{10.767585in}{2.668858in}}%
\pgfpathlineto{\pgfqpoint{10.769919in}{0.989037in}}%
\pgfpathlineto{\pgfqpoint{10.771086in}{2.389061in}}%
\pgfpathlineto{\pgfqpoint{10.772253in}{2.410750in}}%
\pgfpathlineto{\pgfqpoint{10.773421in}{1.002719in}}%
\pgfpathlineto{\pgfqpoint{10.774588in}{1.350213in}}%
\pgfpathlineto{\pgfqpoint{10.775755in}{2.667940in}}%
\pgfpathlineto{\pgfqpoint{10.778089in}{0.837034in}}%
\pgfpathlineto{\pgfqpoint{10.780424in}{2.695101in}}%
\pgfpathlineto{\pgfqpoint{10.782758in}{0.964662in}}%
\pgfpathlineto{\pgfqpoint{10.783925in}{2.371933in}}%
\pgfpathlineto{\pgfqpoint{10.785092in}{2.409415in}}%
\pgfpathlineto{\pgfqpoint{10.786260in}{0.988282in}}%
\pgfpathlineto{\pgfqpoint{10.787427in}{1.395392in}}%
\pgfpathlineto{\pgfqpoint{10.788594in}{2.689567in}}%
\pgfpathlineto{\pgfqpoint{10.790928in}{0.838127in}}%
\pgfpathlineto{\pgfqpoint{10.793263in}{2.640831in}}%
\pgfpathlineto{\pgfqpoint{10.795597in}{1.096835in}}%
\pgfpathlineto{\pgfqpoint{10.796764in}{2.539676in}}%
\pgfpathlineto{\pgfqpoint{10.797931in}{2.187059in}}%
\pgfpathlineto{\pgfqpoint{10.799099in}{0.868721in}}%
\pgfpathlineto{\pgfqpoint{10.801433in}{2.714988in}}%
\pgfpathlineto{\pgfqpoint{10.803767in}{0.946179in}}%
\pgfpathlineto{\pgfqpoint{10.804935in}{2.373662in}}%
\pgfpathlineto{\pgfqpoint{10.806102in}{2.377305in}}%
\pgfpathlineto{\pgfqpoint{10.807269in}{0.946270in}}%
\pgfpathlineto{\pgfqpoint{10.808436in}{1.510673in}}%
\pgfpathlineto{\pgfqpoint{10.809603in}{2.717113in}}%
\pgfpathlineto{\pgfqpoint{10.811938in}{0.887333in}}%
\pgfpathlineto{\pgfqpoint{10.814272in}{2.461787in}}%
\pgfpathlineto{\pgfqpoint{10.815000in}{1.548632in}}%
\pgfpathlineto{\pgfqpoint{10.815000in}{1.548632in}}%
\pgfusepath{stroke}%
\end{pgfscope}%
\begin{pgfscope}%
\pgfsetrectcap%
\pgfsetmiterjoin%
\pgfsetlinewidth{0.803000pt}%
\definecolor{currentstroke}{rgb}{0.000000,0.000000,0.000000}%
\pgfsetstrokecolor{currentstroke}%
\pgfsetdash{}{0pt}%
\pgfpathmoveto{\pgfqpoint{0.390138in}{0.835548in}}%
\pgfpathlineto{\pgfqpoint{0.390138in}{2.720366in}}%
\pgfusepath{stroke}%
\end{pgfscope}%
\begin{pgfscope}%
\pgfsetrectcap%
\pgfsetmiterjoin%
\pgfsetlinewidth{0.803000pt}%
\definecolor{currentstroke}{rgb}{0.000000,0.000000,0.000000}%
\pgfsetstrokecolor{currentstroke}%
\pgfsetdash{}{0pt}%
\pgfpathmoveto{\pgfqpoint{10.805000in}{0.835548in}}%
\pgfpathlineto{\pgfqpoint{10.805000in}{2.720366in}}%
\pgfusepath{stroke}%
\end{pgfscope}%
\begin{pgfscope}%
\pgfsetrectcap%
\pgfsetmiterjoin%
\pgfsetlinewidth{0.803000pt}%
\definecolor{currentstroke}{rgb}{0.000000,0.000000,0.000000}%
\pgfsetstrokecolor{currentstroke}%
\pgfsetdash{}{0pt}%
\pgfpathmoveto{\pgfqpoint{0.390138in}{0.835548in}}%
\pgfpathlineto{\pgfqpoint{10.805000in}{0.835548in}}%
\pgfusepath{stroke}%
\end{pgfscope}%
\begin{pgfscope}%
\pgfsetrectcap%
\pgfsetmiterjoin%
\pgfsetlinewidth{0.803000pt}%
\definecolor{currentstroke}{rgb}{0.000000,0.000000,0.000000}%
\pgfsetstrokecolor{currentstroke}%
\pgfsetdash{}{0pt}%
\pgfpathmoveto{\pgfqpoint{0.390138in}{2.720366in}}%
\pgfpathlineto{\pgfqpoint{10.805000in}{2.720366in}}%
\pgfusepath{stroke}%
\end{pgfscope}%
\begin{pgfscope}%
\definecolor{textcolor}{rgb}{0.000000,0.000000,0.000000}%
\pgfsetstrokecolor{textcolor}%
\pgfsetfillcolor{textcolor}%
\pgftext[x=4.859621in,y=2.249161in,left,base]{\color{textcolor}{\rmfamily\fontsize{16.000000}{19.200000}\selectfont\catcode`\^=\active\def^{\ifmmode\sp\else\^{}\fi}\catcode`\%=\active\def%{\%}$P_{0+}$}}%
\end{pgfscope}%
\begin{pgfscope}%
\definecolor{textcolor}{rgb}{0.000000,0.000000,1.000000}%
\pgfsetstrokecolor{textcolor}%
\pgfsetfillcolor{textcolor}%
\pgftext[x=4.859621in,y=1.212511in,left,base]{\color{textcolor}{\rmfamily\fontsize{16.000000}{19.200000}\selectfont\catcode`\^=\active\def^{\ifmmode\sp\else\^{}\fi}\catcode`\%=\active\def%{\%}$P_{2-}$}}%
\end{pgfscope}%
\begin{pgfscope}%
\definecolor{textcolor}{rgb}{0.000000,0.000000,0.000000}%
\pgfsetstrokecolor{textcolor}%
\pgfsetfillcolor{textcolor}%
\pgftext[x=0.494286in,y=2.626125in,left,top]{\color{textcolor}{\rmfamily\fontsize{16.000000}{19.200000}\selectfont\catcode`\^=\active\def^{\ifmmode\sp\else\^{}\fi}\catcode`\%=\active\def%{\%}$(c)$}}%
\end{pgfscope}%
\begin{pgfscope}%
\pgfsetbuttcap%
\pgfsetmiterjoin%
\definecolor{currentfill}{rgb}{1.000000,1.000000,1.000000}%
\pgfsetfillcolor{currentfill}%
\pgfsetlinewidth{0.000000pt}%
\definecolor{currentstroke}{rgb}{0.000000,0.000000,0.000000}%
\pgfsetstrokecolor{currentstroke}%
\pgfsetstrokeopacity{0.000000}%
\pgfsetdash{}{0pt}%
\pgfpathmoveto{\pgfqpoint{1.521902in}{5.797633in}}%
\pgfpathlineto{\pgfqpoint{4.438063in}{5.797633in}}%
\pgfpathlineto{\pgfqpoint{4.438063in}{7.051037in}}%
\pgfpathlineto{\pgfqpoint{1.521902in}{7.051037in}}%
\pgfpathlineto{\pgfqpoint{1.521902in}{5.797633in}}%
\pgfpathclose%
\pgfusepath{fill}%
\end{pgfscope}%
\begin{pgfscope}%
\pgfsetbuttcap%
\pgfsetroundjoin%
\definecolor{currentfill}{rgb}{0.000000,0.000000,0.000000}%
\pgfsetfillcolor{currentfill}%
\pgfsetlinewidth{0.803000pt}%
\definecolor{currentstroke}{rgb}{0.000000,0.000000,0.000000}%
\pgfsetstrokecolor{currentstroke}%
\pgfsetdash{}{0pt}%
\pgfsys@defobject{currentmarker}{\pgfqpoint{0.000000in}{-0.048611in}}{\pgfqpoint{0.000000in}{0.000000in}}{%
\pgfpathmoveto{\pgfqpoint{0.000000in}{0.000000in}}%
\pgfpathlineto{\pgfqpoint{0.000000in}{-0.048611in}}%
\pgfusepath{stroke,fill}%
}%
\begin{pgfscope}%
\pgfsys@transformshift{2.280312in}{5.797633in}%
\pgfsys@useobject{currentmarker}{}%
\end{pgfscope}%
\end{pgfscope}%
\begin{pgfscope}%
\definecolor{textcolor}{rgb}{0.000000,0.000000,0.000000}%
\pgfsetstrokecolor{textcolor}%
\pgfsetfillcolor{textcolor}%
\pgftext[x=2.280312in,y=5.700411in,,top]{\color{textcolor}{\rmfamily\fontsize{16.000000}{19.200000}\selectfont\catcode`\^=\active\def^{\ifmmode\sp\else\^{}\fi}\catcode`\%=\active\def%{\%}$\mathdefault{10^{5}}$}}%
\end{pgfscope}%
\begin{pgfscope}%
\pgfsetbuttcap%
\pgfsetroundjoin%
\definecolor{currentfill}{rgb}{0.000000,0.000000,0.000000}%
\pgfsetfillcolor{currentfill}%
\pgfsetlinewidth{0.803000pt}%
\definecolor{currentstroke}{rgb}{0.000000,0.000000,0.000000}%
\pgfsetstrokecolor{currentstroke}%
\pgfsetdash{}{0pt}%
\pgfsys@defobject{currentmarker}{\pgfqpoint{0.000000in}{-0.048611in}}{\pgfqpoint{0.000000in}{0.000000in}}{%
\pgfpathmoveto{\pgfqpoint{0.000000in}{0.000000in}}%
\pgfpathlineto{\pgfqpoint{0.000000in}{-0.048611in}}%
\pgfusepath{stroke,fill}%
}%
\begin{pgfscope}%
\pgfsys@transformshift{3.797132in}{5.797633in}%
\pgfsys@useobject{currentmarker}{}%
\end{pgfscope}%
\end{pgfscope}%
\begin{pgfscope}%
\definecolor{textcolor}{rgb}{0.000000,0.000000,0.000000}%
\pgfsetstrokecolor{textcolor}%
\pgfsetfillcolor{textcolor}%
\pgftext[x=3.797132in,y=5.700411in,,top]{\color{textcolor}{\rmfamily\fontsize{16.000000}{19.200000}\selectfont\catcode`\^=\active\def^{\ifmmode\sp\else\^{}\fi}\catcode`\%=\active\def%{\%}$\mathdefault{10^{7}}$}}%
\end{pgfscope}%
\begin{pgfscope}%
\pgfsetbuttcap%
\pgfsetroundjoin%
\definecolor{currentfill}{rgb}{0.000000,0.000000,0.000000}%
\pgfsetfillcolor{currentfill}%
\pgfsetlinewidth{0.602250pt}%
\definecolor{currentstroke}{rgb}{0.000000,0.000000,0.000000}%
\pgfsetstrokecolor{currentstroke}%
\pgfsetdash{}{0pt}%
\pgfsys@defobject{currentmarker}{\pgfqpoint{0.000000in}{-0.027778in}}{\pgfqpoint{0.000000in}{0.000000in}}{%
\pgfpathmoveto{\pgfqpoint{0.000000in}{0.000000in}}%
\pgfpathlineto{\pgfqpoint{0.000000in}{-0.027778in}}%
\pgfusepath{stroke,fill}%
}%
\begin{pgfscope}%
\pgfsys@transformshift{1.750206in}{5.797633in}%
\pgfsys@useobject{currentmarker}{}%
\end{pgfscope}%
\end{pgfscope}%
\begin{pgfscope}%
\pgfsetbuttcap%
\pgfsetroundjoin%
\definecolor{currentfill}{rgb}{0.000000,0.000000,0.000000}%
\pgfsetfillcolor{currentfill}%
\pgfsetlinewidth{0.602250pt}%
\definecolor{currentstroke}{rgb}{0.000000,0.000000,0.000000}%
\pgfsetstrokecolor{currentstroke}%
\pgfsetdash{}{0pt}%
\pgfsys@defobject{currentmarker}{\pgfqpoint{0.000000in}{-0.027778in}}{\pgfqpoint{0.000000in}{0.000000in}}{%
\pgfpathmoveto{\pgfqpoint{0.000000in}{0.000000in}}%
\pgfpathlineto{\pgfqpoint{0.000000in}{-0.027778in}}%
\pgfusepath{stroke,fill}%
}%
\begin{pgfscope}%
\pgfsys@transformshift{1.883755in}{5.797633in}%
\pgfsys@useobject{currentmarker}{}%
\end{pgfscope}%
\end{pgfscope}%
\begin{pgfscope}%
\pgfsetbuttcap%
\pgfsetroundjoin%
\definecolor{currentfill}{rgb}{0.000000,0.000000,0.000000}%
\pgfsetfillcolor{currentfill}%
\pgfsetlinewidth{0.602250pt}%
\definecolor{currentstroke}{rgb}{0.000000,0.000000,0.000000}%
\pgfsetstrokecolor{currentstroke}%
\pgfsetdash{}{0pt}%
\pgfsys@defobject{currentmarker}{\pgfqpoint{0.000000in}{-0.027778in}}{\pgfqpoint{0.000000in}{0.000000in}}{%
\pgfpathmoveto{\pgfqpoint{0.000000in}{0.000000in}}%
\pgfpathlineto{\pgfqpoint{0.000000in}{-0.027778in}}%
\pgfusepath{stroke,fill}%
}%
\begin{pgfscope}%
\pgfsys@transformshift{1.978510in}{5.797633in}%
\pgfsys@useobject{currentmarker}{}%
\end{pgfscope}%
\end{pgfscope}%
\begin{pgfscope}%
\pgfsetbuttcap%
\pgfsetroundjoin%
\definecolor{currentfill}{rgb}{0.000000,0.000000,0.000000}%
\pgfsetfillcolor{currentfill}%
\pgfsetlinewidth{0.602250pt}%
\definecolor{currentstroke}{rgb}{0.000000,0.000000,0.000000}%
\pgfsetstrokecolor{currentstroke}%
\pgfsetdash{}{0pt}%
\pgfsys@defobject{currentmarker}{\pgfqpoint{0.000000in}{-0.027778in}}{\pgfqpoint{0.000000in}{0.000000in}}{%
\pgfpathmoveto{\pgfqpoint{0.000000in}{0.000000in}}%
\pgfpathlineto{\pgfqpoint{0.000000in}{-0.027778in}}%
\pgfusepath{stroke,fill}%
}%
\begin{pgfscope}%
\pgfsys@transformshift{2.052008in}{5.797633in}%
\pgfsys@useobject{currentmarker}{}%
\end{pgfscope}%
\end{pgfscope}%
\begin{pgfscope}%
\pgfsetbuttcap%
\pgfsetroundjoin%
\definecolor{currentfill}{rgb}{0.000000,0.000000,0.000000}%
\pgfsetfillcolor{currentfill}%
\pgfsetlinewidth{0.602250pt}%
\definecolor{currentstroke}{rgb}{0.000000,0.000000,0.000000}%
\pgfsetstrokecolor{currentstroke}%
\pgfsetdash{}{0pt}%
\pgfsys@defobject{currentmarker}{\pgfqpoint{0.000000in}{-0.027778in}}{\pgfqpoint{0.000000in}{0.000000in}}{%
\pgfpathmoveto{\pgfqpoint{0.000000in}{0.000000in}}%
\pgfpathlineto{\pgfqpoint{0.000000in}{-0.027778in}}%
\pgfusepath{stroke,fill}%
}%
\begin{pgfscope}%
\pgfsys@transformshift{2.112059in}{5.797633in}%
\pgfsys@useobject{currentmarker}{}%
\end{pgfscope}%
\end{pgfscope}%
\begin{pgfscope}%
\pgfsetbuttcap%
\pgfsetroundjoin%
\definecolor{currentfill}{rgb}{0.000000,0.000000,0.000000}%
\pgfsetfillcolor{currentfill}%
\pgfsetlinewidth{0.602250pt}%
\definecolor{currentstroke}{rgb}{0.000000,0.000000,0.000000}%
\pgfsetstrokecolor{currentstroke}%
\pgfsetdash{}{0pt}%
\pgfsys@defobject{currentmarker}{\pgfqpoint{0.000000in}{-0.027778in}}{\pgfqpoint{0.000000in}{0.000000in}}{%
\pgfpathmoveto{\pgfqpoint{0.000000in}{0.000000in}}%
\pgfpathlineto{\pgfqpoint{0.000000in}{-0.027778in}}%
\pgfusepath{stroke,fill}%
}%
\begin{pgfscope}%
\pgfsys@transformshift{2.162833in}{5.797633in}%
\pgfsys@useobject{currentmarker}{}%
\end{pgfscope}%
\end{pgfscope}%
\begin{pgfscope}%
\pgfsetbuttcap%
\pgfsetroundjoin%
\definecolor{currentfill}{rgb}{0.000000,0.000000,0.000000}%
\pgfsetfillcolor{currentfill}%
\pgfsetlinewidth{0.602250pt}%
\definecolor{currentstroke}{rgb}{0.000000,0.000000,0.000000}%
\pgfsetstrokecolor{currentstroke}%
\pgfsetdash{}{0pt}%
\pgfsys@defobject{currentmarker}{\pgfqpoint{0.000000in}{-0.027778in}}{\pgfqpoint{0.000000in}{0.000000in}}{%
\pgfpathmoveto{\pgfqpoint{0.000000in}{0.000000in}}%
\pgfpathlineto{\pgfqpoint{0.000000in}{-0.027778in}}%
\pgfusepath{stroke,fill}%
}%
\begin{pgfscope}%
\pgfsys@transformshift{2.206814in}{5.797633in}%
\pgfsys@useobject{currentmarker}{}%
\end{pgfscope}%
\end{pgfscope}%
\begin{pgfscope}%
\pgfsetbuttcap%
\pgfsetroundjoin%
\definecolor{currentfill}{rgb}{0.000000,0.000000,0.000000}%
\pgfsetfillcolor{currentfill}%
\pgfsetlinewidth{0.602250pt}%
\definecolor{currentstroke}{rgb}{0.000000,0.000000,0.000000}%
\pgfsetstrokecolor{currentstroke}%
\pgfsetdash{}{0pt}%
\pgfsys@defobject{currentmarker}{\pgfqpoint{0.000000in}{-0.027778in}}{\pgfqpoint{0.000000in}{0.000000in}}{%
\pgfpathmoveto{\pgfqpoint{0.000000in}{0.000000in}}%
\pgfpathlineto{\pgfqpoint{0.000000in}{-0.027778in}}%
\pgfusepath{stroke,fill}%
}%
\begin{pgfscope}%
\pgfsys@transformshift{2.245609in}{5.797633in}%
\pgfsys@useobject{currentmarker}{}%
\end{pgfscope}%
\end{pgfscope}%
\begin{pgfscope}%
\pgfsetbuttcap%
\pgfsetroundjoin%
\definecolor{currentfill}{rgb}{0.000000,0.000000,0.000000}%
\pgfsetfillcolor{currentfill}%
\pgfsetlinewidth{0.602250pt}%
\definecolor{currentstroke}{rgb}{0.000000,0.000000,0.000000}%
\pgfsetstrokecolor{currentstroke}%
\pgfsetdash{}{0pt}%
\pgfsys@defobject{currentmarker}{\pgfqpoint{0.000000in}{-0.027778in}}{\pgfqpoint{0.000000in}{0.000000in}}{%
\pgfpathmoveto{\pgfqpoint{0.000000in}{0.000000in}}%
\pgfpathlineto{\pgfqpoint{0.000000in}{-0.027778in}}%
\pgfusepath{stroke,fill}%
}%
\begin{pgfscope}%
\pgfsys@transformshift{2.508616in}{5.797633in}%
\pgfsys@useobject{currentmarker}{}%
\end{pgfscope}%
\end{pgfscope}%
\begin{pgfscope}%
\pgfsetbuttcap%
\pgfsetroundjoin%
\definecolor{currentfill}{rgb}{0.000000,0.000000,0.000000}%
\pgfsetfillcolor{currentfill}%
\pgfsetlinewidth{0.602250pt}%
\definecolor{currentstroke}{rgb}{0.000000,0.000000,0.000000}%
\pgfsetstrokecolor{currentstroke}%
\pgfsetdash{}{0pt}%
\pgfsys@defobject{currentmarker}{\pgfqpoint{0.000000in}{-0.027778in}}{\pgfqpoint{0.000000in}{0.000000in}}{%
\pgfpathmoveto{\pgfqpoint{0.000000in}{0.000000in}}%
\pgfpathlineto{\pgfqpoint{0.000000in}{-0.027778in}}%
\pgfusepath{stroke,fill}%
}%
\begin{pgfscope}%
\pgfsys@transformshift{2.642165in}{5.797633in}%
\pgfsys@useobject{currentmarker}{}%
\end{pgfscope}%
\end{pgfscope}%
\begin{pgfscope}%
\pgfsetbuttcap%
\pgfsetroundjoin%
\definecolor{currentfill}{rgb}{0.000000,0.000000,0.000000}%
\pgfsetfillcolor{currentfill}%
\pgfsetlinewidth{0.602250pt}%
\definecolor{currentstroke}{rgb}{0.000000,0.000000,0.000000}%
\pgfsetstrokecolor{currentstroke}%
\pgfsetdash{}{0pt}%
\pgfsys@defobject{currentmarker}{\pgfqpoint{0.000000in}{-0.027778in}}{\pgfqpoint{0.000000in}{0.000000in}}{%
\pgfpathmoveto{\pgfqpoint{0.000000in}{0.000000in}}%
\pgfpathlineto{\pgfqpoint{0.000000in}{-0.027778in}}%
\pgfusepath{stroke,fill}%
}%
\begin{pgfscope}%
\pgfsys@transformshift{2.736920in}{5.797633in}%
\pgfsys@useobject{currentmarker}{}%
\end{pgfscope}%
\end{pgfscope}%
\begin{pgfscope}%
\pgfsetbuttcap%
\pgfsetroundjoin%
\definecolor{currentfill}{rgb}{0.000000,0.000000,0.000000}%
\pgfsetfillcolor{currentfill}%
\pgfsetlinewidth{0.602250pt}%
\definecolor{currentstroke}{rgb}{0.000000,0.000000,0.000000}%
\pgfsetstrokecolor{currentstroke}%
\pgfsetdash{}{0pt}%
\pgfsys@defobject{currentmarker}{\pgfqpoint{0.000000in}{-0.027778in}}{\pgfqpoint{0.000000in}{0.000000in}}{%
\pgfpathmoveto{\pgfqpoint{0.000000in}{0.000000in}}%
\pgfpathlineto{\pgfqpoint{0.000000in}{-0.027778in}}%
\pgfusepath{stroke,fill}%
}%
\begin{pgfscope}%
\pgfsys@transformshift{2.810418in}{5.797633in}%
\pgfsys@useobject{currentmarker}{}%
\end{pgfscope}%
\end{pgfscope}%
\begin{pgfscope}%
\pgfsetbuttcap%
\pgfsetroundjoin%
\definecolor{currentfill}{rgb}{0.000000,0.000000,0.000000}%
\pgfsetfillcolor{currentfill}%
\pgfsetlinewidth{0.602250pt}%
\definecolor{currentstroke}{rgb}{0.000000,0.000000,0.000000}%
\pgfsetstrokecolor{currentstroke}%
\pgfsetdash{}{0pt}%
\pgfsys@defobject{currentmarker}{\pgfqpoint{0.000000in}{-0.027778in}}{\pgfqpoint{0.000000in}{0.000000in}}{%
\pgfpathmoveto{\pgfqpoint{0.000000in}{0.000000in}}%
\pgfpathlineto{\pgfqpoint{0.000000in}{-0.027778in}}%
\pgfusepath{stroke,fill}%
}%
\begin{pgfscope}%
\pgfsys@transformshift{2.870470in}{5.797633in}%
\pgfsys@useobject{currentmarker}{}%
\end{pgfscope}%
\end{pgfscope}%
\begin{pgfscope}%
\pgfsetbuttcap%
\pgfsetroundjoin%
\definecolor{currentfill}{rgb}{0.000000,0.000000,0.000000}%
\pgfsetfillcolor{currentfill}%
\pgfsetlinewidth{0.602250pt}%
\definecolor{currentstroke}{rgb}{0.000000,0.000000,0.000000}%
\pgfsetstrokecolor{currentstroke}%
\pgfsetdash{}{0pt}%
\pgfsys@defobject{currentmarker}{\pgfqpoint{0.000000in}{-0.027778in}}{\pgfqpoint{0.000000in}{0.000000in}}{%
\pgfpathmoveto{\pgfqpoint{0.000000in}{0.000000in}}%
\pgfpathlineto{\pgfqpoint{0.000000in}{-0.027778in}}%
\pgfusepath{stroke,fill}%
}%
\begin{pgfscope}%
\pgfsys@transformshift{2.921243in}{5.797633in}%
\pgfsys@useobject{currentmarker}{}%
\end{pgfscope}%
\end{pgfscope}%
\begin{pgfscope}%
\pgfsetbuttcap%
\pgfsetroundjoin%
\definecolor{currentfill}{rgb}{0.000000,0.000000,0.000000}%
\pgfsetfillcolor{currentfill}%
\pgfsetlinewidth{0.602250pt}%
\definecolor{currentstroke}{rgb}{0.000000,0.000000,0.000000}%
\pgfsetstrokecolor{currentstroke}%
\pgfsetdash{}{0pt}%
\pgfsys@defobject{currentmarker}{\pgfqpoint{0.000000in}{-0.027778in}}{\pgfqpoint{0.000000in}{0.000000in}}{%
\pgfpathmoveto{\pgfqpoint{0.000000in}{0.000000in}}%
\pgfpathlineto{\pgfqpoint{0.000000in}{-0.027778in}}%
\pgfusepath{stroke,fill}%
}%
\begin{pgfscope}%
\pgfsys@transformshift{2.965224in}{5.797633in}%
\pgfsys@useobject{currentmarker}{}%
\end{pgfscope}%
\end{pgfscope}%
\begin{pgfscope}%
\pgfsetbuttcap%
\pgfsetroundjoin%
\definecolor{currentfill}{rgb}{0.000000,0.000000,0.000000}%
\pgfsetfillcolor{currentfill}%
\pgfsetlinewidth{0.602250pt}%
\definecolor{currentstroke}{rgb}{0.000000,0.000000,0.000000}%
\pgfsetstrokecolor{currentstroke}%
\pgfsetdash{}{0pt}%
\pgfsys@defobject{currentmarker}{\pgfqpoint{0.000000in}{-0.027778in}}{\pgfqpoint{0.000000in}{0.000000in}}{%
\pgfpathmoveto{\pgfqpoint{0.000000in}{0.000000in}}%
\pgfpathlineto{\pgfqpoint{0.000000in}{-0.027778in}}%
\pgfusepath{stroke,fill}%
}%
\begin{pgfscope}%
\pgfsys@transformshift{3.004019in}{5.797633in}%
\pgfsys@useobject{currentmarker}{}%
\end{pgfscope}%
\end{pgfscope}%
\begin{pgfscope}%
\pgfsetbuttcap%
\pgfsetroundjoin%
\definecolor{currentfill}{rgb}{0.000000,0.000000,0.000000}%
\pgfsetfillcolor{currentfill}%
\pgfsetlinewidth{0.602250pt}%
\definecolor{currentstroke}{rgb}{0.000000,0.000000,0.000000}%
\pgfsetstrokecolor{currentstroke}%
\pgfsetdash{}{0pt}%
\pgfsys@defobject{currentmarker}{\pgfqpoint{0.000000in}{-0.027778in}}{\pgfqpoint{0.000000in}{0.000000in}}{%
\pgfpathmoveto{\pgfqpoint{0.000000in}{0.000000in}}%
\pgfpathlineto{\pgfqpoint{0.000000in}{-0.027778in}}%
\pgfusepath{stroke,fill}%
}%
\begin{pgfscope}%
\pgfsys@transformshift{3.267026in}{5.797633in}%
\pgfsys@useobject{currentmarker}{}%
\end{pgfscope}%
\end{pgfscope}%
\begin{pgfscope}%
\pgfsetbuttcap%
\pgfsetroundjoin%
\definecolor{currentfill}{rgb}{0.000000,0.000000,0.000000}%
\pgfsetfillcolor{currentfill}%
\pgfsetlinewidth{0.602250pt}%
\definecolor{currentstroke}{rgb}{0.000000,0.000000,0.000000}%
\pgfsetstrokecolor{currentstroke}%
\pgfsetdash{}{0pt}%
\pgfsys@defobject{currentmarker}{\pgfqpoint{0.000000in}{-0.027778in}}{\pgfqpoint{0.000000in}{0.000000in}}{%
\pgfpathmoveto{\pgfqpoint{0.000000in}{0.000000in}}%
\pgfpathlineto{\pgfqpoint{0.000000in}{-0.027778in}}%
\pgfusepath{stroke,fill}%
}%
\begin{pgfscope}%
\pgfsys@transformshift{3.400576in}{5.797633in}%
\pgfsys@useobject{currentmarker}{}%
\end{pgfscope}%
\end{pgfscope}%
\begin{pgfscope}%
\pgfsetbuttcap%
\pgfsetroundjoin%
\definecolor{currentfill}{rgb}{0.000000,0.000000,0.000000}%
\pgfsetfillcolor{currentfill}%
\pgfsetlinewidth{0.602250pt}%
\definecolor{currentstroke}{rgb}{0.000000,0.000000,0.000000}%
\pgfsetstrokecolor{currentstroke}%
\pgfsetdash{}{0pt}%
\pgfsys@defobject{currentmarker}{\pgfqpoint{0.000000in}{-0.027778in}}{\pgfqpoint{0.000000in}{0.000000in}}{%
\pgfpathmoveto{\pgfqpoint{0.000000in}{0.000000in}}%
\pgfpathlineto{\pgfqpoint{0.000000in}{-0.027778in}}%
\pgfusepath{stroke,fill}%
}%
\begin{pgfscope}%
\pgfsys@transformshift{3.495330in}{5.797633in}%
\pgfsys@useobject{currentmarker}{}%
\end{pgfscope}%
\end{pgfscope}%
\begin{pgfscope}%
\pgfsetbuttcap%
\pgfsetroundjoin%
\definecolor{currentfill}{rgb}{0.000000,0.000000,0.000000}%
\pgfsetfillcolor{currentfill}%
\pgfsetlinewidth{0.602250pt}%
\definecolor{currentstroke}{rgb}{0.000000,0.000000,0.000000}%
\pgfsetstrokecolor{currentstroke}%
\pgfsetdash{}{0pt}%
\pgfsys@defobject{currentmarker}{\pgfqpoint{0.000000in}{-0.027778in}}{\pgfqpoint{0.000000in}{0.000000in}}{%
\pgfpathmoveto{\pgfqpoint{0.000000in}{0.000000in}}%
\pgfpathlineto{\pgfqpoint{0.000000in}{-0.027778in}}%
\pgfusepath{stroke,fill}%
}%
\begin{pgfscope}%
\pgfsys@transformshift{3.568828in}{5.797633in}%
\pgfsys@useobject{currentmarker}{}%
\end{pgfscope}%
\end{pgfscope}%
\begin{pgfscope}%
\pgfsetbuttcap%
\pgfsetroundjoin%
\definecolor{currentfill}{rgb}{0.000000,0.000000,0.000000}%
\pgfsetfillcolor{currentfill}%
\pgfsetlinewidth{0.602250pt}%
\definecolor{currentstroke}{rgb}{0.000000,0.000000,0.000000}%
\pgfsetstrokecolor{currentstroke}%
\pgfsetdash{}{0pt}%
\pgfsys@defobject{currentmarker}{\pgfqpoint{0.000000in}{-0.027778in}}{\pgfqpoint{0.000000in}{0.000000in}}{%
\pgfpathmoveto{\pgfqpoint{0.000000in}{0.000000in}}%
\pgfpathlineto{\pgfqpoint{0.000000in}{-0.027778in}}%
\pgfusepath{stroke,fill}%
}%
\begin{pgfscope}%
\pgfsys@transformshift{3.628880in}{5.797633in}%
\pgfsys@useobject{currentmarker}{}%
\end{pgfscope}%
\end{pgfscope}%
\begin{pgfscope}%
\pgfsetbuttcap%
\pgfsetroundjoin%
\definecolor{currentfill}{rgb}{0.000000,0.000000,0.000000}%
\pgfsetfillcolor{currentfill}%
\pgfsetlinewidth{0.602250pt}%
\definecolor{currentstroke}{rgb}{0.000000,0.000000,0.000000}%
\pgfsetstrokecolor{currentstroke}%
\pgfsetdash{}{0pt}%
\pgfsys@defobject{currentmarker}{\pgfqpoint{0.000000in}{-0.027778in}}{\pgfqpoint{0.000000in}{0.000000in}}{%
\pgfpathmoveto{\pgfqpoint{0.000000in}{0.000000in}}%
\pgfpathlineto{\pgfqpoint{0.000000in}{-0.027778in}}%
\pgfusepath{stroke,fill}%
}%
\begin{pgfscope}%
\pgfsys@transformshift{3.679653in}{5.797633in}%
\pgfsys@useobject{currentmarker}{}%
\end{pgfscope}%
\end{pgfscope}%
\begin{pgfscope}%
\pgfsetbuttcap%
\pgfsetroundjoin%
\definecolor{currentfill}{rgb}{0.000000,0.000000,0.000000}%
\pgfsetfillcolor{currentfill}%
\pgfsetlinewidth{0.602250pt}%
\definecolor{currentstroke}{rgb}{0.000000,0.000000,0.000000}%
\pgfsetstrokecolor{currentstroke}%
\pgfsetdash{}{0pt}%
\pgfsys@defobject{currentmarker}{\pgfqpoint{0.000000in}{-0.027778in}}{\pgfqpoint{0.000000in}{0.000000in}}{%
\pgfpathmoveto{\pgfqpoint{0.000000in}{0.000000in}}%
\pgfpathlineto{\pgfqpoint{0.000000in}{-0.027778in}}%
\pgfusepath{stroke,fill}%
}%
\begin{pgfscope}%
\pgfsys@transformshift{3.723635in}{5.797633in}%
\pgfsys@useobject{currentmarker}{}%
\end{pgfscope}%
\end{pgfscope}%
\begin{pgfscope}%
\pgfsetbuttcap%
\pgfsetroundjoin%
\definecolor{currentfill}{rgb}{0.000000,0.000000,0.000000}%
\pgfsetfillcolor{currentfill}%
\pgfsetlinewidth{0.602250pt}%
\definecolor{currentstroke}{rgb}{0.000000,0.000000,0.000000}%
\pgfsetstrokecolor{currentstroke}%
\pgfsetdash{}{0pt}%
\pgfsys@defobject{currentmarker}{\pgfqpoint{0.000000in}{-0.027778in}}{\pgfqpoint{0.000000in}{0.000000in}}{%
\pgfpathmoveto{\pgfqpoint{0.000000in}{0.000000in}}%
\pgfpathlineto{\pgfqpoint{0.000000in}{-0.027778in}}%
\pgfusepath{stroke,fill}%
}%
\begin{pgfscope}%
\pgfsys@transformshift{3.762429in}{5.797633in}%
\pgfsys@useobject{currentmarker}{}%
\end{pgfscope}%
\end{pgfscope}%
\begin{pgfscope}%
\pgfsetbuttcap%
\pgfsetroundjoin%
\definecolor{currentfill}{rgb}{0.000000,0.000000,0.000000}%
\pgfsetfillcolor{currentfill}%
\pgfsetlinewidth{0.602250pt}%
\definecolor{currentstroke}{rgb}{0.000000,0.000000,0.000000}%
\pgfsetstrokecolor{currentstroke}%
\pgfsetdash{}{0pt}%
\pgfsys@defobject{currentmarker}{\pgfqpoint{0.000000in}{-0.027778in}}{\pgfqpoint{0.000000in}{0.000000in}}{%
\pgfpathmoveto{\pgfqpoint{0.000000in}{0.000000in}}%
\pgfpathlineto{\pgfqpoint{0.000000in}{-0.027778in}}%
\pgfusepath{stroke,fill}%
}%
\begin{pgfscope}%
\pgfsys@transformshift{4.025436in}{5.797633in}%
\pgfsys@useobject{currentmarker}{}%
\end{pgfscope}%
\end{pgfscope}%
\begin{pgfscope}%
\pgfsetbuttcap%
\pgfsetroundjoin%
\definecolor{currentfill}{rgb}{0.000000,0.000000,0.000000}%
\pgfsetfillcolor{currentfill}%
\pgfsetlinewidth{0.602250pt}%
\definecolor{currentstroke}{rgb}{0.000000,0.000000,0.000000}%
\pgfsetstrokecolor{currentstroke}%
\pgfsetdash{}{0pt}%
\pgfsys@defobject{currentmarker}{\pgfqpoint{0.000000in}{-0.027778in}}{\pgfqpoint{0.000000in}{0.000000in}}{%
\pgfpathmoveto{\pgfqpoint{0.000000in}{0.000000in}}%
\pgfpathlineto{\pgfqpoint{0.000000in}{-0.027778in}}%
\pgfusepath{stroke,fill}%
}%
\begin{pgfscope}%
\pgfsys@transformshift{4.158986in}{5.797633in}%
\pgfsys@useobject{currentmarker}{}%
\end{pgfscope}%
\end{pgfscope}%
\begin{pgfscope}%
\pgfsetbuttcap%
\pgfsetroundjoin%
\definecolor{currentfill}{rgb}{0.000000,0.000000,0.000000}%
\pgfsetfillcolor{currentfill}%
\pgfsetlinewidth{0.602250pt}%
\definecolor{currentstroke}{rgb}{0.000000,0.000000,0.000000}%
\pgfsetstrokecolor{currentstroke}%
\pgfsetdash{}{0pt}%
\pgfsys@defobject{currentmarker}{\pgfqpoint{0.000000in}{-0.027778in}}{\pgfqpoint{0.000000in}{0.000000in}}{%
\pgfpathmoveto{\pgfqpoint{0.000000in}{0.000000in}}%
\pgfpathlineto{\pgfqpoint{0.000000in}{-0.027778in}}%
\pgfusepath{stroke,fill}%
}%
\begin{pgfscope}%
\pgfsys@transformshift{4.253741in}{5.797633in}%
\pgfsys@useobject{currentmarker}{}%
\end{pgfscope}%
\end{pgfscope}%
\begin{pgfscope}%
\pgfsetbuttcap%
\pgfsetroundjoin%
\definecolor{currentfill}{rgb}{0.000000,0.000000,0.000000}%
\pgfsetfillcolor{currentfill}%
\pgfsetlinewidth{0.602250pt}%
\definecolor{currentstroke}{rgb}{0.000000,0.000000,0.000000}%
\pgfsetstrokecolor{currentstroke}%
\pgfsetdash{}{0pt}%
\pgfsys@defobject{currentmarker}{\pgfqpoint{0.000000in}{-0.027778in}}{\pgfqpoint{0.000000in}{0.000000in}}{%
\pgfpathmoveto{\pgfqpoint{0.000000in}{0.000000in}}%
\pgfpathlineto{\pgfqpoint{0.000000in}{-0.027778in}}%
\pgfusepath{stroke,fill}%
}%
\begin{pgfscope}%
\pgfsys@transformshift{4.327238in}{5.797633in}%
\pgfsys@useobject{currentmarker}{}%
\end{pgfscope}%
\end{pgfscope}%
\begin{pgfscope}%
\pgfsetbuttcap%
\pgfsetroundjoin%
\definecolor{currentfill}{rgb}{0.000000,0.000000,0.000000}%
\pgfsetfillcolor{currentfill}%
\pgfsetlinewidth{0.602250pt}%
\definecolor{currentstroke}{rgb}{0.000000,0.000000,0.000000}%
\pgfsetstrokecolor{currentstroke}%
\pgfsetdash{}{0pt}%
\pgfsys@defobject{currentmarker}{\pgfqpoint{0.000000in}{-0.027778in}}{\pgfqpoint{0.000000in}{0.000000in}}{%
\pgfpathmoveto{\pgfqpoint{0.000000in}{0.000000in}}%
\pgfpathlineto{\pgfqpoint{0.000000in}{-0.027778in}}%
\pgfusepath{stroke,fill}%
}%
\begin{pgfscope}%
\pgfsys@transformshift{4.387290in}{5.797633in}%
\pgfsys@useobject{currentmarker}{}%
\end{pgfscope}%
\end{pgfscope}%
\begin{pgfscope}%
\pgfsetbuttcap%
\pgfsetroundjoin%
\definecolor{currentfill}{rgb}{0.000000,0.000000,0.000000}%
\pgfsetfillcolor{currentfill}%
\pgfsetlinewidth{0.602250pt}%
\definecolor{currentstroke}{rgb}{0.000000,0.000000,0.000000}%
\pgfsetstrokecolor{currentstroke}%
\pgfsetdash{}{0pt}%
\pgfsys@defobject{currentmarker}{\pgfqpoint{0.000000in}{-0.027778in}}{\pgfqpoint{0.000000in}{0.000000in}}{%
\pgfpathmoveto{\pgfqpoint{0.000000in}{0.000000in}}%
\pgfpathlineto{\pgfqpoint{0.000000in}{-0.027778in}}%
\pgfusepath{stroke,fill}%
}%
\begin{pgfscope}%
\pgfsys@transformshift{4.438063in}{5.797633in}%
\pgfsys@useobject{currentmarker}{}%
\end{pgfscope}%
\end{pgfscope}%
\begin{pgfscope}%
\pgfsetbuttcap%
\pgfsetroundjoin%
\definecolor{currentfill}{rgb}{0.000000,0.000000,0.000000}%
\pgfsetfillcolor{currentfill}%
\pgfsetlinewidth{0.803000pt}%
\definecolor{currentstroke}{rgb}{0.000000,0.000000,0.000000}%
\pgfsetstrokecolor{currentstroke}%
\pgfsetdash{}{0pt}%
\pgfsys@defobject{currentmarker}{\pgfqpoint{-0.048611in}{0.000000in}}{\pgfqpoint{-0.000000in}{0.000000in}}{%
\pgfpathmoveto{\pgfqpoint{-0.000000in}{0.000000in}}%
\pgfpathlineto{\pgfqpoint{-0.048611in}{0.000000in}}%
\pgfusepath{stroke,fill}%
}%
\begin{pgfscope}%
\pgfsys@transformshift{1.521902in}{5.797633in}%
\pgfsys@useobject{currentmarker}{}%
\end{pgfscope}%
\end{pgfscope}%
\begin{pgfscope}%
\definecolor{textcolor}{rgb}{0.000000,0.000000,0.000000}%
\pgfsetstrokecolor{textcolor}%
\pgfsetfillcolor{textcolor}%
\pgftext[x=1.314611in, y=5.714300in, left, base]{\color{textcolor}{\rmfamily\fontsize{16.000000}{19.200000}\selectfont\catcode`\^=\active\def^{\ifmmode\sp\else\^{}\fi}\catcode`\%=\active\def%{\%}0}}%
\end{pgfscope}%
\begin{pgfscope}%
\pgfsetbuttcap%
\pgfsetroundjoin%
\definecolor{currentfill}{rgb}{0.000000,0.000000,0.000000}%
\pgfsetfillcolor{currentfill}%
\pgfsetlinewidth{0.803000pt}%
\definecolor{currentstroke}{rgb}{0.000000,0.000000,0.000000}%
\pgfsetstrokecolor{currentstroke}%
\pgfsetdash{}{0pt}%
\pgfsys@defobject{currentmarker}{\pgfqpoint{-0.048611in}{0.000000in}}{\pgfqpoint{-0.000000in}{0.000000in}}{%
\pgfpathmoveto{\pgfqpoint{-0.000000in}{0.000000in}}%
\pgfpathlineto{\pgfqpoint{-0.048611in}{0.000000in}}%
\pgfusepath{stroke,fill}%
}%
\begin{pgfscope}%
\pgfsys@transformshift{1.521902in}{7.051037in}%
\pgfsys@useobject{currentmarker}{}%
\end{pgfscope}%
\end{pgfscope}%
\begin{pgfscope}%
\definecolor{textcolor}{rgb}{0.000000,0.000000,0.000000}%
\pgfsetstrokecolor{textcolor}%
\pgfsetfillcolor{textcolor}%
\pgftext[x=1.314611in, y=6.967704in, left, base]{\color{textcolor}{\rmfamily\fontsize{16.000000}{19.200000}\selectfont\catcode`\^=\active\def^{\ifmmode\sp\else\^{}\fi}\catcode`\%=\active\def%{\%}1}}%
\end{pgfscope}%
\begin{pgfscope}%
\pgfpathrectangle{\pgfqpoint{1.521902in}{5.797633in}}{\pgfqpoint{2.916161in}{1.253404in}}%
\pgfusepath{clip}%
\pgfsetrectcap%
\pgfsetroundjoin%
\pgfsetlinewidth{1.003750pt}%
\definecolor{currentstroke}{rgb}{0.000000,0.000000,0.000000}%
\pgfsetstrokecolor{currentstroke}%
\pgfsetdash{}{0pt}%
\pgfpathmoveto{\pgfqpoint{1.521489in}{7.051037in}}%
\pgfpathlineto{\pgfqpoint{2.890082in}{7.049926in}}%
\pgfpathlineto{\pgfqpoint{3.079023in}{7.047539in}}%
\pgfpathlineto{\pgfqpoint{3.195157in}{7.043963in}}%
\pgfpathlineto{\pgfqpoint{3.280024in}{7.039209in}}%
\pgfpathlineto{\pgfqpoint{3.347025in}{7.033299in}}%
\pgfpathlineto{\pgfqpoint{3.402412in}{7.026255in}}%
\pgfpathlineto{\pgfqpoint{3.450205in}{7.017984in}}%
\pgfpathlineto{\pgfqpoint{3.492192in}{7.008496in}}%
\pgfpathlineto{\pgfqpoint{3.529712in}{6.997768in}}%
\pgfpathlineto{\pgfqpoint{3.563659in}{6.985789in}}%
\pgfpathlineto{\pgfqpoint{3.594926in}{6.972440in}}%
\pgfpathlineto{\pgfqpoint{3.623959in}{6.957674in}}%
\pgfpathlineto{\pgfqpoint{3.651653in}{6.941094in}}%
\pgfpathlineto{\pgfqpoint{3.678006in}{6.922692in}}%
\pgfpathlineto{\pgfqpoint{3.703020in}{6.902510in}}%
\pgfpathlineto{\pgfqpoint{3.727140in}{6.880206in}}%
\pgfpathlineto{\pgfqpoint{3.750813in}{6.855260in}}%
\pgfpathlineto{\pgfqpoint{3.774040in}{6.827500in}}%
\pgfpathlineto{\pgfqpoint{3.796820in}{6.796773in}}%
\pgfpathlineto{\pgfqpoint{3.819600in}{6.762239in}}%
\pgfpathlineto{\pgfqpoint{3.842381in}{6.723556in}}%
\pgfpathlineto{\pgfqpoint{3.865161in}{6.680403in}}%
\pgfpathlineto{\pgfqpoint{3.888387in}{6.631499in}}%
\pgfpathlineto{\pgfqpoint{3.912061in}{6.576315in}}%
\pgfpathlineto{\pgfqpoint{3.936628in}{6.513200in}}%
\pgfpathlineto{\pgfqpoint{3.962088in}{6.441552in}}%
\pgfpathlineto{\pgfqpoint{3.989781in}{6.356847in}}%
\pgfpathlineto{\pgfqpoint{4.022388in}{6.249585in}}%
\pgfpathlineto{\pgfqpoint{4.115295in}{5.938704in}}%
\pgfpathlineto{\pgfqpoint{4.133609in}{5.888016in}}%
\pgfpathlineto{\pgfqpoint{4.148349in}{5.853344in}}%
\pgfpathlineto{\pgfqpoint{4.160856in}{5.829388in}}%
\pgfpathlineto{\pgfqpoint{4.171129in}{5.814178in}}%
\pgfpathlineto{\pgfqpoint{4.180062in}{5.804694in}}%
\pgfpathlineto{\pgfqpoint{4.187656in}{5.799662in}}%
\pgfpathlineto{\pgfqpoint{4.194356in}{5.797730in}}%
\pgfpathlineto{\pgfqpoint{4.200609in}{5.798194in}}%
\pgfpathlineto{\pgfqpoint{4.206862in}{5.800975in}}%
\pgfpathlineto{\pgfqpoint{4.213116in}{5.806198in}}%
\pgfpathlineto{\pgfqpoint{4.220263in}{5.815308in}}%
\pgfpathlineto{\pgfqpoint{4.227856in}{5.828835in}}%
\pgfpathlineto{\pgfqpoint{4.236343in}{5.848864in}}%
\pgfpathlineto{\pgfqpoint{4.245723in}{5.877282in}}%
\pgfpathlineto{\pgfqpoint{4.255549in}{5.914363in}}%
\pgfpathlineto{\pgfqpoint{4.266716in}{5.965768in}}%
\pgfpathlineto{\pgfqpoint{4.278776in}{6.032367in}}%
\pgfpathlineto{\pgfqpoint{4.292176in}{6.119439in}}%
\pgfpathlineto{\pgfqpoint{4.307363in}{6.233334in}}%
\pgfpathlineto{\pgfqpoint{4.326123in}{6.391976in}}%
\pgfpathlineto{\pgfqpoint{4.382403in}{6.881180in}}%
\pgfpathlineto{\pgfqpoint{4.394017in}{6.956787in}}%
\pgfpathlineto{\pgfqpoint{4.403397in}{7.003955in}}%
\pgfpathlineto{\pgfqpoint{4.410543in}{7.029840in}}%
\pgfpathlineto{\pgfqpoint{4.416350in}{7.043648in}}%
\pgfpathlineto{\pgfqpoint{4.420817in}{7.049501in}}%
\pgfpathlineto{\pgfqpoint{4.424390in}{7.051035in}}%
\pgfpathlineto{\pgfqpoint{4.427517in}{7.049996in}}%
\pgfpathlineto{\pgfqpoint{4.431090in}{7.046004in}}%
\pgfpathlineto{\pgfqpoint{4.435110in}{7.037846in}}%
\pgfpathlineto{\pgfqpoint{4.438237in}{7.028761in}}%
\pgfpathlineto{\pgfqpoint{4.438237in}{7.028761in}}%
\pgfusepath{stroke}%
\end{pgfscope}%
\begin{pgfscope}%
\pgfpathrectangle{\pgfqpoint{1.521902in}{5.797633in}}{\pgfqpoint{2.916161in}{1.253404in}}%
\pgfusepath{clip}%
\pgfsetrectcap%
\pgfsetroundjoin%
\pgfsetlinewidth{1.003750pt}%
\definecolor{currentstroke}{rgb}{1.000000,0.000000,0.000000}%
\pgfsetstrokecolor{currentstroke}%
\pgfsetdash{}{0pt}%
\pgfpathmoveto{\pgfqpoint{1.521489in}{5.797634in}}%
\pgfpathlineto{\pgfqpoint{2.890082in}{5.798745in}}%
\pgfpathlineto{\pgfqpoint{3.079023in}{5.801131in}}%
\pgfpathlineto{\pgfqpoint{3.195157in}{5.804707in}}%
\pgfpathlineto{\pgfqpoint{3.280024in}{5.809461in}}%
\pgfpathlineto{\pgfqpoint{3.347025in}{5.815371in}}%
\pgfpathlineto{\pgfqpoint{3.402412in}{5.822416in}}%
\pgfpathlineto{\pgfqpoint{3.450205in}{5.830686in}}%
\pgfpathlineto{\pgfqpoint{3.492192in}{5.840174in}}%
\pgfpathlineto{\pgfqpoint{3.529712in}{5.850903in}}%
\pgfpathlineto{\pgfqpoint{3.563659in}{5.862881in}}%
\pgfpathlineto{\pgfqpoint{3.594926in}{5.876231in}}%
\pgfpathlineto{\pgfqpoint{3.623959in}{5.890997in}}%
\pgfpathlineto{\pgfqpoint{3.651653in}{5.907577in}}%
\pgfpathlineto{\pgfqpoint{3.678006in}{5.925979in}}%
\pgfpathlineto{\pgfqpoint{3.703020in}{5.946160in}}%
\pgfpathlineto{\pgfqpoint{3.727140in}{5.968464in}}%
\pgfpathlineto{\pgfqpoint{3.750813in}{5.993410in}}%
\pgfpathlineto{\pgfqpoint{3.774040in}{6.021170in}}%
\pgfpathlineto{\pgfqpoint{3.796820in}{6.051897in}}%
\pgfpathlineto{\pgfqpoint{3.819600in}{6.086432in}}%
\pgfpathlineto{\pgfqpoint{3.842381in}{6.125114in}}%
\pgfpathlineto{\pgfqpoint{3.865161in}{6.168268in}}%
\pgfpathlineto{\pgfqpoint{3.888387in}{6.217171in}}%
\pgfpathlineto{\pgfqpoint{3.912061in}{6.272356in}}%
\pgfpathlineto{\pgfqpoint{3.936628in}{6.335470in}}%
\pgfpathlineto{\pgfqpoint{3.962088in}{6.407119in}}%
\pgfpathlineto{\pgfqpoint{3.989781in}{6.491824in}}%
\pgfpathlineto{\pgfqpoint{4.022388in}{6.599085in}}%
\pgfpathlineto{\pgfqpoint{4.115295in}{6.909967in}}%
\pgfpathlineto{\pgfqpoint{4.133609in}{6.960655in}}%
\pgfpathlineto{\pgfqpoint{4.148349in}{6.995326in}}%
\pgfpathlineto{\pgfqpoint{4.160856in}{7.019283in}}%
\pgfpathlineto{\pgfqpoint{4.171129in}{7.034492in}}%
\pgfpathlineto{\pgfqpoint{4.180062in}{7.043976in}}%
\pgfpathlineto{\pgfqpoint{4.187656in}{7.049008in}}%
\pgfpathlineto{\pgfqpoint{4.194356in}{7.050940in}}%
\pgfpathlineto{\pgfqpoint{4.200609in}{7.050477in}}%
\pgfpathlineto{\pgfqpoint{4.206862in}{7.047695in}}%
\pgfpathlineto{\pgfqpoint{4.213116in}{7.042472in}}%
\pgfpathlineto{\pgfqpoint{4.220263in}{7.033362in}}%
\pgfpathlineto{\pgfqpoint{4.227856in}{7.019835in}}%
\pgfpathlineto{\pgfqpoint{4.236343in}{6.999807in}}%
\pgfpathlineto{\pgfqpoint{4.245723in}{6.971389in}}%
\pgfpathlineto{\pgfqpoint{4.255549in}{6.934307in}}%
\pgfpathlineto{\pgfqpoint{4.266716in}{6.882902in}}%
\pgfpathlineto{\pgfqpoint{4.278776in}{6.816303in}}%
\pgfpathlineto{\pgfqpoint{4.292176in}{6.729232in}}%
\pgfpathlineto{\pgfqpoint{4.307363in}{6.615337in}}%
\pgfpathlineto{\pgfqpoint{4.326123in}{6.456694in}}%
\pgfpathlineto{\pgfqpoint{4.382403in}{5.967490in}}%
\pgfpathlineto{\pgfqpoint{4.394017in}{5.891883in}}%
\pgfpathlineto{\pgfqpoint{4.403397in}{5.844715in}}%
\pgfpathlineto{\pgfqpoint{4.410543in}{5.818830in}}%
\pgfpathlineto{\pgfqpoint{4.416350in}{5.805023in}}%
\pgfpathlineto{\pgfqpoint{4.420817in}{5.799169in}}%
\pgfpathlineto{\pgfqpoint{4.424390in}{5.797635in}}%
\pgfpathlineto{\pgfqpoint{4.427517in}{5.798675in}}%
\pgfpathlineto{\pgfqpoint{4.431090in}{5.802666in}}%
\pgfpathlineto{\pgfqpoint{4.435110in}{5.810824in}}%
\pgfpathlineto{\pgfqpoint{4.438237in}{5.819909in}}%
\pgfpathlineto{\pgfqpoint{4.438237in}{5.819909in}}%
\pgfusepath{stroke}%
\end{pgfscope}%
\begin{pgfscope}%
\pgfsetrectcap%
\pgfsetmiterjoin%
\pgfsetlinewidth{0.803000pt}%
\definecolor{currentstroke}{rgb}{0.000000,0.000000,0.000000}%
\pgfsetstrokecolor{currentstroke}%
\pgfsetdash{}{0pt}%
\pgfpathmoveto{\pgfqpoint{1.521902in}{5.797633in}}%
\pgfpathlineto{\pgfqpoint{1.521902in}{7.051037in}}%
\pgfusepath{stroke}%
\end{pgfscope}%
\begin{pgfscope}%
\pgfsetrectcap%
\pgfsetmiterjoin%
\pgfsetlinewidth{0.803000pt}%
\definecolor{currentstroke}{rgb}{0.000000,0.000000,0.000000}%
\pgfsetstrokecolor{currentstroke}%
\pgfsetdash{}{0pt}%
\pgfpathmoveto{\pgfqpoint{4.438063in}{5.797633in}}%
\pgfpathlineto{\pgfqpoint{4.438063in}{7.051037in}}%
\pgfusepath{stroke}%
\end{pgfscope}%
\begin{pgfscope}%
\pgfsetrectcap%
\pgfsetmiterjoin%
\pgfsetlinewidth{0.803000pt}%
\definecolor{currentstroke}{rgb}{0.000000,0.000000,0.000000}%
\pgfsetstrokecolor{currentstroke}%
\pgfsetdash{}{0pt}%
\pgfpathmoveto{\pgfqpoint{1.521902in}{5.797633in}}%
\pgfpathlineto{\pgfqpoint{4.438063in}{5.797633in}}%
\pgfusepath{stroke}%
\end{pgfscope}%
\begin{pgfscope}%
\pgfsetrectcap%
\pgfsetmiterjoin%
\pgfsetlinewidth{0.803000pt}%
\definecolor{currentstroke}{rgb}{0.000000,0.000000,0.000000}%
\pgfsetstrokecolor{currentstroke}%
\pgfsetdash{}{0pt}%
\pgfpathmoveto{\pgfqpoint{1.521902in}{7.051037in}}%
\pgfpathlineto{\pgfqpoint{4.438063in}{7.051037in}}%
\pgfusepath{stroke}%
\end{pgfscope}%
\begin{pgfscope}%
\definecolor{textcolor}{rgb}{0.000000,0.000000,0.000000}%
\pgfsetstrokecolor{textcolor}%
\pgfsetfillcolor{textcolor}%
\pgftext[x=3.400576in,y=6.737686in,left,base]{\color{textcolor}{\rmfamily\fontsize{16.000000}{19.200000}\selectfont\catcode`\^=\active\def^{\ifmmode\sp\else\^{}\fi}\catcode`\%=\active\def%{\%}$P_{0v}$}}%
\end{pgfscope}%
\begin{pgfscope}%
\definecolor{textcolor}{rgb}{1.000000,0.000000,0.000000}%
\pgfsetstrokecolor{textcolor}%
\pgfsetfillcolor{textcolor}%
\pgftext[x=3.400576in,y=6.048314in,left,base]{\color{textcolor}{\rmfamily\fontsize{16.000000}{19.200000}\selectfont\catcode`\^=\active\def^{\ifmmode\sp\else\^{}\fi}\catcode`\%=\active\def%{\%}$P_{3c}$}}%
\end{pgfscope}%
\begin{pgfscope}%
\pgfsetbuttcap%
\pgfsetmiterjoin%
\definecolor{currentfill}{rgb}{1.000000,1.000000,1.000000}%
\pgfsetfillcolor{currentfill}%
\pgfsetlinewidth{0.000000pt}%
\definecolor{currentstroke}{rgb}{0.000000,0.000000,0.000000}%
\pgfsetstrokecolor{currentstroke}%
\pgfsetstrokeopacity{0.000000}%
\pgfsetdash{}{0pt}%
\pgfpathmoveto{\pgfqpoint{1.313604in}{3.540038in}}%
\pgfpathlineto{\pgfqpoint{4.229766in}{3.540038in}}%
\pgfpathlineto{\pgfqpoint{4.229766in}{4.793442in}}%
\pgfpathlineto{\pgfqpoint{1.313604in}{4.793442in}}%
\pgfpathlineto{\pgfqpoint{1.313604in}{3.540038in}}%
\pgfpathclose%
\pgfusepath{fill}%
\end{pgfscope}%
\begin{pgfscope}%
\pgfsetbuttcap%
\pgfsetroundjoin%
\definecolor{currentfill}{rgb}{0.000000,0.000000,0.000000}%
\pgfsetfillcolor{currentfill}%
\pgfsetlinewidth{0.803000pt}%
\definecolor{currentstroke}{rgb}{0.000000,0.000000,0.000000}%
\pgfsetstrokecolor{currentstroke}%
\pgfsetdash{}{0pt}%
\pgfsys@defobject{currentmarker}{\pgfqpoint{0.000000in}{-0.048611in}}{\pgfqpoint{0.000000in}{0.000000in}}{%
\pgfpathmoveto{\pgfqpoint{0.000000in}{0.000000in}}%
\pgfpathlineto{\pgfqpoint{0.000000in}{-0.048611in}}%
\pgfusepath{stroke,fill}%
}%
\begin{pgfscope}%
\pgfsys@transformshift{2.072015in}{3.540038in}%
\pgfsys@useobject{currentmarker}{}%
\end{pgfscope}%
\end{pgfscope}%
\begin{pgfscope}%
\definecolor{textcolor}{rgb}{0.000000,0.000000,0.000000}%
\pgfsetstrokecolor{textcolor}%
\pgfsetfillcolor{textcolor}%
\pgftext[x=2.072015in,y=3.442816in,,top]{\color{textcolor}{\rmfamily\fontsize{16.000000}{19.200000}\selectfont\catcode`\^=\active\def^{\ifmmode\sp\else\^{}\fi}\catcode`\%=\active\def%{\%}$\mathdefault{10^{5}}$}}%
\end{pgfscope}%
\begin{pgfscope}%
\pgfsetbuttcap%
\pgfsetroundjoin%
\definecolor{currentfill}{rgb}{0.000000,0.000000,0.000000}%
\pgfsetfillcolor{currentfill}%
\pgfsetlinewidth{0.803000pt}%
\definecolor{currentstroke}{rgb}{0.000000,0.000000,0.000000}%
\pgfsetstrokecolor{currentstroke}%
\pgfsetdash{}{0pt}%
\pgfsys@defobject{currentmarker}{\pgfqpoint{0.000000in}{-0.048611in}}{\pgfqpoint{0.000000in}{0.000000in}}{%
\pgfpathmoveto{\pgfqpoint{0.000000in}{0.000000in}}%
\pgfpathlineto{\pgfqpoint{0.000000in}{-0.048611in}}%
\pgfusepath{stroke,fill}%
}%
\begin{pgfscope}%
\pgfsys@transformshift{3.588835in}{3.540038in}%
\pgfsys@useobject{currentmarker}{}%
\end{pgfscope}%
\end{pgfscope}%
\begin{pgfscope}%
\definecolor{textcolor}{rgb}{0.000000,0.000000,0.000000}%
\pgfsetstrokecolor{textcolor}%
\pgfsetfillcolor{textcolor}%
\pgftext[x=3.588835in,y=3.442816in,,top]{\color{textcolor}{\rmfamily\fontsize{16.000000}{19.200000}\selectfont\catcode`\^=\active\def^{\ifmmode\sp\else\^{}\fi}\catcode`\%=\active\def%{\%}$\mathdefault{10^{7}}$}}%
\end{pgfscope}%
\begin{pgfscope}%
\pgfsetbuttcap%
\pgfsetroundjoin%
\definecolor{currentfill}{rgb}{0.000000,0.000000,0.000000}%
\pgfsetfillcolor{currentfill}%
\pgfsetlinewidth{0.602250pt}%
\definecolor{currentstroke}{rgb}{0.000000,0.000000,0.000000}%
\pgfsetstrokecolor{currentstroke}%
\pgfsetdash{}{0pt}%
\pgfsys@defobject{currentmarker}{\pgfqpoint{0.000000in}{-0.027778in}}{\pgfqpoint{0.000000in}{0.000000in}}{%
\pgfpathmoveto{\pgfqpoint{0.000000in}{0.000000in}}%
\pgfpathlineto{\pgfqpoint{0.000000in}{-0.027778in}}%
\pgfusepath{stroke,fill}%
}%
\begin{pgfscope}%
\pgfsys@transformshift{1.541909in}{3.540038in}%
\pgfsys@useobject{currentmarker}{}%
\end{pgfscope}%
\end{pgfscope}%
\begin{pgfscope}%
\pgfsetbuttcap%
\pgfsetroundjoin%
\definecolor{currentfill}{rgb}{0.000000,0.000000,0.000000}%
\pgfsetfillcolor{currentfill}%
\pgfsetlinewidth{0.602250pt}%
\definecolor{currentstroke}{rgb}{0.000000,0.000000,0.000000}%
\pgfsetstrokecolor{currentstroke}%
\pgfsetdash{}{0pt}%
\pgfsys@defobject{currentmarker}{\pgfqpoint{0.000000in}{-0.027778in}}{\pgfqpoint{0.000000in}{0.000000in}}{%
\pgfpathmoveto{\pgfqpoint{0.000000in}{0.000000in}}%
\pgfpathlineto{\pgfqpoint{0.000000in}{-0.027778in}}%
\pgfusepath{stroke,fill}%
}%
\begin{pgfscope}%
\pgfsys@transformshift{1.675458in}{3.540038in}%
\pgfsys@useobject{currentmarker}{}%
\end{pgfscope}%
\end{pgfscope}%
\begin{pgfscope}%
\pgfsetbuttcap%
\pgfsetroundjoin%
\definecolor{currentfill}{rgb}{0.000000,0.000000,0.000000}%
\pgfsetfillcolor{currentfill}%
\pgfsetlinewidth{0.602250pt}%
\definecolor{currentstroke}{rgb}{0.000000,0.000000,0.000000}%
\pgfsetstrokecolor{currentstroke}%
\pgfsetdash{}{0pt}%
\pgfsys@defobject{currentmarker}{\pgfqpoint{0.000000in}{-0.027778in}}{\pgfqpoint{0.000000in}{0.000000in}}{%
\pgfpathmoveto{\pgfqpoint{0.000000in}{0.000000in}}%
\pgfpathlineto{\pgfqpoint{0.000000in}{-0.027778in}}%
\pgfusepath{stroke,fill}%
}%
\begin{pgfscope}%
\pgfsys@transformshift{1.770213in}{3.540038in}%
\pgfsys@useobject{currentmarker}{}%
\end{pgfscope}%
\end{pgfscope}%
\begin{pgfscope}%
\pgfsetbuttcap%
\pgfsetroundjoin%
\definecolor{currentfill}{rgb}{0.000000,0.000000,0.000000}%
\pgfsetfillcolor{currentfill}%
\pgfsetlinewidth{0.602250pt}%
\definecolor{currentstroke}{rgb}{0.000000,0.000000,0.000000}%
\pgfsetstrokecolor{currentstroke}%
\pgfsetdash{}{0pt}%
\pgfsys@defobject{currentmarker}{\pgfqpoint{0.000000in}{-0.027778in}}{\pgfqpoint{0.000000in}{0.000000in}}{%
\pgfpathmoveto{\pgfqpoint{0.000000in}{0.000000in}}%
\pgfpathlineto{\pgfqpoint{0.000000in}{-0.027778in}}%
\pgfusepath{stroke,fill}%
}%
\begin{pgfscope}%
\pgfsys@transformshift{1.843710in}{3.540038in}%
\pgfsys@useobject{currentmarker}{}%
\end{pgfscope}%
\end{pgfscope}%
\begin{pgfscope}%
\pgfsetbuttcap%
\pgfsetroundjoin%
\definecolor{currentfill}{rgb}{0.000000,0.000000,0.000000}%
\pgfsetfillcolor{currentfill}%
\pgfsetlinewidth{0.602250pt}%
\definecolor{currentstroke}{rgb}{0.000000,0.000000,0.000000}%
\pgfsetstrokecolor{currentstroke}%
\pgfsetdash{}{0pt}%
\pgfsys@defobject{currentmarker}{\pgfqpoint{0.000000in}{-0.027778in}}{\pgfqpoint{0.000000in}{0.000000in}}{%
\pgfpathmoveto{\pgfqpoint{0.000000in}{0.000000in}}%
\pgfpathlineto{\pgfqpoint{0.000000in}{-0.027778in}}%
\pgfusepath{stroke,fill}%
}%
\begin{pgfscope}%
\pgfsys@transformshift{1.903762in}{3.540038in}%
\pgfsys@useobject{currentmarker}{}%
\end{pgfscope}%
\end{pgfscope}%
\begin{pgfscope}%
\pgfsetbuttcap%
\pgfsetroundjoin%
\definecolor{currentfill}{rgb}{0.000000,0.000000,0.000000}%
\pgfsetfillcolor{currentfill}%
\pgfsetlinewidth{0.602250pt}%
\definecolor{currentstroke}{rgb}{0.000000,0.000000,0.000000}%
\pgfsetstrokecolor{currentstroke}%
\pgfsetdash{}{0pt}%
\pgfsys@defobject{currentmarker}{\pgfqpoint{0.000000in}{-0.027778in}}{\pgfqpoint{0.000000in}{0.000000in}}{%
\pgfpathmoveto{\pgfqpoint{0.000000in}{0.000000in}}%
\pgfpathlineto{\pgfqpoint{0.000000in}{-0.027778in}}%
\pgfusepath{stroke,fill}%
}%
\begin{pgfscope}%
\pgfsys@transformshift{1.954535in}{3.540038in}%
\pgfsys@useobject{currentmarker}{}%
\end{pgfscope}%
\end{pgfscope}%
\begin{pgfscope}%
\pgfsetbuttcap%
\pgfsetroundjoin%
\definecolor{currentfill}{rgb}{0.000000,0.000000,0.000000}%
\pgfsetfillcolor{currentfill}%
\pgfsetlinewidth{0.602250pt}%
\definecolor{currentstroke}{rgb}{0.000000,0.000000,0.000000}%
\pgfsetstrokecolor{currentstroke}%
\pgfsetdash{}{0pt}%
\pgfsys@defobject{currentmarker}{\pgfqpoint{0.000000in}{-0.027778in}}{\pgfqpoint{0.000000in}{0.000000in}}{%
\pgfpathmoveto{\pgfqpoint{0.000000in}{0.000000in}}%
\pgfpathlineto{\pgfqpoint{0.000000in}{-0.027778in}}%
\pgfusepath{stroke,fill}%
}%
\begin{pgfscope}%
\pgfsys@transformshift{1.998517in}{3.540038in}%
\pgfsys@useobject{currentmarker}{}%
\end{pgfscope}%
\end{pgfscope}%
\begin{pgfscope}%
\pgfsetbuttcap%
\pgfsetroundjoin%
\definecolor{currentfill}{rgb}{0.000000,0.000000,0.000000}%
\pgfsetfillcolor{currentfill}%
\pgfsetlinewidth{0.602250pt}%
\definecolor{currentstroke}{rgb}{0.000000,0.000000,0.000000}%
\pgfsetstrokecolor{currentstroke}%
\pgfsetdash{}{0pt}%
\pgfsys@defobject{currentmarker}{\pgfqpoint{0.000000in}{-0.027778in}}{\pgfqpoint{0.000000in}{0.000000in}}{%
\pgfpathmoveto{\pgfqpoint{0.000000in}{0.000000in}}%
\pgfpathlineto{\pgfqpoint{0.000000in}{-0.027778in}}%
\pgfusepath{stroke,fill}%
}%
\begin{pgfscope}%
\pgfsys@transformshift{2.037312in}{3.540038in}%
\pgfsys@useobject{currentmarker}{}%
\end{pgfscope}%
\end{pgfscope}%
\begin{pgfscope}%
\pgfsetbuttcap%
\pgfsetroundjoin%
\definecolor{currentfill}{rgb}{0.000000,0.000000,0.000000}%
\pgfsetfillcolor{currentfill}%
\pgfsetlinewidth{0.602250pt}%
\definecolor{currentstroke}{rgb}{0.000000,0.000000,0.000000}%
\pgfsetstrokecolor{currentstroke}%
\pgfsetdash{}{0pt}%
\pgfsys@defobject{currentmarker}{\pgfqpoint{0.000000in}{-0.027778in}}{\pgfqpoint{0.000000in}{0.000000in}}{%
\pgfpathmoveto{\pgfqpoint{0.000000in}{0.000000in}}%
\pgfpathlineto{\pgfqpoint{0.000000in}{-0.027778in}}%
\pgfusepath{stroke,fill}%
}%
\begin{pgfscope}%
\pgfsys@transformshift{2.300319in}{3.540038in}%
\pgfsys@useobject{currentmarker}{}%
\end{pgfscope}%
\end{pgfscope}%
\begin{pgfscope}%
\pgfsetbuttcap%
\pgfsetroundjoin%
\definecolor{currentfill}{rgb}{0.000000,0.000000,0.000000}%
\pgfsetfillcolor{currentfill}%
\pgfsetlinewidth{0.602250pt}%
\definecolor{currentstroke}{rgb}{0.000000,0.000000,0.000000}%
\pgfsetstrokecolor{currentstroke}%
\pgfsetdash{}{0pt}%
\pgfsys@defobject{currentmarker}{\pgfqpoint{0.000000in}{-0.027778in}}{\pgfqpoint{0.000000in}{0.000000in}}{%
\pgfpathmoveto{\pgfqpoint{0.000000in}{0.000000in}}%
\pgfpathlineto{\pgfqpoint{0.000000in}{-0.027778in}}%
\pgfusepath{stroke,fill}%
}%
\begin{pgfscope}%
\pgfsys@transformshift{2.433868in}{3.540038in}%
\pgfsys@useobject{currentmarker}{}%
\end{pgfscope}%
\end{pgfscope}%
\begin{pgfscope}%
\pgfsetbuttcap%
\pgfsetroundjoin%
\definecolor{currentfill}{rgb}{0.000000,0.000000,0.000000}%
\pgfsetfillcolor{currentfill}%
\pgfsetlinewidth{0.602250pt}%
\definecolor{currentstroke}{rgb}{0.000000,0.000000,0.000000}%
\pgfsetstrokecolor{currentstroke}%
\pgfsetdash{}{0pt}%
\pgfsys@defobject{currentmarker}{\pgfqpoint{0.000000in}{-0.027778in}}{\pgfqpoint{0.000000in}{0.000000in}}{%
\pgfpathmoveto{\pgfqpoint{0.000000in}{0.000000in}}%
\pgfpathlineto{\pgfqpoint{0.000000in}{-0.027778in}}%
\pgfusepath{stroke,fill}%
}%
\begin{pgfscope}%
\pgfsys@transformshift{2.528623in}{3.540038in}%
\pgfsys@useobject{currentmarker}{}%
\end{pgfscope}%
\end{pgfscope}%
\begin{pgfscope}%
\pgfsetbuttcap%
\pgfsetroundjoin%
\definecolor{currentfill}{rgb}{0.000000,0.000000,0.000000}%
\pgfsetfillcolor{currentfill}%
\pgfsetlinewidth{0.602250pt}%
\definecolor{currentstroke}{rgb}{0.000000,0.000000,0.000000}%
\pgfsetstrokecolor{currentstroke}%
\pgfsetdash{}{0pt}%
\pgfsys@defobject{currentmarker}{\pgfqpoint{0.000000in}{-0.027778in}}{\pgfqpoint{0.000000in}{0.000000in}}{%
\pgfpathmoveto{\pgfqpoint{0.000000in}{0.000000in}}%
\pgfpathlineto{\pgfqpoint{0.000000in}{-0.027778in}}%
\pgfusepath{stroke,fill}%
}%
\begin{pgfscope}%
\pgfsys@transformshift{2.602121in}{3.540038in}%
\pgfsys@useobject{currentmarker}{}%
\end{pgfscope}%
\end{pgfscope}%
\begin{pgfscope}%
\pgfsetbuttcap%
\pgfsetroundjoin%
\definecolor{currentfill}{rgb}{0.000000,0.000000,0.000000}%
\pgfsetfillcolor{currentfill}%
\pgfsetlinewidth{0.602250pt}%
\definecolor{currentstroke}{rgb}{0.000000,0.000000,0.000000}%
\pgfsetstrokecolor{currentstroke}%
\pgfsetdash{}{0pt}%
\pgfsys@defobject{currentmarker}{\pgfqpoint{0.000000in}{-0.027778in}}{\pgfqpoint{0.000000in}{0.000000in}}{%
\pgfpathmoveto{\pgfqpoint{0.000000in}{0.000000in}}%
\pgfpathlineto{\pgfqpoint{0.000000in}{-0.027778in}}%
\pgfusepath{stroke,fill}%
}%
\begin{pgfscope}%
\pgfsys@transformshift{2.662172in}{3.540038in}%
\pgfsys@useobject{currentmarker}{}%
\end{pgfscope}%
\end{pgfscope}%
\begin{pgfscope}%
\pgfsetbuttcap%
\pgfsetroundjoin%
\definecolor{currentfill}{rgb}{0.000000,0.000000,0.000000}%
\pgfsetfillcolor{currentfill}%
\pgfsetlinewidth{0.602250pt}%
\definecolor{currentstroke}{rgb}{0.000000,0.000000,0.000000}%
\pgfsetstrokecolor{currentstroke}%
\pgfsetdash{}{0pt}%
\pgfsys@defobject{currentmarker}{\pgfqpoint{0.000000in}{-0.027778in}}{\pgfqpoint{0.000000in}{0.000000in}}{%
\pgfpathmoveto{\pgfqpoint{0.000000in}{0.000000in}}%
\pgfpathlineto{\pgfqpoint{0.000000in}{-0.027778in}}%
\pgfusepath{stroke,fill}%
}%
\begin{pgfscope}%
\pgfsys@transformshift{2.712946in}{3.540038in}%
\pgfsys@useobject{currentmarker}{}%
\end{pgfscope}%
\end{pgfscope}%
\begin{pgfscope}%
\pgfsetbuttcap%
\pgfsetroundjoin%
\definecolor{currentfill}{rgb}{0.000000,0.000000,0.000000}%
\pgfsetfillcolor{currentfill}%
\pgfsetlinewidth{0.602250pt}%
\definecolor{currentstroke}{rgb}{0.000000,0.000000,0.000000}%
\pgfsetstrokecolor{currentstroke}%
\pgfsetdash{}{0pt}%
\pgfsys@defobject{currentmarker}{\pgfqpoint{0.000000in}{-0.027778in}}{\pgfqpoint{0.000000in}{0.000000in}}{%
\pgfpathmoveto{\pgfqpoint{0.000000in}{0.000000in}}%
\pgfpathlineto{\pgfqpoint{0.000000in}{-0.027778in}}%
\pgfusepath{stroke,fill}%
}%
\begin{pgfscope}%
\pgfsys@transformshift{2.756927in}{3.540038in}%
\pgfsys@useobject{currentmarker}{}%
\end{pgfscope}%
\end{pgfscope}%
\begin{pgfscope}%
\pgfsetbuttcap%
\pgfsetroundjoin%
\definecolor{currentfill}{rgb}{0.000000,0.000000,0.000000}%
\pgfsetfillcolor{currentfill}%
\pgfsetlinewidth{0.602250pt}%
\definecolor{currentstroke}{rgb}{0.000000,0.000000,0.000000}%
\pgfsetstrokecolor{currentstroke}%
\pgfsetdash{}{0pt}%
\pgfsys@defobject{currentmarker}{\pgfqpoint{0.000000in}{-0.027778in}}{\pgfqpoint{0.000000in}{0.000000in}}{%
\pgfpathmoveto{\pgfqpoint{0.000000in}{0.000000in}}%
\pgfpathlineto{\pgfqpoint{0.000000in}{-0.027778in}}%
\pgfusepath{stroke,fill}%
}%
\begin{pgfscope}%
\pgfsys@transformshift{2.795722in}{3.540038in}%
\pgfsys@useobject{currentmarker}{}%
\end{pgfscope}%
\end{pgfscope}%
\begin{pgfscope}%
\pgfsetbuttcap%
\pgfsetroundjoin%
\definecolor{currentfill}{rgb}{0.000000,0.000000,0.000000}%
\pgfsetfillcolor{currentfill}%
\pgfsetlinewidth{0.602250pt}%
\definecolor{currentstroke}{rgb}{0.000000,0.000000,0.000000}%
\pgfsetstrokecolor{currentstroke}%
\pgfsetdash{}{0pt}%
\pgfsys@defobject{currentmarker}{\pgfqpoint{0.000000in}{-0.027778in}}{\pgfqpoint{0.000000in}{0.000000in}}{%
\pgfpathmoveto{\pgfqpoint{0.000000in}{0.000000in}}%
\pgfpathlineto{\pgfqpoint{0.000000in}{-0.027778in}}%
\pgfusepath{stroke,fill}%
}%
\begin{pgfscope}%
\pgfsys@transformshift{3.058729in}{3.540038in}%
\pgfsys@useobject{currentmarker}{}%
\end{pgfscope}%
\end{pgfscope}%
\begin{pgfscope}%
\pgfsetbuttcap%
\pgfsetroundjoin%
\definecolor{currentfill}{rgb}{0.000000,0.000000,0.000000}%
\pgfsetfillcolor{currentfill}%
\pgfsetlinewidth{0.602250pt}%
\definecolor{currentstroke}{rgb}{0.000000,0.000000,0.000000}%
\pgfsetstrokecolor{currentstroke}%
\pgfsetdash{}{0pt}%
\pgfsys@defobject{currentmarker}{\pgfqpoint{0.000000in}{-0.027778in}}{\pgfqpoint{0.000000in}{0.000000in}}{%
\pgfpathmoveto{\pgfqpoint{0.000000in}{0.000000in}}%
\pgfpathlineto{\pgfqpoint{0.000000in}{-0.027778in}}%
\pgfusepath{stroke,fill}%
}%
\begin{pgfscope}%
\pgfsys@transformshift{3.192278in}{3.540038in}%
\pgfsys@useobject{currentmarker}{}%
\end{pgfscope}%
\end{pgfscope}%
\begin{pgfscope}%
\pgfsetbuttcap%
\pgfsetroundjoin%
\definecolor{currentfill}{rgb}{0.000000,0.000000,0.000000}%
\pgfsetfillcolor{currentfill}%
\pgfsetlinewidth{0.602250pt}%
\definecolor{currentstroke}{rgb}{0.000000,0.000000,0.000000}%
\pgfsetstrokecolor{currentstroke}%
\pgfsetdash{}{0pt}%
\pgfsys@defobject{currentmarker}{\pgfqpoint{0.000000in}{-0.027778in}}{\pgfqpoint{0.000000in}{0.000000in}}{%
\pgfpathmoveto{\pgfqpoint{0.000000in}{0.000000in}}%
\pgfpathlineto{\pgfqpoint{0.000000in}{-0.027778in}}%
\pgfusepath{stroke,fill}%
}%
\begin{pgfscope}%
\pgfsys@transformshift{3.287033in}{3.540038in}%
\pgfsys@useobject{currentmarker}{}%
\end{pgfscope}%
\end{pgfscope}%
\begin{pgfscope}%
\pgfsetbuttcap%
\pgfsetroundjoin%
\definecolor{currentfill}{rgb}{0.000000,0.000000,0.000000}%
\pgfsetfillcolor{currentfill}%
\pgfsetlinewidth{0.602250pt}%
\definecolor{currentstroke}{rgb}{0.000000,0.000000,0.000000}%
\pgfsetstrokecolor{currentstroke}%
\pgfsetdash{}{0pt}%
\pgfsys@defobject{currentmarker}{\pgfqpoint{0.000000in}{-0.027778in}}{\pgfqpoint{0.000000in}{0.000000in}}{%
\pgfpathmoveto{\pgfqpoint{0.000000in}{0.000000in}}%
\pgfpathlineto{\pgfqpoint{0.000000in}{-0.027778in}}%
\pgfusepath{stroke,fill}%
}%
\begin{pgfscope}%
\pgfsys@transformshift{3.360531in}{3.540038in}%
\pgfsys@useobject{currentmarker}{}%
\end{pgfscope}%
\end{pgfscope}%
\begin{pgfscope}%
\pgfsetbuttcap%
\pgfsetroundjoin%
\definecolor{currentfill}{rgb}{0.000000,0.000000,0.000000}%
\pgfsetfillcolor{currentfill}%
\pgfsetlinewidth{0.602250pt}%
\definecolor{currentstroke}{rgb}{0.000000,0.000000,0.000000}%
\pgfsetstrokecolor{currentstroke}%
\pgfsetdash{}{0pt}%
\pgfsys@defobject{currentmarker}{\pgfqpoint{0.000000in}{-0.027778in}}{\pgfqpoint{0.000000in}{0.000000in}}{%
\pgfpathmoveto{\pgfqpoint{0.000000in}{0.000000in}}%
\pgfpathlineto{\pgfqpoint{0.000000in}{-0.027778in}}%
\pgfusepath{stroke,fill}%
}%
\begin{pgfscope}%
\pgfsys@transformshift{3.420583in}{3.540038in}%
\pgfsys@useobject{currentmarker}{}%
\end{pgfscope}%
\end{pgfscope}%
\begin{pgfscope}%
\pgfsetbuttcap%
\pgfsetroundjoin%
\definecolor{currentfill}{rgb}{0.000000,0.000000,0.000000}%
\pgfsetfillcolor{currentfill}%
\pgfsetlinewidth{0.602250pt}%
\definecolor{currentstroke}{rgb}{0.000000,0.000000,0.000000}%
\pgfsetstrokecolor{currentstroke}%
\pgfsetdash{}{0pt}%
\pgfsys@defobject{currentmarker}{\pgfqpoint{0.000000in}{-0.027778in}}{\pgfqpoint{0.000000in}{0.000000in}}{%
\pgfpathmoveto{\pgfqpoint{0.000000in}{0.000000in}}%
\pgfpathlineto{\pgfqpoint{0.000000in}{-0.027778in}}%
\pgfusepath{stroke,fill}%
}%
\begin{pgfscope}%
\pgfsys@transformshift{3.471356in}{3.540038in}%
\pgfsys@useobject{currentmarker}{}%
\end{pgfscope}%
\end{pgfscope}%
\begin{pgfscope}%
\pgfsetbuttcap%
\pgfsetroundjoin%
\definecolor{currentfill}{rgb}{0.000000,0.000000,0.000000}%
\pgfsetfillcolor{currentfill}%
\pgfsetlinewidth{0.602250pt}%
\definecolor{currentstroke}{rgb}{0.000000,0.000000,0.000000}%
\pgfsetstrokecolor{currentstroke}%
\pgfsetdash{}{0pt}%
\pgfsys@defobject{currentmarker}{\pgfqpoint{0.000000in}{-0.027778in}}{\pgfqpoint{0.000000in}{0.000000in}}{%
\pgfpathmoveto{\pgfqpoint{0.000000in}{0.000000in}}%
\pgfpathlineto{\pgfqpoint{0.000000in}{-0.027778in}}%
\pgfusepath{stroke,fill}%
}%
\begin{pgfscope}%
\pgfsys@transformshift{3.515337in}{3.540038in}%
\pgfsys@useobject{currentmarker}{}%
\end{pgfscope}%
\end{pgfscope}%
\begin{pgfscope}%
\pgfsetbuttcap%
\pgfsetroundjoin%
\definecolor{currentfill}{rgb}{0.000000,0.000000,0.000000}%
\pgfsetfillcolor{currentfill}%
\pgfsetlinewidth{0.602250pt}%
\definecolor{currentstroke}{rgb}{0.000000,0.000000,0.000000}%
\pgfsetstrokecolor{currentstroke}%
\pgfsetdash{}{0pt}%
\pgfsys@defobject{currentmarker}{\pgfqpoint{0.000000in}{-0.027778in}}{\pgfqpoint{0.000000in}{0.000000in}}{%
\pgfpathmoveto{\pgfqpoint{0.000000in}{0.000000in}}%
\pgfpathlineto{\pgfqpoint{0.000000in}{-0.027778in}}%
\pgfusepath{stroke,fill}%
}%
\begin{pgfscope}%
\pgfsys@transformshift{3.554132in}{3.540038in}%
\pgfsys@useobject{currentmarker}{}%
\end{pgfscope}%
\end{pgfscope}%
\begin{pgfscope}%
\pgfsetbuttcap%
\pgfsetroundjoin%
\definecolor{currentfill}{rgb}{0.000000,0.000000,0.000000}%
\pgfsetfillcolor{currentfill}%
\pgfsetlinewidth{0.602250pt}%
\definecolor{currentstroke}{rgb}{0.000000,0.000000,0.000000}%
\pgfsetstrokecolor{currentstroke}%
\pgfsetdash{}{0pt}%
\pgfsys@defobject{currentmarker}{\pgfqpoint{0.000000in}{-0.027778in}}{\pgfqpoint{0.000000in}{0.000000in}}{%
\pgfpathmoveto{\pgfqpoint{0.000000in}{0.000000in}}%
\pgfpathlineto{\pgfqpoint{0.000000in}{-0.027778in}}%
\pgfusepath{stroke,fill}%
}%
\begin{pgfscope}%
\pgfsys@transformshift{3.817139in}{3.540038in}%
\pgfsys@useobject{currentmarker}{}%
\end{pgfscope}%
\end{pgfscope}%
\begin{pgfscope}%
\pgfsetbuttcap%
\pgfsetroundjoin%
\definecolor{currentfill}{rgb}{0.000000,0.000000,0.000000}%
\pgfsetfillcolor{currentfill}%
\pgfsetlinewidth{0.602250pt}%
\definecolor{currentstroke}{rgb}{0.000000,0.000000,0.000000}%
\pgfsetstrokecolor{currentstroke}%
\pgfsetdash{}{0pt}%
\pgfsys@defobject{currentmarker}{\pgfqpoint{0.000000in}{-0.027778in}}{\pgfqpoint{0.000000in}{0.000000in}}{%
\pgfpathmoveto{\pgfqpoint{0.000000in}{0.000000in}}%
\pgfpathlineto{\pgfqpoint{0.000000in}{-0.027778in}}%
\pgfusepath{stroke,fill}%
}%
\begin{pgfscope}%
\pgfsys@transformshift{3.950689in}{3.540038in}%
\pgfsys@useobject{currentmarker}{}%
\end{pgfscope}%
\end{pgfscope}%
\begin{pgfscope}%
\pgfsetbuttcap%
\pgfsetroundjoin%
\definecolor{currentfill}{rgb}{0.000000,0.000000,0.000000}%
\pgfsetfillcolor{currentfill}%
\pgfsetlinewidth{0.602250pt}%
\definecolor{currentstroke}{rgb}{0.000000,0.000000,0.000000}%
\pgfsetstrokecolor{currentstroke}%
\pgfsetdash{}{0pt}%
\pgfsys@defobject{currentmarker}{\pgfqpoint{0.000000in}{-0.027778in}}{\pgfqpoint{0.000000in}{0.000000in}}{%
\pgfpathmoveto{\pgfqpoint{0.000000in}{0.000000in}}%
\pgfpathlineto{\pgfqpoint{0.000000in}{-0.027778in}}%
\pgfusepath{stroke,fill}%
}%
\begin{pgfscope}%
\pgfsys@transformshift{4.045443in}{3.540038in}%
\pgfsys@useobject{currentmarker}{}%
\end{pgfscope}%
\end{pgfscope}%
\begin{pgfscope}%
\pgfsetbuttcap%
\pgfsetroundjoin%
\definecolor{currentfill}{rgb}{0.000000,0.000000,0.000000}%
\pgfsetfillcolor{currentfill}%
\pgfsetlinewidth{0.602250pt}%
\definecolor{currentstroke}{rgb}{0.000000,0.000000,0.000000}%
\pgfsetstrokecolor{currentstroke}%
\pgfsetdash{}{0pt}%
\pgfsys@defobject{currentmarker}{\pgfqpoint{0.000000in}{-0.027778in}}{\pgfqpoint{0.000000in}{0.000000in}}{%
\pgfpathmoveto{\pgfqpoint{0.000000in}{0.000000in}}%
\pgfpathlineto{\pgfqpoint{0.000000in}{-0.027778in}}%
\pgfusepath{stroke,fill}%
}%
\begin{pgfscope}%
\pgfsys@transformshift{4.118941in}{3.540038in}%
\pgfsys@useobject{currentmarker}{}%
\end{pgfscope}%
\end{pgfscope}%
\begin{pgfscope}%
\pgfsetbuttcap%
\pgfsetroundjoin%
\definecolor{currentfill}{rgb}{0.000000,0.000000,0.000000}%
\pgfsetfillcolor{currentfill}%
\pgfsetlinewidth{0.602250pt}%
\definecolor{currentstroke}{rgb}{0.000000,0.000000,0.000000}%
\pgfsetstrokecolor{currentstroke}%
\pgfsetdash{}{0pt}%
\pgfsys@defobject{currentmarker}{\pgfqpoint{0.000000in}{-0.027778in}}{\pgfqpoint{0.000000in}{0.000000in}}{%
\pgfpathmoveto{\pgfqpoint{0.000000in}{0.000000in}}%
\pgfpathlineto{\pgfqpoint{0.000000in}{-0.027778in}}%
\pgfusepath{stroke,fill}%
}%
\begin{pgfscope}%
\pgfsys@transformshift{4.178993in}{3.540038in}%
\pgfsys@useobject{currentmarker}{}%
\end{pgfscope}%
\end{pgfscope}%
\begin{pgfscope}%
\pgfsetbuttcap%
\pgfsetroundjoin%
\definecolor{currentfill}{rgb}{0.000000,0.000000,0.000000}%
\pgfsetfillcolor{currentfill}%
\pgfsetlinewidth{0.602250pt}%
\definecolor{currentstroke}{rgb}{0.000000,0.000000,0.000000}%
\pgfsetstrokecolor{currentstroke}%
\pgfsetdash{}{0pt}%
\pgfsys@defobject{currentmarker}{\pgfqpoint{0.000000in}{-0.027778in}}{\pgfqpoint{0.000000in}{0.000000in}}{%
\pgfpathmoveto{\pgfqpoint{0.000000in}{0.000000in}}%
\pgfpathlineto{\pgfqpoint{0.000000in}{-0.027778in}}%
\pgfusepath{stroke,fill}%
}%
\begin{pgfscope}%
\pgfsys@transformshift{4.229766in}{3.540038in}%
\pgfsys@useobject{currentmarker}{}%
\end{pgfscope}%
\end{pgfscope}%
\begin{pgfscope}%
\pgfsetbuttcap%
\pgfsetroundjoin%
\definecolor{currentfill}{rgb}{0.000000,0.000000,0.000000}%
\pgfsetfillcolor{currentfill}%
\pgfsetlinewidth{0.803000pt}%
\definecolor{currentstroke}{rgb}{0.000000,0.000000,0.000000}%
\pgfsetstrokecolor{currentstroke}%
\pgfsetdash{}{0pt}%
\pgfsys@defobject{currentmarker}{\pgfqpoint{-0.048611in}{0.000000in}}{\pgfqpoint{-0.000000in}{0.000000in}}{%
\pgfpathmoveto{\pgfqpoint{-0.000000in}{0.000000in}}%
\pgfpathlineto{\pgfqpoint{-0.048611in}{0.000000in}}%
\pgfusepath{stroke,fill}%
}%
\begin{pgfscope}%
\pgfsys@transformshift{1.313604in}{3.540038in}%
\pgfsys@useobject{currentmarker}{}%
\end{pgfscope}%
\end{pgfscope}%
\begin{pgfscope}%
\definecolor{textcolor}{rgb}{0.000000,0.000000,0.000000}%
\pgfsetstrokecolor{textcolor}%
\pgfsetfillcolor{textcolor}%
\pgftext[x=1.106314in, y=3.456705in, left, base]{\color{textcolor}{\rmfamily\fontsize{16.000000}{19.200000}\selectfont\catcode`\^=\active\def^{\ifmmode\sp\else\^{}\fi}\catcode`\%=\active\def%{\%}0}}%
\end{pgfscope}%
\begin{pgfscope}%
\pgfsetbuttcap%
\pgfsetroundjoin%
\definecolor{currentfill}{rgb}{0.000000,0.000000,0.000000}%
\pgfsetfillcolor{currentfill}%
\pgfsetlinewidth{0.803000pt}%
\definecolor{currentstroke}{rgb}{0.000000,0.000000,0.000000}%
\pgfsetstrokecolor{currentstroke}%
\pgfsetdash{}{0pt}%
\pgfsys@defobject{currentmarker}{\pgfqpoint{-0.048611in}{0.000000in}}{\pgfqpoint{-0.000000in}{0.000000in}}{%
\pgfpathmoveto{\pgfqpoint{-0.000000in}{0.000000in}}%
\pgfpathlineto{\pgfqpoint{-0.048611in}{0.000000in}}%
\pgfusepath{stroke,fill}%
}%
\begin{pgfscope}%
\pgfsys@transformshift{1.313604in}{4.793442in}%
\pgfsys@useobject{currentmarker}{}%
\end{pgfscope}%
\end{pgfscope}%
\begin{pgfscope}%
\definecolor{textcolor}{rgb}{0.000000,0.000000,0.000000}%
\pgfsetstrokecolor{textcolor}%
\pgfsetfillcolor{textcolor}%
\pgftext[x=1.106314in, y=4.710109in, left, base]{\color{textcolor}{\rmfamily\fontsize{16.000000}{19.200000}\selectfont\catcode`\^=\active\def^{\ifmmode\sp\else\^{}\fi}\catcode`\%=\active\def%{\%}1}}%
\end{pgfscope}%
\begin{pgfscope}%
\pgfpathrectangle{\pgfqpoint{1.313604in}{3.540038in}}{\pgfqpoint{2.916161in}{1.253404in}}%
\pgfusepath{clip}%
\pgfsetrectcap%
\pgfsetroundjoin%
\pgfsetlinewidth{1.003750pt}%
\definecolor{currentstroke}{rgb}{0.000000,0.000000,0.000000}%
\pgfsetstrokecolor{currentstroke}%
\pgfsetdash{}{0pt}%
\pgfpathmoveto{\pgfqpoint{1.313191in}{4.793071in}}%
\pgfpathlineto{\pgfqpoint{1.600846in}{4.791313in}}%
\pgfpathlineto{\pgfqpoint{1.744674in}{4.788348in}}%
\pgfpathlineto{\pgfqpoint{1.842941in}{4.784201in}}%
\pgfpathlineto{\pgfqpoint{1.917981in}{4.778888in}}%
\pgfpathlineto{\pgfqpoint{1.978728in}{4.772432in}}%
\pgfpathlineto{\pgfqpoint{2.030095in}{4.764801in}}%
\pgfpathlineto{\pgfqpoint{2.074762in}{4.755967in}}%
\pgfpathlineto{\pgfqpoint{2.114516in}{4.745867in}}%
\pgfpathlineto{\pgfqpoint{2.150249in}{4.734523in}}%
\pgfpathlineto{\pgfqpoint{2.182856in}{4.721873in}}%
\pgfpathlineto{\pgfqpoint{2.213229in}{4.707717in}}%
\pgfpathlineto{\pgfqpoint{2.241370in}{4.692187in}}%
\pgfpathlineto{\pgfqpoint{2.268170in}{4.674878in}}%
\pgfpathlineto{\pgfqpoint{2.293630in}{4.655812in}}%
\pgfpathlineto{\pgfqpoint{2.318197in}{4.634654in}}%
\pgfpathlineto{\pgfqpoint{2.342317in}{4.610896in}}%
\pgfpathlineto{\pgfqpoint{2.365543in}{4.584885in}}%
\pgfpathlineto{\pgfqpoint{2.388324in}{4.556055in}}%
\pgfpathlineto{\pgfqpoint{2.411104in}{4.523598in}}%
\pgfpathlineto{\pgfqpoint{2.433884in}{4.487174in}}%
\pgfpathlineto{\pgfqpoint{2.456664in}{4.446449in}}%
\pgfpathlineto{\pgfqpoint{2.479444in}{4.401113in}}%
\pgfpathlineto{\pgfqpoint{2.502671in}{4.349865in}}%
\pgfpathlineto{\pgfqpoint{2.526344in}{4.292212in}}%
\pgfpathlineto{\pgfqpoint{2.551358in}{4.225276in}}%
\pgfpathlineto{\pgfqpoint{2.577711in}{4.148246in}}%
\pgfpathlineto{\pgfqpoint{2.607191in}{4.054940in}}%
\pgfpathlineto{\pgfqpoint{2.644712in}{3.928174in}}%
\pgfpathlineto{\pgfqpoint{2.705012in}{3.723967in}}%
\pgfpathlineto{\pgfqpoint{2.726899in}{3.658291in}}%
\pgfpathlineto{\pgfqpoint{2.743425in}{3.615227in}}%
\pgfpathlineto{\pgfqpoint{2.757272in}{3.585034in}}%
\pgfpathlineto{\pgfqpoint{2.768886in}{3.564824in}}%
\pgfpathlineto{\pgfqpoint{2.778712in}{3.551987in}}%
\pgfpathlineto{\pgfqpoint{2.787199in}{3.544447in}}%
\pgfpathlineto{\pgfqpoint{2.794346in}{3.540898in}}%
\pgfpathlineto{\pgfqpoint{2.800599in}{3.540053in}}%
\pgfpathlineto{\pgfqpoint{2.806852in}{3.541452in}}%
\pgfpathlineto{\pgfqpoint{2.813106in}{3.545219in}}%
\pgfpathlineto{\pgfqpoint{2.819806in}{3.552023in}}%
\pgfpathlineto{\pgfqpoint{2.826953in}{3.562591in}}%
\pgfpathlineto{\pgfqpoint{2.834993in}{3.578760in}}%
\pgfpathlineto{\pgfqpoint{2.843479in}{3.600961in}}%
\pgfpathlineto{\pgfqpoint{2.852859in}{3.631868in}}%
\pgfpathlineto{\pgfqpoint{2.863133in}{3.673611in}}%
\pgfpathlineto{\pgfqpoint{2.874299in}{3.728477in}}%
\pgfpathlineto{\pgfqpoint{2.886806in}{3.801566in}}%
\pgfpathlineto{\pgfqpoint{2.900653in}{3.896172in}}%
\pgfpathlineto{\pgfqpoint{2.916733in}{4.022048in}}%
\pgfpathlineto{\pgfqpoint{2.937726in}{4.205459in}}%
\pgfpathlineto{\pgfqpoint{2.979267in}{4.572175in}}%
\pgfpathlineto{\pgfqpoint{2.992667in}{4.668532in}}%
\pgfpathlineto{\pgfqpoint{3.002940in}{4.727281in}}%
\pgfpathlineto{\pgfqpoint{3.010980in}{4.761660in}}%
\pgfpathlineto{\pgfqpoint{3.017680in}{4.781285in}}%
\pgfpathlineto{\pgfqpoint{3.022594in}{4.789936in}}%
\pgfpathlineto{\pgfqpoint{3.026614in}{4.793161in}}%
\pgfpathlineto{\pgfqpoint{3.029740in}{4.793166in}}%
\pgfpathlineto{\pgfqpoint{3.032867in}{4.790912in}}%
\pgfpathlineto{\pgfqpoint{3.036440in}{4.785497in}}%
\pgfpathlineto{\pgfqpoint{3.040907in}{4.774365in}}%
\pgfpathlineto{\pgfqpoint{3.046267in}{4.754471in}}%
\pgfpathlineto{\pgfqpoint{3.052520in}{4.722118in}}%
\pgfpathlineto{\pgfqpoint{3.059667in}{4.673086in}}%
\pgfpathlineto{\pgfqpoint{3.068154in}{4.598591in}}%
\pgfpathlineto{\pgfqpoint{3.077980in}{4.491847in}}%
\pgfpathlineto{\pgfqpoint{3.090487in}{4.329394in}}%
\pgfpathlineto{\pgfqpoint{3.110587in}{4.032942in}}%
\pgfpathlineto{\pgfqpoint{3.128901in}{3.774378in}}%
\pgfpathlineto{\pgfqpoint{3.139174in}{3.658150in}}%
\pgfpathlineto{\pgfqpoint{3.146767in}{3.594193in}}%
\pgfpathlineto{\pgfqpoint{3.152574in}{3.560876in}}%
\pgfpathlineto{\pgfqpoint{3.157041in}{3.545610in}}%
\pgfpathlineto{\pgfqpoint{3.160168in}{3.540668in}}%
\pgfpathlineto{\pgfqpoint{3.162401in}{3.540151in}}%
\pgfpathlineto{\pgfqpoint{3.164634in}{3.542209in}}%
\pgfpathlineto{\pgfqpoint{3.167314in}{3.548136in}}%
\pgfpathlineto{\pgfqpoint{3.170888in}{3.561995in}}%
\pgfpathlineto{\pgfqpoint{3.175354in}{3.588973in}}%
\pgfpathlineto{\pgfqpoint{3.180714in}{3.635414in}}%
\pgfpathlineto{\pgfqpoint{3.187414in}{3.714265in}}%
\pgfpathlineto{\pgfqpoint{3.195454in}{3.836552in}}%
\pgfpathlineto{\pgfqpoint{3.206174in}{4.036400in}}%
\pgfpathlineto{\pgfqpoint{3.237441in}{4.643544in}}%
\pgfpathlineto{\pgfqpoint{3.244588in}{4.732117in}}%
\pgfpathlineto{\pgfqpoint{3.249948in}{4.774387in}}%
\pgfpathlineto{\pgfqpoint{3.253968in}{4.790560in}}%
\pgfpathlineto{\pgfqpoint{3.256201in}{4.793403in}}%
\pgfpathlineto{\pgfqpoint{3.257988in}{4.792417in}}%
\pgfpathlineto{\pgfqpoint{3.260221in}{4.787039in}}%
\pgfpathlineto{\pgfqpoint{3.263348in}{4.771681in}}%
\pgfpathlineto{\pgfqpoint{3.267368in}{4.738500in}}%
\pgfpathlineto{\pgfqpoint{3.272281in}{4.677869in}}%
\pgfpathlineto{\pgfqpoint{3.278535in}{4.570977in}}%
\pgfpathlineto{\pgfqpoint{3.286575in}{4.392804in}}%
\pgfpathlineto{\pgfqpoint{3.301762in}{3.993160in}}%
\pgfpathlineto{\pgfqpoint{3.312928in}{3.726961in}}%
\pgfpathlineto{\pgfqpoint{3.319628in}{3.612740in}}%
\pgfpathlineto{\pgfqpoint{3.324542in}{3.560732in}}%
\pgfpathlineto{\pgfqpoint{3.328115in}{3.542528in}}%
\pgfpathlineto{\pgfqpoint{3.330348in}{3.540130in}}%
\pgfpathlineto{\pgfqpoint{3.331688in}{3.542103in}}%
\pgfpathlineto{\pgfqpoint{3.333922in}{3.551154in}}%
\pgfpathlineto{\pgfqpoint{3.337048in}{3.575966in}}%
\pgfpathlineto{\pgfqpoint{3.341068in}{3.628351in}}%
\pgfpathlineto{\pgfqpoint{3.346429in}{3.731846in}}%
\pgfpathlineto{\pgfqpoint{3.353575in}{3.919946in}}%
\pgfpathlineto{\pgfqpoint{3.367422in}{4.363724in}}%
\pgfpathlineto{\pgfqpoint{3.376802in}{4.628408in}}%
\pgfpathlineto{\pgfqpoint{3.382609in}{4.738814in}}%
\pgfpathlineto{\pgfqpoint{3.386629in}{4.781646in}}%
\pgfpathlineto{\pgfqpoint{3.389309in}{4.792892in}}%
\pgfpathlineto{\pgfqpoint{3.390649in}{4.793063in}}%
\pgfpathlineto{\pgfqpoint{3.391989in}{4.789533in}}%
\pgfpathlineto{\pgfqpoint{3.394222in}{4.775368in}}%
\pgfpathlineto{\pgfqpoint{3.397349in}{4.738249in}}%
\pgfpathlineto{\pgfqpoint{3.401369in}{4.662024in}}%
\pgfpathlineto{\pgfqpoint{3.407175in}{4.501925in}}%
\pgfpathlineto{\pgfqpoint{3.415662in}{4.195592in}}%
\pgfpathlineto{\pgfqpoint{3.429062in}{3.717199in}}%
\pgfpathlineto{\pgfqpoint{3.434422in}{3.595200in}}%
\pgfpathlineto{\pgfqpoint{3.437996in}{3.551015in}}%
\pgfpathlineto{\pgfqpoint{3.440229in}{3.540517in}}%
\pgfpathlineto{\pgfqpoint{3.441122in}{3.540170in}}%
\pgfpathlineto{\pgfqpoint{3.442462in}{3.543844in}}%
\pgfpathlineto{\pgfqpoint{3.444696in}{3.561203in}}%
\pgfpathlineto{\pgfqpoint{3.447822in}{3.608808in}}%
\pgfpathlineto{\pgfqpoint{3.451842in}{3.707643in}}%
\pgfpathlineto{\pgfqpoint{3.457649in}{3.912554in}}%
\pgfpathlineto{\pgfqpoint{3.478643in}{4.726833in}}%
\pgfpathlineto{\pgfqpoint{3.482216in}{4.781423in}}%
\pgfpathlineto{\pgfqpoint{3.484449in}{4.793222in}}%
\pgfpathlineto{\pgfqpoint{3.485343in}{4.792898in}}%
\pgfpathlineto{\pgfqpoint{3.486683in}{4.786928in}}%
\pgfpathlineto{\pgfqpoint{3.488916in}{4.762356in}}%
\pgfpathlineto{\pgfqpoint{3.492043in}{4.698043in}}%
\pgfpathlineto{\pgfqpoint{3.496509in}{4.551473in}}%
\pgfpathlineto{\pgfqpoint{3.503209in}{4.245383in}}%
\pgfpathlineto{\pgfqpoint{3.515269in}{3.689612in}}%
\pgfpathlineto{\pgfqpoint{3.519736in}{3.573654in}}%
\pgfpathlineto{\pgfqpoint{3.522416in}{3.543214in}}%
\pgfpathlineto{\pgfqpoint{3.523756in}{3.540101in}}%
\pgfpathlineto{\pgfqpoint{3.524649in}{3.542630in}}%
\pgfpathlineto{\pgfqpoint{3.526436in}{3.558793in}}%
\pgfpathlineto{\pgfqpoint{3.529116in}{3.610357in}}%
\pgfpathlineto{\pgfqpoint{3.533136in}{3.744414in}}%
\pgfpathlineto{\pgfqpoint{3.538943in}{4.028882in}}%
\pgfpathlineto{\pgfqpoint{3.551450in}{4.667116in}}%
\pgfpathlineto{\pgfqpoint{3.555470in}{4.771025in}}%
\pgfpathlineto{\pgfqpoint{3.557703in}{4.792447in}}%
\pgfpathlineto{\pgfqpoint{3.558596in}{4.793182in}}%
\pgfpathlineto{\pgfqpoint{3.559936in}{4.785733in}}%
\pgfpathlineto{\pgfqpoint{3.562170in}{4.750619in}}%
\pgfpathlineto{\pgfqpoint{3.565296in}{4.656007in}}%
\pgfpathlineto{\pgfqpoint{3.569763in}{4.443633in}}%
\pgfpathlineto{\pgfqpoint{3.585843in}{3.589741in}}%
\pgfpathlineto{\pgfqpoint{3.588523in}{3.544688in}}%
\pgfpathlineto{\pgfqpoint{3.589863in}{3.540145in}}%
\pgfpathlineto{\pgfqpoint{3.590756in}{3.543993in}}%
\pgfpathlineto{\pgfqpoint{3.592543in}{3.568209in}}%
\pgfpathlineto{\pgfqpoint{3.595223in}{3.644515in}}%
\pgfpathlineto{\pgfqpoint{3.599243in}{3.837314in}}%
\pgfpathlineto{\pgfqpoint{3.607730in}{4.396319in}}%
\pgfpathlineto{\pgfqpoint{3.613537in}{4.702164in}}%
\pgfpathlineto{\pgfqpoint{3.616663in}{4.781910in}}%
\pgfpathlineto{\pgfqpoint{3.618450in}{4.793398in}}%
\pgfpathlineto{\pgfqpoint{3.619343in}{4.789353in}}%
\pgfpathlineto{\pgfqpoint{3.621130in}{4.761633in}}%
\pgfpathlineto{\pgfqpoint{3.623810in}{4.672916in}}%
\pgfpathlineto{\pgfqpoint{3.627830in}{4.450301in}}%
\pgfpathlineto{\pgfqpoint{3.641677in}{3.583451in}}%
\pgfpathlineto{\pgfqpoint{3.644357in}{3.540639in}}%
\pgfpathlineto{\pgfqpoint{3.644803in}{3.540080in}}%
\pgfpathlineto{\pgfqpoint{3.645697in}{3.544736in}}%
\pgfpathlineto{\pgfqpoint{3.647483in}{3.577036in}}%
\pgfpathlineto{\pgfqpoint{3.650163in}{3.680150in}}%
\pgfpathlineto{\pgfqpoint{3.654183in}{3.934812in}}%
\pgfpathlineto{\pgfqpoint{3.665797in}{4.733368in}}%
\pgfpathlineto{\pgfqpoint{3.668477in}{4.791142in}}%
\pgfpathlineto{\pgfqpoint{3.668924in}{4.793228in}}%
\pgfpathlineto{\pgfqpoint{3.669370in}{4.793090in}}%
\pgfpathlineto{\pgfqpoint{3.670264in}{4.786115in}}%
\pgfpathlineto{\pgfqpoint{3.672050in}{4.745636in}}%
\pgfpathlineto{\pgfqpoint{3.674730in}{4.622864in}}%
\pgfpathlineto{\pgfqpoint{3.679197in}{4.293887in}}%
\pgfpathlineto{\pgfqpoint{3.688130in}{3.625480in}}%
\pgfpathlineto{\pgfqpoint{3.690810in}{3.546854in}}%
\pgfpathlineto{\pgfqpoint{3.691704in}{3.540164in}}%
\pgfpathlineto{\pgfqpoint{3.692150in}{3.540642in}}%
\pgfpathlineto{\pgfqpoint{3.693044in}{3.549284in}}%
\pgfpathlineto{\pgfqpoint{3.694830in}{3.596892in}}%
\pgfpathlineto{\pgfqpoint{3.697510in}{3.738159in}}%
\pgfpathlineto{\pgfqpoint{3.702424in}{4.146357in}}%
\pgfpathlineto{\pgfqpoint{3.709570in}{4.705577in}}%
\pgfpathlineto{\pgfqpoint{3.712250in}{4.788183in}}%
\pgfpathlineto{\pgfqpoint{3.713144in}{4.793429in}}%
\pgfpathlineto{\pgfqpoint{3.713590in}{4.791692in}}%
\pgfpathlineto{\pgfqpoint{3.714930in}{4.769041in}}%
\pgfpathlineto{\pgfqpoint{3.717164in}{4.675390in}}%
\pgfpathlineto{\pgfqpoint{3.720737in}{4.406068in}}%
\pgfpathlineto{\pgfqpoint{3.730564in}{3.590659in}}%
\pgfpathlineto{\pgfqpoint{3.732797in}{3.540647in}}%
\pgfpathlineto{\pgfqpoint{3.733244in}{3.540290in}}%
\pgfpathlineto{\pgfqpoint{3.734137in}{3.549444in}}%
\pgfpathlineto{\pgfqpoint{3.735924in}{3.606553in}}%
\pgfpathlineto{\pgfqpoint{3.738604in}{3.779651in}}%
\pgfpathlineto{\pgfqpoint{3.744411in}{4.352076in}}%
\pgfpathlineto{\pgfqpoint{3.749324in}{4.733805in}}%
\pgfpathlineto{\pgfqpoint{3.751557in}{4.792386in}}%
\pgfpathlineto{\pgfqpoint{3.752004in}{4.793333in}}%
\pgfpathlineto{\pgfqpoint{3.752897in}{4.784169in}}%
\pgfpathlineto{\pgfqpoint{3.754684in}{4.722402in}}%
\pgfpathlineto{\pgfqpoint{3.757364in}{4.532971in}}%
\pgfpathlineto{\pgfqpoint{3.768977in}{3.545450in}}%
\pgfpathlineto{\pgfqpoint{3.769871in}{3.540325in}}%
\pgfpathlineto{\pgfqpoint{3.770764in}{3.551607in}}%
\pgfpathlineto{\pgfqpoint{3.772551in}{3.622302in}}%
\pgfpathlineto{\pgfqpoint{3.775677in}{3.878246in}}%
\pgfpathlineto{\pgfqpoint{3.785057in}{4.766681in}}%
\pgfpathlineto{\pgfqpoint{3.786397in}{4.792981in}}%
\pgfpathlineto{\pgfqpoint{3.786844in}{4.792754in}}%
\pgfpathlineto{\pgfqpoint{3.787737in}{4.778657in}}%
\pgfpathlineto{\pgfqpoint{3.789524in}{4.697522in}}%
\pgfpathlineto{\pgfqpoint{3.792651in}{4.414003in}}%
\pgfpathlineto{\pgfqpoint{3.801138in}{3.569147in}}%
\pgfpathlineto{\pgfqpoint{3.802478in}{3.540497in}}%
\pgfpathlineto{\pgfqpoint{3.802924in}{3.540859in}}%
\pgfpathlineto{\pgfqpoint{3.803818in}{3.556615in}}%
\pgfpathlineto{\pgfqpoint{3.805604in}{3.646202in}}%
\pgfpathlineto{\pgfqpoint{3.808731in}{3.954619in}}%
\pgfpathlineto{\pgfqpoint{3.816324in}{4.755685in}}%
\pgfpathlineto{\pgfqpoint{3.818111in}{4.793244in}}%
\pgfpathlineto{\pgfqpoint{3.819004in}{4.779289in}}%
\pgfpathlineto{\pgfqpoint{3.820791in}{4.687488in}}%
\pgfpathlineto{\pgfqpoint{3.823918in}{4.361460in}}%
\pgfpathlineto{\pgfqpoint{3.831064in}{3.576579in}}%
\pgfpathlineto{\pgfqpoint{3.832851in}{3.540760in}}%
\pgfpathlineto{\pgfqpoint{3.833744in}{3.558655in}}%
\pgfpathlineto{\pgfqpoint{3.835531in}{3.663581in}}%
\pgfpathlineto{\pgfqpoint{3.838658in}{4.019711in}}%
\pgfpathlineto{\pgfqpoint{3.844911in}{4.744676in}}%
\pgfpathlineto{\pgfqpoint{3.846698in}{4.793405in}}%
\pgfpathlineto{\pgfqpoint{3.847591in}{4.778926in}}%
\pgfpathlineto{\pgfqpoint{3.849378in}{4.674263in}}%
\pgfpathlineto{\pgfqpoint{3.852504in}{4.302731in}}%
\pgfpathlineto{\pgfqpoint{3.858311in}{3.595651in}}%
\pgfpathlineto{\pgfqpoint{3.860098in}{3.540038in}}%
\pgfpathlineto{\pgfqpoint{3.860991in}{3.554298in}}%
\pgfpathlineto{\pgfqpoint{3.862778in}{3.664848in}}%
\pgfpathlineto{\pgfqpoint{3.865905in}{4.058572in}}%
\pgfpathlineto{\pgfqpoint{3.871265in}{4.735664in}}%
\pgfpathlineto{\pgfqpoint{3.873051in}{4.793416in}}%
\pgfpathlineto{\pgfqpoint{3.873945in}{4.776792in}}%
\pgfpathlineto{\pgfqpoint{3.875731in}{4.655399in}}%
\pgfpathlineto{\pgfqpoint{3.878858in}{4.234731in}}%
\pgfpathlineto{\pgfqpoint{3.883771in}{3.597071in}}%
\pgfpathlineto{\pgfqpoint{3.885558in}{3.540304in}}%
\pgfpathlineto{\pgfqpoint{3.886451in}{3.561030in}}%
\pgfpathlineto{\pgfqpoint{3.888238in}{3.696549in}}%
\pgfpathlineto{\pgfqpoint{3.891811in}{4.219739in}}%
\pgfpathlineto{\pgfqpoint{3.895831in}{4.737900in}}%
\pgfpathlineto{\pgfqpoint{3.897618in}{4.792596in}}%
\pgfpathlineto{\pgfqpoint{3.898511in}{4.767092in}}%
\pgfpathlineto{\pgfqpoint{3.900298in}{4.616197in}}%
\pgfpathlineto{\pgfqpoint{3.904318in}{3.987584in}}%
\pgfpathlineto{\pgfqpoint{3.907891in}{3.568227in}}%
\pgfpathlineto{\pgfqpoint{3.908785in}{3.540967in}}%
\pgfpathlineto{\pgfqpoint{3.909231in}{3.541534in}}%
\pgfpathlineto{\pgfqpoint{3.910125in}{3.571213in}}%
\pgfpathlineto{\pgfqpoint{3.911911in}{3.736452in}}%
\pgfpathlineto{\pgfqpoint{3.919951in}{4.792453in}}%
\pgfpathlineto{\pgfqpoint{3.920398in}{4.791835in}}%
\pgfpathlineto{\pgfqpoint{3.921291in}{4.760074in}}%
\pgfpathlineto{\pgfqpoint{3.923078in}{4.584009in}}%
\pgfpathlineto{\pgfqpoint{3.931118in}{3.540832in}}%
\pgfpathlineto{\pgfqpoint{3.932012in}{3.570869in}}%
\pgfpathlineto{\pgfqpoint{3.933798in}{3.751438in}}%
\pgfpathlineto{\pgfqpoint{3.941392in}{4.793437in}}%
\pgfpathlineto{\pgfqpoint{3.941838in}{4.787968in}}%
\pgfpathlineto{\pgfqpoint{3.943178in}{4.703417in}}%
\pgfpathlineto{\pgfqpoint{3.945858in}{4.292721in}}%
\pgfpathlineto{\pgfqpoint{3.950772in}{3.558261in}}%
\pgfpathlineto{\pgfqpoint{3.951665in}{3.540519in}}%
\pgfpathlineto{\pgfqpoint{3.952112in}{3.550153in}}%
\pgfpathlineto{\pgfqpoint{3.953452in}{3.650664in}}%
\pgfpathlineto{\pgfqpoint{3.956132in}{4.094458in}}%
\pgfpathlineto{\pgfqpoint{3.960598in}{4.773800in}}%
\pgfpathlineto{\pgfqpoint{3.961492in}{4.792980in}}%
\pgfpathlineto{\pgfqpoint{3.961938in}{4.782932in}}%
\pgfpathlineto{\pgfqpoint{3.963278in}{4.676950in}}%
\pgfpathlineto{\pgfqpoint{3.965958in}{4.212754in}}%
\pgfpathlineto{\pgfqpoint{3.969978in}{3.570726in}}%
\pgfpathlineto{\pgfqpoint{3.970872in}{3.540123in}}%
\pgfpathlineto{\pgfqpoint{3.971318in}{3.545551in}}%
\pgfpathlineto{\pgfqpoint{3.972658in}{3.643051in}}%
\pgfpathlineto{\pgfqpoint{3.975338in}{4.112788in}}%
\pgfpathlineto{\pgfqpoint{3.979358in}{4.767538in}}%
\pgfpathlineto{\pgfqpoint{3.980252in}{4.793342in}}%
\pgfpathlineto{\pgfqpoint{3.980698in}{4.784263in}}%
\pgfpathlineto{\pgfqpoint{3.982038in}{4.671980in}}%
\pgfpathlineto{\pgfqpoint{3.985165in}{4.074896in}}%
\pgfpathlineto{\pgfqpoint{3.988738in}{3.549212in}}%
\pgfpathlineto{\pgfqpoint{3.989185in}{3.540098in}}%
\pgfpathlineto{\pgfqpoint{3.989632in}{3.546517in}}%
\pgfpathlineto{\pgfqpoint{3.990972in}{3.656088in}}%
\pgfpathlineto{\pgfqpoint{3.993652in}{4.168173in}}%
\pgfpathlineto{\pgfqpoint{3.997225in}{4.768440in}}%
\pgfpathlineto{\pgfqpoint{3.998119in}{4.792940in}}%
\pgfpathlineto{\pgfqpoint{3.998565in}{4.780687in}}%
\pgfpathlineto{\pgfqpoint{3.999905in}{4.650243in}}%
\pgfpathlineto{\pgfqpoint{4.003479in}{3.906327in}}%
\pgfpathlineto{\pgfqpoint{4.006159in}{3.547133in}}%
\pgfpathlineto{\pgfqpoint{4.006605in}{3.540113in}}%
\pgfpathlineto{\pgfqpoint{4.007052in}{3.550359in}}%
\pgfpathlineto{\pgfqpoint{4.008392in}{3.680207in}}%
\pgfpathlineto{\pgfqpoint{4.011519in}{4.343715in}}%
\pgfpathlineto{\pgfqpoint{4.014645in}{4.790424in}}%
\pgfpathlineto{\pgfqpoint{4.015092in}{4.791808in}}%
\pgfpathlineto{\pgfqpoint{4.015985in}{4.740572in}}%
\pgfpathlineto{\pgfqpoint{4.017772in}{4.451307in}}%
\pgfpathlineto{\pgfqpoint{4.022685in}{3.545909in}}%
\pgfpathlineto{\pgfqpoint{4.023132in}{3.540482in}}%
\pgfpathlineto{\pgfqpoint{4.024025in}{3.586515in}}%
\pgfpathlineto{\pgfqpoint{4.025812in}{3.877041in}}%
\pgfpathlineto{\pgfqpoint{4.030725in}{4.790366in}}%
\pgfpathlineto{\pgfqpoint{4.031172in}{4.791448in}}%
\pgfpathlineto{\pgfqpoint{4.032065in}{4.734184in}}%
\pgfpathlineto{\pgfqpoint{4.034299in}{4.310666in}}%
\pgfpathlineto{\pgfqpoint{4.038319in}{3.548631in}}%
\pgfpathlineto{\pgfqpoint{4.038765in}{3.540132in}}%
\pgfpathlineto{\pgfqpoint{4.039212in}{3.552611in}}%
\pgfpathlineto{\pgfqpoint{4.040552in}{3.709119in}}%
\pgfpathlineto{\pgfqpoint{4.046359in}{4.793220in}}%
\pgfpathlineto{\pgfqpoint{4.046805in}{4.779114in}}%
\pgfpathlineto{\pgfqpoint{4.048145in}{4.612875in}}%
\pgfpathlineto{\pgfqpoint{4.053505in}{3.542160in}}%
\pgfpathlineto{\pgfqpoint{4.054399in}{3.568292in}}%
\pgfpathlineto{\pgfqpoint{4.056185in}{3.866950in}}%
\pgfpathlineto{\pgfqpoint{4.060652in}{4.788574in}}%
\pgfpathlineto{\pgfqpoint{4.061099in}{4.791853in}}%
\pgfpathlineto{\pgfqpoint{4.061992in}{4.727210in}}%
\pgfpathlineto{\pgfqpoint{4.064226in}{4.242094in}}%
\pgfpathlineto{\pgfqpoint{4.067799in}{3.543117in}}%
\pgfpathlineto{\pgfqpoint{4.068246in}{3.543237in}}%
\pgfpathlineto{\pgfqpoint{4.069139in}{3.617473in}}%
\pgfpathlineto{\pgfqpoint{4.071372in}{4.128902in}}%
\pgfpathlineto{\pgfqpoint{4.074499in}{4.779873in}}%
\pgfpathlineto{\pgfqpoint{4.074946in}{4.793437in}}%
\pgfpathlineto{\pgfqpoint{4.075392in}{4.780885in}}%
\pgfpathlineto{\pgfqpoint{4.076732in}{4.596084in}}%
\pgfpathlineto{\pgfqpoint{4.081646in}{3.540728in}}%
\pgfpathlineto{\pgfqpoint{4.082092in}{3.548236in}}%
\pgfpathlineto{\pgfqpoint{4.083432in}{3.725697in}}%
\pgfpathlineto{\pgfqpoint{4.088346in}{4.793230in}}%
\pgfpathlineto{\pgfqpoint{4.088792in}{4.782506in}}%
\pgfpathlineto{\pgfqpoint{4.090132in}{4.590512in}}%
\pgfpathlineto{\pgfqpoint{4.095046in}{3.541160in}}%
\pgfpathlineto{\pgfqpoint{4.095492in}{3.563986in}}%
\pgfpathlineto{\pgfqpoint{4.096832in}{3.792899in}}%
\pgfpathlineto{\pgfqpoint{4.101299in}{4.793413in}}%
\pgfpathlineto{\pgfqpoint{4.102192in}{4.735340in}}%
\pgfpathlineto{\pgfqpoint{4.103979in}{4.319278in}}%
\pgfpathlineto{\pgfqpoint{4.107552in}{3.540167in}}%
\pgfpathlineto{\pgfqpoint{4.108446in}{3.597589in}}%
\pgfpathlineto{\pgfqpoint{4.110232in}{4.023325in}}%
\pgfpathlineto{\pgfqpoint{4.113359in}{4.780852in}}%
\pgfpathlineto{\pgfqpoint{4.113806in}{4.793176in}}%
\pgfpathlineto{\pgfqpoint{4.114699in}{4.719704in}}%
\pgfpathlineto{\pgfqpoint{4.116932in}{4.118699in}}%
\pgfpathlineto{\pgfqpoint{4.119612in}{3.542841in}}%
\pgfpathlineto{\pgfqpoint{4.120059in}{3.546147in}}%
\pgfpathlineto{\pgfqpoint{4.121399in}{3.750638in}}%
\pgfpathlineto{\pgfqpoint{4.125866in}{4.791736in}}%
\pgfpathlineto{\pgfqpoint{4.126759in}{4.700424in}}%
\pgfpathlineto{\pgfqpoint{4.128993in}{4.053477in}}%
\pgfpathlineto{\pgfqpoint{4.131673in}{3.541265in}}%
\pgfpathlineto{\pgfqpoint{4.132566in}{3.632161in}}%
\pgfpathlineto{\pgfqpoint{4.134799in}{4.291464in}}%
\pgfpathlineto{\pgfqpoint{4.137033in}{4.787305in}}%
\pgfpathlineto{\pgfqpoint{4.137479in}{4.789857in}}%
\pgfpathlineto{\pgfqpoint{4.138373in}{4.683206in}}%
\pgfpathlineto{\pgfqpoint{4.142839in}{3.540627in}}%
\pgfpathlineto{\pgfqpoint{4.143733in}{3.605167in}}%
\pgfpathlineto{\pgfqpoint{4.145519in}{4.104067in}}%
\pgfpathlineto{\pgfqpoint{4.148199in}{4.786899in}}%
\pgfpathlineto{\pgfqpoint{4.148646in}{4.789587in}}%
\pgfpathlineto{\pgfqpoint{4.149539in}{4.675628in}}%
\pgfpathlineto{\pgfqpoint{4.154006in}{3.541312in}}%
\pgfpathlineto{\pgfqpoint{4.154453in}{3.572626in}}%
\pgfpathlineto{\pgfqpoint{4.155793in}{3.886084in}}%
\pgfpathlineto{\pgfqpoint{4.159366in}{4.792023in}}%
\pgfpathlineto{\pgfqpoint{4.160259in}{4.685215in}}%
\pgfpathlineto{\pgfqpoint{4.164726in}{3.544732in}}%
\pgfpathlineto{\pgfqpoint{4.165173in}{3.587390in}}%
\pgfpathlineto{\pgfqpoint{4.166959in}{4.103604in}}%
\pgfpathlineto{\pgfqpoint{4.169639in}{4.792870in}}%
\pgfpathlineto{\pgfqpoint{4.170086in}{4.776928in}}%
\pgfpathlineto{\pgfqpoint{4.171426in}{4.478204in}}%
\pgfpathlineto{\pgfqpoint{4.174999in}{3.544086in}}%
\pgfpathlineto{\pgfqpoint{4.175893in}{3.674986in}}%
\pgfpathlineto{\pgfqpoint{4.179913in}{4.792867in}}%
\pgfpathlineto{\pgfqpoint{4.180359in}{4.760780in}}%
\pgfpathlineto{\pgfqpoint{4.181699in}{4.410750in}}%
\pgfpathlineto{\pgfqpoint{4.184826in}{3.540141in}}%
\pgfpathlineto{\pgfqpoint{4.185719in}{3.645923in}}%
\pgfpathlineto{\pgfqpoint{4.189740in}{4.792935in}}%
\pgfpathlineto{\pgfqpoint{4.190633in}{4.676612in}}%
\pgfpathlineto{\pgfqpoint{4.194653in}{3.543817in}}%
\pgfpathlineto{\pgfqpoint{4.195100in}{3.590534in}}%
\pgfpathlineto{\pgfqpoint{4.196886in}{4.175912in}}%
\pgfpathlineto{\pgfqpoint{4.199120in}{4.791883in}}%
\pgfpathlineto{\pgfqpoint{4.199566in}{4.777254in}}%
\pgfpathlineto{\pgfqpoint{4.200906in}{4.436841in}}%
\pgfpathlineto{\pgfqpoint{4.204033in}{3.542681in}}%
\pgfpathlineto{\pgfqpoint{4.204926in}{3.686570in}}%
\pgfpathlineto{\pgfqpoint{4.208500in}{4.793439in}}%
\pgfpathlineto{\pgfqpoint{4.208946in}{4.764716in}}%
\pgfpathlineto{\pgfqpoint{4.210286in}{4.377802in}}%
\pgfpathlineto{\pgfqpoint{4.212966in}{3.541030in}}%
\pgfpathlineto{\pgfqpoint{4.213413in}{3.560282in}}%
\pgfpathlineto{\pgfqpoint{4.214753in}{3.934299in}}%
\pgfpathlineto{\pgfqpoint{4.217433in}{4.792020in}}%
\pgfpathlineto{\pgfqpoint{4.217880in}{4.774268in}}%
\pgfpathlineto{\pgfqpoint{4.219220in}{4.395732in}}%
\pgfpathlineto{\pgfqpoint{4.221900in}{3.540439in}}%
\pgfpathlineto{\pgfqpoint{4.222793in}{3.650946in}}%
\pgfpathlineto{\pgfqpoint{4.226366in}{4.792960in}}%
\pgfpathlineto{\pgfqpoint{4.226813in}{4.752544in}}%
\pgfpathlineto{\pgfqpoint{4.228600in}{4.108467in}}%
\pgfpathlineto{\pgfqpoint{4.229940in}{3.616024in}}%
\pgfpathlineto{\pgfqpoint{4.229940in}{3.616024in}}%
\pgfusepath{stroke}%
\end{pgfscope}%
\begin{pgfscope}%
\pgfpathrectangle{\pgfqpoint{1.313604in}{3.540038in}}{\pgfqpoint{2.916161in}{1.253404in}}%
\pgfusepath{clip}%
\pgfsetrectcap%
\pgfsetroundjoin%
\pgfsetlinewidth{1.003750pt}%
\definecolor{currentstroke}{rgb}{1.000000,0.000000,0.000000}%
\pgfsetstrokecolor{currentstroke}%
\pgfsetdash{}{0pt}%
\pgfpathmoveto{\pgfqpoint{1.313191in}{3.540410in}}%
\pgfpathlineto{\pgfqpoint{1.600846in}{3.542167in}}%
\pgfpathlineto{\pgfqpoint{1.744674in}{3.545132in}}%
\pgfpathlineto{\pgfqpoint{1.842941in}{3.549279in}}%
\pgfpathlineto{\pgfqpoint{1.917981in}{3.554592in}}%
\pgfpathlineto{\pgfqpoint{1.978728in}{3.561048in}}%
\pgfpathlineto{\pgfqpoint{2.030095in}{3.568679in}}%
\pgfpathlineto{\pgfqpoint{2.074762in}{3.577513in}}%
\pgfpathlineto{\pgfqpoint{2.114516in}{3.587613in}}%
\pgfpathlineto{\pgfqpoint{2.150249in}{3.598957in}}%
\pgfpathlineto{\pgfqpoint{2.182856in}{3.611607in}}%
\pgfpathlineto{\pgfqpoint{2.213229in}{3.625763in}}%
\pgfpathlineto{\pgfqpoint{2.241370in}{3.641293in}}%
\pgfpathlineto{\pgfqpoint{2.268170in}{3.658602in}}%
\pgfpathlineto{\pgfqpoint{2.293630in}{3.677668in}}%
\pgfpathlineto{\pgfqpoint{2.318197in}{3.698826in}}%
\pgfpathlineto{\pgfqpoint{2.342317in}{3.722584in}}%
\pgfpathlineto{\pgfqpoint{2.365543in}{3.748595in}}%
\pgfpathlineto{\pgfqpoint{2.388324in}{3.777426in}}%
\pgfpathlineto{\pgfqpoint{2.411104in}{3.809882in}}%
\pgfpathlineto{\pgfqpoint{2.433884in}{3.846306in}}%
\pgfpathlineto{\pgfqpoint{2.456664in}{3.887031in}}%
\pgfpathlineto{\pgfqpoint{2.479444in}{3.932368in}}%
\pgfpathlineto{\pgfqpoint{2.502671in}{3.983615in}}%
\pgfpathlineto{\pgfqpoint{2.526344in}{4.041268in}}%
\pgfpathlineto{\pgfqpoint{2.551358in}{4.108204in}}%
\pgfpathlineto{\pgfqpoint{2.577711in}{4.185234in}}%
\pgfpathlineto{\pgfqpoint{2.607191in}{4.278540in}}%
\pgfpathlineto{\pgfqpoint{2.644712in}{4.405306in}}%
\pgfpathlineto{\pgfqpoint{2.705012in}{4.609513in}}%
\pgfpathlineto{\pgfqpoint{2.726899in}{4.675189in}}%
\pgfpathlineto{\pgfqpoint{2.743425in}{4.718253in}}%
\pgfpathlineto{\pgfqpoint{2.757272in}{4.748446in}}%
\pgfpathlineto{\pgfqpoint{2.768886in}{4.768656in}}%
\pgfpathlineto{\pgfqpoint{2.778712in}{4.781493in}}%
\pgfpathlineto{\pgfqpoint{2.787199in}{4.789033in}}%
\pgfpathlineto{\pgfqpoint{2.794346in}{4.792582in}}%
\pgfpathlineto{\pgfqpoint{2.800599in}{4.793427in}}%
\pgfpathlineto{\pgfqpoint{2.806852in}{4.792029in}}%
\pgfpathlineto{\pgfqpoint{2.813106in}{4.788261in}}%
\pgfpathlineto{\pgfqpoint{2.819806in}{4.781458in}}%
\pgfpathlineto{\pgfqpoint{2.826953in}{4.770890in}}%
\pgfpathlineto{\pgfqpoint{2.834993in}{4.754720in}}%
\pgfpathlineto{\pgfqpoint{2.843479in}{4.732519in}}%
\pgfpathlineto{\pgfqpoint{2.852859in}{4.701612in}}%
\pgfpathlineto{\pgfqpoint{2.863133in}{4.659869in}}%
\pgfpathlineto{\pgfqpoint{2.874299in}{4.605003in}}%
\pgfpathlineto{\pgfqpoint{2.886806in}{4.531914in}}%
\pgfpathlineto{\pgfqpoint{2.900653in}{4.437308in}}%
\pgfpathlineto{\pgfqpoint{2.916733in}{4.311432in}}%
\pgfpathlineto{\pgfqpoint{2.937726in}{4.128021in}}%
\pgfpathlineto{\pgfqpoint{2.979267in}{3.761305in}}%
\pgfpathlineto{\pgfqpoint{2.992667in}{3.664949in}}%
\pgfpathlineto{\pgfqpoint{3.002940in}{3.606199in}}%
\pgfpathlineto{\pgfqpoint{3.010980in}{3.571821in}}%
\pgfpathlineto{\pgfqpoint{3.017680in}{3.552196in}}%
\pgfpathlineto{\pgfqpoint{3.022594in}{3.543544in}}%
\pgfpathlineto{\pgfqpoint{3.026614in}{3.540320in}}%
\pgfpathlineto{\pgfqpoint{3.029740in}{3.540314in}}%
\pgfpathlineto{\pgfqpoint{3.032867in}{3.542568in}}%
\pgfpathlineto{\pgfqpoint{3.036440in}{3.547984in}}%
\pgfpathlineto{\pgfqpoint{3.040907in}{3.559115in}}%
\pgfpathlineto{\pgfqpoint{3.046267in}{3.579009in}}%
\pgfpathlineto{\pgfqpoint{3.052520in}{3.611362in}}%
\pgfpathlineto{\pgfqpoint{3.059667in}{3.660394in}}%
\pgfpathlineto{\pgfqpoint{3.068154in}{3.734889in}}%
\pgfpathlineto{\pgfqpoint{3.077980in}{3.841633in}}%
\pgfpathlineto{\pgfqpoint{3.090487in}{4.004086in}}%
\pgfpathlineto{\pgfqpoint{3.110587in}{4.300538in}}%
\pgfpathlineto{\pgfqpoint{3.128901in}{4.559102in}}%
\pgfpathlineto{\pgfqpoint{3.139174in}{4.675330in}}%
\pgfpathlineto{\pgfqpoint{3.146767in}{4.739288in}}%
\pgfpathlineto{\pgfqpoint{3.152574in}{4.772604in}}%
\pgfpathlineto{\pgfqpoint{3.157041in}{4.787870in}}%
\pgfpathlineto{\pgfqpoint{3.160168in}{4.792813in}}%
\pgfpathlineto{\pgfqpoint{3.162401in}{4.793329in}}%
\pgfpathlineto{\pgfqpoint{3.164634in}{4.791272in}}%
\pgfpathlineto{\pgfqpoint{3.167314in}{4.785344in}}%
\pgfpathlineto{\pgfqpoint{3.170888in}{4.771485in}}%
\pgfpathlineto{\pgfqpoint{3.175354in}{4.744508in}}%
\pgfpathlineto{\pgfqpoint{3.180714in}{4.698066in}}%
\pgfpathlineto{\pgfqpoint{3.187414in}{4.619216in}}%
\pgfpathlineto{\pgfqpoint{3.195454in}{4.496928in}}%
\pgfpathlineto{\pgfqpoint{3.206174in}{4.297081in}}%
\pgfpathlineto{\pgfqpoint{3.237441in}{3.689936in}}%
\pgfpathlineto{\pgfqpoint{3.244588in}{3.601363in}}%
\pgfpathlineto{\pgfqpoint{3.249948in}{3.559093in}}%
\pgfpathlineto{\pgfqpoint{3.253968in}{3.542921in}}%
\pgfpathlineto{\pgfqpoint{3.256201in}{3.540077in}}%
\pgfpathlineto{\pgfqpoint{3.257988in}{3.541063in}}%
\pgfpathlineto{\pgfqpoint{3.260221in}{3.546442in}}%
\pgfpathlineto{\pgfqpoint{3.263348in}{3.561800in}}%
\pgfpathlineto{\pgfqpoint{3.267368in}{3.594981in}}%
\pgfpathlineto{\pgfqpoint{3.272281in}{3.655611in}}%
\pgfpathlineto{\pgfqpoint{3.278535in}{3.762503in}}%
\pgfpathlineto{\pgfqpoint{3.286575in}{3.940676in}}%
\pgfpathlineto{\pgfqpoint{3.301762in}{4.340320in}}%
\pgfpathlineto{\pgfqpoint{3.312928in}{4.606519in}}%
\pgfpathlineto{\pgfqpoint{3.319628in}{4.720740in}}%
\pgfpathlineto{\pgfqpoint{3.324542in}{4.772748in}}%
\pgfpathlineto{\pgfqpoint{3.328115in}{4.790952in}}%
\pgfpathlineto{\pgfqpoint{3.330348in}{4.793350in}}%
\pgfpathlineto{\pgfqpoint{3.331688in}{4.791377in}}%
\pgfpathlineto{\pgfqpoint{3.333922in}{4.782326in}}%
\pgfpathlineto{\pgfqpoint{3.337048in}{4.757514in}}%
\pgfpathlineto{\pgfqpoint{3.341068in}{4.705129in}}%
\pgfpathlineto{\pgfqpoint{3.346429in}{4.601634in}}%
\pgfpathlineto{\pgfqpoint{3.353575in}{4.413535in}}%
\pgfpathlineto{\pgfqpoint{3.367422in}{3.969756in}}%
\pgfpathlineto{\pgfqpoint{3.376802in}{3.705073in}}%
\pgfpathlineto{\pgfqpoint{3.382609in}{3.594666in}}%
\pgfpathlineto{\pgfqpoint{3.386629in}{3.551834in}}%
\pgfpathlineto{\pgfqpoint{3.389309in}{3.540589in}}%
\pgfpathlineto{\pgfqpoint{3.390649in}{3.540417in}}%
\pgfpathlineto{\pgfqpoint{3.391989in}{3.543947in}}%
\pgfpathlineto{\pgfqpoint{3.394222in}{3.558112in}}%
\pgfpathlineto{\pgfqpoint{3.397349in}{3.595232in}}%
\pgfpathlineto{\pgfqpoint{3.401369in}{3.671456in}}%
\pgfpathlineto{\pgfqpoint{3.407175in}{3.831556in}}%
\pgfpathlineto{\pgfqpoint{3.415662in}{4.137888in}}%
\pgfpathlineto{\pgfqpoint{3.429062in}{4.616281in}}%
\pgfpathlineto{\pgfqpoint{3.434422in}{4.738281in}}%
\pgfpathlineto{\pgfqpoint{3.437996in}{4.782465in}}%
\pgfpathlineto{\pgfqpoint{3.440229in}{4.792963in}}%
\pgfpathlineto{\pgfqpoint{3.441122in}{4.793310in}}%
\pgfpathlineto{\pgfqpoint{3.442462in}{4.789636in}}%
\pgfpathlineto{\pgfqpoint{3.444696in}{4.772277in}}%
\pgfpathlineto{\pgfqpoint{3.447822in}{4.724672in}}%
\pgfpathlineto{\pgfqpoint{3.451842in}{4.625837in}}%
\pgfpathlineto{\pgfqpoint{3.457649in}{4.420926in}}%
\pgfpathlineto{\pgfqpoint{3.478643in}{3.606647in}}%
\pgfpathlineto{\pgfqpoint{3.482216in}{3.552057in}}%
\pgfpathlineto{\pgfqpoint{3.484449in}{3.540258in}}%
\pgfpathlineto{\pgfqpoint{3.485343in}{3.540583in}}%
\pgfpathlineto{\pgfqpoint{3.486683in}{3.546553in}}%
\pgfpathlineto{\pgfqpoint{3.488916in}{3.571124in}}%
\pgfpathlineto{\pgfqpoint{3.492043in}{3.635437in}}%
\pgfpathlineto{\pgfqpoint{3.496509in}{3.782007in}}%
\pgfpathlineto{\pgfqpoint{3.503209in}{4.088098in}}%
\pgfpathlineto{\pgfqpoint{3.515269in}{4.643868in}}%
\pgfpathlineto{\pgfqpoint{3.519736in}{4.759826in}}%
\pgfpathlineto{\pgfqpoint{3.522416in}{4.790267in}}%
\pgfpathlineto{\pgfqpoint{3.523756in}{4.793379in}}%
\pgfpathlineto{\pgfqpoint{3.524649in}{4.790850in}}%
\pgfpathlineto{\pgfqpoint{3.526436in}{4.774687in}}%
\pgfpathlineto{\pgfqpoint{3.529116in}{4.723123in}}%
\pgfpathlineto{\pgfqpoint{3.533136in}{4.589066in}}%
\pgfpathlineto{\pgfqpoint{3.538943in}{4.304598in}}%
\pgfpathlineto{\pgfqpoint{3.551450in}{3.666365in}}%
\pgfpathlineto{\pgfqpoint{3.555470in}{3.562455in}}%
\pgfpathlineto{\pgfqpoint{3.557703in}{3.541033in}}%
\pgfpathlineto{\pgfqpoint{3.558596in}{3.540298in}}%
\pgfpathlineto{\pgfqpoint{3.559936in}{3.547747in}}%
\pgfpathlineto{\pgfqpoint{3.562170in}{3.582861in}}%
\pgfpathlineto{\pgfqpoint{3.565296in}{3.677473in}}%
\pgfpathlineto{\pgfqpoint{3.569763in}{3.889847in}}%
\pgfpathlineto{\pgfqpoint{3.585843in}{4.743739in}}%
\pgfpathlineto{\pgfqpoint{3.588523in}{4.788792in}}%
\pgfpathlineto{\pgfqpoint{3.589863in}{4.793335in}}%
\pgfpathlineto{\pgfqpoint{3.590756in}{4.789488in}}%
\pgfpathlineto{\pgfqpoint{3.592543in}{4.765271in}}%
\pgfpathlineto{\pgfqpoint{3.595223in}{4.688965in}}%
\pgfpathlineto{\pgfqpoint{3.599243in}{4.496167in}}%
\pgfpathlineto{\pgfqpoint{3.607730in}{3.937161in}}%
\pgfpathlineto{\pgfqpoint{3.613537in}{3.631316in}}%
\pgfpathlineto{\pgfqpoint{3.616663in}{3.551570in}}%
\pgfpathlineto{\pgfqpoint{3.618450in}{3.540082in}}%
\pgfpathlineto{\pgfqpoint{3.619343in}{3.544127in}}%
\pgfpathlineto{\pgfqpoint{3.621130in}{3.571847in}}%
\pgfpathlineto{\pgfqpoint{3.623810in}{3.660564in}}%
\pgfpathlineto{\pgfqpoint{3.627830in}{3.883179in}}%
\pgfpathlineto{\pgfqpoint{3.641677in}{4.750029in}}%
\pgfpathlineto{\pgfqpoint{3.644357in}{4.792841in}}%
\pgfpathlineto{\pgfqpoint{3.644803in}{4.793400in}}%
\pgfpathlineto{\pgfqpoint{3.645697in}{4.788744in}}%
\pgfpathlineto{\pgfqpoint{3.647483in}{4.756444in}}%
\pgfpathlineto{\pgfqpoint{3.650163in}{4.653331in}}%
\pgfpathlineto{\pgfqpoint{3.654183in}{4.398668in}}%
\pgfpathlineto{\pgfqpoint{3.665797in}{3.600112in}}%
\pgfpathlineto{\pgfqpoint{3.668477in}{3.542338in}}%
\pgfpathlineto{\pgfqpoint{3.668924in}{3.540252in}}%
\pgfpathlineto{\pgfqpoint{3.669370in}{3.540390in}}%
\pgfpathlineto{\pgfqpoint{3.670264in}{3.547365in}}%
\pgfpathlineto{\pgfqpoint{3.672050in}{3.587844in}}%
\pgfpathlineto{\pgfqpoint{3.674730in}{3.710616in}}%
\pgfpathlineto{\pgfqpoint{3.679197in}{4.039593in}}%
\pgfpathlineto{\pgfqpoint{3.688130in}{4.708001in}}%
\pgfpathlineto{\pgfqpoint{3.690810in}{4.786626in}}%
\pgfpathlineto{\pgfqpoint{3.691704in}{4.793316in}}%
\pgfpathlineto{\pgfqpoint{3.692150in}{4.792839in}}%
\pgfpathlineto{\pgfqpoint{3.693044in}{4.784196in}}%
\pgfpathlineto{\pgfqpoint{3.694830in}{4.736589in}}%
\pgfpathlineto{\pgfqpoint{3.697510in}{4.595321in}}%
\pgfpathlineto{\pgfqpoint{3.702424in}{4.187123in}}%
\pgfpathlineto{\pgfqpoint{3.709570in}{3.627903in}}%
\pgfpathlineto{\pgfqpoint{3.712250in}{3.545297in}}%
\pgfpathlineto{\pgfqpoint{3.713144in}{3.540052in}}%
\pgfpathlineto{\pgfqpoint{3.713590in}{3.541788in}}%
\pgfpathlineto{\pgfqpoint{3.714930in}{3.564439in}}%
\pgfpathlineto{\pgfqpoint{3.717164in}{3.658090in}}%
\pgfpathlineto{\pgfqpoint{3.720737in}{3.927412in}}%
\pgfpathlineto{\pgfqpoint{3.730564in}{4.742822in}}%
\pgfpathlineto{\pgfqpoint{3.732797in}{4.792833in}}%
\pgfpathlineto{\pgfqpoint{3.733244in}{4.793190in}}%
\pgfpathlineto{\pgfqpoint{3.734137in}{4.784037in}}%
\pgfpathlineto{\pgfqpoint{3.735924in}{4.726927in}}%
\pgfpathlineto{\pgfqpoint{3.738604in}{4.553829in}}%
\pgfpathlineto{\pgfqpoint{3.744411in}{3.981404in}}%
\pgfpathlineto{\pgfqpoint{3.749324in}{3.599676in}}%
\pgfpathlineto{\pgfqpoint{3.751557in}{3.541094in}}%
\pgfpathlineto{\pgfqpoint{3.752004in}{3.540147in}}%
\pgfpathlineto{\pgfqpoint{3.752897in}{3.549312in}}%
\pgfpathlineto{\pgfqpoint{3.754684in}{3.611078in}}%
\pgfpathlineto{\pgfqpoint{3.757364in}{3.800509in}}%
\pgfpathlineto{\pgfqpoint{3.768977in}{4.788030in}}%
\pgfpathlineto{\pgfqpoint{3.769871in}{4.793155in}}%
\pgfpathlineto{\pgfqpoint{3.770764in}{4.781874in}}%
\pgfpathlineto{\pgfqpoint{3.772551in}{4.711178in}}%
\pgfpathlineto{\pgfqpoint{3.775677in}{4.455234in}}%
\pgfpathlineto{\pgfqpoint{3.785057in}{3.566800in}}%
\pgfpathlineto{\pgfqpoint{3.786397in}{3.540499in}}%
\pgfpathlineto{\pgfqpoint{3.786844in}{3.540726in}}%
\pgfpathlineto{\pgfqpoint{3.787737in}{3.554823in}}%
\pgfpathlineto{\pgfqpoint{3.789524in}{3.635958in}}%
\pgfpathlineto{\pgfqpoint{3.792651in}{3.919477in}}%
\pgfpathlineto{\pgfqpoint{3.801138in}{4.764333in}}%
\pgfpathlineto{\pgfqpoint{3.802478in}{4.792983in}}%
\pgfpathlineto{\pgfqpoint{3.802924in}{4.792621in}}%
\pgfpathlineto{\pgfqpoint{3.803818in}{4.776865in}}%
\pgfpathlineto{\pgfqpoint{3.805604in}{4.687278in}}%
\pgfpathlineto{\pgfqpoint{3.808731in}{4.378862in}}%
\pgfpathlineto{\pgfqpoint{3.816324in}{3.577795in}}%
\pgfpathlineto{\pgfqpoint{3.818111in}{3.540236in}}%
\pgfpathlineto{\pgfqpoint{3.819004in}{3.554191in}}%
\pgfpathlineto{\pgfqpoint{3.820791in}{3.645992in}}%
\pgfpathlineto{\pgfqpoint{3.823918in}{3.972020in}}%
\pgfpathlineto{\pgfqpoint{3.831064in}{4.756901in}}%
\pgfpathlineto{\pgfqpoint{3.832851in}{4.792721in}}%
\pgfpathlineto{\pgfqpoint{3.833744in}{4.774825in}}%
\pgfpathlineto{\pgfqpoint{3.835531in}{4.669900in}}%
\pgfpathlineto{\pgfqpoint{3.838658in}{4.313769in}}%
\pgfpathlineto{\pgfqpoint{3.844911in}{3.588804in}}%
\pgfpathlineto{\pgfqpoint{3.846698in}{3.540075in}}%
\pgfpathlineto{\pgfqpoint{3.847591in}{3.554555in}}%
\pgfpathlineto{\pgfqpoint{3.849378in}{3.659218in}}%
\pgfpathlineto{\pgfqpoint{3.852504in}{4.030750in}}%
\pgfpathlineto{\pgfqpoint{3.858311in}{4.737829in}}%
\pgfpathlineto{\pgfqpoint{3.860098in}{4.793442in}}%
\pgfpathlineto{\pgfqpoint{3.860991in}{4.779182in}}%
\pgfpathlineto{\pgfqpoint{3.862778in}{4.668632in}}%
\pgfpathlineto{\pgfqpoint{3.865905in}{4.274908in}}%
\pgfpathlineto{\pgfqpoint{3.871265in}{3.597816in}}%
\pgfpathlineto{\pgfqpoint{3.873051in}{3.540064in}}%
\pgfpathlineto{\pgfqpoint{3.873945in}{3.556688in}}%
\pgfpathlineto{\pgfqpoint{3.875731in}{3.678081in}}%
\pgfpathlineto{\pgfqpoint{3.878858in}{4.098749in}}%
\pgfpathlineto{\pgfqpoint{3.883771in}{4.736409in}}%
\pgfpathlineto{\pgfqpoint{3.885558in}{4.793177in}}%
\pgfpathlineto{\pgfqpoint{3.886451in}{4.772450in}}%
\pgfpathlineto{\pgfqpoint{3.888238in}{4.636931in}}%
\pgfpathlineto{\pgfqpoint{3.891811in}{4.113741in}}%
\pgfpathlineto{\pgfqpoint{3.895831in}{3.595580in}}%
\pgfpathlineto{\pgfqpoint{3.897618in}{3.540884in}}%
\pgfpathlineto{\pgfqpoint{3.898511in}{3.566388in}}%
\pgfpathlineto{\pgfqpoint{3.900298in}{3.717283in}}%
\pgfpathlineto{\pgfqpoint{3.904318in}{4.345896in}}%
\pgfpathlineto{\pgfqpoint{3.907891in}{4.765253in}}%
\pgfpathlineto{\pgfqpoint{3.908785in}{4.792513in}}%
\pgfpathlineto{\pgfqpoint{3.909231in}{4.791946in}}%
\pgfpathlineto{\pgfqpoint{3.910125in}{4.762268in}}%
\pgfpathlineto{\pgfqpoint{3.911911in}{4.597028in}}%
\pgfpathlineto{\pgfqpoint{3.919951in}{3.541027in}}%
\pgfpathlineto{\pgfqpoint{3.920398in}{3.541645in}}%
\pgfpathlineto{\pgfqpoint{3.921291in}{3.573406in}}%
\pgfpathlineto{\pgfqpoint{3.923078in}{3.749471in}}%
\pgfpathlineto{\pgfqpoint{3.931118in}{4.792648in}}%
\pgfpathlineto{\pgfqpoint{3.932012in}{4.762612in}}%
\pgfpathlineto{\pgfqpoint{3.933798in}{4.582042in}}%
\pgfpathlineto{\pgfqpoint{3.941392in}{3.540044in}}%
\pgfpathlineto{\pgfqpoint{3.941838in}{3.545512in}}%
\pgfpathlineto{\pgfqpoint{3.943178in}{3.630063in}}%
\pgfpathlineto{\pgfqpoint{3.945858in}{4.040759in}}%
\pgfpathlineto{\pgfqpoint{3.950772in}{4.775219in}}%
\pgfpathlineto{\pgfqpoint{3.951665in}{4.792961in}}%
\pgfpathlineto{\pgfqpoint{3.952112in}{4.783327in}}%
\pgfpathlineto{\pgfqpoint{3.953452in}{4.682817in}}%
\pgfpathlineto{\pgfqpoint{3.956132in}{4.239023in}}%
\pgfpathlineto{\pgfqpoint{3.960598in}{3.559680in}}%
\pgfpathlineto{\pgfqpoint{3.961492in}{3.540500in}}%
\pgfpathlineto{\pgfqpoint{3.961938in}{3.550548in}}%
\pgfpathlineto{\pgfqpoint{3.963278in}{3.656530in}}%
\pgfpathlineto{\pgfqpoint{3.965958in}{4.120726in}}%
\pgfpathlineto{\pgfqpoint{3.969978in}{4.762754in}}%
\pgfpathlineto{\pgfqpoint{3.970872in}{4.793357in}}%
\pgfpathlineto{\pgfqpoint{3.971318in}{4.787930in}}%
\pgfpathlineto{\pgfqpoint{3.972658in}{4.690429in}}%
\pgfpathlineto{\pgfqpoint{3.975338in}{4.220692in}}%
\pgfpathlineto{\pgfqpoint{3.979358in}{3.565943in}}%
\pgfpathlineto{\pgfqpoint{3.980252in}{3.540138in}}%
\pgfpathlineto{\pgfqpoint{3.980698in}{3.549217in}}%
\pgfpathlineto{\pgfqpoint{3.982038in}{3.661500in}}%
\pgfpathlineto{\pgfqpoint{3.985165in}{4.258584in}}%
\pgfpathlineto{\pgfqpoint{3.988738in}{4.784268in}}%
\pgfpathlineto{\pgfqpoint{3.989185in}{4.793382in}}%
\pgfpathlineto{\pgfqpoint{3.989632in}{4.786963in}}%
\pgfpathlineto{\pgfqpoint{3.990972in}{4.677392in}}%
\pgfpathlineto{\pgfqpoint{3.993652in}{4.165307in}}%
\pgfpathlineto{\pgfqpoint{3.997225in}{3.565040in}}%
\pgfpathlineto{\pgfqpoint{3.998119in}{3.540541in}}%
\pgfpathlineto{\pgfqpoint{3.998565in}{3.552793in}}%
\pgfpathlineto{\pgfqpoint{3.999905in}{3.683237in}}%
\pgfpathlineto{\pgfqpoint{4.003479in}{4.427153in}}%
\pgfpathlineto{\pgfqpoint{4.006159in}{4.786347in}}%
\pgfpathlineto{\pgfqpoint{4.006605in}{4.793367in}}%
\pgfpathlineto{\pgfqpoint{4.007052in}{4.783121in}}%
\pgfpathlineto{\pgfqpoint{4.008392in}{4.653274in}}%
\pgfpathlineto{\pgfqpoint{4.011519in}{3.989765in}}%
\pgfpathlineto{\pgfqpoint{4.014645in}{3.543056in}}%
\pgfpathlineto{\pgfqpoint{4.015092in}{3.541672in}}%
\pgfpathlineto{\pgfqpoint{4.015985in}{3.592909in}}%
\pgfpathlineto{\pgfqpoint{4.017772in}{3.882173in}}%
\pgfpathlineto{\pgfqpoint{4.022685in}{4.787571in}}%
\pgfpathlineto{\pgfqpoint{4.023132in}{4.792999in}}%
\pgfpathlineto{\pgfqpoint{4.024025in}{4.746965in}}%
\pgfpathlineto{\pgfqpoint{4.025812in}{4.456440in}}%
\pgfpathlineto{\pgfqpoint{4.030725in}{3.543115in}}%
\pgfpathlineto{\pgfqpoint{4.031172in}{3.542032in}}%
\pgfpathlineto{\pgfqpoint{4.032065in}{3.599297in}}%
\pgfpathlineto{\pgfqpoint{4.034299in}{4.022814in}}%
\pgfpathlineto{\pgfqpoint{4.038319in}{4.784849in}}%
\pgfpathlineto{\pgfqpoint{4.038765in}{4.793348in}}%
\pgfpathlineto{\pgfqpoint{4.039212in}{4.780869in}}%
\pgfpathlineto{\pgfqpoint{4.040552in}{4.624362in}}%
\pgfpathlineto{\pgfqpoint{4.046359in}{3.540260in}}%
\pgfpathlineto{\pgfqpoint{4.046805in}{3.554366in}}%
\pgfpathlineto{\pgfqpoint{4.048145in}{3.720605in}}%
\pgfpathlineto{\pgfqpoint{4.053505in}{4.791320in}}%
\pgfpathlineto{\pgfqpoint{4.054399in}{4.765188in}}%
\pgfpathlineto{\pgfqpoint{4.056185in}{4.466531in}}%
\pgfpathlineto{\pgfqpoint{4.060652in}{3.544906in}}%
\pgfpathlineto{\pgfqpoint{4.061099in}{3.541627in}}%
\pgfpathlineto{\pgfqpoint{4.061992in}{3.606270in}}%
\pgfpathlineto{\pgfqpoint{4.064226in}{4.091387in}}%
\pgfpathlineto{\pgfqpoint{4.067799in}{4.790364in}}%
\pgfpathlineto{\pgfqpoint{4.068246in}{4.790243in}}%
\pgfpathlineto{\pgfqpoint{4.069139in}{4.716007in}}%
\pgfpathlineto{\pgfqpoint{4.071372in}{4.204579in}}%
\pgfpathlineto{\pgfqpoint{4.074499in}{3.553607in}}%
\pgfpathlineto{\pgfqpoint{4.074946in}{3.540044in}}%
\pgfpathlineto{\pgfqpoint{4.075392in}{3.552595in}}%
\pgfpathlineto{\pgfqpoint{4.076732in}{3.737396in}}%
\pgfpathlineto{\pgfqpoint{4.081646in}{4.792752in}}%
\pgfpathlineto{\pgfqpoint{4.082092in}{4.785244in}}%
\pgfpathlineto{\pgfqpoint{4.083432in}{4.607783in}}%
\pgfpathlineto{\pgfqpoint{4.088346in}{3.540251in}}%
\pgfpathlineto{\pgfqpoint{4.088792in}{3.550975in}}%
\pgfpathlineto{\pgfqpoint{4.090132in}{3.742968in}}%
\pgfpathlineto{\pgfqpoint{4.095046in}{4.792321in}}%
\pgfpathlineto{\pgfqpoint{4.095492in}{4.769494in}}%
\pgfpathlineto{\pgfqpoint{4.096832in}{4.540582in}}%
\pgfpathlineto{\pgfqpoint{4.101299in}{3.540067in}}%
\pgfpathlineto{\pgfqpoint{4.102192in}{3.598140in}}%
\pgfpathlineto{\pgfqpoint{4.103979in}{4.014202in}}%
\pgfpathlineto{\pgfqpoint{4.107552in}{4.793314in}}%
\pgfpathlineto{\pgfqpoint{4.108446in}{4.735891in}}%
\pgfpathlineto{\pgfqpoint{4.110232in}{4.310155in}}%
\pgfpathlineto{\pgfqpoint{4.113359in}{3.552628in}}%
\pgfpathlineto{\pgfqpoint{4.113806in}{3.540305in}}%
\pgfpathlineto{\pgfqpoint{4.114699in}{3.613777in}}%
\pgfpathlineto{\pgfqpoint{4.116932in}{4.214781in}}%
\pgfpathlineto{\pgfqpoint{4.119612in}{4.790639in}}%
\pgfpathlineto{\pgfqpoint{4.120059in}{4.787333in}}%
\pgfpathlineto{\pgfqpoint{4.121399in}{4.582842in}}%
\pgfpathlineto{\pgfqpoint{4.125866in}{3.541744in}}%
\pgfpathlineto{\pgfqpoint{4.126759in}{3.633056in}}%
\pgfpathlineto{\pgfqpoint{4.128993in}{4.280003in}}%
\pgfpathlineto{\pgfqpoint{4.131673in}{4.792216in}}%
\pgfpathlineto{\pgfqpoint{4.132566in}{4.701319in}}%
\pgfpathlineto{\pgfqpoint{4.134799in}{4.042016in}}%
\pgfpathlineto{\pgfqpoint{4.137033in}{3.546175in}}%
\pgfpathlineto{\pgfqpoint{4.137479in}{3.543624in}}%
\pgfpathlineto{\pgfqpoint{4.138373in}{3.650274in}}%
\pgfpathlineto{\pgfqpoint{4.142839in}{4.792853in}}%
\pgfpathlineto{\pgfqpoint{4.143733in}{4.728313in}}%
\pgfpathlineto{\pgfqpoint{4.145519in}{4.229413in}}%
\pgfpathlineto{\pgfqpoint{4.148199in}{3.546581in}}%
\pgfpathlineto{\pgfqpoint{4.148646in}{3.543893in}}%
\pgfpathlineto{\pgfqpoint{4.149539in}{3.657852in}}%
\pgfpathlineto{\pgfqpoint{4.154006in}{4.792169in}}%
\pgfpathlineto{\pgfqpoint{4.154453in}{4.760854in}}%
\pgfpathlineto{\pgfqpoint{4.155793in}{4.447396in}}%
\pgfpathlineto{\pgfqpoint{4.159366in}{3.541458in}}%
\pgfpathlineto{\pgfqpoint{4.160259in}{3.648265in}}%
\pgfpathlineto{\pgfqpoint{4.164726in}{4.788748in}}%
\pgfpathlineto{\pgfqpoint{4.165173in}{4.746090in}}%
\pgfpathlineto{\pgfqpoint{4.166959in}{4.229877in}}%
\pgfpathlineto{\pgfqpoint{4.169639in}{3.540610in}}%
\pgfpathlineto{\pgfqpoint{4.170086in}{3.556553in}}%
\pgfpathlineto{\pgfqpoint{4.171426in}{3.855277in}}%
\pgfpathlineto{\pgfqpoint{4.174999in}{4.789394in}}%
\pgfpathlineto{\pgfqpoint{4.175893in}{4.658494in}}%
\pgfpathlineto{\pgfqpoint{4.179913in}{3.540613in}}%
\pgfpathlineto{\pgfqpoint{4.180359in}{3.572700in}}%
\pgfpathlineto{\pgfqpoint{4.181699in}{3.922730in}}%
\pgfpathlineto{\pgfqpoint{4.184826in}{4.793339in}}%
\pgfpathlineto{\pgfqpoint{4.185719in}{4.687557in}}%
\pgfpathlineto{\pgfqpoint{4.189740in}{3.540545in}}%
\pgfpathlineto{\pgfqpoint{4.190633in}{3.656868in}}%
\pgfpathlineto{\pgfqpoint{4.194653in}{4.789664in}}%
\pgfpathlineto{\pgfqpoint{4.195100in}{4.742946in}}%
\pgfpathlineto{\pgfqpoint{4.196886in}{4.157568in}}%
\pgfpathlineto{\pgfqpoint{4.199120in}{3.541597in}}%
\pgfpathlineto{\pgfqpoint{4.199566in}{3.556226in}}%
\pgfpathlineto{\pgfqpoint{4.200906in}{3.896639in}}%
\pgfpathlineto{\pgfqpoint{4.204033in}{4.790799in}}%
\pgfpathlineto{\pgfqpoint{4.204926in}{4.646910in}}%
\pgfpathlineto{\pgfqpoint{4.208500in}{3.540041in}}%
\pgfpathlineto{\pgfqpoint{4.208946in}{3.568765in}}%
\pgfpathlineto{\pgfqpoint{4.210286in}{3.955679in}}%
\pgfpathlineto{\pgfqpoint{4.212966in}{4.792450in}}%
\pgfpathlineto{\pgfqpoint{4.213413in}{4.773198in}}%
\pgfpathlineto{\pgfqpoint{4.214753in}{4.399181in}}%
\pgfpathlineto{\pgfqpoint{4.217433in}{3.541461in}}%
\pgfpathlineto{\pgfqpoint{4.217880in}{3.559212in}}%
\pgfpathlineto{\pgfqpoint{4.219220in}{3.937748in}}%
\pgfpathlineto{\pgfqpoint{4.221900in}{4.793041in}}%
\pgfpathlineto{\pgfqpoint{4.222793in}{4.682534in}}%
\pgfpathlineto{\pgfqpoint{4.226366in}{3.540520in}}%
\pgfpathlineto{\pgfqpoint{4.226813in}{3.580936in}}%
\pgfpathlineto{\pgfqpoint{4.228600in}{4.225013in}}%
\pgfpathlineto{\pgfqpoint{4.229940in}{4.717457in}}%
\pgfpathlineto{\pgfqpoint{4.229940in}{4.717457in}}%
\pgfusepath{stroke}%
\end{pgfscope}%
\begin{pgfscope}%
\pgfsetrectcap%
\pgfsetmiterjoin%
\pgfsetlinewidth{0.803000pt}%
\definecolor{currentstroke}{rgb}{0.000000,0.000000,0.000000}%
\pgfsetstrokecolor{currentstroke}%
\pgfsetdash{}{0pt}%
\pgfpathmoveto{\pgfqpoint{1.313604in}{3.540038in}}%
\pgfpathlineto{\pgfqpoint{1.313604in}{4.793442in}}%
\pgfusepath{stroke}%
\end{pgfscope}%
\begin{pgfscope}%
\pgfsetrectcap%
\pgfsetmiterjoin%
\pgfsetlinewidth{0.803000pt}%
\definecolor{currentstroke}{rgb}{0.000000,0.000000,0.000000}%
\pgfsetstrokecolor{currentstroke}%
\pgfsetdash{}{0pt}%
\pgfpathmoveto{\pgfqpoint{4.229766in}{3.540038in}}%
\pgfpathlineto{\pgfqpoint{4.229766in}{4.793442in}}%
\pgfusepath{stroke}%
\end{pgfscope}%
\begin{pgfscope}%
\pgfsetrectcap%
\pgfsetmiterjoin%
\pgfsetlinewidth{0.803000pt}%
\definecolor{currentstroke}{rgb}{0.000000,0.000000,0.000000}%
\pgfsetstrokecolor{currentstroke}%
\pgfsetdash{}{0pt}%
\pgfpathmoveto{\pgfqpoint{1.313604in}{3.540038in}}%
\pgfpathlineto{\pgfqpoint{4.229766in}{3.540038in}}%
\pgfusepath{stroke}%
\end{pgfscope}%
\begin{pgfscope}%
\pgfsetrectcap%
\pgfsetmiterjoin%
\pgfsetlinewidth{0.803000pt}%
\definecolor{currentstroke}{rgb}{0.000000,0.000000,0.000000}%
\pgfsetstrokecolor{currentstroke}%
\pgfsetdash{}{0pt}%
\pgfpathmoveto{\pgfqpoint{1.313604in}{4.793442in}}%
\pgfpathlineto{\pgfqpoint{4.229766in}{4.793442in}}%
\pgfusepath{stroke}%
\end{pgfscope}%
\begin{pgfscope}%
\definecolor{textcolor}{rgb}{0.000000,0.000000,0.000000}%
\pgfsetstrokecolor{textcolor}%
\pgfsetfillcolor{textcolor}%
\pgftext[x=1.998517in,y=4.480091in,left,base]{\color{textcolor}{\rmfamily\fontsize{16.000000}{19.200000}\selectfont\catcode`\^=\active\def^{\ifmmode\sp\else\^{}\fi}\catcode`\%=\active\def%{\%}$P_{0v}$}}%
\end{pgfscope}%
\begin{pgfscope}%
\definecolor{textcolor}{rgb}{1.000000,0.000000,0.000000}%
\pgfsetstrokecolor{textcolor}%
\pgfsetfillcolor{textcolor}%
\pgftext[x=1.998517in,y=3.790719in,left,base]{\color{textcolor}{\rmfamily\fontsize{16.000000}{19.200000}\selectfont\catcode`\^=\active\def^{\ifmmode\sp\else\^{}\fi}\catcode`\%=\active\def%{\%}$P_{3c}$}}%
\end{pgfscope}%
\begin{pgfscope}%
\pgfsetbuttcap%
\pgfsetmiterjoin%
\definecolor{currentfill}{rgb}{1.000000,1.000000,1.000000}%
\pgfsetfillcolor{currentfill}%
\pgfsetlinewidth{0.000000pt}%
\definecolor{currentstroke}{rgb}{0.000000,0.000000,0.000000}%
\pgfsetstrokecolor{currentstroke}%
\pgfsetstrokeopacity{0.000000}%
\pgfsetdash{}{0pt}%
\pgfpathmoveto{\pgfqpoint{1.105307in}{1.282443in}}%
\pgfpathlineto{\pgfqpoint{4.021469in}{1.282443in}}%
\pgfpathlineto{\pgfqpoint{4.021469in}{2.535847in}}%
\pgfpathlineto{\pgfqpoint{1.105307in}{2.535847in}}%
\pgfpathlineto{\pgfqpoint{1.105307in}{1.282443in}}%
\pgfpathclose%
\pgfusepath{fill}%
\end{pgfscope}%
\begin{pgfscope}%
\pgfsetbuttcap%
\pgfsetroundjoin%
\definecolor{currentfill}{rgb}{0.000000,0.000000,0.000000}%
\pgfsetfillcolor{currentfill}%
\pgfsetlinewidth{0.803000pt}%
\definecolor{currentstroke}{rgb}{0.000000,0.000000,0.000000}%
\pgfsetstrokecolor{currentstroke}%
\pgfsetdash{}{0pt}%
\pgfsys@defobject{currentmarker}{\pgfqpoint{0.000000in}{-0.048611in}}{\pgfqpoint{0.000000in}{0.000000in}}{%
\pgfpathmoveto{\pgfqpoint{0.000000in}{0.000000in}}%
\pgfpathlineto{\pgfqpoint{0.000000in}{-0.048611in}}%
\pgfusepath{stroke,fill}%
}%
\begin{pgfscope}%
\pgfsys@transformshift{1.863717in}{1.282443in}%
\pgfsys@useobject{currentmarker}{}%
\end{pgfscope}%
\end{pgfscope}%
\begin{pgfscope}%
\definecolor{textcolor}{rgb}{0.000000,0.000000,0.000000}%
\pgfsetstrokecolor{textcolor}%
\pgfsetfillcolor{textcolor}%
\pgftext[x=1.863717in,y=1.185221in,,top]{\color{textcolor}{\rmfamily\fontsize{16.000000}{19.200000}\selectfont\catcode`\^=\active\def^{\ifmmode\sp\else\^{}\fi}\catcode`\%=\active\def%{\%}$\mathdefault{10^{5}}$}}%
\end{pgfscope}%
\begin{pgfscope}%
\pgfsetbuttcap%
\pgfsetroundjoin%
\definecolor{currentfill}{rgb}{0.000000,0.000000,0.000000}%
\pgfsetfillcolor{currentfill}%
\pgfsetlinewidth{0.803000pt}%
\definecolor{currentstroke}{rgb}{0.000000,0.000000,0.000000}%
\pgfsetstrokecolor{currentstroke}%
\pgfsetdash{}{0pt}%
\pgfsys@defobject{currentmarker}{\pgfqpoint{0.000000in}{-0.048611in}}{\pgfqpoint{0.000000in}{0.000000in}}{%
\pgfpathmoveto{\pgfqpoint{0.000000in}{0.000000in}}%
\pgfpathlineto{\pgfqpoint{0.000000in}{-0.048611in}}%
\pgfusepath{stroke,fill}%
}%
\begin{pgfscope}%
\pgfsys@transformshift{3.380538in}{1.282443in}%
\pgfsys@useobject{currentmarker}{}%
\end{pgfscope}%
\end{pgfscope}%
\begin{pgfscope}%
\definecolor{textcolor}{rgb}{0.000000,0.000000,0.000000}%
\pgfsetstrokecolor{textcolor}%
\pgfsetfillcolor{textcolor}%
\pgftext[x=3.380538in,y=1.185221in,,top]{\color{textcolor}{\rmfamily\fontsize{16.000000}{19.200000}\selectfont\catcode`\^=\active\def^{\ifmmode\sp\else\^{}\fi}\catcode`\%=\active\def%{\%}$\mathdefault{10^{7}}$}}%
\end{pgfscope}%
\begin{pgfscope}%
\pgfsetbuttcap%
\pgfsetroundjoin%
\definecolor{currentfill}{rgb}{0.000000,0.000000,0.000000}%
\pgfsetfillcolor{currentfill}%
\pgfsetlinewidth{0.602250pt}%
\definecolor{currentstroke}{rgb}{0.000000,0.000000,0.000000}%
\pgfsetstrokecolor{currentstroke}%
\pgfsetdash{}{0pt}%
\pgfsys@defobject{currentmarker}{\pgfqpoint{0.000000in}{-0.027778in}}{\pgfqpoint{0.000000in}{0.000000in}}{%
\pgfpathmoveto{\pgfqpoint{0.000000in}{0.000000in}}%
\pgfpathlineto{\pgfqpoint{0.000000in}{-0.027778in}}%
\pgfusepath{stroke,fill}%
}%
\begin{pgfscope}%
\pgfsys@transformshift{1.333611in}{1.282443in}%
\pgfsys@useobject{currentmarker}{}%
\end{pgfscope}%
\end{pgfscope}%
\begin{pgfscope}%
\pgfsetbuttcap%
\pgfsetroundjoin%
\definecolor{currentfill}{rgb}{0.000000,0.000000,0.000000}%
\pgfsetfillcolor{currentfill}%
\pgfsetlinewidth{0.602250pt}%
\definecolor{currentstroke}{rgb}{0.000000,0.000000,0.000000}%
\pgfsetstrokecolor{currentstroke}%
\pgfsetdash{}{0pt}%
\pgfsys@defobject{currentmarker}{\pgfqpoint{0.000000in}{-0.027778in}}{\pgfqpoint{0.000000in}{0.000000in}}{%
\pgfpathmoveto{\pgfqpoint{0.000000in}{0.000000in}}%
\pgfpathlineto{\pgfqpoint{0.000000in}{-0.027778in}}%
\pgfusepath{stroke,fill}%
}%
\begin{pgfscope}%
\pgfsys@transformshift{1.467161in}{1.282443in}%
\pgfsys@useobject{currentmarker}{}%
\end{pgfscope}%
\end{pgfscope}%
\begin{pgfscope}%
\pgfsetbuttcap%
\pgfsetroundjoin%
\definecolor{currentfill}{rgb}{0.000000,0.000000,0.000000}%
\pgfsetfillcolor{currentfill}%
\pgfsetlinewidth{0.602250pt}%
\definecolor{currentstroke}{rgb}{0.000000,0.000000,0.000000}%
\pgfsetstrokecolor{currentstroke}%
\pgfsetdash{}{0pt}%
\pgfsys@defobject{currentmarker}{\pgfqpoint{0.000000in}{-0.027778in}}{\pgfqpoint{0.000000in}{0.000000in}}{%
\pgfpathmoveto{\pgfqpoint{0.000000in}{0.000000in}}%
\pgfpathlineto{\pgfqpoint{0.000000in}{-0.027778in}}%
\pgfusepath{stroke,fill}%
}%
\begin{pgfscope}%
\pgfsys@transformshift{1.561916in}{1.282443in}%
\pgfsys@useobject{currentmarker}{}%
\end{pgfscope}%
\end{pgfscope}%
\begin{pgfscope}%
\pgfsetbuttcap%
\pgfsetroundjoin%
\definecolor{currentfill}{rgb}{0.000000,0.000000,0.000000}%
\pgfsetfillcolor{currentfill}%
\pgfsetlinewidth{0.602250pt}%
\definecolor{currentstroke}{rgb}{0.000000,0.000000,0.000000}%
\pgfsetstrokecolor{currentstroke}%
\pgfsetdash{}{0pt}%
\pgfsys@defobject{currentmarker}{\pgfqpoint{0.000000in}{-0.027778in}}{\pgfqpoint{0.000000in}{0.000000in}}{%
\pgfpathmoveto{\pgfqpoint{0.000000in}{0.000000in}}%
\pgfpathlineto{\pgfqpoint{0.000000in}{-0.027778in}}%
\pgfusepath{stroke,fill}%
}%
\begin{pgfscope}%
\pgfsys@transformshift{1.635413in}{1.282443in}%
\pgfsys@useobject{currentmarker}{}%
\end{pgfscope}%
\end{pgfscope}%
\begin{pgfscope}%
\pgfsetbuttcap%
\pgfsetroundjoin%
\definecolor{currentfill}{rgb}{0.000000,0.000000,0.000000}%
\pgfsetfillcolor{currentfill}%
\pgfsetlinewidth{0.602250pt}%
\definecolor{currentstroke}{rgb}{0.000000,0.000000,0.000000}%
\pgfsetstrokecolor{currentstroke}%
\pgfsetdash{}{0pt}%
\pgfsys@defobject{currentmarker}{\pgfqpoint{0.000000in}{-0.027778in}}{\pgfqpoint{0.000000in}{0.000000in}}{%
\pgfpathmoveto{\pgfqpoint{0.000000in}{0.000000in}}%
\pgfpathlineto{\pgfqpoint{0.000000in}{-0.027778in}}%
\pgfusepath{stroke,fill}%
}%
\begin{pgfscope}%
\pgfsys@transformshift{1.695465in}{1.282443in}%
\pgfsys@useobject{currentmarker}{}%
\end{pgfscope}%
\end{pgfscope}%
\begin{pgfscope}%
\pgfsetbuttcap%
\pgfsetroundjoin%
\definecolor{currentfill}{rgb}{0.000000,0.000000,0.000000}%
\pgfsetfillcolor{currentfill}%
\pgfsetlinewidth{0.602250pt}%
\definecolor{currentstroke}{rgb}{0.000000,0.000000,0.000000}%
\pgfsetstrokecolor{currentstroke}%
\pgfsetdash{}{0pt}%
\pgfsys@defobject{currentmarker}{\pgfqpoint{0.000000in}{-0.027778in}}{\pgfqpoint{0.000000in}{0.000000in}}{%
\pgfpathmoveto{\pgfqpoint{0.000000in}{0.000000in}}%
\pgfpathlineto{\pgfqpoint{0.000000in}{-0.027778in}}%
\pgfusepath{stroke,fill}%
}%
\begin{pgfscope}%
\pgfsys@transformshift{1.746238in}{1.282443in}%
\pgfsys@useobject{currentmarker}{}%
\end{pgfscope}%
\end{pgfscope}%
\begin{pgfscope}%
\pgfsetbuttcap%
\pgfsetroundjoin%
\definecolor{currentfill}{rgb}{0.000000,0.000000,0.000000}%
\pgfsetfillcolor{currentfill}%
\pgfsetlinewidth{0.602250pt}%
\definecolor{currentstroke}{rgb}{0.000000,0.000000,0.000000}%
\pgfsetstrokecolor{currentstroke}%
\pgfsetdash{}{0pt}%
\pgfsys@defobject{currentmarker}{\pgfqpoint{0.000000in}{-0.027778in}}{\pgfqpoint{0.000000in}{0.000000in}}{%
\pgfpathmoveto{\pgfqpoint{0.000000in}{0.000000in}}%
\pgfpathlineto{\pgfqpoint{0.000000in}{-0.027778in}}%
\pgfusepath{stroke,fill}%
}%
\begin{pgfscope}%
\pgfsys@transformshift{1.790220in}{1.282443in}%
\pgfsys@useobject{currentmarker}{}%
\end{pgfscope}%
\end{pgfscope}%
\begin{pgfscope}%
\pgfsetbuttcap%
\pgfsetroundjoin%
\definecolor{currentfill}{rgb}{0.000000,0.000000,0.000000}%
\pgfsetfillcolor{currentfill}%
\pgfsetlinewidth{0.602250pt}%
\definecolor{currentstroke}{rgb}{0.000000,0.000000,0.000000}%
\pgfsetstrokecolor{currentstroke}%
\pgfsetdash{}{0pt}%
\pgfsys@defobject{currentmarker}{\pgfqpoint{0.000000in}{-0.027778in}}{\pgfqpoint{0.000000in}{0.000000in}}{%
\pgfpathmoveto{\pgfqpoint{0.000000in}{0.000000in}}%
\pgfpathlineto{\pgfqpoint{0.000000in}{-0.027778in}}%
\pgfusepath{stroke,fill}%
}%
\begin{pgfscope}%
\pgfsys@transformshift{1.829014in}{1.282443in}%
\pgfsys@useobject{currentmarker}{}%
\end{pgfscope}%
\end{pgfscope}%
\begin{pgfscope}%
\pgfsetbuttcap%
\pgfsetroundjoin%
\definecolor{currentfill}{rgb}{0.000000,0.000000,0.000000}%
\pgfsetfillcolor{currentfill}%
\pgfsetlinewidth{0.602250pt}%
\definecolor{currentstroke}{rgb}{0.000000,0.000000,0.000000}%
\pgfsetstrokecolor{currentstroke}%
\pgfsetdash{}{0pt}%
\pgfsys@defobject{currentmarker}{\pgfqpoint{0.000000in}{-0.027778in}}{\pgfqpoint{0.000000in}{0.000000in}}{%
\pgfpathmoveto{\pgfqpoint{0.000000in}{0.000000in}}%
\pgfpathlineto{\pgfqpoint{0.000000in}{-0.027778in}}%
\pgfusepath{stroke,fill}%
}%
\begin{pgfscope}%
\pgfsys@transformshift{2.092022in}{1.282443in}%
\pgfsys@useobject{currentmarker}{}%
\end{pgfscope}%
\end{pgfscope}%
\begin{pgfscope}%
\pgfsetbuttcap%
\pgfsetroundjoin%
\definecolor{currentfill}{rgb}{0.000000,0.000000,0.000000}%
\pgfsetfillcolor{currentfill}%
\pgfsetlinewidth{0.602250pt}%
\definecolor{currentstroke}{rgb}{0.000000,0.000000,0.000000}%
\pgfsetstrokecolor{currentstroke}%
\pgfsetdash{}{0pt}%
\pgfsys@defobject{currentmarker}{\pgfqpoint{0.000000in}{-0.027778in}}{\pgfqpoint{0.000000in}{0.000000in}}{%
\pgfpathmoveto{\pgfqpoint{0.000000in}{0.000000in}}%
\pgfpathlineto{\pgfqpoint{0.000000in}{-0.027778in}}%
\pgfusepath{stroke,fill}%
}%
\begin{pgfscope}%
\pgfsys@transformshift{2.225571in}{1.282443in}%
\pgfsys@useobject{currentmarker}{}%
\end{pgfscope}%
\end{pgfscope}%
\begin{pgfscope}%
\pgfsetbuttcap%
\pgfsetroundjoin%
\definecolor{currentfill}{rgb}{0.000000,0.000000,0.000000}%
\pgfsetfillcolor{currentfill}%
\pgfsetlinewidth{0.602250pt}%
\definecolor{currentstroke}{rgb}{0.000000,0.000000,0.000000}%
\pgfsetstrokecolor{currentstroke}%
\pgfsetdash{}{0pt}%
\pgfsys@defobject{currentmarker}{\pgfqpoint{0.000000in}{-0.027778in}}{\pgfqpoint{0.000000in}{0.000000in}}{%
\pgfpathmoveto{\pgfqpoint{0.000000in}{0.000000in}}%
\pgfpathlineto{\pgfqpoint{0.000000in}{-0.027778in}}%
\pgfusepath{stroke,fill}%
}%
\begin{pgfscope}%
\pgfsys@transformshift{2.320326in}{1.282443in}%
\pgfsys@useobject{currentmarker}{}%
\end{pgfscope}%
\end{pgfscope}%
\begin{pgfscope}%
\pgfsetbuttcap%
\pgfsetroundjoin%
\definecolor{currentfill}{rgb}{0.000000,0.000000,0.000000}%
\pgfsetfillcolor{currentfill}%
\pgfsetlinewidth{0.602250pt}%
\definecolor{currentstroke}{rgb}{0.000000,0.000000,0.000000}%
\pgfsetstrokecolor{currentstroke}%
\pgfsetdash{}{0pt}%
\pgfsys@defobject{currentmarker}{\pgfqpoint{0.000000in}{-0.027778in}}{\pgfqpoint{0.000000in}{0.000000in}}{%
\pgfpathmoveto{\pgfqpoint{0.000000in}{0.000000in}}%
\pgfpathlineto{\pgfqpoint{0.000000in}{-0.027778in}}%
\pgfusepath{stroke,fill}%
}%
\begin{pgfscope}%
\pgfsys@transformshift{2.393823in}{1.282443in}%
\pgfsys@useobject{currentmarker}{}%
\end{pgfscope}%
\end{pgfscope}%
\begin{pgfscope}%
\pgfsetbuttcap%
\pgfsetroundjoin%
\definecolor{currentfill}{rgb}{0.000000,0.000000,0.000000}%
\pgfsetfillcolor{currentfill}%
\pgfsetlinewidth{0.602250pt}%
\definecolor{currentstroke}{rgb}{0.000000,0.000000,0.000000}%
\pgfsetstrokecolor{currentstroke}%
\pgfsetdash{}{0pt}%
\pgfsys@defobject{currentmarker}{\pgfqpoint{0.000000in}{-0.027778in}}{\pgfqpoint{0.000000in}{0.000000in}}{%
\pgfpathmoveto{\pgfqpoint{0.000000in}{0.000000in}}%
\pgfpathlineto{\pgfqpoint{0.000000in}{-0.027778in}}%
\pgfusepath{stroke,fill}%
}%
\begin{pgfscope}%
\pgfsys@transformshift{2.453875in}{1.282443in}%
\pgfsys@useobject{currentmarker}{}%
\end{pgfscope}%
\end{pgfscope}%
\begin{pgfscope}%
\pgfsetbuttcap%
\pgfsetroundjoin%
\definecolor{currentfill}{rgb}{0.000000,0.000000,0.000000}%
\pgfsetfillcolor{currentfill}%
\pgfsetlinewidth{0.602250pt}%
\definecolor{currentstroke}{rgb}{0.000000,0.000000,0.000000}%
\pgfsetstrokecolor{currentstroke}%
\pgfsetdash{}{0pt}%
\pgfsys@defobject{currentmarker}{\pgfqpoint{0.000000in}{-0.027778in}}{\pgfqpoint{0.000000in}{0.000000in}}{%
\pgfpathmoveto{\pgfqpoint{0.000000in}{0.000000in}}%
\pgfpathlineto{\pgfqpoint{0.000000in}{-0.027778in}}%
\pgfusepath{stroke,fill}%
}%
\begin{pgfscope}%
\pgfsys@transformshift{2.504648in}{1.282443in}%
\pgfsys@useobject{currentmarker}{}%
\end{pgfscope}%
\end{pgfscope}%
\begin{pgfscope}%
\pgfsetbuttcap%
\pgfsetroundjoin%
\definecolor{currentfill}{rgb}{0.000000,0.000000,0.000000}%
\pgfsetfillcolor{currentfill}%
\pgfsetlinewidth{0.602250pt}%
\definecolor{currentstroke}{rgb}{0.000000,0.000000,0.000000}%
\pgfsetstrokecolor{currentstroke}%
\pgfsetdash{}{0pt}%
\pgfsys@defobject{currentmarker}{\pgfqpoint{0.000000in}{-0.027778in}}{\pgfqpoint{0.000000in}{0.000000in}}{%
\pgfpathmoveto{\pgfqpoint{0.000000in}{0.000000in}}%
\pgfpathlineto{\pgfqpoint{0.000000in}{-0.027778in}}%
\pgfusepath{stroke,fill}%
}%
\begin{pgfscope}%
\pgfsys@transformshift{2.548630in}{1.282443in}%
\pgfsys@useobject{currentmarker}{}%
\end{pgfscope}%
\end{pgfscope}%
\begin{pgfscope}%
\pgfsetbuttcap%
\pgfsetroundjoin%
\definecolor{currentfill}{rgb}{0.000000,0.000000,0.000000}%
\pgfsetfillcolor{currentfill}%
\pgfsetlinewidth{0.602250pt}%
\definecolor{currentstroke}{rgb}{0.000000,0.000000,0.000000}%
\pgfsetstrokecolor{currentstroke}%
\pgfsetdash{}{0pt}%
\pgfsys@defobject{currentmarker}{\pgfqpoint{0.000000in}{-0.027778in}}{\pgfqpoint{0.000000in}{0.000000in}}{%
\pgfpathmoveto{\pgfqpoint{0.000000in}{0.000000in}}%
\pgfpathlineto{\pgfqpoint{0.000000in}{-0.027778in}}%
\pgfusepath{stroke,fill}%
}%
\begin{pgfscope}%
\pgfsys@transformshift{2.587425in}{1.282443in}%
\pgfsys@useobject{currentmarker}{}%
\end{pgfscope}%
\end{pgfscope}%
\begin{pgfscope}%
\pgfsetbuttcap%
\pgfsetroundjoin%
\definecolor{currentfill}{rgb}{0.000000,0.000000,0.000000}%
\pgfsetfillcolor{currentfill}%
\pgfsetlinewidth{0.602250pt}%
\definecolor{currentstroke}{rgb}{0.000000,0.000000,0.000000}%
\pgfsetstrokecolor{currentstroke}%
\pgfsetdash{}{0pt}%
\pgfsys@defobject{currentmarker}{\pgfqpoint{0.000000in}{-0.027778in}}{\pgfqpoint{0.000000in}{0.000000in}}{%
\pgfpathmoveto{\pgfqpoint{0.000000in}{0.000000in}}%
\pgfpathlineto{\pgfqpoint{0.000000in}{-0.027778in}}%
\pgfusepath{stroke,fill}%
}%
\begin{pgfscope}%
\pgfsys@transformshift{2.850432in}{1.282443in}%
\pgfsys@useobject{currentmarker}{}%
\end{pgfscope}%
\end{pgfscope}%
\begin{pgfscope}%
\pgfsetbuttcap%
\pgfsetroundjoin%
\definecolor{currentfill}{rgb}{0.000000,0.000000,0.000000}%
\pgfsetfillcolor{currentfill}%
\pgfsetlinewidth{0.602250pt}%
\definecolor{currentstroke}{rgb}{0.000000,0.000000,0.000000}%
\pgfsetstrokecolor{currentstroke}%
\pgfsetdash{}{0pt}%
\pgfsys@defobject{currentmarker}{\pgfqpoint{0.000000in}{-0.027778in}}{\pgfqpoint{0.000000in}{0.000000in}}{%
\pgfpathmoveto{\pgfqpoint{0.000000in}{0.000000in}}%
\pgfpathlineto{\pgfqpoint{0.000000in}{-0.027778in}}%
\pgfusepath{stroke,fill}%
}%
\begin{pgfscope}%
\pgfsys@transformshift{2.983981in}{1.282443in}%
\pgfsys@useobject{currentmarker}{}%
\end{pgfscope}%
\end{pgfscope}%
\begin{pgfscope}%
\pgfsetbuttcap%
\pgfsetroundjoin%
\definecolor{currentfill}{rgb}{0.000000,0.000000,0.000000}%
\pgfsetfillcolor{currentfill}%
\pgfsetlinewidth{0.602250pt}%
\definecolor{currentstroke}{rgb}{0.000000,0.000000,0.000000}%
\pgfsetstrokecolor{currentstroke}%
\pgfsetdash{}{0pt}%
\pgfsys@defobject{currentmarker}{\pgfqpoint{0.000000in}{-0.027778in}}{\pgfqpoint{0.000000in}{0.000000in}}{%
\pgfpathmoveto{\pgfqpoint{0.000000in}{0.000000in}}%
\pgfpathlineto{\pgfqpoint{0.000000in}{-0.027778in}}%
\pgfusepath{stroke,fill}%
}%
\begin{pgfscope}%
\pgfsys@transformshift{3.078736in}{1.282443in}%
\pgfsys@useobject{currentmarker}{}%
\end{pgfscope}%
\end{pgfscope}%
\begin{pgfscope}%
\pgfsetbuttcap%
\pgfsetroundjoin%
\definecolor{currentfill}{rgb}{0.000000,0.000000,0.000000}%
\pgfsetfillcolor{currentfill}%
\pgfsetlinewidth{0.602250pt}%
\definecolor{currentstroke}{rgb}{0.000000,0.000000,0.000000}%
\pgfsetstrokecolor{currentstroke}%
\pgfsetdash{}{0pt}%
\pgfsys@defobject{currentmarker}{\pgfqpoint{0.000000in}{-0.027778in}}{\pgfqpoint{0.000000in}{0.000000in}}{%
\pgfpathmoveto{\pgfqpoint{0.000000in}{0.000000in}}%
\pgfpathlineto{\pgfqpoint{0.000000in}{-0.027778in}}%
\pgfusepath{stroke,fill}%
}%
\begin{pgfscope}%
\pgfsys@transformshift{3.152233in}{1.282443in}%
\pgfsys@useobject{currentmarker}{}%
\end{pgfscope}%
\end{pgfscope}%
\begin{pgfscope}%
\pgfsetbuttcap%
\pgfsetroundjoin%
\definecolor{currentfill}{rgb}{0.000000,0.000000,0.000000}%
\pgfsetfillcolor{currentfill}%
\pgfsetlinewidth{0.602250pt}%
\definecolor{currentstroke}{rgb}{0.000000,0.000000,0.000000}%
\pgfsetstrokecolor{currentstroke}%
\pgfsetdash{}{0pt}%
\pgfsys@defobject{currentmarker}{\pgfqpoint{0.000000in}{-0.027778in}}{\pgfqpoint{0.000000in}{0.000000in}}{%
\pgfpathmoveto{\pgfqpoint{0.000000in}{0.000000in}}%
\pgfpathlineto{\pgfqpoint{0.000000in}{-0.027778in}}%
\pgfusepath{stroke,fill}%
}%
\begin{pgfscope}%
\pgfsys@transformshift{3.212285in}{1.282443in}%
\pgfsys@useobject{currentmarker}{}%
\end{pgfscope}%
\end{pgfscope}%
\begin{pgfscope}%
\pgfsetbuttcap%
\pgfsetroundjoin%
\definecolor{currentfill}{rgb}{0.000000,0.000000,0.000000}%
\pgfsetfillcolor{currentfill}%
\pgfsetlinewidth{0.602250pt}%
\definecolor{currentstroke}{rgb}{0.000000,0.000000,0.000000}%
\pgfsetstrokecolor{currentstroke}%
\pgfsetdash{}{0pt}%
\pgfsys@defobject{currentmarker}{\pgfqpoint{0.000000in}{-0.027778in}}{\pgfqpoint{0.000000in}{0.000000in}}{%
\pgfpathmoveto{\pgfqpoint{0.000000in}{0.000000in}}%
\pgfpathlineto{\pgfqpoint{0.000000in}{-0.027778in}}%
\pgfusepath{stroke,fill}%
}%
\begin{pgfscope}%
\pgfsys@transformshift{3.263058in}{1.282443in}%
\pgfsys@useobject{currentmarker}{}%
\end{pgfscope}%
\end{pgfscope}%
\begin{pgfscope}%
\pgfsetbuttcap%
\pgfsetroundjoin%
\definecolor{currentfill}{rgb}{0.000000,0.000000,0.000000}%
\pgfsetfillcolor{currentfill}%
\pgfsetlinewidth{0.602250pt}%
\definecolor{currentstroke}{rgb}{0.000000,0.000000,0.000000}%
\pgfsetstrokecolor{currentstroke}%
\pgfsetdash{}{0pt}%
\pgfsys@defobject{currentmarker}{\pgfqpoint{0.000000in}{-0.027778in}}{\pgfqpoint{0.000000in}{0.000000in}}{%
\pgfpathmoveto{\pgfqpoint{0.000000in}{0.000000in}}%
\pgfpathlineto{\pgfqpoint{0.000000in}{-0.027778in}}%
\pgfusepath{stroke,fill}%
}%
\begin{pgfscope}%
\pgfsys@transformshift{3.307040in}{1.282443in}%
\pgfsys@useobject{currentmarker}{}%
\end{pgfscope}%
\end{pgfscope}%
\begin{pgfscope}%
\pgfsetbuttcap%
\pgfsetroundjoin%
\definecolor{currentfill}{rgb}{0.000000,0.000000,0.000000}%
\pgfsetfillcolor{currentfill}%
\pgfsetlinewidth{0.602250pt}%
\definecolor{currentstroke}{rgb}{0.000000,0.000000,0.000000}%
\pgfsetstrokecolor{currentstroke}%
\pgfsetdash{}{0pt}%
\pgfsys@defobject{currentmarker}{\pgfqpoint{0.000000in}{-0.027778in}}{\pgfqpoint{0.000000in}{0.000000in}}{%
\pgfpathmoveto{\pgfqpoint{0.000000in}{0.000000in}}%
\pgfpathlineto{\pgfqpoint{0.000000in}{-0.027778in}}%
\pgfusepath{stroke,fill}%
}%
\begin{pgfscope}%
\pgfsys@transformshift{3.345835in}{1.282443in}%
\pgfsys@useobject{currentmarker}{}%
\end{pgfscope}%
\end{pgfscope}%
\begin{pgfscope}%
\pgfsetbuttcap%
\pgfsetroundjoin%
\definecolor{currentfill}{rgb}{0.000000,0.000000,0.000000}%
\pgfsetfillcolor{currentfill}%
\pgfsetlinewidth{0.602250pt}%
\definecolor{currentstroke}{rgb}{0.000000,0.000000,0.000000}%
\pgfsetstrokecolor{currentstroke}%
\pgfsetdash{}{0pt}%
\pgfsys@defobject{currentmarker}{\pgfqpoint{0.000000in}{-0.027778in}}{\pgfqpoint{0.000000in}{0.000000in}}{%
\pgfpathmoveto{\pgfqpoint{0.000000in}{0.000000in}}%
\pgfpathlineto{\pgfqpoint{0.000000in}{-0.027778in}}%
\pgfusepath{stroke,fill}%
}%
\begin{pgfscope}%
\pgfsys@transformshift{3.608842in}{1.282443in}%
\pgfsys@useobject{currentmarker}{}%
\end{pgfscope}%
\end{pgfscope}%
\begin{pgfscope}%
\pgfsetbuttcap%
\pgfsetroundjoin%
\definecolor{currentfill}{rgb}{0.000000,0.000000,0.000000}%
\pgfsetfillcolor{currentfill}%
\pgfsetlinewidth{0.602250pt}%
\definecolor{currentstroke}{rgb}{0.000000,0.000000,0.000000}%
\pgfsetstrokecolor{currentstroke}%
\pgfsetdash{}{0pt}%
\pgfsys@defobject{currentmarker}{\pgfqpoint{0.000000in}{-0.027778in}}{\pgfqpoint{0.000000in}{0.000000in}}{%
\pgfpathmoveto{\pgfqpoint{0.000000in}{0.000000in}}%
\pgfpathlineto{\pgfqpoint{0.000000in}{-0.027778in}}%
\pgfusepath{stroke,fill}%
}%
\begin{pgfscope}%
\pgfsys@transformshift{3.742391in}{1.282443in}%
\pgfsys@useobject{currentmarker}{}%
\end{pgfscope}%
\end{pgfscope}%
\begin{pgfscope}%
\pgfsetbuttcap%
\pgfsetroundjoin%
\definecolor{currentfill}{rgb}{0.000000,0.000000,0.000000}%
\pgfsetfillcolor{currentfill}%
\pgfsetlinewidth{0.602250pt}%
\definecolor{currentstroke}{rgb}{0.000000,0.000000,0.000000}%
\pgfsetstrokecolor{currentstroke}%
\pgfsetdash{}{0pt}%
\pgfsys@defobject{currentmarker}{\pgfqpoint{0.000000in}{-0.027778in}}{\pgfqpoint{0.000000in}{0.000000in}}{%
\pgfpathmoveto{\pgfqpoint{0.000000in}{0.000000in}}%
\pgfpathlineto{\pgfqpoint{0.000000in}{-0.027778in}}%
\pgfusepath{stroke,fill}%
}%
\begin{pgfscope}%
\pgfsys@transformshift{3.837146in}{1.282443in}%
\pgfsys@useobject{currentmarker}{}%
\end{pgfscope}%
\end{pgfscope}%
\begin{pgfscope}%
\pgfsetbuttcap%
\pgfsetroundjoin%
\definecolor{currentfill}{rgb}{0.000000,0.000000,0.000000}%
\pgfsetfillcolor{currentfill}%
\pgfsetlinewidth{0.602250pt}%
\definecolor{currentstroke}{rgb}{0.000000,0.000000,0.000000}%
\pgfsetstrokecolor{currentstroke}%
\pgfsetdash{}{0pt}%
\pgfsys@defobject{currentmarker}{\pgfqpoint{0.000000in}{-0.027778in}}{\pgfqpoint{0.000000in}{0.000000in}}{%
\pgfpathmoveto{\pgfqpoint{0.000000in}{0.000000in}}%
\pgfpathlineto{\pgfqpoint{0.000000in}{-0.027778in}}%
\pgfusepath{stroke,fill}%
}%
\begin{pgfscope}%
\pgfsys@transformshift{3.910644in}{1.282443in}%
\pgfsys@useobject{currentmarker}{}%
\end{pgfscope}%
\end{pgfscope}%
\begin{pgfscope}%
\pgfsetbuttcap%
\pgfsetroundjoin%
\definecolor{currentfill}{rgb}{0.000000,0.000000,0.000000}%
\pgfsetfillcolor{currentfill}%
\pgfsetlinewidth{0.602250pt}%
\definecolor{currentstroke}{rgb}{0.000000,0.000000,0.000000}%
\pgfsetstrokecolor{currentstroke}%
\pgfsetdash{}{0pt}%
\pgfsys@defobject{currentmarker}{\pgfqpoint{0.000000in}{-0.027778in}}{\pgfqpoint{0.000000in}{0.000000in}}{%
\pgfpathmoveto{\pgfqpoint{0.000000in}{0.000000in}}%
\pgfpathlineto{\pgfqpoint{0.000000in}{-0.027778in}}%
\pgfusepath{stroke,fill}%
}%
\begin{pgfscope}%
\pgfsys@transformshift{3.970695in}{1.282443in}%
\pgfsys@useobject{currentmarker}{}%
\end{pgfscope}%
\end{pgfscope}%
\begin{pgfscope}%
\pgfsetbuttcap%
\pgfsetroundjoin%
\definecolor{currentfill}{rgb}{0.000000,0.000000,0.000000}%
\pgfsetfillcolor{currentfill}%
\pgfsetlinewidth{0.602250pt}%
\definecolor{currentstroke}{rgb}{0.000000,0.000000,0.000000}%
\pgfsetstrokecolor{currentstroke}%
\pgfsetdash{}{0pt}%
\pgfsys@defobject{currentmarker}{\pgfqpoint{0.000000in}{-0.027778in}}{\pgfqpoint{0.000000in}{0.000000in}}{%
\pgfpathmoveto{\pgfqpoint{0.000000in}{0.000000in}}%
\pgfpathlineto{\pgfqpoint{0.000000in}{-0.027778in}}%
\pgfusepath{stroke,fill}%
}%
\begin{pgfscope}%
\pgfsys@transformshift{4.021469in}{1.282443in}%
\pgfsys@useobject{currentmarker}{}%
\end{pgfscope}%
\end{pgfscope}%
\begin{pgfscope}%
\pgfsetbuttcap%
\pgfsetroundjoin%
\definecolor{currentfill}{rgb}{0.000000,0.000000,0.000000}%
\pgfsetfillcolor{currentfill}%
\pgfsetlinewidth{0.803000pt}%
\definecolor{currentstroke}{rgb}{0.000000,0.000000,0.000000}%
\pgfsetstrokecolor{currentstroke}%
\pgfsetdash{}{0pt}%
\pgfsys@defobject{currentmarker}{\pgfqpoint{-0.048611in}{0.000000in}}{\pgfqpoint{-0.000000in}{0.000000in}}{%
\pgfpathmoveto{\pgfqpoint{-0.000000in}{0.000000in}}%
\pgfpathlineto{\pgfqpoint{-0.048611in}{0.000000in}}%
\pgfusepath{stroke,fill}%
}%
\begin{pgfscope}%
\pgfsys@transformshift{1.105307in}{1.282443in}%
\pgfsys@useobject{currentmarker}{}%
\end{pgfscope}%
\end{pgfscope}%
\begin{pgfscope}%
\definecolor{textcolor}{rgb}{0.000000,0.000000,0.000000}%
\pgfsetstrokecolor{textcolor}%
\pgfsetfillcolor{textcolor}%
\pgftext[x=0.898017in, y=1.199110in, left, base]{\color{textcolor}{\rmfamily\fontsize{16.000000}{19.200000}\selectfont\catcode`\^=\active\def^{\ifmmode\sp\else\^{}\fi}\catcode`\%=\active\def%{\%}0}}%
\end{pgfscope}%
\begin{pgfscope}%
\pgfsetbuttcap%
\pgfsetroundjoin%
\definecolor{currentfill}{rgb}{0.000000,0.000000,0.000000}%
\pgfsetfillcolor{currentfill}%
\pgfsetlinewidth{0.803000pt}%
\definecolor{currentstroke}{rgb}{0.000000,0.000000,0.000000}%
\pgfsetstrokecolor{currentstroke}%
\pgfsetdash{}{0pt}%
\pgfsys@defobject{currentmarker}{\pgfqpoint{-0.048611in}{0.000000in}}{\pgfqpoint{-0.000000in}{0.000000in}}{%
\pgfpathmoveto{\pgfqpoint{-0.000000in}{0.000000in}}%
\pgfpathlineto{\pgfqpoint{-0.048611in}{0.000000in}}%
\pgfusepath{stroke,fill}%
}%
\begin{pgfscope}%
\pgfsys@transformshift{1.105307in}{2.535847in}%
\pgfsys@useobject{currentmarker}{}%
\end{pgfscope}%
\end{pgfscope}%
\begin{pgfscope}%
\definecolor{textcolor}{rgb}{0.000000,0.000000,0.000000}%
\pgfsetstrokecolor{textcolor}%
\pgfsetfillcolor{textcolor}%
\pgftext[x=0.898017in, y=2.452514in, left, base]{\color{textcolor}{\rmfamily\fontsize{16.000000}{19.200000}\selectfont\catcode`\^=\active\def^{\ifmmode\sp\else\^{}\fi}\catcode`\%=\active\def%{\%}1}}%
\end{pgfscope}%
\begin{pgfscope}%
\pgfpathrectangle{\pgfqpoint{1.105307in}{1.282443in}}{\pgfqpoint{2.916161in}{1.253404in}}%
\pgfusepath{clip}%
\pgfsetrectcap%
\pgfsetroundjoin%
\pgfsetlinewidth{1.003750pt}%
\definecolor{currentstroke}{rgb}{0.000000,0.000000,0.000000}%
\pgfsetstrokecolor{currentstroke}%
\pgfsetdash{}{0pt}%
\pgfpathmoveto{\pgfqpoint{1.104894in}{2.530860in}}%
\pgfpathlineto{\pgfqpoint{1.204055in}{2.526750in}}%
\pgfpathlineto{\pgfqpoint{1.279542in}{2.521481in}}%
\pgfpathlineto{\pgfqpoint{1.340735in}{2.515052in}}%
\pgfpathlineto{\pgfqpoint{1.392549in}{2.507422in}}%
\pgfpathlineto{\pgfqpoint{1.437216in}{2.498654in}}%
\pgfpathlineto{\pgfqpoint{1.476969in}{2.488629in}}%
\pgfpathlineto{\pgfqpoint{1.512703in}{2.477369in}}%
\pgfpathlineto{\pgfqpoint{1.545310in}{2.464812in}}%
\pgfpathlineto{\pgfqpoint{1.575683in}{2.450758in}}%
\pgfpathlineto{\pgfqpoint{1.604270in}{2.435075in}}%
\pgfpathlineto{\pgfqpoint{1.631070in}{2.417845in}}%
\pgfpathlineto{\pgfqpoint{1.656530in}{2.398867in}}%
\pgfpathlineto{\pgfqpoint{1.681097in}{2.377803in}}%
\pgfpathlineto{\pgfqpoint{1.705217in}{2.354151in}}%
\pgfpathlineto{\pgfqpoint{1.728444in}{2.328253in}}%
\pgfpathlineto{\pgfqpoint{1.751224in}{2.299545in}}%
\pgfpathlineto{\pgfqpoint{1.774004in}{2.267223in}}%
\pgfpathlineto{\pgfqpoint{1.796784in}{2.230946in}}%
\pgfpathlineto{\pgfqpoint{1.819565in}{2.190380in}}%
\pgfpathlineto{\pgfqpoint{1.842345in}{2.145212in}}%
\pgfpathlineto{\pgfqpoint{1.865571in}{2.094145in}}%
\pgfpathlineto{\pgfqpoint{1.889245in}{2.036681in}}%
\pgfpathlineto{\pgfqpoint{1.914258in}{1.969944in}}%
\pgfpathlineto{\pgfqpoint{1.940612in}{1.893113in}}%
\pgfpathlineto{\pgfqpoint{1.970092in}{1.799997in}}%
\pgfpathlineto{\pgfqpoint{2.007165in}{1.674914in}}%
\pgfpathlineto{\pgfqpoint{2.069252in}{1.464693in}}%
\pgfpathlineto{\pgfqpoint{2.090693in}{1.400463in}}%
\pgfpathlineto{\pgfqpoint{2.107219in}{1.357433in}}%
\pgfpathlineto{\pgfqpoint{2.121066in}{1.327276in}}%
\pgfpathlineto{\pgfqpoint{2.132679in}{1.307103in}}%
\pgfpathlineto{\pgfqpoint{2.142506in}{1.294301in}}%
\pgfpathlineto{\pgfqpoint{2.150993in}{1.286796in}}%
\pgfpathlineto{\pgfqpoint{2.158140in}{1.283277in}}%
\pgfpathlineto{\pgfqpoint{2.164393in}{1.282461in}}%
\pgfpathlineto{\pgfqpoint{2.170646in}{1.283891in}}%
\pgfpathlineto{\pgfqpoint{2.176900in}{1.287691in}}%
\pgfpathlineto{\pgfqpoint{2.183600in}{1.294531in}}%
\pgfpathlineto{\pgfqpoint{2.190746in}{1.305140in}}%
\pgfpathlineto{\pgfqpoint{2.198786in}{1.321357in}}%
\pgfpathlineto{\pgfqpoint{2.207273in}{1.343611in}}%
\pgfpathlineto{\pgfqpoint{2.216653in}{1.374578in}}%
\pgfpathlineto{\pgfqpoint{2.226927in}{1.416389in}}%
\pgfpathlineto{\pgfqpoint{2.238093in}{1.471329in}}%
\pgfpathlineto{\pgfqpoint{2.250600in}{1.544499in}}%
\pgfpathlineto{\pgfqpoint{2.264447in}{1.639187in}}%
\pgfpathlineto{\pgfqpoint{2.280527in}{1.765145in}}%
\pgfpathlineto{\pgfqpoint{2.301520in}{1.948620in}}%
\pgfpathlineto{\pgfqpoint{2.342614in}{2.311719in}}%
\pgfpathlineto{\pgfqpoint{2.356014in}{2.408592in}}%
\pgfpathlineto{\pgfqpoint{2.366287in}{2.467876in}}%
\pgfpathlineto{\pgfqpoint{2.374327in}{2.502756in}}%
\pgfpathlineto{\pgfqpoint{2.381027in}{2.522852in}}%
\pgfpathlineto{\pgfqpoint{2.386387in}{2.532444in}}%
\pgfpathlineto{\pgfqpoint{2.390407in}{2.535594in}}%
\pgfpathlineto{\pgfqpoint{2.393534in}{2.535541in}}%
\pgfpathlineto{\pgfqpoint{2.396661in}{2.533225in}}%
\pgfpathlineto{\pgfqpoint{2.400234in}{2.527739in}}%
\pgfpathlineto{\pgfqpoint{2.404701in}{2.516516in}}%
\pgfpathlineto{\pgfqpoint{2.410061in}{2.496510in}}%
\pgfpathlineto{\pgfqpoint{2.416314in}{2.464025in}}%
\pgfpathlineto{\pgfqpoint{2.423461in}{2.414844in}}%
\pgfpathlineto{\pgfqpoint{2.431948in}{2.340180in}}%
\pgfpathlineto{\pgfqpoint{2.441774in}{2.233260in}}%
\pgfpathlineto{\pgfqpoint{2.454281in}{2.070636in}}%
\pgfpathlineto{\pgfqpoint{2.474381in}{1.774097in}}%
\pgfpathlineto{\pgfqpoint{2.492695in}{1.515729in}}%
\pgfpathlineto{\pgfqpoint{2.502968in}{1.399740in}}%
\pgfpathlineto{\pgfqpoint{2.510561in}{1.336018in}}%
\pgfpathlineto{\pgfqpoint{2.516368in}{1.302910in}}%
\pgfpathlineto{\pgfqpoint{2.520835in}{1.287820in}}%
\pgfpathlineto{\pgfqpoint{2.523961in}{1.283008in}}%
\pgfpathlineto{\pgfqpoint{2.526195in}{1.282586in}}%
\pgfpathlineto{\pgfqpoint{2.528428in}{1.284741in}}%
\pgfpathlineto{\pgfqpoint{2.531108in}{1.290786in}}%
\pgfpathlineto{\pgfqpoint{2.534682in}{1.304805in}}%
\pgfpathlineto{\pgfqpoint{2.539148in}{1.331983in}}%
\pgfpathlineto{\pgfqpoint{2.544955in}{1.383231in}}%
\pgfpathlineto{\pgfqpoint{2.551655in}{1.463845in}}%
\pgfpathlineto{\pgfqpoint{2.560142in}{1.595629in}}%
\pgfpathlineto{\pgfqpoint{2.571308in}{1.807571in}}%
\pgfpathlineto{\pgfqpoint{2.599449in}{2.360192in}}%
\pgfpathlineto{\pgfqpoint{2.607042in}{2.461446in}}%
\pgfpathlineto{\pgfqpoint{2.612849in}{2.511875in}}%
\pgfpathlineto{\pgfqpoint{2.616869in}{2.530833in}}%
\pgfpathlineto{\pgfqpoint{2.619549in}{2.535652in}}%
\pgfpathlineto{\pgfqpoint{2.621335in}{2.535260in}}%
\pgfpathlineto{\pgfqpoint{2.623122in}{2.531930in}}%
\pgfpathlineto{\pgfqpoint{2.625802in}{2.521359in}}%
\pgfpathlineto{\pgfqpoint{2.629375in}{2.496800in}}%
\pgfpathlineto{\pgfqpoint{2.633842in}{2.449461in}}%
\pgfpathlineto{\pgfqpoint{2.639649in}{2.361569in}}%
\pgfpathlineto{\pgfqpoint{2.646795in}{2.217356in}}%
\pgfpathlineto{\pgfqpoint{2.657515in}{1.948898in}}%
\pgfpathlineto{\pgfqpoint{2.675829in}{1.486014in}}%
\pgfpathlineto{\pgfqpoint{2.682976in}{1.360139in}}%
\pgfpathlineto{\pgfqpoint{2.687889in}{1.305945in}}%
\pgfpathlineto{\pgfqpoint{2.691462in}{1.285986in}}%
\pgfpathlineto{\pgfqpoint{2.693696in}{1.282443in}}%
\pgfpathlineto{\pgfqpoint{2.695036in}{1.283718in}}%
\pgfpathlineto{\pgfqpoint{2.696822in}{1.289442in}}%
\pgfpathlineto{\pgfqpoint{2.699502in}{1.306701in}}%
\pgfpathlineto{\pgfqpoint{2.703076in}{1.345798in}}%
\pgfpathlineto{\pgfqpoint{2.707989in}{1.428375in}}%
\pgfpathlineto{\pgfqpoint{2.714242in}{1.576274in}}%
\pgfpathlineto{\pgfqpoint{2.723622in}{1.860947in}}%
\pgfpathlineto{\pgfqpoint{2.739703in}{2.351514in}}%
\pgfpathlineto{\pgfqpoint{2.745956in}{2.475902in}}%
\pgfpathlineto{\pgfqpoint{2.749976in}{2.521433in}}%
\pgfpathlineto{\pgfqpoint{2.752656in}{2.534618in}}%
\pgfpathlineto{\pgfqpoint{2.753996in}{2.535786in}}%
\pgfpathlineto{\pgfqpoint{2.755336in}{2.533263in}}%
\pgfpathlineto{\pgfqpoint{2.757569in}{2.520782in}}%
\pgfpathlineto{\pgfqpoint{2.760696in}{2.485988in}}%
\pgfpathlineto{\pgfqpoint{2.764716in}{2.412572in}}%
\pgfpathlineto{\pgfqpoint{2.770076in}{2.269745in}}%
\pgfpathlineto{\pgfqpoint{2.778116in}{1.986432in}}%
\pgfpathlineto{\pgfqpoint{2.793303in}{1.444947in}}%
\pgfpathlineto{\pgfqpoint{2.798663in}{1.329021in}}%
\pgfpathlineto{\pgfqpoint{2.802236in}{1.289700in}}%
\pgfpathlineto{\pgfqpoint{2.804470in}{1.282448in}}%
\pgfpathlineto{\pgfqpoint{2.805363in}{1.283425in}}%
\pgfpathlineto{\pgfqpoint{2.807150in}{1.292113in}}%
\pgfpathlineto{\pgfqpoint{2.809830in}{1.321956in}}%
\pgfpathlineto{\pgfqpoint{2.813403in}{1.392178in}}%
\pgfpathlineto{\pgfqpoint{2.818316in}{1.539679in}}%
\pgfpathlineto{\pgfqpoint{2.825910in}{1.849269in}}%
\pgfpathlineto{\pgfqpoint{2.838863in}{2.378136in}}%
\pgfpathlineto{\pgfqpoint{2.843776in}{2.496025in}}%
\pgfpathlineto{\pgfqpoint{2.846903in}{2.531181in}}%
\pgfpathlineto{\pgfqpoint{2.848690in}{2.535786in}}%
\pgfpathlineto{\pgfqpoint{2.849583in}{2.533729in}}%
\pgfpathlineto{\pgfqpoint{2.851370in}{2.520827in}}%
\pgfpathlineto{\pgfqpoint{2.854050in}{2.479706in}}%
\pgfpathlineto{\pgfqpoint{2.857623in}{2.386326in}}%
\pgfpathlineto{\pgfqpoint{2.862983in}{2.176979in}}%
\pgfpathlineto{\pgfqpoint{2.882637in}{1.331344in}}%
\pgfpathlineto{\pgfqpoint{2.885763in}{1.287876in}}%
\pgfpathlineto{\pgfqpoint{2.887550in}{1.282585in}}%
\pgfpathlineto{\pgfqpoint{2.888443in}{1.285460in}}%
\pgfpathlineto{\pgfqpoint{2.890230in}{1.302312in}}%
\pgfpathlineto{\pgfqpoint{2.892910in}{1.354876in}}%
\pgfpathlineto{\pgfqpoint{2.896930in}{1.490242in}}%
\pgfpathlineto{\pgfqpoint{2.902737in}{1.775875in}}%
\pgfpathlineto{\pgfqpoint{2.915243in}{2.412444in}}%
\pgfpathlineto{\pgfqpoint{2.919263in}{2.514721in}}%
\pgfpathlineto{\pgfqpoint{2.921497in}{2.535113in}}%
\pgfpathlineto{\pgfqpoint{2.921943in}{2.535836in}}%
\pgfpathlineto{\pgfqpoint{2.922390in}{2.535423in}}%
\pgfpathlineto{\pgfqpoint{2.923730in}{2.527335in}}%
\pgfpathlineto{\pgfqpoint{2.925964in}{2.491173in}}%
\pgfpathlineto{\pgfqpoint{2.929090in}{2.395215in}}%
\pgfpathlineto{\pgfqpoint{2.933557in}{2.181399in}}%
\pgfpathlineto{\pgfqpoint{2.949637in}{1.330053in}}%
\pgfpathlineto{\pgfqpoint{2.952317in}{1.286452in}}%
\pgfpathlineto{\pgfqpoint{2.953210in}{1.282564in}}%
\pgfpathlineto{\pgfqpoint{2.953657in}{1.282676in}}%
\pgfpathlineto{\pgfqpoint{2.954550in}{1.287038in}}%
\pgfpathlineto{\pgfqpoint{2.956337in}{1.312277in}}%
\pgfpathlineto{\pgfqpoint{2.959017in}{1.390023in}}%
\pgfpathlineto{\pgfqpoint{2.963037in}{1.584533in}}%
\pgfpathlineto{\pgfqpoint{2.972417in}{2.200468in}}%
\pgfpathlineto{\pgfqpoint{2.977777in}{2.463087in}}%
\pgfpathlineto{\pgfqpoint{2.980904in}{2.530418in}}%
\pgfpathlineto{\pgfqpoint{2.982244in}{2.535703in}}%
\pgfpathlineto{\pgfqpoint{2.983137in}{2.531045in}}%
\pgfpathlineto{\pgfqpoint{2.984924in}{2.502113in}}%
\pgfpathlineto{\pgfqpoint{2.987604in}{2.411712in}}%
\pgfpathlineto{\pgfqpoint{2.991624in}{2.187203in}}%
\pgfpathlineto{\pgfqpoint{3.005471in}{1.323539in}}%
\pgfpathlineto{\pgfqpoint{3.008151in}{1.282794in}}%
\pgfpathlineto{\pgfqpoint{3.008597in}{1.282594in}}%
\pgfpathlineto{\pgfqpoint{3.009491in}{1.287969in}}%
\pgfpathlineto{\pgfqpoint{3.011277in}{1.321685in}}%
\pgfpathlineto{\pgfqpoint{3.013957in}{1.426733in}}%
\pgfpathlineto{\pgfqpoint{3.018424in}{1.717130in}}%
\pgfpathlineto{\pgfqpoint{3.029144in}{2.462273in}}%
\pgfpathlineto{\pgfqpoint{3.031824in}{2.530221in}}%
\pgfpathlineto{\pgfqpoint{3.032717in}{2.535776in}}%
\pgfpathlineto{\pgfqpoint{3.033164in}{2.535222in}}%
\pgfpathlineto{\pgfqpoint{3.034504in}{2.520170in}}%
\pgfpathlineto{\pgfqpoint{3.036737in}{2.451475in}}%
\pgfpathlineto{\pgfqpoint{3.039864in}{2.273985in}}%
\pgfpathlineto{\pgfqpoint{3.046564in}{1.722764in}}%
\pgfpathlineto{\pgfqpoint{3.051924in}{1.364196in}}%
\pgfpathlineto{\pgfqpoint{3.054604in}{1.288207in}}%
\pgfpathlineto{\pgfqpoint{3.055498in}{1.282464in}}%
\pgfpathlineto{\pgfqpoint{3.055944in}{1.283420in}}%
\pgfpathlineto{\pgfqpoint{3.057284in}{1.301642in}}%
\pgfpathlineto{\pgfqpoint{3.059518in}{1.381681in}}%
\pgfpathlineto{\pgfqpoint{3.063091in}{1.619701in}}%
\pgfpathlineto{\pgfqpoint{3.074704in}{2.503924in}}%
\pgfpathlineto{\pgfqpoint{3.076938in}{2.535731in}}%
\pgfpathlineto{\pgfqpoint{3.077831in}{2.528254in}}%
\pgfpathlineto{\pgfqpoint{3.079618in}{2.478776in}}%
\pgfpathlineto{\pgfqpoint{3.082298in}{2.325457in}}%
\pgfpathlineto{\pgfqpoint{3.087211in}{1.883088in}}%
\pgfpathlineto{\pgfqpoint{3.093464in}{1.369969in}}%
\pgfpathlineto{\pgfqpoint{3.096144in}{1.285758in}}%
\pgfpathlineto{\pgfqpoint{3.096591in}{1.282736in}}%
\pgfpathlineto{\pgfqpoint{3.097038in}{1.282992in}}%
\pgfpathlineto{\pgfqpoint{3.097931in}{1.293372in}}%
\pgfpathlineto{\pgfqpoint{3.099718in}{1.352842in}}%
\pgfpathlineto{\pgfqpoint{3.102844in}{1.566592in}}%
\pgfpathlineto{\pgfqpoint{3.114458in}{2.523750in}}%
\pgfpathlineto{\pgfqpoint{3.115798in}{2.535507in}}%
\pgfpathlineto{\pgfqpoint{3.116691in}{2.524968in}}%
\pgfpathlineto{\pgfqpoint{3.118478in}{2.460565in}}%
\pgfpathlineto{\pgfqpoint{3.121605in}{2.226783in}}%
\pgfpathlineto{\pgfqpoint{3.132325in}{1.294586in}}%
\pgfpathlineto{\pgfqpoint{3.133218in}{1.282840in}}%
\pgfpathlineto{\pgfqpoint{3.133665in}{1.283088in}}%
\pgfpathlineto{\pgfqpoint{3.134558in}{1.295899in}}%
\pgfpathlineto{\pgfqpoint{3.136345in}{1.369500in}}%
\pgfpathlineto{\pgfqpoint{3.139471in}{1.629245in}}%
\pgfpathlineto{\pgfqpoint{3.148851in}{2.511878in}}%
\pgfpathlineto{\pgfqpoint{3.150191in}{2.535689in}}%
\pgfpathlineto{\pgfqpoint{3.150638in}{2.534612in}}%
\pgfpathlineto{\pgfqpoint{3.151978in}{2.504193in}}%
\pgfpathlineto{\pgfqpoint{3.154211in}{2.368190in}}%
\pgfpathlineto{\pgfqpoint{3.158231in}{1.934953in}}%
\pgfpathlineto{\pgfqpoint{3.164038in}{1.349624in}}%
\pgfpathlineto{\pgfqpoint{3.166271in}{1.282589in}}%
\pgfpathlineto{\pgfqpoint{3.166718in}{1.283888in}}%
\pgfpathlineto{\pgfqpoint{3.168058in}{1.317723in}}%
\pgfpathlineto{\pgfqpoint{3.170291in}{1.467284in}}%
\pgfpathlineto{\pgfqpoint{3.174758in}{1.989784in}}%
\pgfpathlineto{\pgfqpoint{3.179671in}{2.480173in}}%
\pgfpathlineto{\pgfqpoint{3.181905in}{2.535277in}}%
\pgfpathlineto{\pgfqpoint{3.182798in}{2.519276in}}%
\pgfpathlineto{\pgfqpoint{3.184585in}{2.423660in}}%
\pgfpathlineto{\pgfqpoint{3.187712in}{2.093224in}}%
\pgfpathlineto{\pgfqpoint{3.194412in}{1.337538in}}%
\pgfpathlineto{\pgfqpoint{3.196198in}{1.282759in}}%
\pgfpathlineto{\pgfqpoint{3.196645in}{1.283819in}}%
\pgfpathlineto{\pgfqpoint{3.197538in}{1.303945in}}%
\pgfpathlineto{\pgfqpoint{3.199325in}{1.412965in}}%
\pgfpathlineto{\pgfqpoint{3.202452in}{1.773490in}}%
\pgfpathlineto{\pgfqpoint{3.208705in}{2.491579in}}%
\pgfpathlineto{\pgfqpoint{3.210492in}{2.535566in}}%
\pgfpathlineto{\pgfqpoint{3.211385in}{2.518652in}}%
\pgfpathlineto{\pgfqpoint{3.213172in}{2.409498in}}%
\pgfpathlineto{\pgfqpoint{3.216298in}{2.033225in}}%
\pgfpathlineto{\pgfqpoint{3.222105in}{1.333065in}}%
\pgfpathlineto{\pgfqpoint{3.223892in}{1.282572in}}%
\pgfpathlineto{\pgfqpoint{3.224785in}{1.299475in}}%
\pgfpathlineto{\pgfqpoint{3.226572in}{1.414876in}}%
\pgfpathlineto{\pgfqpoint{3.229698in}{1.813489in}}%
\pgfpathlineto{\pgfqpoint{3.235058in}{2.483354in}}%
\pgfpathlineto{\pgfqpoint{3.236845in}{2.535568in}}%
\pgfpathlineto{\pgfqpoint{3.237738in}{2.516091in}}%
\pgfpathlineto{\pgfqpoint{3.239525in}{2.389518in}}%
\pgfpathlineto{\pgfqpoint{3.243098in}{1.893354in}}%
\pgfpathlineto{\pgfqpoint{3.247565in}{1.334026in}}%
\pgfpathlineto{\pgfqpoint{3.249352in}{1.283245in}}%
\pgfpathlineto{\pgfqpoint{3.250245in}{1.307039in}}%
\pgfpathlineto{\pgfqpoint{3.252032in}{1.448036in}}%
\pgfpathlineto{\pgfqpoint{3.256052in}{2.048321in}}%
\pgfpathlineto{\pgfqpoint{3.260072in}{2.510883in}}%
\pgfpathlineto{\pgfqpoint{3.260965in}{2.535211in}}%
\pgfpathlineto{\pgfqpoint{3.261412in}{2.534120in}}%
\pgfpathlineto{\pgfqpoint{3.262305in}{2.505330in}}%
\pgfpathlineto{\pgfqpoint{3.264092in}{2.348679in}}%
\pgfpathlineto{\pgfqpoint{3.268559in}{1.645285in}}%
\pgfpathlineto{\pgfqpoint{3.271685in}{1.306518in}}%
\pgfpathlineto{\pgfqpoint{3.272579in}{1.282752in}}%
\pgfpathlineto{\pgfqpoint{3.273025in}{1.285105in}}%
\pgfpathlineto{\pgfqpoint{3.274365in}{1.348698in}}%
\pgfpathlineto{\pgfqpoint{3.276599in}{1.617631in}}%
\pgfpathlineto{\pgfqpoint{3.283299in}{2.527839in}}%
\pgfpathlineto{\pgfqpoint{3.283745in}{2.535519in}}%
\pgfpathlineto{\pgfqpoint{3.284192in}{2.532991in}}%
\pgfpathlineto{\pgfqpoint{3.285532in}{2.464996in}}%
\pgfpathlineto{\pgfqpoint{3.287765in}{2.179590in}}%
\pgfpathlineto{\pgfqpoint{3.294019in}{1.293602in}}%
\pgfpathlineto{\pgfqpoint{3.294465in}{1.283464in}}%
\pgfpathlineto{\pgfqpoint{3.294912in}{1.284205in}}%
\pgfpathlineto{\pgfqpoint{3.295805in}{1.318257in}}%
\pgfpathlineto{\pgfqpoint{3.297592in}{1.505641in}}%
\pgfpathlineto{\pgfqpoint{3.305185in}{2.535705in}}%
\pgfpathlineto{\pgfqpoint{3.306079in}{2.508886in}}%
\pgfpathlineto{\pgfqpoint{3.307865in}{2.326138in}}%
\pgfpathlineto{\pgfqpoint{3.315459in}{1.283784in}}%
\pgfpathlineto{\pgfqpoint{3.316352in}{1.319832in}}%
\pgfpathlineto{\pgfqpoint{3.318139in}{1.525971in}}%
\pgfpathlineto{\pgfqpoint{3.325286in}{2.534506in}}%
\pgfpathlineto{\pgfqpoint{3.326179in}{2.496614in}}%
\pgfpathlineto{\pgfqpoint{3.327966in}{2.279456in}}%
\pgfpathlineto{\pgfqpoint{3.334666in}{1.282484in}}%
\pgfpathlineto{\pgfqpoint{3.335559in}{1.312322in}}%
\pgfpathlineto{\pgfqpoint{3.337346in}{1.524406in}}%
\pgfpathlineto{\pgfqpoint{3.344046in}{2.535170in}}%
\pgfpathlineto{\pgfqpoint{3.344939in}{2.497035in}}%
\pgfpathlineto{\pgfqpoint{3.346726in}{2.262972in}}%
\pgfpathlineto{\pgfqpoint{3.352979in}{1.282520in}}%
\pgfpathlineto{\pgfqpoint{3.353872in}{1.316535in}}%
\pgfpathlineto{\pgfqpoint{3.355659in}{1.552500in}}%
\pgfpathlineto{\pgfqpoint{3.361466in}{2.533219in}}%
\pgfpathlineto{\pgfqpoint{3.361912in}{2.534300in}}%
\pgfpathlineto{\pgfqpoint{3.362806in}{2.487663in}}%
\pgfpathlineto{\pgfqpoint{3.364592in}{2.223052in}}%
\pgfpathlineto{\pgfqpoint{3.369952in}{1.286923in}}%
\pgfpathlineto{\pgfqpoint{3.370399in}{1.283120in}}%
\pgfpathlineto{\pgfqpoint{3.371292in}{1.327000in}}%
\pgfpathlineto{\pgfqpoint{3.373079in}{1.596145in}}%
\pgfpathlineto{\pgfqpoint{3.378439in}{2.534466in}}%
\pgfpathlineto{\pgfqpoint{3.378886in}{2.532458in}}%
\pgfpathlineto{\pgfqpoint{3.379779in}{2.474645in}}%
\pgfpathlineto{\pgfqpoint{3.382013in}{2.074599in}}%
\pgfpathlineto{\pgfqpoint{3.386033in}{1.306648in}}%
\pgfpathlineto{\pgfqpoint{3.386926in}{1.283988in}}%
\pgfpathlineto{\pgfqpoint{3.387819in}{1.336975in}}%
\pgfpathlineto{\pgfqpoint{3.389606in}{1.637872in}}%
\pgfpathlineto{\pgfqpoint{3.394519in}{2.534493in}}%
\pgfpathlineto{\pgfqpoint{3.394966in}{2.531835in}}%
\pgfpathlineto{\pgfqpoint{3.395859in}{2.467357in}}%
\pgfpathlineto{\pgfqpoint{3.398093in}{2.032413in}}%
\pgfpathlineto{\pgfqpoint{3.402113in}{1.287865in}}%
\pgfpathlineto{\pgfqpoint{3.402559in}{1.283274in}}%
\pgfpathlineto{\pgfqpoint{3.403453in}{1.336537in}}%
\pgfpathlineto{\pgfqpoint{3.405239in}{1.656344in}}%
\pgfpathlineto{\pgfqpoint{3.409706in}{2.530895in}}%
\pgfpathlineto{\pgfqpoint{3.410153in}{2.534658in}}%
\pgfpathlineto{\pgfqpoint{3.411046in}{2.476898in}}%
\pgfpathlineto{\pgfqpoint{3.412833in}{2.141222in}}%
\pgfpathlineto{\pgfqpoint{3.417299in}{1.283122in}}%
\pgfpathlineto{\pgfqpoint{3.417746in}{1.289041in}}%
\pgfpathlineto{\pgfqpoint{3.419086in}{1.438877in}}%
\pgfpathlineto{\pgfqpoint{3.424446in}{2.533412in}}%
\pgfpathlineto{\pgfqpoint{3.425339in}{2.507097in}}%
\pgfpathlineto{\pgfqpoint{3.427126in}{2.198004in}}%
\pgfpathlineto{\pgfqpoint{3.431593in}{1.283640in}}%
\pgfpathlineto{\pgfqpoint{3.432039in}{1.288443in}}%
\pgfpathlineto{\pgfqpoint{3.433379in}{1.446762in}}%
\pgfpathlineto{\pgfqpoint{3.438739in}{2.535484in}}%
\pgfpathlineto{\pgfqpoint{3.439186in}{2.518072in}}%
\pgfpathlineto{\pgfqpoint{3.440526in}{2.320656in}}%
\pgfpathlineto{\pgfqpoint{3.445440in}{1.282463in}}%
\pgfpathlineto{\pgfqpoint{3.445886in}{1.295051in}}%
\pgfpathlineto{\pgfqpoint{3.447226in}{1.485875in}}%
\pgfpathlineto{\pgfqpoint{3.452140in}{2.535788in}}%
\pgfpathlineto{\pgfqpoint{3.452586in}{2.519783in}}%
\pgfpathlineto{\pgfqpoint{3.453926in}{2.314128in}}%
\pgfpathlineto{\pgfqpoint{3.458840in}{1.285600in}}%
\pgfpathlineto{\pgfqpoint{3.459733in}{1.370278in}}%
\pgfpathlineto{\pgfqpoint{3.461966in}{1.945950in}}%
\pgfpathlineto{\pgfqpoint{3.464646in}{2.524601in}}%
\pgfpathlineto{\pgfqpoint{3.465093in}{2.535532in}}%
\pgfpathlineto{\pgfqpoint{3.465986in}{2.466332in}}%
\pgfpathlineto{\pgfqpoint{3.468220in}{1.897009in}}%
\pgfpathlineto{\pgfqpoint{3.470900in}{1.295411in}}%
\pgfpathlineto{\pgfqpoint{3.471346in}{1.282593in}}%
\pgfpathlineto{\pgfqpoint{3.471793in}{1.301598in}}%
\pgfpathlineto{\pgfqpoint{3.473133in}{1.533310in}}%
\pgfpathlineto{\pgfqpoint{3.477600in}{2.534218in}}%
\pgfpathlineto{\pgfqpoint{3.478493in}{2.448897in}}%
\pgfpathlineto{\pgfqpoint{3.480726in}{1.834063in}}%
\pgfpathlineto{\pgfqpoint{3.483406in}{1.283255in}}%
\pgfpathlineto{\pgfqpoint{3.483853in}{1.292966in}}%
\pgfpathlineto{\pgfqpoint{3.485193in}{1.514036in}}%
\pgfpathlineto{\pgfqpoint{3.489660in}{2.531463in}}%
\pgfpathlineto{\pgfqpoint{3.490553in}{2.427624in}}%
\pgfpathlineto{\pgfqpoint{3.495466in}{1.286089in}}%
\pgfpathlineto{\pgfqpoint{3.496360in}{1.389986in}}%
\pgfpathlineto{\pgfqpoint{3.500826in}{2.533082in}}%
\pgfpathlineto{\pgfqpoint{3.501720in}{2.486218in}}%
\pgfpathlineto{\pgfqpoint{3.503506in}{2.019418in}}%
\pgfpathlineto{\pgfqpoint{3.506633in}{1.282447in}}%
\pgfpathlineto{\pgfqpoint{3.507526in}{1.361273in}}%
\pgfpathlineto{\pgfqpoint{3.509760in}{2.034090in}}%
\pgfpathlineto{\pgfqpoint{3.511993in}{2.532904in}}%
\pgfpathlineto{\pgfqpoint{3.512440in}{2.527970in}}%
\pgfpathlineto{\pgfqpoint{3.513780in}{2.284919in}}%
\pgfpathlineto{\pgfqpoint{3.517800in}{1.286390in}}%
\pgfpathlineto{\pgfqpoint{3.518693in}{1.403885in}}%
\pgfpathlineto{\pgfqpoint{3.523160in}{2.531589in}}%
\pgfpathlineto{\pgfqpoint{3.523607in}{2.490881in}}%
\pgfpathlineto{\pgfqpoint{3.525393in}{1.989245in}}%
\pgfpathlineto{\pgfqpoint{3.528073in}{1.285320in}}%
\pgfpathlineto{\pgfqpoint{3.528520in}{1.291755in}}%
\pgfpathlineto{\pgfqpoint{3.529860in}{1.560140in}}%
\pgfpathlineto{\pgfqpoint{3.533433in}{2.535826in}}%
\pgfpathlineto{\pgfqpoint{3.534327in}{2.442023in}}%
\pgfpathlineto{\pgfqpoint{3.538347in}{1.286336in}}%
\pgfpathlineto{\pgfqpoint{3.539240in}{1.342880in}}%
\pgfpathlineto{\pgfqpoint{3.541027in}{1.899807in}}%
\pgfpathlineto{\pgfqpoint{3.543707in}{2.533001in}}%
\pgfpathlineto{\pgfqpoint{3.544600in}{2.404723in}}%
\pgfpathlineto{\pgfqpoint{3.548620in}{1.284039in}}%
\pgfpathlineto{\pgfqpoint{3.549067in}{1.321984in}}%
\pgfpathlineto{\pgfqpoint{3.550853in}{1.867282in}}%
\pgfpathlineto{\pgfqpoint{3.553533in}{2.533063in}}%
\pgfpathlineto{\pgfqpoint{3.554427in}{2.398514in}}%
\pgfpathlineto{\pgfqpoint{3.558000in}{1.287669in}}%
\pgfpathlineto{\pgfqpoint{3.558893in}{1.347352in}}%
\pgfpathlineto{\pgfqpoint{3.560680in}{1.952923in}}%
\pgfpathlineto{\pgfqpoint{3.562913in}{2.535777in}}%
\pgfpathlineto{\pgfqpoint{3.563807in}{2.432560in}}%
\pgfpathlineto{\pgfqpoint{3.567827in}{1.289327in}}%
\pgfpathlineto{\pgfqpoint{3.568273in}{1.345174in}}%
\pgfpathlineto{\pgfqpoint{3.570060in}{1.969303in}}%
\pgfpathlineto{\pgfqpoint{3.572293in}{2.534925in}}%
\pgfpathlineto{\pgfqpoint{3.573187in}{2.400862in}}%
\pgfpathlineto{\pgfqpoint{3.576760in}{1.282444in}}%
\pgfpathlineto{\pgfqpoint{3.577207in}{1.312889in}}%
\pgfpathlineto{\pgfqpoint{3.578547in}{1.711044in}}%
\pgfpathlineto{\pgfqpoint{3.581227in}{2.535824in}}%
\pgfpathlineto{\pgfqpoint{3.582120in}{2.418178in}}%
\pgfpathlineto{\pgfqpoint{3.585694in}{1.282622in}}%
\pgfpathlineto{\pgfqpoint{3.586140in}{1.319088in}}%
\pgfpathlineto{\pgfqpoint{3.587927in}{1.944420in}}%
\pgfpathlineto{\pgfqpoint{3.590160in}{2.532737in}}%
\pgfpathlineto{\pgfqpoint{3.591054in}{2.368386in}}%
\pgfpathlineto{\pgfqpoint{3.594180in}{1.286146in}}%
\pgfpathlineto{\pgfqpoint{3.594627in}{1.297461in}}%
\pgfpathlineto{\pgfqpoint{3.595967in}{1.685800in}}%
\pgfpathlineto{\pgfqpoint{3.598647in}{2.535033in}}%
\pgfpathlineto{\pgfqpoint{3.599540in}{2.381687in}}%
\pgfpathlineto{\pgfqpoint{3.602667in}{1.284492in}}%
\pgfpathlineto{\pgfqpoint{3.603114in}{1.302837in}}%
\pgfpathlineto{\pgfqpoint{3.604454in}{1.722639in}}%
\pgfpathlineto{\pgfqpoint{3.607134in}{2.527331in}}%
\pgfpathlineto{\pgfqpoint{3.608027in}{2.323663in}}%
\pgfpathlineto{\pgfqpoint{3.611154in}{1.286352in}}%
\pgfpathlineto{\pgfqpoint{3.612047in}{1.473940in}}%
\pgfpathlineto{\pgfqpoint{3.615174in}{2.532940in}}%
\pgfpathlineto{\pgfqpoint{3.615620in}{2.474522in}}%
\pgfpathlineto{\pgfqpoint{3.617854in}{1.539420in}}%
\pgfpathlineto{\pgfqpoint{3.619194in}{1.286903in}}%
\pgfpathlineto{\pgfqpoint{3.620087in}{1.485216in}}%
\pgfpathlineto{\pgfqpoint{3.622767in}{2.526095in}}%
\pgfpathlineto{\pgfqpoint{3.623214in}{2.525583in}}%
\pgfpathlineto{\pgfqpoint{3.624554in}{2.102129in}}%
\pgfpathlineto{\pgfqpoint{3.626787in}{1.284439in}}%
\pgfpathlineto{\pgfqpoint{3.627234in}{1.307275in}}%
\pgfpathlineto{\pgfqpoint{3.628574in}{1.785501in}}%
\pgfpathlineto{\pgfqpoint{3.630807in}{2.534838in}}%
\pgfpathlineto{\pgfqpoint{3.631700in}{2.349941in}}%
\pgfpathlineto{\pgfqpoint{3.634380in}{1.288326in}}%
\pgfpathlineto{\pgfqpoint{3.634827in}{1.299463in}}%
\pgfpathlineto{\pgfqpoint{3.636167in}{1.771136in}}%
\pgfpathlineto{\pgfqpoint{3.638400in}{2.534042in}}%
\pgfpathlineto{\pgfqpoint{3.639294in}{2.333498in}}%
\pgfpathlineto{\pgfqpoint{3.641974in}{1.283203in}}%
\pgfpathlineto{\pgfqpoint{3.642421in}{1.316285in}}%
\pgfpathlineto{\pgfqpoint{3.643761in}{1.846900in}}%
\pgfpathlineto{\pgfqpoint{3.645547in}{2.529586in}}%
\pgfpathlineto{\pgfqpoint{3.645994in}{2.517580in}}%
\pgfpathlineto{\pgfqpoint{3.647334in}{2.018340in}}%
\pgfpathlineto{\pgfqpoint{3.649567in}{1.293398in}}%
\pgfpathlineto{\pgfqpoint{3.650907in}{1.774010in}}%
\pgfpathlineto{\pgfqpoint{3.653141in}{2.527218in}}%
\pgfpathlineto{\pgfqpoint{3.654034in}{2.272894in}}%
\pgfpathlineto{\pgfqpoint{3.656714in}{1.292462in}}%
\pgfpathlineto{\pgfqpoint{3.658054in}{1.784670in}}%
\pgfpathlineto{\pgfqpoint{3.659841in}{2.526370in}}%
\pgfpathlineto{\pgfqpoint{3.660287in}{2.519856in}}%
\pgfpathlineto{\pgfqpoint{3.661627in}{1.996360in}}%
\pgfpathlineto{\pgfqpoint{3.663414in}{1.285352in}}%
\pgfpathlineto{\pgfqpoint{3.663861in}{1.312030in}}%
\pgfpathlineto{\pgfqpoint{3.665201in}{1.881353in}}%
\pgfpathlineto{\pgfqpoint{3.666987in}{2.535762in}}%
\pgfpathlineto{\pgfqpoint{3.667881in}{2.329480in}}%
\pgfpathlineto{\pgfqpoint{3.670561in}{1.290939in}}%
\pgfpathlineto{\pgfqpoint{3.671454in}{1.566816in}}%
\pgfpathlineto{\pgfqpoint{3.673687in}{2.534373in}}%
\pgfpathlineto{\pgfqpoint{3.674134in}{2.498045in}}%
\pgfpathlineto{\pgfqpoint{3.675921in}{1.633844in}}%
\pgfpathlineto{\pgfqpoint{3.677261in}{1.289236in}}%
\pgfpathlineto{\pgfqpoint{3.678154in}{1.566571in}}%
\pgfpathlineto{\pgfqpoint{3.680387in}{2.535630in}}%
\pgfpathlineto{\pgfqpoint{3.680834in}{2.486385in}}%
\pgfpathlineto{\pgfqpoint{3.683514in}{1.291660in}}%
\pgfpathlineto{\pgfqpoint{3.684407in}{1.425636in}}%
\pgfpathlineto{\pgfqpoint{3.687087in}{2.529185in}}%
\pgfpathlineto{\pgfqpoint{3.687534in}{2.433018in}}%
\pgfpathlineto{\pgfqpoint{3.690214in}{1.283652in}}%
\pgfpathlineto{\pgfqpoint{3.690661in}{1.359561in}}%
\pgfpathlineto{\pgfqpoint{3.693341in}{2.535829in}}%
\pgfpathlineto{\pgfqpoint{3.693787in}{2.473221in}}%
\pgfpathlineto{\pgfqpoint{3.696467in}{1.282542in}}%
\pgfpathlineto{\pgfqpoint{3.697361in}{1.508931in}}%
\pgfpathlineto{\pgfqpoint{3.699594in}{2.535792in}}%
\pgfpathlineto{\pgfqpoint{3.700041in}{2.476507in}}%
\pgfpathlineto{\pgfqpoint{3.702721in}{1.282586in}}%
\pgfpathlineto{\pgfqpoint{3.703167in}{1.352546in}}%
\pgfpathlineto{\pgfqpoint{3.705848in}{2.533523in}}%
\pgfpathlineto{\pgfqpoint{3.706294in}{2.444507in}}%
\pgfpathlineto{\pgfqpoint{3.708974in}{1.292742in}}%
\pgfpathlineto{\pgfqpoint{3.709421in}{1.408907in}}%
\pgfpathlineto{\pgfqpoint{3.711654in}{2.527750in}}%
\pgfpathlineto{\pgfqpoint{3.712101in}{2.506390in}}%
\pgfpathlineto{\pgfqpoint{3.713888in}{1.545850in}}%
\pgfpathlineto{\pgfqpoint{3.714781in}{1.282463in}}%
\pgfpathlineto{\pgfqpoint{3.715228in}{1.349060in}}%
\pgfpathlineto{\pgfqpoint{3.717908in}{2.526116in}}%
\pgfpathlineto{\pgfqpoint{3.718354in}{2.406463in}}%
\pgfpathlineto{\pgfqpoint{3.720588in}{1.284926in}}%
\pgfpathlineto{\pgfqpoint{3.721034in}{1.330119in}}%
\pgfpathlineto{\pgfqpoint{3.723714in}{2.528381in}}%
\pgfpathlineto{\pgfqpoint{3.724161in}{2.411548in}}%
\pgfpathlineto{\pgfqpoint{3.726394in}{1.283565in}}%
\pgfpathlineto{\pgfqpoint{3.726841in}{1.339586in}}%
\pgfpathlineto{\pgfqpoint{3.729521in}{2.517296in}}%
\pgfpathlineto{\pgfqpoint{3.729968in}{2.372457in}}%
\pgfpathlineto{\pgfqpoint{3.732201in}{1.284530in}}%
\pgfpathlineto{\pgfqpoint{3.732648in}{1.385119in}}%
\pgfpathlineto{\pgfqpoint{3.734881in}{2.534853in}}%
\pgfpathlineto{\pgfqpoint{3.735328in}{2.474361in}}%
\pgfpathlineto{\pgfqpoint{3.737561in}{1.291448in}}%
\pgfpathlineto{\pgfqpoint{3.738454in}{1.493259in}}%
\pgfpathlineto{\pgfqpoint{3.740688in}{2.515920in}}%
\pgfpathlineto{\pgfqpoint{3.742028in}{1.772415in}}%
\pgfpathlineto{\pgfqpoint{3.743368in}{1.294116in}}%
\pgfpathlineto{\pgfqpoint{3.744708in}{2.017000in}}%
\pgfpathlineto{\pgfqpoint{3.746048in}{2.527948in}}%
\pgfpathlineto{\pgfqpoint{3.747388in}{1.814712in}}%
\pgfpathlineto{\pgfqpoint{3.748728in}{1.289476in}}%
\pgfpathlineto{\pgfqpoint{3.749621in}{1.681958in}}%
\pgfpathlineto{\pgfqpoint{3.751408in}{2.527173in}}%
\pgfpathlineto{\pgfqpoint{3.752748in}{1.794153in}}%
\pgfpathlineto{\pgfqpoint{3.754088in}{1.296078in}}%
\pgfpathlineto{\pgfqpoint{3.755428in}{2.058068in}}%
\pgfpathlineto{\pgfqpoint{3.756321in}{2.514878in}}%
\pgfpathlineto{\pgfqpoint{3.756768in}{2.511888in}}%
\pgfpathlineto{\pgfqpoint{3.759001in}{1.291387in}}%
\pgfpathlineto{\pgfqpoint{3.759894in}{1.526489in}}%
\pgfpathlineto{\pgfqpoint{3.761681in}{2.534984in}}%
\pgfpathlineto{\pgfqpoint{3.762128in}{2.461174in}}%
\pgfpathlineto{\pgfqpoint{3.764361in}{1.285025in}}%
\pgfpathlineto{\pgfqpoint{3.764808in}{1.406805in}}%
\pgfpathlineto{\pgfqpoint{3.767041in}{2.514632in}}%
\pgfpathlineto{\pgfqpoint{3.769275in}{1.285589in}}%
\pgfpathlineto{\pgfqpoint{3.770168in}{1.579672in}}%
\pgfpathlineto{\pgfqpoint{3.771955in}{2.531365in}}%
\pgfpathlineto{\pgfqpoint{3.772848in}{2.108983in}}%
\pgfpathlineto{\pgfqpoint{3.774188in}{1.294900in}}%
\pgfpathlineto{\pgfqpoint{3.774635in}{1.324183in}}%
\pgfpathlineto{\pgfqpoint{3.776868in}{2.534281in}}%
\pgfpathlineto{\pgfqpoint{3.777315in}{2.411041in}}%
\pgfpathlineto{\pgfqpoint{3.779101in}{1.295727in}}%
\pgfpathlineto{\pgfqpoint{3.779548in}{1.324627in}}%
\pgfpathlineto{\pgfqpoint{3.781781in}{2.531676in}}%
\pgfpathlineto{\pgfqpoint{3.782228in}{2.390326in}}%
\pgfpathlineto{\pgfqpoint{3.784015in}{1.286836in}}%
\pgfpathlineto{\pgfqpoint{3.784461in}{1.349107in}}%
\pgfpathlineto{\pgfqpoint{3.786695in}{2.515538in}}%
\pgfpathlineto{\pgfqpoint{3.787141in}{2.325598in}}%
\pgfpathlineto{\pgfqpoint{3.788928in}{1.283769in}}%
\pgfpathlineto{\pgfqpoint{3.789375in}{1.413190in}}%
\pgfpathlineto{\pgfqpoint{3.791161in}{2.532391in}}%
\pgfpathlineto{\pgfqpoint{3.791608in}{2.461619in}}%
\pgfpathlineto{\pgfqpoint{3.793395in}{1.302706in}}%
\pgfpathlineto{\pgfqpoint{3.793841in}{1.319538in}}%
\pgfpathlineto{\pgfqpoint{3.796075in}{2.520699in}}%
\pgfpathlineto{\pgfqpoint{3.796521in}{2.333940in}}%
\pgfpathlineto{\pgfqpoint{3.798308in}{1.286609in}}%
\pgfpathlineto{\pgfqpoint{3.798755in}{1.440126in}}%
\pgfpathlineto{\pgfqpoint{3.800541in}{2.535545in}}%
\pgfpathlineto{\pgfqpoint{3.800988in}{2.409936in}}%
\pgfpathlineto{\pgfqpoint{3.802775in}{1.282773in}}%
\pgfpathlineto{\pgfqpoint{3.803221in}{1.386899in}}%
\pgfpathlineto{\pgfqpoint{3.805008in}{2.534086in}}%
\pgfpathlineto{\pgfqpoint{3.805455in}{2.444288in}}%
\pgfpathlineto{\pgfqpoint{3.807241in}{1.285355in}}%
\pgfpathlineto{\pgfqpoint{3.807688in}{1.368462in}}%
\pgfpathlineto{\pgfqpoint{3.809475in}{2.532920in}}%
\pgfpathlineto{\pgfqpoint{3.809921in}{2.448548in}}%
\pgfpathlineto{\pgfqpoint{3.811708in}{1.284231in}}%
\pgfpathlineto{\pgfqpoint{3.812155in}{1.378028in}}%
\pgfpathlineto{\pgfqpoint{3.813941in}{2.535508in}}%
\pgfpathlineto{\pgfqpoint{3.814388in}{2.424055in}}%
\pgfpathlineto{\pgfqpoint{3.816175in}{1.282753in}}%
\pgfpathlineto{\pgfqpoint{3.816621in}{1.419961in}}%
\pgfpathlineto{\pgfqpoint{3.818408in}{2.531522in}}%
\pgfpathlineto{\pgfqpoint{3.818855in}{2.360920in}}%
\pgfpathlineto{\pgfqpoint{3.820641in}{1.298330in}}%
\pgfpathlineto{\pgfqpoint{3.822428in}{2.507802in}}%
\pgfpathlineto{\pgfqpoint{3.823321in}{2.240940in}}%
\pgfpathlineto{\pgfqpoint{3.824661in}{1.289697in}}%
\pgfpathlineto{\pgfqpoint{3.825108in}{1.361676in}}%
\pgfpathlineto{\pgfqpoint{3.826895in}{2.535745in}}%
\pgfpathlineto{\pgfqpoint{3.827341in}{2.395199in}}%
\pgfpathlineto{\pgfqpoint{3.829128in}{1.296341in}}%
\pgfpathlineto{\pgfqpoint{3.829575in}{1.510135in}}%
\pgfpathlineto{\pgfqpoint{3.830915in}{2.516877in}}%
\pgfpathlineto{\pgfqpoint{3.831361in}{2.479684in}}%
\pgfpathlineto{\pgfqpoint{3.833148in}{1.282495in}}%
\pgfpathlineto{\pgfqpoint{3.833595in}{1.415992in}}%
\pgfpathlineto{\pgfqpoint{3.835382in}{2.517283in}}%
\pgfpathlineto{\pgfqpoint{3.837168in}{1.290671in}}%
\pgfpathlineto{\pgfqpoint{3.838062in}{1.688520in}}%
\pgfpathlineto{\pgfqpoint{3.839402in}{2.530031in}}%
\pgfpathlineto{\pgfqpoint{3.839848in}{2.333398in}}%
\pgfpathlineto{\pgfqpoint{3.841188in}{1.299858in}}%
\pgfpathlineto{\pgfqpoint{3.841635in}{1.347114in}}%
\pgfpathlineto{\pgfqpoint{3.843422in}{2.532428in}}%
\pgfpathlineto{\pgfqpoint{3.843868in}{2.344051in}}%
\pgfpathlineto{\pgfqpoint{3.845208in}{1.299541in}}%
\pgfpathlineto{\pgfqpoint{3.845655in}{1.350113in}}%
\pgfpathlineto{\pgfqpoint{3.847442in}{2.528862in}}%
\pgfpathlineto{\pgfqpoint{3.847888in}{2.319289in}}%
\pgfpathlineto{\pgfqpoint{3.849228in}{1.289934in}}%
\pgfpathlineto{\pgfqpoint{3.849675in}{1.377167in}}%
\pgfpathlineto{\pgfqpoint{3.851462in}{2.512793in}}%
\pgfpathlineto{\pgfqpoint{3.851908in}{2.253703in}}%
\pgfpathlineto{\pgfqpoint{3.853248in}{1.282445in}}%
\pgfpathlineto{\pgfqpoint{3.853695in}{1.439316in}}%
\pgfpathlineto{\pgfqpoint{3.855035in}{2.516204in}}%
\pgfpathlineto{\pgfqpoint{3.855482in}{2.466929in}}%
\pgfpathlineto{\pgfqpoint{3.857268in}{1.300300in}}%
\pgfpathlineto{\pgfqpoint{3.857715in}{1.554164in}}%
\pgfpathlineto{\pgfqpoint{3.859055in}{2.535709in}}%
\pgfpathlineto{\pgfqpoint{3.859502in}{2.365847in}}%
\pgfpathlineto{\pgfqpoint{3.860842in}{1.292723in}}%
\pgfpathlineto{\pgfqpoint{3.861288in}{1.376817in}}%
\pgfpathlineto{\pgfqpoint{3.862628in}{2.493941in}}%
\pgfpathlineto{\pgfqpoint{3.863075in}{2.492630in}}%
\pgfpathlineto{\pgfqpoint{3.864862in}{1.295855in}}%
\pgfpathlineto{\pgfqpoint{3.865308in}{1.546444in}}%
\pgfpathlineto{\pgfqpoint{3.866648in}{2.534824in}}%
\pgfpathlineto{\pgfqpoint{3.867095in}{2.342360in}}%
\pgfpathlineto{\pgfqpoint{3.868435in}{1.284285in}}%
\pgfpathlineto{\pgfqpoint{3.868882in}{1.420977in}}%
\pgfpathlineto{\pgfqpoint{3.870222in}{2.524072in}}%
\pgfpathlineto{\pgfqpoint{3.870668in}{2.438638in}}%
\pgfpathlineto{\pgfqpoint{3.872008in}{1.309548in}}%
\pgfpathlineto{\pgfqpoint{3.872455in}{1.349702in}}%
\pgfpathlineto{\pgfqpoint{3.873795in}{2.491219in}}%
\pgfpathlineto{\pgfqpoint{3.874242in}{2.489451in}}%
\pgfpathlineto{\pgfqpoint{3.876028in}{1.314955in}}%
\pgfpathlineto{\pgfqpoint{3.876475in}{1.622426in}}%
\pgfpathlineto{\pgfqpoint{3.877815in}{2.512030in}}%
\pgfpathlineto{\pgfqpoint{3.879602in}{1.301381in}}%
\pgfpathlineto{\pgfqpoint{3.880048in}{1.584883in}}%
\pgfpathlineto{\pgfqpoint{3.881388in}{2.518873in}}%
\pgfpathlineto{\pgfqpoint{3.883175in}{1.299977in}}%
\pgfpathlineto{\pgfqpoint{3.883622in}{1.584428in}}%
\pgfpathlineto{\pgfqpoint{3.884962in}{2.515098in}}%
\pgfpathlineto{\pgfqpoint{3.886748in}{1.309730in}}%
\pgfpathlineto{\pgfqpoint{3.887195in}{1.621422in}}%
\pgfpathlineto{\pgfqpoint{3.888535in}{2.497505in}}%
\pgfpathlineto{\pgfqpoint{3.889875in}{1.327337in}}%
\pgfpathlineto{\pgfqpoint{3.890322in}{1.338048in}}%
\pgfpathlineto{\pgfqpoint{3.891662in}{2.508796in}}%
\pgfpathlineto{\pgfqpoint{3.892108in}{2.454640in}}%
\pgfpathlineto{\pgfqpoint{3.893448in}{1.293888in}}%
\pgfpathlineto{\pgfqpoint{3.893895in}{1.400042in}}%
\pgfpathlineto{\pgfqpoint{3.895235in}{2.534300in}}%
\pgfpathlineto{\pgfqpoint{3.895682in}{2.368472in}}%
\pgfpathlineto{\pgfqpoint{3.897022in}{1.283918in}}%
\pgfpathlineto{\pgfqpoint{3.897468in}{1.515428in}}%
\pgfpathlineto{\pgfqpoint{3.898808in}{2.520004in}}%
\pgfpathlineto{\pgfqpoint{3.900595in}{1.331913in}}%
\pgfpathlineto{\pgfqpoint{3.901042in}{1.700784in}}%
\pgfpathlineto{\pgfqpoint{3.901935in}{2.517499in}}%
\pgfpathlineto{\pgfqpoint{3.902382in}{2.428956in}}%
\pgfpathlineto{\pgfqpoint{3.903722in}{1.282856in}}%
\pgfpathlineto{\pgfqpoint{3.904169in}{1.474107in}}%
\pgfpathlineto{\pgfqpoint{3.905509in}{2.524779in}}%
\pgfpathlineto{\pgfqpoint{3.905955in}{2.230385in}}%
\pgfpathlineto{\pgfqpoint{3.906849in}{1.336094in}}%
\pgfpathlineto{\pgfqpoint{3.907295in}{1.339764in}}%
\pgfpathlineto{\pgfqpoint{3.908635in}{2.527824in}}%
\pgfpathlineto{\pgfqpoint{3.909082in}{2.391588in}}%
\pgfpathlineto{\pgfqpoint{3.910422in}{1.286576in}}%
\pgfpathlineto{\pgfqpoint{3.910869in}{1.555894in}}%
\pgfpathlineto{\pgfqpoint{3.912209in}{2.484260in}}%
\pgfpathlineto{\pgfqpoint{3.913549in}{1.288957in}}%
\pgfpathlineto{\pgfqpoint{3.913995in}{1.438377in}}%
\pgfpathlineto{\pgfqpoint{3.915335in}{2.525480in}}%
\pgfpathlineto{\pgfqpoint{3.915782in}{2.219836in}}%
\pgfpathlineto{\pgfqpoint{3.916675in}{1.319595in}}%
\pgfpathlineto{\pgfqpoint{3.917122in}{1.367615in}}%
\pgfpathlineto{\pgfqpoint{3.918462in}{2.535790in}}%
\pgfpathlineto{\pgfqpoint{3.918909in}{2.302763in}}%
\pgfpathlineto{\pgfqpoint{3.920249in}{1.329988in}}%
\pgfpathlineto{\pgfqpoint{3.921589in}{2.532908in}}%
\pgfpathlineto{\pgfqpoint{3.922035in}{2.350362in}}%
\pgfpathlineto{\pgfqpoint{3.923375in}{1.313328in}}%
\pgfpathlineto{\pgfqpoint{3.924715in}{2.528962in}}%
\pgfpathlineto{\pgfqpoint{3.925162in}{2.369095in}}%
\pgfpathlineto{\pgfqpoint{3.926502in}{1.310004in}}%
\pgfpathlineto{\pgfqpoint{3.927842in}{2.529680in}}%
\pgfpathlineto{\pgfqpoint{3.928289in}{2.362010in}}%
\pgfpathlineto{\pgfqpoint{3.929629in}{1.318288in}}%
\pgfpathlineto{\pgfqpoint{3.930969in}{2.534204in}}%
\pgfpathlineto{\pgfqpoint{3.931415in}{2.327537in}}%
\pgfpathlineto{\pgfqpoint{3.932755in}{1.342768in}}%
\pgfpathlineto{\pgfqpoint{3.934095in}{2.534861in}}%
\pgfpathlineto{\pgfqpoint{3.934542in}{2.259767in}}%
\pgfpathlineto{\pgfqpoint{3.935435in}{1.314044in}}%
\pgfpathlineto{\pgfqpoint{3.935882in}{1.393960in}}%
\pgfpathlineto{\pgfqpoint{3.937222in}{2.517009in}}%
\pgfpathlineto{\pgfqpoint{3.937669in}{2.150459in}}%
\pgfpathlineto{\pgfqpoint{3.938562in}{1.285753in}}%
\pgfpathlineto{\pgfqpoint{3.939009in}{1.486517in}}%
\pgfpathlineto{\pgfqpoint{3.939902in}{2.476134in}}%
\pgfpathlineto{\pgfqpoint{3.940349in}{2.459954in}}%
\pgfpathlineto{\pgfqpoint{3.941689in}{1.292042in}}%
\pgfpathlineto{\pgfqpoint{3.942135in}{1.634787in}}%
\pgfpathlineto{\pgfqpoint{3.943029in}{2.530124in}}%
\pgfpathlineto{\pgfqpoint{3.943475in}{2.340797in}}%
\pgfpathlineto{\pgfqpoint{3.944369in}{1.340599in}}%
\pgfpathlineto{\pgfqpoint{3.944815in}{1.364011in}}%
\pgfpathlineto{\pgfqpoint{3.946155in}{2.518997in}}%
\pgfpathlineto{\pgfqpoint{3.946602in}{2.143405in}}%
\pgfpathlineto{\pgfqpoint{3.947495in}{1.282930in}}%
\pgfpathlineto{\pgfqpoint{3.947942in}{1.529349in}}%
\pgfpathlineto{\pgfqpoint{3.948835in}{2.507934in}}%
\pgfpathlineto{\pgfqpoint{3.949282in}{2.402884in}}%
\pgfpathlineto{\pgfqpoint{3.950622in}{1.336664in}}%
\pgfpathlineto{\pgfqpoint{3.951962in}{2.524995in}}%
\pgfpathlineto{\pgfqpoint{3.952409in}{2.160189in}}%
\pgfpathlineto{\pgfqpoint{3.953302in}{1.282815in}}%
\pgfpathlineto{\pgfqpoint{3.953749in}{1.540438in}}%
\pgfpathlineto{\pgfqpoint{3.954642in}{2.517385in}}%
\pgfpathlineto{\pgfqpoint{3.955089in}{2.373258in}}%
\pgfpathlineto{\pgfqpoint{3.955982in}{1.342209in}}%
\pgfpathlineto{\pgfqpoint{3.956429in}{1.372891in}}%
\pgfpathlineto{\pgfqpoint{3.957769in}{2.495005in}}%
\pgfpathlineto{\pgfqpoint{3.959109in}{1.294219in}}%
\pgfpathlineto{\pgfqpoint{3.959555in}{1.674923in}}%
\pgfpathlineto{\pgfqpoint{3.960449in}{2.535408in}}%
\pgfpathlineto{\pgfqpoint{3.960895in}{2.229053in}}%
\pgfpathlineto{\pgfqpoint{3.961789in}{1.286037in}}%
\pgfpathlineto{\pgfqpoint{3.962235in}{1.516922in}}%
\pgfpathlineto{\pgfqpoint{3.963129in}{2.517959in}}%
\pgfpathlineto{\pgfqpoint{3.963575in}{2.360670in}}%
\pgfpathlineto{\pgfqpoint{3.964469in}{1.322525in}}%
\pgfpathlineto{\pgfqpoint{3.964916in}{1.410359in}}%
\pgfpathlineto{\pgfqpoint{3.965809in}{2.468643in}}%
\pgfpathlineto{\pgfqpoint{3.966256in}{2.444511in}}%
\pgfpathlineto{\pgfqpoint{3.967596in}{1.346315in}}%
\pgfpathlineto{\pgfqpoint{3.968936in}{2.491933in}}%
\pgfpathlineto{\pgfqpoint{3.970276in}{1.312360in}}%
\pgfpathlineto{\pgfqpoint{3.970722in}{1.764626in}}%
\pgfpathlineto{\pgfqpoint{3.971616in}{2.515355in}}%
\pgfpathlineto{\pgfqpoint{3.972062in}{2.083731in}}%
\pgfpathlineto{\pgfqpoint{3.972956in}{1.296900in}}%
\pgfpathlineto{\pgfqpoint{3.973402in}{1.712407in}}%
\pgfpathlineto{\pgfqpoint{3.974296in}{2.524965in}}%
\pgfpathlineto{\pgfqpoint{3.974742in}{2.120316in}}%
\pgfpathlineto{\pgfqpoint{3.975636in}{1.291556in}}%
\pgfpathlineto{\pgfqpoint{3.976082in}{1.691116in}}%
\pgfpathlineto{\pgfqpoint{3.976976in}{2.527067in}}%
\pgfpathlineto{\pgfqpoint{3.977422in}{2.126521in}}%
\pgfpathlineto{\pgfqpoint{3.978316in}{1.292261in}}%
\pgfpathlineto{\pgfqpoint{3.978762in}{1.699974in}}%
\pgfpathlineto{\pgfqpoint{3.979656in}{2.523388in}}%
\pgfpathlineto{\pgfqpoint{3.980102in}{2.102418in}}%
\pgfpathlineto{\pgfqpoint{3.980996in}{1.299686in}}%
\pgfpathlineto{\pgfqpoint{3.981442in}{1.739688in}}%
\pgfpathlineto{\pgfqpoint{3.982336in}{2.510836in}}%
\pgfpathlineto{\pgfqpoint{3.983676in}{1.319332in}}%
\pgfpathlineto{\pgfqpoint{3.985016in}{2.481579in}}%
\pgfpathlineto{\pgfqpoint{3.986356in}{1.361198in}}%
\pgfpathlineto{\pgfqpoint{3.987249in}{2.466521in}}%
\pgfpathlineto{\pgfqpoint{3.987696in}{2.423746in}}%
\pgfpathlineto{\pgfqpoint{3.988589in}{1.323800in}}%
\pgfpathlineto{\pgfqpoint{3.989036in}{1.438529in}}%
\pgfpathlineto{\pgfqpoint{3.989929in}{2.517497in}}%
\pgfpathlineto{\pgfqpoint{3.990376in}{2.323473in}}%
\pgfpathlineto{\pgfqpoint{3.991269in}{1.285996in}}%
\pgfpathlineto{\pgfqpoint{3.991716in}{1.564743in}}%
\pgfpathlineto{\pgfqpoint{3.992609in}{2.535271in}}%
\pgfpathlineto{\pgfqpoint{3.993056in}{2.169229in}}%
\pgfpathlineto{\pgfqpoint{3.993949in}{1.295654in}}%
\pgfpathlineto{\pgfqpoint{3.994396in}{1.747645in}}%
\pgfpathlineto{\pgfqpoint{3.995289in}{2.490677in}}%
\pgfpathlineto{\pgfqpoint{3.996182in}{1.371671in}}%
\pgfpathlineto{\pgfqpoint{3.996629in}{1.382137in}}%
\pgfpathlineto{\pgfqpoint{3.997522in}{2.499699in}}%
\pgfpathlineto{\pgfqpoint{3.997969in}{2.356781in}}%
\pgfpathlineto{\pgfqpoint{3.998862in}{1.287635in}}%
\pgfpathlineto{\pgfqpoint{3.999309in}{1.566460in}}%
\pgfpathlineto{\pgfqpoint{4.000202in}{2.532880in}}%
\pgfpathlineto{\pgfqpoint{4.000649in}{2.122753in}}%
\pgfpathlineto{\pgfqpoint{4.001542in}{1.317744in}}%
\pgfpathlineto{\pgfqpoint{4.002436in}{2.445905in}}%
\pgfpathlineto{\pgfqpoint{4.002882in}{2.429579in}}%
\pgfpathlineto{\pgfqpoint{4.003776in}{1.306760in}}%
\pgfpathlineto{\pgfqpoint{4.004222in}{1.499515in}}%
\pgfpathlineto{\pgfqpoint{4.005116in}{2.535845in}}%
\pgfpathlineto{\pgfqpoint{4.005562in}{2.170943in}}%
\pgfpathlineto{\pgfqpoint{4.006456in}{1.308255in}}%
\pgfpathlineto{\pgfqpoint{4.006902in}{1.824432in}}%
\pgfpathlineto{\pgfqpoint{4.007349in}{2.440912in}}%
\pgfpathlineto{\pgfqpoint{4.007796in}{2.428953in}}%
\pgfpathlineto{\pgfqpoint{4.008689in}{1.300940in}}%
\pgfpathlineto{\pgfqpoint{4.009136in}{1.525230in}}%
\pgfpathlineto{\pgfqpoint{4.010029in}{2.533606in}}%
\pgfpathlineto{\pgfqpoint{4.010476in}{2.110137in}}%
\pgfpathlineto{\pgfqpoint{4.011369in}{1.338684in}}%
\pgfpathlineto{\pgfqpoint{4.012262in}{2.490153in}}%
\pgfpathlineto{\pgfqpoint{4.012709in}{2.353182in}}%
\pgfpathlineto{\pgfqpoint{4.013602in}{1.282547in}}%
\pgfpathlineto{\pgfqpoint{4.014049in}{1.655516in}}%
\pgfpathlineto{\pgfqpoint{4.014942in}{2.495216in}}%
\pgfpathlineto{\pgfqpoint{4.015836in}{1.337841in}}%
\pgfpathlineto{\pgfqpoint{4.016282in}{1.453475in}}%
\pgfpathlineto{\pgfqpoint{4.017176in}{2.535729in}}%
\pgfpathlineto{\pgfqpoint{4.017622in}{2.156081in}}%
\pgfpathlineto{\pgfqpoint{4.018516in}{1.331754in}}%
\pgfpathlineto{\pgfqpoint{4.019409in}{2.495529in}}%
\pgfpathlineto{\pgfqpoint{4.019856in}{2.330901in}}%
\pgfpathlineto{\pgfqpoint{4.020749in}{1.284564in}}%
\pgfpathlineto{\pgfqpoint{4.021196in}{1.729683in}}%
\pgfpathlineto{\pgfqpoint{4.021642in}{2.403655in}}%
\pgfpathlineto{\pgfqpoint{4.021642in}{2.403655in}}%
\pgfusepath{stroke}%
\end{pgfscope}%
\begin{pgfscope}%
\pgfpathrectangle{\pgfqpoint{1.105307in}{1.282443in}}{\pgfqpoint{2.916161in}{1.253404in}}%
\pgfusepath{clip}%
\pgfsetrectcap%
\pgfsetroundjoin%
\pgfsetlinewidth{1.003750pt}%
\definecolor{currentstroke}{rgb}{1.000000,0.000000,0.000000}%
\pgfsetstrokecolor{currentstroke}%
\pgfsetdash{}{0pt}%
\pgfpathmoveto{\pgfqpoint{1.104894in}{1.287430in}}%
\pgfpathlineto{\pgfqpoint{1.204055in}{1.291540in}}%
\pgfpathlineto{\pgfqpoint{1.279542in}{1.296809in}}%
\pgfpathlineto{\pgfqpoint{1.340735in}{1.303238in}}%
\pgfpathlineto{\pgfqpoint{1.392549in}{1.310868in}}%
\pgfpathlineto{\pgfqpoint{1.437216in}{1.319636in}}%
\pgfpathlineto{\pgfqpoint{1.476969in}{1.329661in}}%
\pgfpathlineto{\pgfqpoint{1.512703in}{1.340921in}}%
\pgfpathlineto{\pgfqpoint{1.545310in}{1.353478in}}%
\pgfpathlineto{\pgfqpoint{1.575683in}{1.367532in}}%
\pgfpathlineto{\pgfqpoint{1.604270in}{1.383215in}}%
\pgfpathlineto{\pgfqpoint{1.631070in}{1.400445in}}%
\pgfpathlineto{\pgfqpoint{1.656530in}{1.419423in}}%
\pgfpathlineto{\pgfqpoint{1.681097in}{1.440487in}}%
\pgfpathlineto{\pgfqpoint{1.705217in}{1.464140in}}%
\pgfpathlineto{\pgfqpoint{1.728444in}{1.490037in}}%
\pgfpathlineto{\pgfqpoint{1.751224in}{1.518745in}}%
\pgfpathlineto{\pgfqpoint{1.774004in}{1.551067in}}%
\pgfpathlineto{\pgfqpoint{1.796784in}{1.587344in}}%
\pgfpathlineto{\pgfqpoint{1.819565in}{1.627910in}}%
\pgfpathlineto{\pgfqpoint{1.842345in}{1.673078in}}%
\pgfpathlineto{\pgfqpoint{1.865571in}{1.724145in}}%
\pgfpathlineto{\pgfqpoint{1.889245in}{1.781609in}}%
\pgfpathlineto{\pgfqpoint{1.914258in}{1.848346in}}%
\pgfpathlineto{\pgfqpoint{1.940612in}{1.925177in}}%
\pgfpathlineto{\pgfqpoint{1.970092in}{2.018293in}}%
\pgfpathlineto{\pgfqpoint{2.007165in}{2.143376in}}%
\pgfpathlineto{\pgfqpoint{2.069252in}{2.353597in}}%
\pgfpathlineto{\pgfqpoint{2.090693in}{2.417828in}}%
\pgfpathlineto{\pgfqpoint{2.107219in}{2.460857in}}%
\pgfpathlineto{\pgfqpoint{2.121066in}{2.491014in}}%
\pgfpathlineto{\pgfqpoint{2.132679in}{2.511187in}}%
\pgfpathlineto{\pgfqpoint{2.142506in}{2.523989in}}%
\pgfpathlineto{\pgfqpoint{2.150993in}{2.531494in}}%
\pgfpathlineto{\pgfqpoint{2.158140in}{2.535013in}}%
\pgfpathlineto{\pgfqpoint{2.164393in}{2.535829in}}%
\pgfpathlineto{\pgfqpoint{2.170646in}{2.534399in}}%
\pgfpathlineto{\pgfqpoint{2.176900in}{2.530599in}}%
\pgfpathlineto{\pgfqpoint{2.183600in}{2.523759in}}%
\pgfpathlineto{\pgfqpoint{2.190746in}{2.513150in}}%
\pgfpathlineto{\pgfqpoint{2.198786in}{2.496933in}}%
\pgfpathlineto{\pgfqpoint{2.207273in}{2.474679in}}%
\pgfpathlineto{\pgfqpoint{2.216653in}{2.443712in}}%
\pgfpathlineto{\pgfqpoint{2.226927in}{2.401901in}}%
\pgfpathlineto{\pgfqpoint{2.238093in}{2.346961in}}%
\pgfpathlineto{\pgfqpoint{2.250600in}{2.273791in}}%
\pgfpathlineto{\pgfqpoint{2.264447in}{2.179103in}}%
\pgfpathlineto{\pgfqpoint{2.280527in}{2.053145in}}%
\pgfpathlineto{\pgfqpoint{2.301520in}{1.869670in}}%
\pgfpathlineto{\pgfqpoint{2.342614in}{1.506571in}}%
\pgfpathlineto{\pgfqpoint{2.356014in}{1.409698in}}%
\pgfpathlineto{\pgfqpoint{2.366287in}{1.350414in}}%
\pgfpathlineto{\pgfqpoint{2.374327in}{1.315534in}}%
\pgfpathlineto{\pgfqpoint{2.381027in}{1.295439in}}%
\pgfpathlineto{\pgfqpoint{2.386387in}{1.285846in}}%
\pgfpathlineto{\pgfqpoint{2.390407in}{1.282696in}}%
\pgfpathlineto{\pgfqpoint{2.393534in}{1.282750in}}%
\pgfpathlineto{\pgfqpoint{2.396661in}{1.285065in}}%
\pgfpathlineto{\pgfqpoint{2.400234in}{1.290552in}}%
\pgfpathlineto{\pgfqpoint{2.404701in}{1.301774in}}%
\pgfpathlineto{\pgfqpoint{2.410061in}{1.321780in}}%
\pgfpathlineto{\pgfqpoint{2.416314in}{1.354265in}}%
\pgfpathlineto{\pgfqpoint{2.423461in}{1.403446in}}%
\pgfpathlineto{\pgfqpoint{2.431948in}{1.478110in}}%
\pgfpathlineto{\pgfqpoint{2.441774in}{1.585030in}}%
\pgfpathlineto{\pgfqpoint{2.454281in}{1.747654in}}%
\pgfpathlineto{\pgfqpoint{2.474381in}{2.044193in}}%
\pgfpathlineto{\pgfqpoint{2.492695in}{2.302561in}}%
\pgfpathlineto{\pgfqpoint{2.502968in}{2.418550in}}%
\pgfpathlineto{\pgfqpoint{2.510561in}{2.482272in}}%
\pgfpathlineto{\pgfqpoint{2.516368in}{2.515380in}}%
\pgfpathlineto{\pgfqpoint{2.520835in}{2.530470in}}%
\pgfpathlineto{\pgfqpoint{2.523961in}{2.535282in}}%
\pgfpathlineto{\pgfqpoint{2.526195in}{2.535704in}}%
\pgfpathlineto{\pgfqpoint{2.528428in}{2.533549in}}%
\pgfpathlineto{\pgfqpoint{2.531108in}{2.527504in}}%
\pgfpathlineto{\pgfqpoint{2.534682in}{2.513485in}}%
\pgfpathlineto{\pgfqpoint{2.539148in}{2.486307in}}%
\pgfpathlineto{\pgfqpoint{2.544955in}{2.435059in}}%
\pgfpathlineto{\pgfqpoint{2.551655in}{2.354445in}}%
\pgfpathlineto{\pgfqpoint{2.560142in}{2.222661in}}%
\pgfpathlineto{\pgfqpoint{2.571308in}{2.010719in}}%
\pgfpathlineto{\pgfqpoint{2.599449in}{1.458098in}}%
\pgfpathlineto{\pgfqpoint{2.607042in}{1.356845in}}%
\pgfpathlineto{\pgfqpoint{2.612849in}{1.306416in}}%
\pgfpathlineto{\pgfqpoint{2.616869in}{1.287457in}}%
\pgfpathlineto{\pgfqpoint{2.619549in}{1.282638in}}%
\pgfpathlineto{\pgfqpoint{2.621335in}{1.283030in}}%
\pgfpathlineto{\pgfqpoint{2.623122in}{1.286360in}}%
\pgfpathlineto{\pgfqpoint{2.625802in}{1.296931in}}%
\pgfpathlineto{\pgfqpoint{2.629375in}{1.321490in}}%
\pgfpathlineto{\pgfqpoint{2.633842in}{1.368829in}}%
\pgfpathlineto{\pgfqpoint{2.639649in}{1.456722in}}%
\pgfpathlineto{\pgfqpoint{2.646795in}{1.600934in}}%
\pgfpathlineto{\pgfqpoint{2.657515in}{1.869392in}}%
\pgfpathlineto{\pgfqpoint{2.675829in}{2.332276in}}%
\pgfpathlineto{\pgfqpoint{2.682976in}{2.458151in}}%
\pgfpathlineto{\pgfqpoint{2.687889in}{2.512345in}}%
\pgfpathlineto{\pgfqpoint{2.691462in}{2.532304in}}%
\pgfpathlineto{\pgfqpoint{2.693696in}{2.535847in}}%
\pgfpathlineto{\pgfqpoint{2.695036in}{2.534572in}}%
\pgfpathlineto{\pgfqpoint{2.696822in}{2.528849in}}%
\pgfpathlineto{\pgfqpoint{2.699502in}{2.511589in}}%
\pgfpathlineto{\pgfqpoint{2.703076in}{2.472492in}}%
\pgfpathlineto{\pgfqpoint{2.707989in}{2.389915in}}%
\pgfpathlineto{\pgfqpoint{2.714242in}{2.242016in}}%
\pgfpathlineto{\pgfqpoint{2.723622in}{1.957343in}}%
\pgfpathlineto{\pgfqpoint{2.739703in}{1.466776in}}%
\pgfpathlineto{\pgfqpoint{2.745956in}{1.342388in}}%
\pgfpathlineto{\pgfqpoint{2.749976in}{1.296857in}}%
\pgfpathlineto{\pgfqpoint{2.752656in}{1.283673in}}%
\pgfpathlineto{\pgfqpoint{2.753996in}{1.282504in}}%
\pgfpathlineto{\pgfqpoint{2.755336in}{1.285027in}}%
\pgfpathlineto{\pgfqpoint{2.757569in}{1.297508in}}%
\pgfpathlineto{\pgfqpoint{2.760696in}{1.332302in}}%
\pgfpathlineto{\pgfqpoint{2.764716in}{1.405718in}}%
\pgfpathlineto{\pgfqpoint{2.770076in}{1.548545in}}%
\pgfpathlineto{\pgfqpoint{2.778116in}{1.831858in}}%
\pgfpathlineto{\pgfqpoint{2.793303in}{2.373344in}}%
\pgfpathlineto{\pgfqpoint{2.798663in}{2.489269in}}%
\pgfpathlineto{\pgfqpoint{2.802236in}{2.528590in}}%
\pgfpathlineto{\pgfqpoint{2.804470in}{2.535843in}}%
\pgfpathlineto{\pgfqpoint{2.805363in}{2.534865in}}%
\pgfpathlineto{\pgfqpoint{2.807150in}{2.526177in}}%
\pgfpathlineto{\pgfqpoint{2.809830in}{2.496334in}}%
\pgfpathlineto{\pgfqpoint{2.813403in}{2.426112in}}%
\pgfpathlineto{\pgfqpoint{2.818316in}{2.278611in}}%
\pgfpathlineto{\pgfqpoint{2.825910in}{1.969021in}}%
\pgfpathlineto{\pgfqpoint{2.838863in}{1.440154in}}%
\pgfpathlineto{\pgfqpoint{2.843776in}{1.322266in}}%
\pgfpathlineto{\pgfqpoint{2.846903in}{1.287110in}}%
\pgfpathlineto{\pgfqpoint{2.848690in}{1.282504in}}%
\pgfpathlineto{\pgfqpoint{2.849583in}{1.284561in}}%
\pgfpathlineto{\pgfqpoint{2.851370in}{1.297463in}}%
\pgfpathlineto{\pgfqpoint{2.854050in}{1.338584in}}%
\pgfpathlineto{\pgfqpoint{2.857623in}{1.431964in}}%
\pgfpathlineto{\pgfqpoint{2.862983in}{1.641311in}}%
\pgfpathlineto{\pgfqpoint{2.882637in}{2.486947in}}%
\pgfpathlineto{\pgfqpoint{2.885763in}{2.530414in}}%
\pgfpathlineto{\pgfqpoint{2.887550in}{2.535705in}}%
\pgfpathlineto{\pgfqpoint{2.888443in}{2.532830in}}%
\pgfpathlineto{\pgfqpoint{2.890230in}{2.515978in}}%
\pgfpathlineto{\pgfqpoint{2.892910in}{2.463414in}}%
\pgfpathlineto{\pgfqpoint{2.896930in}{2.328048in}}%
\pgfpathlineto{\pgfqpoint{2.902737in}{2.042415in}}%
\pgfpathlineto{\pgfqpoint{2.915243in}{1.405846in}}%
\pgfpathlineto{\pgfqpoint{2.919263in}{1.303569in}}%
\pgfpathlineto{\pgfqpoint{2.921497in}{1.283178in}}%
\pgfpathlineto{\pgfqpoint{2.921943in}{1.282454in}}%
\pgfpathlineto{\pgfqpoint{2.922390in}{1.282867in}}%
\pgfpathlineto{\pgfqpoint{2.923730in}{1.290955in}}%
\pgfpathlineto{\pgfqpoint{2.925964in}{1.327117in}}%
\pgfpathlineto{\pgfqpoint{2.929090in}{1.423075in}}%
\pgfpathlineto{\pgfqpoint{2.933557in}{1.636891in}}%
\pgfpathlineto{\pgfqpoint{2.949637in}{2.488237in}}%
\pgfpathlineto{\pgfqpoint{2.952317in}{2.531838in}}%
\pgfpathlineto{\pgfqpoint{2.953210in}{2.535726in}}%
\pgfpathlineto{\pgfqpoint{2.953657in}{2.535614in}}%
\pgfpathlineto{\pgfqpoint{2.954550in}{2.531252in}}%
\pgfpathlineto{\pgfqpoint{2.956337in}{2.506013in}}%
\pgfpathlineto{\pgfqpoint{2.959017in}{2.428267in}}%
\pgfpathlineto{\pgfqpoint{2.963037in}{2.233757in}}%
\pgfpathlineto{\pgfqpoint{2.972417in}{1.617822in}}%
\pgfpathlineto{\pgfqpoint{2.977777in}{1.355203in}}%
\pgfpathlineto{\pgfqpoint{2.980904in}{1.287872in}}%
\pgfpathlineto{\pgfqpoint{2.982244in}{1.282587in}}%
\pgfpathlineto{\pgfqpoint{2.983137in}{1.287245in}}%
\pgfpathlineto{\pgfqpoint{2.984924in}{1.316177in}}%
\pgfpathlineto{\pgfqpoint{2.987604in}{1.406578in}}%
\pgfpathlineto{\pgfqpoint{2.991624in}{1.631087in}}%
\pgfpathlineto{\pgfqpoint{3.005471in}{2.494751in}}%
\pgfpathlineto{\pgfqpoint{3.008151in}{2.535496in}}%
\pgfpathlineto{\pgfqpoint{3.008597in}{2.535697in}}%
\pgfpathlineto{\pgfqpoint{3.009491in}{2.530321in}}%
\pgfpathlineto{\pgfqpoint{3.011277in}{2.496605in}}%
\pgfpathlineto{\pgfqpoint{3.013957in}{2.391557in}}%
\pgfpathlineto{\pgfqpoint{3.018424in}{2.101160in}}%
\pgfpathlineto{\pgfqpoint{3.029144in}{1.356017in}}%
\pgfpathlineto{\pgfqpoint{3.031824in}{1.288069in}}%
\pgfpathlineto{\pgfqpoint{3.032717in}{1.282514in}}%
\pgfpathlineto{\pgfqpoint{3.033164in}{1.283068in}}%
\pgfpathlineto{\pgfqpoint{3.034504in}{1.298120in}}%
\pgfpathlineto{\pgfqpoint{3.036737in}{1.366815in}}%
\pgfpathlineto{\pgfqpoint{3.039864in}{1.544305in}}%
\pgfpathlineto{\pgfqpoint{3.046564in}{2.095526in}}%
\pgfpathlineto{\pgfqpoint{3.051924in}{2.454094in}}%
\pgfpathlineto{\pgfqpoint{3.054604in}{2.530083in}}%
\pgfpathlineto{\pgfqpoint{3.055498in}{2.535826in}}%
\pgfpathlineto{\pgfqpoint{3.055944in}{2.534871in}}%
\pgfpathlineto{\pgfqpoint{3.057284in}{2.516648in}}%
\pgfpathlineto{\pgfqpoint{3.059518in}{2.436609in}}%
\pgfpathlineto{\pgfqpoint{3.063091in}{2.198589in}}%
\pgfpathlineto{\pgfqpoint{3.074704in}{1.314367in}}%
\pgfpathlineto{\pgfqpoint{3.076938in}{1.282559in}}%
\pgfpathlineto{\pgfqpoint{3.077831in}{1.290036in}}%
\pgfpathlineto{\pgfqpoint{3.079618in}{1.339514in}}%
\pgfpathlineto{\pgfqpoint{3.082298in}{1.492833in}}%
\pgfpathlineto{\pgfqpoint{3.087211in}{1.935202in}}%
\pgfpathlineto{\pgfqpoint{3.093464in}{2.448321in}}%
\pgfpathlineto{\pgfqpoint{3.096144in}{2.532532in}}%
\pgfpathlineto{\pgfqpoint{3.096591in}{2.535554in}}%
\pgfpathlineto{\pgfqpoint{3.097038in}{2.535298in}}%
\pgfpathlineto{\pgfqpoint{3.097931in}{2.524918in}}%
\pgfpathlineto{\pgfqpoint{3.099718in}{2.465448in}}%
\pgfpathlineto{\pgfqpoint{3.102844in}{2.251698in}}%
\pgfpathlineto{\pgfqpoint{3.114458in}{1.294540in}}%
\pgfpathlineto{\pgfqpoint{3.115798in}{1.282783in}}%
\pgfpathlineto{\pgfqpoint{3.116691in}{1.293322in}}%
\pgfpathlineto{\pgfqpoint{3.118478in}{1.357725in}}%
\pgfpathlineto{\pgfqpoint{3.121605in}{1.591507in}}%
\pgfpathlineto{\pgfqpoint{3.132325in}{2.523704in}}%
\pgfpathlineto{\pgfqpoint{3.133218in}{2.535450in}}%
\pgfpathlineto{\pgfqpoint{3.133665in}{2.535202in}}%
\pgfpathlineto{\pgfqpoint{3.134558in}{2.522391in}}%
\pgfpathlineto{\pgfqpoint{3.136345in}{2.448790in}}%
\pgfpathlineto{\pgfqpoint{3.139471in}{2.189045in}}%
\pgfpathlineto{\pgfqpoint{3.148851in}{1.306412in}}%
\pgfpathlineto{\pgfqpoint{3.150191in}{1.282601in}}%
\pgfpathlineto{\pgfqpoint{3.150638in}{1.283678in}}%
\pgfpathlineto{\pgfqpoint{3.151978in}{1.314097in}}%
\pgfpathlineto{\pgfqpoint{3.154211in}{1.450100in}}%
\pgfpathlineto{\pgfqpoint{3.158231in}{1.883337in}}%
\pgfpathlineto{\pgfqpoint{3.164038in}{2.468666in}}%
\pgfpathlineto{\pgfqpoint{3.166271in}{2.535701in}}%
\pgfpathlineto{\pgfqpoint{3.166718in}{2.534402in}}%
\pgfpathlineto{\pgfqpoint{3.168058in}{2.500567in}}%
\pgfpathlineto{\pgfqpoint{3.170291in}{2.351006in}}%
\pgfpathlineto{\pgfqpoint{3.174758in}{1.828506in}}%
\pgfpathlineto{\pgfqpoint{3.179671in}{1.338117in}}%
\pgfpathlineto{\pgfqpoint{3.181905in}{1.283013in}}%
\pgfpathlineto{\pgfqpoint{3.182798in}{1.299014in}}%
\pgfpathlineto{\pgfqpoint{3.184585in}{1.394630in}}%
\pgfpathlineto{\pgfqpoint{3.187712in}{1.725066in}}%
\pgfpathlineto{\pgfqpoint{3.194412in}{2.480753in}}%
\pgfpathlineto{\pgfqpoint{3.196198in}{2.535532in}}%
\pgfpathlineto{\pgfqpoint{3.196645in}{2.534471in}}%
\pgfpathlineto{\pgfqpoint{3.197538in}{2.514345in}}%
\pgfpathlineto{\pgfqpoint{3.199325in}{2.405325in}}%
\pgfpathlineto{\pgfqpoint{3.202452in}{2.044800in}}%
\pgfpathlineto{\pgfqpoint{3.208705in}{1.326712in}}%
\pgfpathlineto{\pgfqpoint{3.210492in}{1.282724in}}%
\pgfpathlineto{\pgfqpoint{3.211385in}{1.299638in}}%
\pgfpathlineto{\pgfqpoint{3.213172in}{1.408792in}}%
\pgfpathlineto{\pgfqpoint{3.216298in}{1.785065in}}%
\pgfpathlineto{\pgfqpoint{3.222105in}{2.485225in}}%
\pgfpathlineto{\pgfqpoint{3.223892in}{2.535718in}}%
\pgfpathlineto{\pgfqpoint{3.224785in}{2.518815in}}%
\pgfpathlineto{\pgfqpoint{3.226572in}{2.403414in}}%
\pgfpathlineto{\pgfqpoint{3.229698in}{2.004801in}}%
\pgfpathlineto{\pgfqpoint{3.235058in}{1.334936in}}%
\pgfpathlineto{\pgfqpoint{3.236845in}{1.282722in}}%
\pgfpathlineto{\pgfqpoint{3.237738in}{1.302199in}}%
\pgfpathlineto{\pgfqpoint{3.239525in}{1.428772in}}%
\pgfpathlineto{\pgfqpoint{3.243098in}{1.924936in}}%
\pgfpathlineto{\pgfqpoint{3.247565in}{2.484264in}}%
\pgfpathlineto{\pgfqpoint{3.249352in}{2.535045in}}%
\pgfpathlineto{\pgfqpoint{3.250245in}{2.511251in}}%
\pgfpathlineto{\pgfqpoint{3.252032in}{2.370254in}}%
\pgfpathlineto{\pgfqpoint{3.256052in}{1.769969in}}%
\pgfpathlineto{\pgfqpoint{3.260072in}{1.307407in}}%
\pgfpathlineto{\pgfqpoint{3.260965in}{1.283079in}}%
\pgfpathlineto{\pgfqpoint{3.261412in}{1.284170in}}%
\pgfpathlineto{\pgfqpoint{3.262305in}{1.312960in}}%
\pgfpathlineto{\pgfqpoint{3.264092in}{1.469611in}}%
\pgfpathlineto{\pgfqpoint{3.268559in}{2.173005in}}%
\pgfpathlineto{\pgfqpoint{3.271685in}{2.511772in}}%
\pgfpathlineto{\pgfqpoint{3.272579in}{2.535538in}}%
\pgfpathlineto{\pgfqpoint{3.273025in}{2.533185in}}%
\pgfpathlineto{\pgfqpoint{3.274365in}{2.469592in}}%
\pgfpathlineto{\pgfqpoint{3.276599in}{2.200659in}}%
\pgfpathlineto{\pgfqpoint{3.283299in}{1.290451in}}%
\pgfpathlineto{\pgfqpoint{3.283745in}{1.282771in}}%
\pgfpathlineto{\pgfqpoint{3.284192in}{1.285299in}}%
\pgfpathlineto{\pgfqpoint{3.285532in}{1.353295in}}%
\pgfpathlineto{\pgfqpoint{3.287765in}{1.638700in}}%
\pgfpathlineto{\pgfqpoint{3.294019in}{2.524688in}}%
\pgfpathlineto{\pgfqpoint{3.294465in}{2.534826in}}%
\pgfpathlineto{\pgfqpoint{3.294912in}{2.534085in}}%
\pgfpathlineto{\pgfqpoint{3.295805in}{2.500033in}}%
\pgfpathlineto{\pgfqpoint{3.297592in}{2.312649in}}%
\pgfpathlineto{\pgfqpoint{3.305185in}{1.282585in}}%
\pgfpathlineto{\pgfqpoint{3.306079in}{1.309404in}}%
\pgfpathlineto{\pgfqpoint{3.307865in}{1.492152in}}%
\pgfpathlineto{\pgfqpoint{3.315459in}{2.534506in}}%
\pgfpathlineto{\pgfqpoint{3.316352in}{2.498458in}}%
\pgfpathlineto{\pgfqpoint{3.318139in}{2.292319in}}%
\pgfpathlineto{\pgfqpoint{3.325286in}{1.283785in}}%
\pgfpathlineto{\pgfqpoint{3.326179in}{1.321676in}}%
\pgfpathlineto{\pgfqpoint{3.327966in}{1.538834in}}%
\pgfpathlineto{\pgfqpoint{3.334666in}{2.535806in}}%
\pgfpathlineto{\pgfqpoint{3.335559in}{2.505968in}}%
\pgfpathlineto{\pgfqpoint{3.337346in}{2.293885in}}%
\pgfpathlineto{\pgfqpoint{3.344046in}{1.283120in}}%
\pgfpathlineto{\pgfqpoint{3.344939in}{1.321255in}}%
\pgfpathlineto{\pgfqpoint{3.346726in}{1.555318in}}%
\pgfpathlineto{\pgfqpoint{3.352979in}{2.535770in}}%
\pgfpathlineto{\pgfqpoint{3.353872in}{2.501755in}}%
\pgfpathlineto{\pgfqpoint{3.355659in}{2.265790in}}%
\pgfpathlineto{\pgfqpoint{3.361466in}{1.285071in}}%
\pgfpathlineto{\pgfqpoint{3.361912in}{1.283990in}}%
\pgfpathlineto{\pgfqpoint{3.362806in}{1.330627in}}%
\pgfpathlineto{\pgfqpoint{3.364592in}{1.595238in}}%
\pgfpathlineto{\pgfqpoint{3.369952in}{2.531368in}}%
\pgfpathlineto{\pgfqpoint{3.370399in}{2.535170in}}%
\pgfpathlineto{\pgfqpoint{3.371292in}{2.491290in}}%
\pgfpathlineto{\pgfqpoint{3.373079in}{2.222145in}}%
\pgfpathlineto{\pgfqpoint{3.378439in}{1.283824in}}%
\pgfpathlineto{\pgfqpoint{3.378886in}{1.285832in}}%
\pgfpathlineto{\pgfqpoint{3.379779in}{1.343645in}}%
\pgfpathlineto{\pgfqpoint{3.382013in}{1.743692in}}%
\pgfpathlineto{\pgfqpoint{3.386033in}{2.511642in}}%
\pgfpathlineto{\pgfqpoint{3.386926in}{2.534302in}}%
\pgfpathlineto{\pgfqpoint{3.387819in}{2.481316in}}%
\pgfpathlineto{\pgfqpoint{3.389606in}{2.180418in}}%
\pgfpathlineto{\pgfqpoint{3.394519in}{1.283797in}}%
\pgfpathlineto{\pgfqpoint{3.394966in}{1.286456in}}%
\pgfpathlineto{\pgfqpoint{3.395859in}{1.350933in}}%
\pgfpathlineto{\pgfqpoint{3.398093in}{1.785877in}}%
\pgfpathlineto{\pgfqpoint{3.402113in}{2.530425in}}%
\pgfpathlineto{\pgfqpoint{3.402559in}{2.535016in}}%
\pgfpathlineto{\pgfqpoint{3.403453in}{2.481754in}}%
\pgfpathlineto{\pgfqpoint{3.405239in}{2.161946in}}%
\pgfpathlineto{\pgfqpoint{3.409706in}{1.287395in}}%
\pgfpathlineto{\pgfqpoint{3.410153in}{1.283633in}}%
\pgfpathlineto{\pgfqpoint{3.411046in}{1.341393in}}%
\pgfpathlineto{\pgfqpoint{3.412833in}{1.677068in}}%
\pgfpathlineto{\pgfqpoint{3.417299in}{2.535168in}}%
\pgfpathlineto{\pgfqpoint{3.417746in}{2.529249in}}%
\pgfpathlineto{\pgfqpoint{3.419086in}{2.379413in}}%
\pgfpathlineto{\pgfqpoint{3.424446in}{1.284878in}}%
\pgfpathlineto{\pgfqpoint{3.425339in}{1.311193in}}%
\pgfpathlineto{\pgfqpoint{3.427126in}{1.620286in}}%
\pgfpathlineto{\pgfqpoint{3.431593in}{2.534650in}}%
\pgfpathlineto{\pgfqpoint{3.432039in}{2.529847in}}%
\pgfpathlineto{\pgfqpoint{3.433379in}{2.371528in}}%
\pgfpathlineto{\pgfqpoint{3.438739in}{1.282807in}}%
\pgfpathlineto{\pgfqpoint{3.439186in}{1.300218in}}%
\pgfpathlineto{\pgfqpoint{3.440526in}{1.497634in}}%
\pgfpathlineto{\pgfqpoint{3.445440in}{2.535827in}}%
\pgfpathlineto{\pgfqpoint{3.445886in}{2.523239in}}%
\pgfpathlineto{\pgfqpoint{3.447226in}{2.332415in}}%
\pgfpathlineto{\pgfqpoint{3.452140in}{1.282502in}}%
\pgfpathlineto{\pgfqpoint{3.452586in}{1.298507in}}%
\pgfpathlineto{\pgfqpoint{3.453926in}{1.504162in}}%
\pgfpathlineto{\pgfqpoint{3.458840in}{2.532691in}}%
\pgfpathlineto{\pgfqpoint{3.459733in}{2.448013in}}%
\pgfpathlineto{\pgfqpoint{3.461966in}{1.872340in}}%
\pgfpathlineto{\pgfqpoint{3.464646in}{1.293689in}}%
\pgfpathlineto{\pgfqpoint{3.465093in}{1.282758in}}%
\pgfpathlineto{\pgfqpoint{3.465986in}{1.351958in}}%
\pgfpathlineto{\pgfqpoint{3.468220in}{1.921281in}}%
\pgfpathlineto{\pgfqpoint{3.470900in}{2.522879in}}%
\pgfpathlineto{\pgfqpoint{3.471346in}{2.535697in}}%
\pgfpathlineto{\pgfqpoint{3.471793in}{2.516692in}}%
\pgfpathlineto{\pgfqpoint{3.473133in}{2.284980in}}%
\pgfpathlineto{\pgfqpoint{3.477600in}{1.284072in}}%
\pgfpathlineto{\pgfqpoint{3.478493in}{1.369393in}}%
\pgfpathlineto{\pgfqpoint{3.480726in}{1.984227in}}%
\pgfpathlineto{\pgfqpoint{3.483406in}{2.535035in}}%
\pgfpathlineto{\pgfqpoint{3.483853in}{2.525324in}}%
\pgfpathlineto{\pgfqpoint{3.485193in}{2.304254in}}%
\pgfpathlineto{\pgfqpoint{3.489660in}{1.286828in}}%
\pgfpathlineto{\pgfqpoint{3.490553in}{1.390666in}}%
\pgfpathlineto{\pgfqpoint{3.495466in}{2.532201in}}%
\pgfpathlineto{\pgfqpoint{3.496360in}{2.428304in}}%
\pgfpathlineto{\pgfqpoint{3.500826in}{1.285208in}}%
\pgfpathlineto{\pgfqpoint{3.501720in}{1.332072in}}%
\pgfpathlineto{\pgfqpoint{3.503506in}{1.798872in}}%
\pgfpathlineto{\pgfqpoint{3.506633in}{2.535843in}}%
\pgfpathlineto{\pgfqpoint{3.507526in}{2.457017in}}%
\pgfpathlineto{\pgfqpoint{3.509760in}{1.784200in}}%
\pgfpathlineto{\pgfqpoint{3.511993in}{1.285386in}}%
\pgfpathlineto{\pgfqpoint{3.512440in}{1.290320in}}%
\pgfpathlineto{\pgfqpoint{3.513780in}{1.533371in}}%
\pgfpathlineto{\pgfqpoint{3.517800in}{2.531900in}}%
\pgfpathlineto{\pgfqpoint{3.518693in}{2.414405in}}%
\pgfpathlineto{\pgfqpoint{3.523160in}{1.286701in}}%
\pgfpathlineto{\pgfqpoint{3.523607in}{1.327409in}}%
\pgfpathlineto{\pgfqpoint{3.525393in}{1.829045in}}%
\pgfpathlineto{\pgfqpoint{3.528073in}{2.532970in}}%
\pgfpathlineto{\pgfqpoint{3.528520in}{2.526535in}}%
\pgfpathlineto{\pgfqpoint{3.529860in}{2.258150in}}%
\pgfpathlineto{\pgfqpoint{3.533433in}{1.282464in}}%
\pgfpathlineto{\pgfqpoint{3.534327in}{1.376267in}}%
\pgfpathlineto{\pgfqpoint{3.538347in}{2.531954in}}%
\pgfpathlineto{\pgfqpoint{3.539240in}{2.475410in}}%
\pgfpathlineto{\pgfqpoint{3.541027in}{1.918483in}}%
\pgfpathlineto{\pgfqpoint{3.543707in}{1.285289in}}%
\pgfpathlineto{\pgfqpoint{3.544600in}{1.413567in}}%
\pgfpathlineto{\pgfqpoint{3.548620in}{2.534251in}}%
\pgfpathlineto{\pgfqpoint{3.549067in}{2.496306in}}%
\pgfpathlineto{\pgfqpoint{3.550853in}{1.951008in}}%
\pgfpathlineto{\pgfqpoint{3.553533in}{1.285227in}}%
\pgfpathlineto{\pgfqpoint{3.554427in}{1.419777in}}%
\pgfpathlineto{\pgfqpoint{3.558000in}{2.530622in}}%
\pgfpathlineto{\pgfqpoint{3.558893in}{2.470938in}}%
\pgfpathlineto{\pgfqpoint{3.560680in}{1.865367in}}%
\pgfpathlineto{\pgfqpoint{3.562913in}{1.282513in}}%
\pgfpathlineto{\pgfqpoint{3.563807in}{1.385730in}}%
\pgfpathlineto{\pgfqpoint{3.567827in}{2.528963in}}%
\pgfpathlineto{\pgfqpoint{3.568273in}{2.473116in}}%
\pgfpathlineto{\pgfqpoint{3.570060in}{1.848987in}}%
\pgfpathlineto{\pgfqpoint{3.572293in}{1.283365in}}%
\pgfpathlineto{\pgfqpoint{3.573187in}{1.417428in}}%
\pgfpathlineto{\pgfqpoint{3.576760in}{2.535846in}}%
\pgfpathlineto{\pgfqpoint{3.577207in}{2.505402in}}%
\pgfpathlineto{\pgfqpoint{3.578547in}{2.107246in}}%
\pgfpathlineto{\pgfqpoint{3.581227in}{1.282466in}}%
\pgfpathlineto{\pgfqpoint{3.582120in}{1.400112in}}%
\pgfpathlineto{\pgfqpoint{3.585694in}{2.535668in}}%
\pgfpathlineto{\pgfqpoint{3.586140in}{2.499202in}}%
\pgfpathlineto{\pgfqpoint{3.587927in}{1.873870in}}%
\pgfpathlineto{\pgfqpoint{3.590160in}{1.285553in}}%
\pgfpathlineto{\pgfqpoint{3.591054in}{1.449904in}}%
\pgfpathlineto{\pgfqpoint{3.594180in}{2.532144in}}%
\pgfpathlineto{\pgfqpoint{3.594627in}{2.520829in}}%
\pgfpathlineto{\pgfqpoint{3.595967in}{2.132490in}}%
\pgfpathlineto{\pgfqpoint{3.598647in}{1.283258in}}%
\pgfpathlineto{\pgfqpoint{3.599540in}{1.436603in}}%
\pgfpathlineto{\pgfqpoint{3.602667in}{2.533799in}}%
\pgfpathlineto{\pgfqpoint{3.603114in}{2.515453in}}%
\pgfpathlineto{\pgfqpoint{3.604454in}{2.095651in}}%
\pgfpathlineto{\pgfqpoint{3.607134in}{1.290959in}}%
\pgfpathlineto{\pgfqpoint{3.608027in}{1.494628in}}%
\pgfpathlineto{\pgfqpoint{3.611154in}{2.531939in}}%
\pgfpathlineto{\pgfqpoint{3.612047in}{2.344350in}}%
\pgfpathlineto{\pgfqpoint{3.615174in}{1.285350in}}%
\pgfpathlineto{\pgfqpoint{3.615620in}{1.343768in}}%
\pgfpathlineto{\pgfqpoint{3.617854in}{2.278870in}}%
\pgfpathlineto{\pgfqpoint{3.619194in}{2.531387in}}%
\pgfpathlineto{\pgfqpoint{3.620087in}{2.333074in}}%
\pgfpathlineto{\pgfqpoint{3.622767in}{1.292195in}}%
\pgfpathlineto{\pgfqpoint{3.623214in}{1.292707in}}%
\pgfpathlineto{\pgfqpoint{3.624554in}{1.716161in}}%
\pgfpathlineto{\pgfqpoint{3.626787in}{2.533851in}}%
\pgfpathlineto{\pgfqpoint{3.627234in}{2.511015in}}%
\pgfpathlineto{\pgfqpoint{3.628574in}{2.032789in}}%
\pgfpathlineto{\pgfqpoint{3.630807in}{1.283452in}}%
\pgfpathlineto{\pgfqpoint{3.631700in}{1.468349in}}%
\pgfpathlineto{\pgfqpoint{3.634380in}{2.529964in}}%
\pgfpathlineto{\pgfqpoint{3.634827in}{2.518827in}}%
\pgfpathlineto{\pgfqpoint{3.636167in}{2.047154in}}%
\pgfpathlineto{\pgfqpoint{3.638400in}{1.284248in}}%
\pgfpathlineto{\pgfqpoint{3.639294in}{1.484792in}}%
\pgfpathlineto{\pgfqpoint{3.641974in}{2.535088in}}%
\pgfpathlineto{\pgfqpoint{3.642421in}{2.502005in}}%
\pgfpathlineto{\pgfqpoint{3.643761in}{1.971390in}}%
\pgfpathlineto{\pgfqpoint{3.645547in}{1.288704in}}%
\pgfpathlineto{\pgfqpoint{3.645994in}{1.300710in}}%
\pgfpathlineto{\pgfqpoint{3.647334in}{1.799950in}}%
\pgfpathlineto{\pgfqpoint{3.649567in}{2.524892in}}%
\pgfpathlineto{\pgfqpoint{3.650907in}{2.044280in}}%
\pgfpathlineto{\pgfqpoint{3.653141in}{1.291072in}}%
\pgfpathlineto{\pgfqpoint{3.654034in}{1.545396in}}%
\pgfpathlineto{\pgfqpoint{3.656714in}{2.525828in}}%
\pgfpathlineto{\pgfqpoint{3.658054in}{2.033621in}}%
\pgfpathlineto{\pgfqpoint{3.659841in}{1.291920in}}%
\pgfpathlineto{\pgfqpoint{3.660287in}{1.298434in}}%
\pgfpathlineto{\pgfqpoint{3.661627in}{1.821930in}}%
\pgfpathlineto{\pgfqpoint{3.663414in}{2.532939in}}%
\pgfpathlineto{\pgfqpoint{3.663861in}{2.506260in}}%
\pgfpathlineto{\pgfqpoint{3.665201in}{1.936937in}}%
\pgfpathlineto{\pgfqpoint{3.666987in}{1.282528in}}%
\pgfpathlineto{\pgfqpoint{3.667881in}{1.488810in}}%
\pgfpathlineto{\pgfqpoint{3.670561in}{2.527351in}}%
\pgfpathlineto{\pgfqpoint{3.671454in}{2.251474in}}%
\pgfpathlineto{\pgfqpoint{3.673687in}{1.283918in}}%
\pgfpathlineto{\pgfqpoint{3.674134in}{1.320245in}}%
\pgfpathlineto{\pgfqpoint{3.675921in}{2.184446in}}%
\pgfpathlineto{\pgfqpoint{3.677261in}{2.529054in}}%
\pgfpathlineto{\pgfqpoint{3.678154in}{2.251719in}}%
\pgfpathlineto{\pgfqpoint{3.680387in}{1.282660in}}%
\pgfpathlineto{\pgfqpoint{3.680834in}{1.331905in}}%
\pgfpathlineto{\pgfqpoint{3.683514in}{2.526630in}}%
\pgfpathlineto{\pgfqpoint{3.684407in}{2.392654in}}%
\pgfpathlineto{\pgfqpoint{3.687087in}{1.289105in}}%
\pgfpathlineto{\pgfqpoint{3.687534in}{1.385272in}}%
\pgfpathlineto{\pgfqpoint{3.690214in}{2.534638in}}%
\pgfpathlineto{\pgfqpoint{3.690661in}{2.458729in}}%
\pgfpathlineto{\pgfqpoint{3.693341in}{1.282461in}}%
\pgfpathlineto{\pgfqpoint{3.693787in}{1.345069in}}%
\pgfpathlineto{\pgfqpoint{3.696467in}{2.535748in}}%
\pgfpathlineto{\pgfqpoint{3.697361in}{2.309359in}}%
\pgfpathlineto{\pgfqpoint{3.699594in}{1.282498in}}%
\pgfpathlineto{\pgfqpoint{3.700041in}{1.341783in}}%
\pgfpathlineto{\pgfqpoint{3.702721in}{2.535705in}}%
\pgfpathlineto{\pgfqpoint{3.703167in}{2.465744in}}%
\pgfpathlineto{\pgfqpoint{3.705848in}{1.284767in}}%
\pgfpathlineto{\pgfqpoint{3.706294in}{1.373784in}}%
\pgfpathlineto{\pgfqpoint{3.708974in}{2.525548in}}%
\pgfpathlineto{\pgfqpoint{3.709421in}{2.409383in}}%
\pgfpathlineto{\pgfqpoint{3.711654in}{1.290540in}}%
\pgfpathlineto{\pgfqpoint{3.712101in}{1.311900in}}%
\pgfpathlineto{\pgfqpoint{3.713888in}{2.272440in}}%
\pgfpathlineto{\pgfqpoint{3.714781in}{2.535827in}}%
\pgfpathlineto{\pgfqpoint{3.715228in}{2.469230in}}%
\pgfpathlineto{\pgfqpoint{3.717908in}{1.292174in}}%
\pgfpathlineto{\pgfqpoint{3.718354in}{1.411827in}}%
\pgfpathlineto{\pgfqpoint{3.720588in}{2.533364in}}%
\pgfpathlineto{\pgfqpoint{3.721034in}{2.488171in}}%
\pgfpathlineto{\pgfqpoint{3.723714in}{1.289909in}}%
\pgfpathlineto{\pgfqpoint{3.724161in}{1.406742in}}%
\pgfpathlineto{\pgfqpoint{3.726394in}{2.534725in}}%
\pgfpathlineto{\pgfqpoint{3.726841in}{2.478704in}}%
\pgfpathlineto{\pgfqpoint{3.729521in}{1.300994in}}%
\pgfpathlineto{\pgfqpoint{3.729968in}{1.445833in}}%
\pgfpathlineto{\pgfqpoint{3.732201in}{2.533760in}}%
\pgfpathlineto{\pgfqpoint{3.732648in}{2.433171in}}%
\pgfpathlineto{\pgfqpoint{3.734881in}{1.283437in}}%
\pgfpathlineto{\pgfqpoint{3.735328in}{1.343930in}}%
\pgfpathlineto{\pgfqpoint{3.737561in}{2.526842in}}%
\pgfpathlineto{\pgfqpoint{3.738454in}{2.325031in}}%
\pgfpathlineto{\pgfqpoint{3.740688in}{1.302371in}}%
\pgfpathlineto{\pgfqpoint{3.742028in}{2.045875in}}%
\pgfpathlineto{\pgfqpoint{3.743368in}{2.524175in}}%
\pgfpathlineto{\pgfqpoint{3.744708in}{1.801290in}}%
\pgfpathlineto{\pgfqpoint{3.746048in}{1.290342in}}%
\pgfpathlineto{\pgfqpoint{3.747388in}{2.003578in}}%
\pgfpathlineto{\pgfqpoint{3.748728in}{2.528814in}}%
\pgfpathlineto{\pgfqpoint{3.749621in}{2.136332in}}%
\pgfpathlineto{\pgfqpoint{3.751408in}{1.291117in}}%
\pgfpathlineto{\pgfqpoint{3.752748in}{2.024137in}}%
\pgfpathlineto{\pgfqpoint{3.754088in}{2.522212in}}%
\pgfpathlineto{\pgfqpoint{3.755428in}{1.760222in}}%
\pgfpathlineto{\pgfqpoint{3.756321in}{1.303413in}}%
\pgfpathlineto{\pgfqpoint{3.756768in}{1.306402in}}%
\pgfpathlineto{\pgfqpoint{3.759001in}{2.526903in}}%
\pgfpathlineto{\pgfqpoint{3.759894in}{2.291802in}}%
\pgfpathlineto{\pgfqpoint{3.761681in}{1.283306in}}%
\pgfpathlineto{\pgfqpoint{3.762128in}{1.357116in}}%
\pgfpathlineto{\pgfqpoint{3.764361in}{2.533265in}}%
\pgfpathlineto{\pgfqpoint{3.764808in}{2.411485in}}%
\pgfpathlineto{\pgfqpoint{3.767041in}{1.303658in}}%
\pgfpathlineto{\pgfqpoint{3.769275in}{2.532702in}}%
\pgfpathlineto{\pgfqpoint{3.770168in}{2.238619in}}%
\pgfpathlineto{\pgfqpoint{3.771955in}{1.286925in}}%
\pgfpathlineto{\pgfqpoint{3.772848in}{1.709307in}}%
\pgfpathlineto{\pgfqpoint{3.774188in}{2.523390in}}%
\pgfpathlineto{\pgfqpoint{3.774635in}{2.494107in}}%
\pgfpathlineto{\pgfqpoint{3.776868in}{1.284009in}}%
\pgfpathlineto{\pgfqpoint{3.777315in}{1.407249in}}%
\pgfpathlineto{\pgfqpoint{3.779101in}{2.522563in}}%
\pgfpathlineto{\pgfqpoint{3.779548in}{2.493663in}}%
\pgfpathlineto{\pgfqpoint{3.781781in}{1.286614in}}%
\pgfpathlineto{\pgfqpoint{3.782228in}{1.427964in}}%
\pgfpathlineto{\pgfqpoint{3.784015in}{2.531454in}}%
\pgfpathlineto{\pgfqpoint{3.784461in}{2.469183in}}%
\pgfpathlineto{\pgfqpoint{3.786695in}{1.302752in}}%
\pgfpathlineto{\pgfqpoint{3.787141in}{1.492692in}}%
\pgfpathlineto{\pgfqpoint{3.788928in}{2.534521in}}%
\pgfpathlineto{\pgfqpoint{3.789375in}{2.405100in}}%
\pgfpathlineto{\pgfqpoint{3.791161in}{1.285899in}}%
\pgfpathlineto{\pgfqpoint{3.791608in}{1.356671in}}%
\pgfpathlineto{\pgfqpoint{3.793395in}{2.515584in}}%
\pgfpathlineto{\pgfqpoint{3.793841in}{2.498752in}}%
\pgfpathlineto{\pgfqpoint{3.796075in}{1.297591in}}%
\pgfpathlineto{\pgfqpoint{3.796521in}{1.484350in}}%
\pgfpathlineto{\pgfqpoint{3.798308in}{2.531682in}}%
\pgfpathlineto{\pgfqpoint{3.798755in}{2.378164in}}%
\pgfpathlineto{\pgfqpoint{3.800541in}{1.282746in}}%
\pgfpathlineto{\pgfqpoint{3.800988in}{1.408354in}}%
\pgfpathlineto{\pgfqpoint{3.802775in}{2.535517in}}%
\pgfpathlineto{\pgfqpoint{3.803221in}{2.431391in}}%
\pgfpathlineto{\pgfqpoint{3.805008in}{1.284204in}}%
\pgfpathlineto{\pgfqpoint{3.805455in}{1.374002in}}%
\pgfpathlineto{\pgfqpoint{3.807241in}{2.532935in}}%
\pgfpathlineto{\pgfqpoint{3.807688in}{2.449828in}}%
\pgfpathlineto{\pgfqpoint{3.809475in}{1.285370in}}%
\pgfpathlineto{\pgfqpoint{3.809921in}{1.369742in}}%
\pgfpathlineto{\pgfqpoint{3.811708in}{2.534059in}}%
\pgfpathlineto{\pgfqpoint{3.812155in}{2.440262in}}%
\pgfpathlineto{\pgfqpoint{3.813941in}{1.282782in}}%
\pgfpathlineto{\pgfqpoint{3.814388in}{1.394235in}}%
\pgfpathlineto{\pgfqpoint{3.816175in}{2.535538in}}%
\pgfpathlineto{\pgfqpoint{3.816621in}{2.398329in}}%
\pgfpathlineto{\pgfqpoint{3.818408in}{1.286768in}}%
\pgfpathlineto{\pgfqpoint{3.818855in}{1.457370in}}%
\pgfpathlineto{\pgfqpoint{3.820641in}{2.519960in}}%
\pgfpathlineto{\pgfqpoint{3.822428in}{1.310488in}}%
\pgfpathlineto{\pgfqpoint{3.823321in}{1.577350in}}%
\pgfpathlineto{\pgfqpoint{3.824661in}{2.528593in}}%
\pgfpathlineto{\pgfqpoint{3.825108in}{2.456614in}}%
\pgfpathlineto{\pgfqpoint{3.826895in}{1.282546in}}%
\pgfpathlineto{\pgfqpoint{3.827341in}{1.423092in}}%
\pgfpathlineto{\pgfqpoint{3.829128in}{2.521949in}}%
\pgfpathlineto{\pgfqpoint{3.829575in}{2.308155in}}%
\pgfpathlineto{\pgfqpoint{3.830915in}{1.301413in}}%
\pgfpathlineto{\pgfqpoint{3.831361in}{1.338606in}}%
\pgfpathlineto{\pgfqpoint{3.833148in}{2.535795in}}%
\pgfpathlineto{\pgfqpoint{3.833595in}{2.402298in}}%
\pgfpathlineto{\pgfqpoint{3.835382in}{1.301007in}}%
\pgfpathlineto{\pgfqpoint{3.837168in}{2.527619in}}%
\pgfpathlineto{\pgfqpoint{3.838062in}{2.129771in}}%
\pgfpathlineto{\pgfqpoint{3.839402in}{1.288259in}}%
\pgfpathlineto{\pgfqpoint{3.839848in}{1.484893in}}%
\pgfpathlineto{\pgfqpoint{3.841188in}{2.518432in}}%
\pgfpathlineto{\pgfqpoint{3.841635in}{2.471176in}}%
\pgfpathlineto{\pgfqpoint{3.843422in}{1.285862in}}%
\pgfpathlineto{\pgfqpoint{3.843868in}{1.474239in}}%
\pgfpathlineto{\pgfqpoint{3.845208in}{2.518749in}}%
\pgfpathlineto{\pgfqpoint{3.845655in}{2.468177in}}%
\pgfpathlineto{\pgfqpoint{3.847442in}{1.289428in}}%
\pgfpathlineto{\pgfqpoint{3.847888in}{1.499001in}}%
\pgfpathlineto{\pgfqpoint{3.849228in}{2.528356in}}%
\pgfpathlineto{\pgfqpoint{3.849675in}{2.441123in}}%
\pgfpathlineto{\pgfqpoint{3.851462in}{1.305497in}}%
\pgfpathlineto{\pgfqpoint{3.851908in}{1.564587in}}%
\pgfpathlineto{\pgfqpoint{3.853248in}{2.535845in}}%
\pgfpathlineto{\pgfqpoint{3.853695in}{2.378975in}}%
\pgfpathlineto{\pgfqpoint{3.855035in}{1.302086in}}%
\pgfpathlineto{\pgfqpoint{3.855482in}{1.351361in}}%
\pgfpathlineto{\pgfqpoint{3.857268in}{2.517990in}}%
\pgfpathlineto{\pgfqpoint{3.857715in}{2.264127in}}%
\pgfpathlineto{\pgfqpoint{3.859055in}{1.282581in}}%
\pgfpathlineto{\pgfqpoint{3.859502in}{1.452443in}}%
\pgfpathlineto{\pgfqpoint{3.860842in}{2.525567in}}%
\pgfpathlineto{\pgfqpoint{3.861288in}{2.441473in}}%
\pgfpathlineto{\pgfqpoint{3.862628in}{1.324349in}}%
\pgfpathlineto{\pgfqpoint{3.863075in}{1.325660in}}%
\pgfpathlineto{\pgfqpoint{3.864862in}{2.522435in}}%
\pgfpathlineto{\pgfqpoint{3.865308in}{2.271846in}}%
\pgfpathlineto{\pgfqpoint{3.866648in}{1.283466in}}%
\pgfpathlineto{\pgfqpoint{3.867095in}{1.475930in}}%
\pgfpathlineto{\pgfqpoint{3.868435in}{2.534005in}}%
\pgfpathlineto{\pgfqpoint{3.868882in}{2.397313in}}%
\pgfpathlineto{\pgfqpoint{3.870222in}{1.294218in}}%
\pgfpathlineto{\pgfqpoint{3.870668in}{1.379653in}}%
\pgfpathlineto{\pgfqpoint{3.872008in}{2.508742in}}%
\pgfpathlineto{\pgfqpoint{3.872455in}{2.468588in}}%
\pgfpathlineto{\pgfqpoint{3.873795in}{1.327071in}}%
\pgfpathlineto{\pgfqpoint{3.874242in}{1.328840in}}%
\pgfpathlineto{\pgfqpoint{3.876028in}{2.503335in}}%
\pgfpathlineto{\pgfqpoint{3.876475in}{2.195864in}}%
\pgfpathlineto{\pgfqpoint{3.877815in}{1.306260in}}%
\pgfpathlineto{\pgfqpoint{3.879602in}{2.516909in}}%
\pgfpathlineto{\pgfqpoint{3.880048in}{2.233408in}}%
\pgfpathlineto{\pgfqpoint{3.881388in}{1.299418in}}%
\pgfpathlineto{\pgfqpoint{3.883175in}{2.518314in}}%
\pgfpathlineto{\pgfqpoint{3.883622in}{2.233862in}}%
\pgfpathlineto{\pgfqpoint{3.884962in}{1.303192in}}%
\pgfpathlineto{\pgfqpoint{3.886748in}{2.508560in}}%
\pgfpathlineto{\pgfqpoint{3.887195in}{2.196868in}}%
\pgfpathlineto{\pgfqpoint{3.888535in}{1.320785in}}%
\pgfpathlineto{\pgfqpoint{3.889875in}{2.490953in}}%
\pgfpathlineto{\pgfqpoint{3.890322in}{2.480242in}}%
\pgfpathlineto{\pgfqpoint{3.891662in}{1.309494in}}%
\pgfpathlineto{\pgfqpoint{3.892108in}{1.363650in}}%
\pgfpathlineto{\pgfqpoint{3.893448in}{2.524403in}}%
\pgfpathlineto{\pgfqpoint{3.893895in}{2.418248in}}%
\pgfpathlineto{\pgfqpoint{3.895235in}{1.283990in}}%
\pgfpathlineto{\pgfqpoint{3.895682in}{1.449818in}}%
\pgfpathlineto{\pgfqpoint{3.897022in}{2.534372in}}%
\pgfpathlineto{\pgfqpoint{3.897468in}{2.302862in}}%
\pgfpathlineto{\pgfqpoint{3.898808in}{1.298286in}}%
\pgfpathlineto{\pgfqpoint{3.900595in}{2.486377in}}%
\pgfpathlineto{\pgfqpoint{3.901042in}{2.117506in}}%
\pgfpathlineto{\pgfqpoint{3.901935in}{1.300791in}}%
\pgfpathlineto{\pgfqpoint{3.902382in}{1.389334in}}%
\pgfpathlineto{\pgfqpoint{3.903722in}{2.535434in}}%
\pgfpathlineto{\pgfqpoint{3.904169in}{2.344183in}}%
\pgfpathlineto{\pgfqpoint{3.905509in}{1.293511in}}%
\pgfpathlineto{\pgfqpoint{3.905955in}{1.587905in}}%
\pgfpathlineto{\pgfqpoint{3.906849in}{2.482196in}}%
\pgfpathlineto{\pgfqpoint{3.907295in}{2.478526in}}%
\pgfpathlineto{\pgfqpoint{3.908635in}{1.290466in}}%
\pgfpathlineto{\pgfqpoint{3.909082in}{1.426703in}}%
\pgfpathlineto{\pgfqpoint{3.910422in}{2.531714in}}%
\pgfpathlineto{\pgfqpoint{3.910869in}{2.262396in}}%
\pgfpathlineto{\pgfqpoint{3.912209in}{1.334030in}}%
\pgfpathlineto{\pgfqpoint{3.913549in}{2.529333in}}%
\pgfpathlineto{\pgfqpoint{3.913995in}{2.379913in}}%
\pgfpathlineto{\pgfqpoint{3.915335in}{1.292810in}}%
\pgfpathlineto{\pgfqpoint{3.915782in}{1.598454in}}%
\pgfpathlineto{\pgfqpoint{3.916675in}{2.498695in}}%
\pgfpathlineto{\pgfqpoint{3.917122in}{2.450676in}}%
\pgfpathlineto{\pgfqpoint{3.918462in}{1.282500in}}%
\pgfpathlineto{\pgfqpoint{3.918909in}{1.515528in}}%
\pgfpathlineto{\pgfqpoint{3.920249in}{2.488303in}}%
\pgfpathlineto{\pgfqpoint{3.921589in}{1.285382in}}%
\pgfpathlineto{\pgfqpoint{3.922035in}{1.467928in}}%
\pgfpathlineto{\pgfqpoint{3.923375in}{2.504962in}}%
\pgfpathlineto{\pgfqpoint{3.924715in}{1.289328in}}%
\pgfpathlineto{\pgfqpoint{3.925162in}{1.449195in}}%
\pgfpathlineto{\pgfqpoint{3.926502in}{2.508286in}}%
\pgfpathlineto{\pgfqpoint{3.927842in}{1.288610in}}%
\pgfpathlineto{\pgfqpoint{3.928289in}{1.456281in}}%
\pgfpathlineto{\pgfqpoint{3.929629in}{2.500002in}}%
\pgfpathlineto{\pgfqpoint{3.930969in}{1.284087in}}%
\pgfpathlineto{\pgfqpoint{3.931415in}{1.490753in}}%
\pgfpathlineto{\pgfqpoint{3.932755in}{2.475523in}}%
\pgfpathlineto{\pgfqpoint{3.934095in}{1.283429in}}%
\pgfpathlineto{\pgfqpoint{3.934542in}{1.558523in}}%
\pgfpathlineto{\pgfqpoint{3.935435in}{2.504246in}}%
\pgfpathlineto{\pgfqpoint{3.935882in}{2.424330in}}%
\pgfpathlineto{\pgfqpoint{3.937222in}{1.301281in}}%
\pgfpathlineto{\pgfqpoint{3.937669in}{1.667831in}}%
\pgfpathlineto{\pgfqpoint{3.938562in}{2.532537in}}%
\pgfpathlineto{\pgfqpoint{3.939009in}{2.331773in}}%
\pgfpathlineto{\pgfqpoint{3.939902in}{1.342156in}}%
\pgfpathlineto{\pgfqpoint{3.940349in}{1.358337in}}%
\pgfpathlineto{\pgfqpoint{3.941689in}{2.526248in}}%
\pgfpathlineto{\pgfqpoint{3.942135in}{2.183503in}}%
\pgfpathlineto{\pgfqpoint{3.943029in}{1.288166in}}%
\pgfpathlineto{\pgfqpoint{3.943475in}{1.477493in}}%
\pgfpathlineto{\pgfqpoint{3.944369in}{2.477691in}}%
\pgfpathlineto{\pgfqpoint{3.944815in}{2.454279in}}%
\pgfpathlineto{\pgfqpoint{3.946155in}{1.299293in}}%
\pgfpathlineto{\pgfqpoint{3.946602in}{1.674885in}}%
\pgfpathlineto{\pgfqpoint{3.947495in}{2.535360in}}%
\pgfpathlineto{\pgfqpoint{3.947942in}{2.288941in}}%
\pgfpathlineto{\pgfqpoint{3.948835in}{1.310357in}}%
\pgfpathlineto{\pgfqpoint{3.949282in}{1.415406in}}%
\pgfpathlineto{\pgfqpoint{3.950622in}{2.481626in}}%
\pgfpathlineto{\pgfqpoint{3.951962in}{1.293296in}}%
\pgfpathlineto{\pgfqpoint{3.952409in}{1.658101in}}%
\pgfpathlineto{\pgfqpoint{3.953302in}{2.535475in}}%
\pgfpathlineto{\pgfqpoint{3.953749in}{2.277852in}}%
\pgfpathlineto{\pgfqpoint{3.954642in}{1.300906in}}%
\pgfpathlineto{\pgfqpoint{3.955089in}{1.445032in}}%
\pgfpathlineto{\pgfqpoint{3.955982in}{2.476081in}}%
\pgfpathlineto{\pgfqpoint{3.956429in}{2.445399in}}%
\pgfpathlineto{\pgfqpoint{3.957769in}{1.323285in}}%
\pgfpathlineto{\pgfqpoint{3.959109in}{2.524071in}}%
\pgfpathlineto{\pgfqpoint{3.959555in}{2.143367in}}%
\pgfpathlineto{\pgfqpoint{3.960449in}{1.282882in}}%
\pgfpathlineto{\pgfqpoint{3.960895in}{1.589237in}}%
\pgfpathlineto{\pgfqpoint{3.961789in}{2.532253in}}%
\pgfpathlineto{\pgfqpoint{3.962235in}{2.301368in}}%
\pgfpathlineto{\pgfqpoint{3.963129in}{1.300331in}}%
\pgfpathlineto{\pgfqpoint{3.963575in}{1.457620in}}%
\pgfpathlineto{\pgfqpoint{3.964469in}{2.495765in}}%
\pgfpathlineto{\pgfqpoint{3.964916in}{2.407931in}}%
\pgfpathlineto{\pgfqpoint{3.965809in}{1.349647in}}%
\pgfpathlineto{\pgfqpoint{3.966256in}{1.373779in}}%
\pgfpathlineto{\pgfqpoint{3.967596in}{2.471975in}}%
\pgfpathlineto{\pgfqpoint{3.968936in}{1.326357in}}%
\pgfpathlineto{\pgfqpoint{3.970276in}{2.505930in}}%
\pgfpathlineto{\pgfqpoint{3.970722in}{2.053664in}}%
\pgfpathlineto{\pgfqpoint{3.971616in}{1.302935in}}%
\pgfpathlineto{\pgfqpoint{3.972062in}{1.734559in}}%
\pgfpathlineto{\pgfqpoint{3.972956in}{2.521390in}}%
\pgfpathlineto{\pgfqpoint{3.973402in}{2.105883in}}%
\pgfpathlineto{\pgfqpoint{3.974296in}{1.293325in}}%
\pgfpathlineto{\pgfqpoint{3.974742in}{1.697974in}}%
\pgfpathlineto{\pgfqpoint{3.975636in}{2.526735in}}%
\pgfpathlineto{\pgfqpoint{3.976082in}{2.127174in}}%
\pgfpathlineto{\pgfqpoint{3.976976in}{1.291223in}}%
\pgfpathlineto{\pgfqpoint{3.977422in}{1.691769in}}%
\pgfpathlineto{\pgfqpoint{3.978316in}{2.526029in}}%
\pgfpathlineto{\pgfqpoint{3.978762in}{2.118316in}}%
\pgfpathlineto{\pgfqpoint{3.979656in}{1.294902in}}%
\pgfpathlineto{\pgfqpoint{3.980102in}{1.715872in}}%
\pgfpathlineto{\pgfqpoint{3.980996in}{2.518604in}}%
\pgfpathlineto{\pgfqpoint{3.981442in}{2.078602in}}%
\pgfpathlineto{\pgfqpoint{3.982336in}{1.307454in}}%
\pgfpathlineto{\pgfqpoint{3.983676in}{2.498958in}}%
\pgfpathlineto{\pgfqpoint{3.985016in}{1.336711in}}%
\pgfpathlineto{\pgfqpoint{3.986356in}{2.457092in}}%
\pgfpathlineto{\pgfqpoint{3.987249in}{1.351769in}}%
\pgfpathlineto{\pgfqpoint{3.987696in}{1.394544in}}%
\pgfpathlineto{\pgfqpoint{3.988589in}{2.494490in}}%
\pgfpathlineto{\pgfqpoint{3.989036in}{2.379761in}}%
\pgfpathlineto{\pgfqpoint{3.989929in}{1.300793in}}%
\pgfpathlineto{\pgfqpoint{3.990376in}{1.494817in}}%
\pgfpathlineto{\pgfqpoint{3.991269in}{2.532294in}}%
\pgfpathlineto{\pgfqpoint{3.991716in}{2.253547in}}%
\pgfpathlineto{\pgfqpoint{3.992609in}{1.283019in}}%
\pgfpathlineto{\pgfqpoint{3.993056in}{1.649061in}}%
\pgfpathlineto{\pgfqpoint{3.993949in}{2.522636in}}%
\pgfpathlineto{\pgfqpoint{3.994396in}{2.070645in}}%
\pgfpathlineto{\pgfqpoint{3.995289in}{1.327613in}}%
\pgfpathlineto{\pgfqpoint{3.996182in}{2.446619in}}%
\pgfpathlineto{\pgfqpoint{3.996629in}{2.436153in}}%
\pgfpathlineto{\pgfqpoint{3.997522in}{1.318591in}}%
\pgfpathlineto{\pgfqpoint{3.997969in}{1.461509in}}%
\pgfpathlineto{\pgfqpoint{3.998862in}{2.530655in}}%
\pgfpathlineto{\pgfqpoint{3.999309in}{2.251830in}}%
\pgfpathlineto{\pgfqpoint{4.000202in}{1.285410in}}%
\pgfpathlineto{\pgfqpoint{4.000649in}{1.695537in}}%
\pgfpathlineto{\pgfqpoint{4.001542in}{2.500546in}}%
\pgfpathlineto{\pgfqpoint{4.002436in}{1.372385in}}%
\pgfpathlineto{\pgfqpoint{4.002882in}{1.388711in}}%
\pgfpathlineto{\pgfqpoint{4.003776in}{2.511530in}}%
\pgfpathlineto{\pgfqpoint{4.004222in}{2.318775in}}%
\pgfpathlineto{\pgfqpoint{4.005116in}{1.282445in}}%
\pgfpathlineto{\pgfqpoint{4.005562in}{1.647347in}}%
\pgfpathlineto{\pgfqpoint{4.006456in}{2.510035in}}%
\pgfpathlineto{\pgfqpoint{4.006902in}{1.993858in}}%
\pgfpathlineto{\pgfqpoint{4.007349in}{1.377378in}}%
\pgfpathlineto{\pgfqpoint{4.007796in}{1.389337in}}%
\pgfpathlineto{\pgfqpoint{4.008689in}{2.517351in}}%
\pgfpathlineto{\pgfqpoint{4.009136in}{2.293060in}}%
\pgfpathlineto{\pgfqpoint{4.010029in}{1.284684in}}%
\pgfpathlineto{\pgfqpoint{4.010476in}{1.708153in}}%
\pgfpathlineto{\pgfqpoint{4.011369in}{2.479606in}}%
\pgfpathlineto{\pgfqpoint{4.012262in}{1.328137in}}%
\pgfpathlineto{\pgfqpoint{4.012709in}{1.465108in}}%
\pgfpathlineto{\pgfqpoint{4.013602in}{2.535743in}}%
\pgfpathlineto{\pgfqpoint{4.014049in}{2.162774in}}%
\pgfpathlineto{\pgfqpoint{4.014942in}{1.323074in}}%
\pgfpathlineto{\pgfqpoint{4.015836in}{2.480449in}}%
\pgfpathlineto{\pgfqpoint{4.016282in}{2.364815in}}%
\pgfpathlineto{\pgfqpoint{4.017176in}{1.282561in}}%
\pgfpathlineto{\pgfqpoint{4.017622in}{1.662209in}}%
\pgfpathlineto{\pgfqpoint{4.018516in}{2.486536in}}%
\pgfpathlineto{\pgfqpoint{4.019409in}{1.322761in}}%
\pgfpathlineto{\pgfqpoint{4.019856in}{1.487389in}}%
\pgfpathlineto{\pgfqpoint{4.020749in}{2.533727in}}%
\pgfpathlineto{\pgfqpoint{4.021196in}{2.088607in}}%
\pgfpathlineto{\pgfqpoint{4.021642in}{1.414635in}}%
\pgfpathlineto{\pgfqpoint{4.021642in}{1.414635in}}%
\pgfusepath{stroke}%
\end{pgfscope}%
\begin{pgfscope}%
\pgfsetrectcap%
\pgfsetmiterjoin%
\pgfsetlinewidth{0.803000pt}%
\definecolor{currentstroke}{rgb}{0.000000,0.000000,0.000000}%
\pgfsetstrokecolor{currentstroke}%
\pgfsetdash{}{0pt}%
\pgfpathmoveto{\pgfqpoint{1.105307in}{1.282443in}}%
\pgfpathlineto{\pgfqpoint{1.105307in}{2.535847in}}%
\pgfusepath{stroke}%
\end{pgfscope}%
\begin{pgfscope}%
\pgfsetrectcap%
\pgfsetmiterjoin%
\pgfsetlinewidth{0.803000pt}%
\definecolor{currentstroke}{rgb}{0.000000,0.000000,0.000000}%
\pgfsetstrokecolor{currentstroke}%
\pgfsetdash{}{0pt}%
\pgfpathmoveto{\pgfqpoint{4.021469in}{1.282443in}}%
\pgfpathlineto{\pgfqpoint{4.021469in}{2.535847in}}%
\pgfusepath{stroke}%
\end{pgfscope}%
\begin{pgfscope}%
\pgfsetrectcap%
\pgfsetmiterjoin%
\pgfsetlinewidth{0.803000pt}%
\definecolor{currentstroke}{rgb}{0.000000,0.000000,0.000000}%
\pgfsetstrokecolor{currentstroke}%
\pgfsetdash{}{0pt}%
\pgfpathmoveto{\pgfqpoint{1.105307in}{1.282443in}}%
\pgfpathlineto{\pgfqpoint{4.021469in}{1.282443in}}%
\pgfusepath{stroke}%
\end{pgfscope}%
\begin{pgfscope}%
\pgfsetrectcap%
\pgfsetmiterjoin%
\pgfsetlinewidth{0.803000pt}%
\definecolor{currentstroke}{rgb}{0.000000,0.000000,0.000000}%
\pgfsetstrokecolor{currentstroke}%
\pgfsetdash{}{0pt}%
\pgfpathmoveto{\pgfqpoint{1.105307in}{2.535847in}}%
\pgfpathlineto{\pgfqpoint{4.021469in}{2.535847in}}%
\pgfusepath{stroke}%
\end{pgfscope}%
\begin{pgfscope}%
\definecolor{textcolor}{rgb}{0.000000,0.000000,0.000000}%
\pgfsetstrokecolor{textcolor}%
\pgfsetfillcolor{textcolor}%
\pgftext[x=1.333611in,y=2.222496in,left,base]{\color{textcolor}{\rmfamily\fontsize{16.000000}{19.200000}\selectfont\catcode`\^=\active\def^{\ifmmode\sp\else\^{}\fi}\catcode`\%=\active\def%{\%}$P_{0+}$}}%
\end{pgfscope}%
\begin{pgfscope}%
\definecolor{textcolor}{rgb}{1.000000,0.000000,0.000000}%
\pgfsetstrokecolor{textcolor}%
\pgfsetfillcolor{textcolor}%
\pgftext[x=1.333611in,y=1.533124in,left,base]{\color{textcolor}{\rmfamily\fontsize{16.000000}{19.200000}\selectfont\catcode`\^=\active\def^{\ifmmode\sp\else\^{}\fi}\catcode`\%=\active\def%{\%}$P_{3-}$}}%
\end{pgfscope}%
\end{pgfpicture}%
\makeatother%
\endgroup%

	}
	\caption{
		Dinámica coherente tipo Rabi para la emisión de paquetes de $n$ fonones
		en una molécula excitónica acoplada a un modo acústico común.}
	\label{fig:rabi_all_regimes}
\end{figure}

\begin{figure}[H]
	\centering
	\resizebox{0.6\textwidth}{!}{%
		%% Creator: Matplotlib, PGF backend
%%
%% To include the figure in your LaTeX document, write
%%   \input{<filename>.pgf}
%%
%% Make sure the required packages are loaded in your preamble
%%   \usepackage{pgf}
%%
%% Also ensure that all the required font packages are loaded; for instance,
%% the lmodern package is sometimes necessary when using math font.
%%   \usepackage{lmodern}
%%
%% Figures using additional raster images can only be included by \input if
%% they are in the same directory as the main LaTeX file. For loading figures
%% from other directories you can use the `import` package
%%   \usepackage{import}
%%
%% and then include the figures with
%%   \import{<path to file>}{<filename>.pgf}
%%
%% Matplotlib used the following preamble
%%   \def\mathdefault#1{#1}
%%   \everymath=\expandafter{\the\everymath\displaystyle}
%%   \IfFileExists{scrextend.sty}{
%%     \usepackage[fontsize=13.000000pt]{scrextend}
%%   }{
%%     \renewcommand{\normalsize}{\fontsize{13.000000}{15.600000}\selectfont}
%%     \normalsize
%%   }
%%   
%%   \makeatletter\@ifpackageloaded{underscore}{}{\usepackage[strings]{underscore}}\makeatother
%%
\begingroup%
\makeatletter%
\begin{pgfpicture}%
\pgfpathrectangle{\pgfpointorigin}{\pgfqpoint{6.810000in}{11.810000in}}%
\pgfusepath{use as bounding box, clip}%
\begin{pgfscope}%
\pgfsetbuttcap%
\pgfsetmiterjoin%
\definecolor{currentfill}{rgb}{1.000000,1.000000,1.000000}%
\pgfsetfillcolor{currentfill}%
\pgfsetlinewidth{0.000000pt}%
\definecolor{currentstroke}{rgb}{1.000000,1.000000,1.000000}%
\pgfsetstrokecolor{currentstroke}%
\pgfsetdash{}{0pt}%
\pgfpathmoveto{\pgfqpoint{0.000000in}{0.000000in}}%
\pgfpathlineto{\pgfqpoint{6.810000in}{0.000000in}}%
\pgfpathlineto{\pgfqpoint{6.810000in}{11.810000in}}%
\pgfpathlineto{\pgfqpoint{0.000000in}{11.810000in}}%
\pgfpathlineto{\pgfqpoint{0.000000in}{0.000000in}}%
\pgfpathclose%
\pgfusepath{fill}%
\end{pgfscope}%
\begin{pgfscope}%
\pgfsetbuttcap%
\pgfsetmiterjoin%
\definecolor{currentfill}{rgb}{1.000000,1.000000,1.000000}%
\pgfsetfillcolor{currentfill}%
\pgfsetlinewidth{0.000000pt}%
\definecolor{currentstroke}{rgb}{0.000000,0.000000,0.000000}%
\pgfsetstrokecolor{currentstroke}%
\pgfsetstrokeopacity{0.000000}%
\pgfsetdash{}{0pt}%
\pgfpathmoveto{\pgfqpoint{0.922838in}{9.205081in}}%
\pgfpathlineto{\pgfqpoint{6.583264in}{9.205081in}}%
\pgfpathlineto{\pgfqpoint{6.583264in}{11.639151in}}%
\pgfpathlineto{\pgfqpoint{0.922838in}{11.639151in}}%
\pgfpathlineto{\pgfqpoint{0.922838in}{9.205081in}}%
\pgfpathclose%
\pgfusepath{fill}%
\end{pgfscope}%
\begin{pgfscope}%
\pgfsetbuttcap%
\pgfsetroundjoin%
\definecolor{currentfill}{rgb}{0.000000,0.000000,0.000000}%
\pgfsetfillcolor{currentfill}%
\pgfsetlinewidth{0.803000pt}%
\definecolor{currentstroke}{rgb}{0.000000,0.000000,0.000000}%
\pgfsetstrokecolor{currentstroke}%
\pgfsetdash{}{0pt}%
\pgfsys@defobject{currentmarker}{\pgfqpoint{0.000000in}{-0.048611in}}{\pgfqpoint{0.000000in}{0.000000in}}{%
\pgfpathmoveto{\pgfqpoint{0.000000in}{0.000000in}}%
\pgfpathlineto{\pgfqpoint{0.000000in}{-0.048611in}}%
\pgfusepath{stroke,fill}%
}%
\begin{pgfscope}%
\pgfsys@transformshift{6.583264in}{9.205081in}%
\pgfsys@useobject{currentmarker}{}%
\end{pgfscope}%
\end{pgfscope}%
\begin{pgfscope}%
\pgfsetbuttcap%
\pgfsetroundjoin%
\definecolor{currentfill}{rgb}{0.000000,0.000000,0.000000}%
\pgfsetfillcolor{currentfill}%
\pgfsetlinewidth{0.803000pt}%
\definecolor{currentstroke}{rgb}{0.000000,0.000000,0.000000}%
\pgfsetstrokecolor{currentstroke}%
\pgfsetdash{}{0pt}%
\pgfsys@defobject{currentmarker}{\pgfqpoint{0.000000in}{-0.048611in}}{\pgfqpoint{0.000000in}{0.000000in}}{%
\pgfpathmoveto{\pgfqpoint{0.000000in}{0.000000in}}%
\pgfpathlineto{\pgfqpoint{0.000000in}{-0.048611in}}%
\pgfusepath{stroke,fill}%
}%
\begin{pgfscope}%
\pgfsys@transformshift{5.639860in}{9.205081in}%
\pgfsys@useobject{currentmarker}{}%
\end{pgfscope}%
\end{pgfscope}%
\begin{pgfscope}%
\pgfsetbuttcap%
\pgfsetroundjoin%
\definecolor{currentfill}{rgb}{0.000000,0.000000,0.000000}%
\pgfsetfillcolor{currentfill}%
\pgfsetlinewidth{0.803000pt}%
\definecolor{currentstroke}{rgb}{0.000000,0.000000,0.000000}%
\pgfsetstrokecolor{currentstroke}%
\pgfsetdash{}{0pt}%
\pgfsys@defobject{currentmarker}{\pgfqpoint{0.000000in}{-0.048611in}}{\pgfqpoint{0.000000in}{0.000000in}}{%
\pgfpathmoveto{\pgfqpoint{0.000000in}{0.000000in}}%
\pgfpathlineto{\pgfqpoint{0.000000in}{-0.048611in}}%
\pgfusepath{stroke,fill}%
}%
\begin{pgfscope}%
\pgfsys@transformshift{4.696455in}{9.205081in}%
\pgfsys@useobject{currentmarker}{}%
\end{pgfscope}%
\end{pgfscope}%
\begin{pgfscope}%
\pgfsetbuttcap%
\pgfsetroundjoin%
\definecolor{currentfill}{rgb}{0.000000,0.000000,0.000000}%
\pgfsetfillcolor{currentfill}%
\pgfsetlinewidth{0.803000pt}%
\definecolor{currentstroke}{rgb}{0.000000,0.000000,0.000000}%
\pgfsetstrokecolor{currentstroke}%
\pgfsetdash{}{0pt}%
\pgfsys@defobject{currentmarker}{\pgfqpoint{0.000000in}{-0.048611in}}{\pgfqpoint{0.000000in}{0.000000in}}{%
\pgfpathmoveto{\pgfqpoint{0.000000in}{0.000000in}}%
\pgfpathlineto{\pgfqpoint{0.000000in}{-0.048611in}}%
\pgfusepath{stroke,fill}%
}%
\begin{pgfscope}%
\pgfsys@transformshift{3.753051in}{9.205081in}%
\pgfsys@useobject{currentmarker}{}%
\end{pgfscope}%
\end{pgfscope}%
\begin{pgfscope}%
\pgfsetbuttcap%
\pgfsetroundjoin%
\definecolor{currentfill}{rgb}{0.000000,0.000000,0.000000}%
\pgfsetfillcolor{currentfill}%
\pgfsetlinewidth{0.803000pt}%
\definecolor{currentstroke}{rgb}{0.000000,0.000000,0.000000}%
\pgfsetstrokecolor{currentstroke}%
\pgfsetdash{}{0pt}%
\pgfsys@defobject{currentmarker}{\pgfqpoint{0.000000in}{-0.048611in}}{\pgfqpoint{0.000000in}{0.000000in}}{%
\pgfpathmoveto{\pgfqpoint{0.000000in}{0.000000in}}%
\pgfpathlineto{\pgfqpoint{0.000000in}{-0.048611in}}%
\pgfusepath{stroke,fill}%
}%
\begin{pgfscope}%
\pgfsys@transformshift{2.809647in}{9.205081in}%
\pgfsys@useobject{currentmarker}{}%
\end{pgfscope}%
\end{pgfscope}%
\begin{pgfscope}%
\pgfsetbuttcap%
\pgfsetroundjoin%
\definecolor{currentfill}{rgb}{0.000000,0.000000,0.000000}%
\pgfsetfillcolor{currentfill}%
\pgfsetlinewidth{0.803000pt}%
\definecolor{currentstroke}{rgb}{0.000000,0.000000,0.000000}%
\pgfsetstrokecolor{currentstroke}%
\pgfsetdash{}{0pt}%
\pgfsys@defobject{currentmarker}{\pgfqpoint{0.000000in}{-0.048611in}}{\pgfqpoint{0.000000in}{0.000000in}}{%
\pgfpathmoveto{\pgfqpoint{0.000000in}{0.000000in}}%
\pgfpathlineto{\pgfqpoint{0.000000in}{-0.048611in}}%
\pgfusepath{stroke,fill}%
}%
\begin{pgfscope}%
\pgfsys@transformshift{1.866242in}{9.205081in}%
\pgfsys@useobject{currentmarker}{}%
\end{pgfscope}%
\end{pgfscope}%
\begin{pgfscope}%
\pgfsetbuttcap%
\pgfsetroundjoin%
\definecolor{currentfill}{rgb}{0.000000,0.000000,0.000000}%
\pgfsetfillcolor{currentfill}%
\pgfsetlinewidth{0.803000pt}%
\definecolor{currentstroke}{rgb}{0.000000,0.000000,0.000000}%
\pgfsetstrokecolor{currentstroke}%
\pgfsetdash{}{0pt}%
\pgfsys@defobject{currentmarker}{\pgfqpoint{0.000000in}{-0.048611in}}{\pgfqpoint{0.000000in}{0.000000in}}{%
\pgfpathmoveto{\pgfqpoint{0.000000in}{0.000000in}}%
\pgfpathlineto{\pgfqpoint{0.000000in}{-0.048611in}}%
\pgfusepath{stroke,fill}%
}%
\begin{pgfscope}%
\pgfsys@transformshift{0.922838in}{9.205081in}%
\pgfsys@useobject{currentmarker}{}%
\end{pgfscope}%
\end{pgfscope}%
\begin{pgfscope}%
\pgfsetbuttcap%
\pgfsetroundjoin%
\definecolor{currentfill}{rgb}{0.000000,0.000000,0.000000}%
\pgfsetfillcolor{currentfill}%
\pgfsetlinewidth{0.803000pt}%
\definecolor{currentstroke}{rgb}{0.000000,0.000000,0.000000}%
\pgfsetstrokecolor{currentstroke}%
\pgfsetdash{}{0pt}%
\pgfsys@defobject{currentmarker}{\pgfqpoint{-0.048611in}{0.000000in}}{\pgfqpoint{-0.000000in}{0.000000in}}{%
\pgfpathmoveto{\pgfqpoint{-0.000000in}{0.000000in}}%
\pgfpathlineto{\pgfqpoint{-0.048611in}{0.000000in}}%
\pgfusepath{stroke,fill}%
}%
\begin{pgfscope}%
\pgfsys@transformshift{0.922838in}{9.205081in}%
\pgfsys@useobject{currentmarker}{}%
\end{pgfscope}%
\end{pgfscope}%
\begin{pgfscope}%
\definecolor{textcolor}{rgb}{0.000000,0.000000,0.000000}%
\pgfsetstrokecolor{textcolor}%
\pgfsetfillcolor{textcolor}%
\pgftext[x=0.428511in, y=9.134232in, left, base]{\color{textcolor}{\rmfamily\fontsize{15.000000}{18.000000}\selectfont\catcode`\^=\active\def^{\ifmmode\sp\else\^{}\fi}\catcode`\%=\active\def%{\%}$\mathdefault{10^{-1}}$}}%
\end{pgfscope}%
\begin{pgfscope}%
\pgfsetbuttcap%
\pgfsetroundjoin%
\definecolor{currentfill}{rgb}{0.000000,0.000000,0.000000}%
\pgfsetfillcolor{currentfill}%
\pgfsetlinewidth{0.803000pt}%
\definecolor{currentstroke}{rgb}{0.000000,0.000000,0.000000}%
\pgfsetstrokecolor{currentstroke}%
\pgfsetdash{}{0pt}%
\pgfsys@defobject{currentmarker}{\pgfqpoint{-0.048611in}{0.000000in}}{\pgfqpoint{-0.000000in}{0.000000in}}{%
\pgfpathmoveto{\pgfqpoint{-0.000000in}{0.000000in}}%
\pgfpathlineto{\pgfqpoint{-0.048611in}{0.000000in}}%
\pgfusepath{stroke,fill}%
}%
\begin{pgfscope}%
\pgfsys@transformshift{0.922838in}{9.813599in}%
\pgfsys@useobject{currentmarker}{}%
\end{pgfscope}%
\end{pgfscope}%
\begin{pgfscope}%
\definecolor{textcolor}{rgb}{0.000000,0.000000,0.000000}%
\pgfsetstrokecolor{textcolor}%
\pgfsetfillcolor{textcolor}%
\pgftext[x=0.546798in, y=9.742749in, left, base]{\color{textcolor}{\rmfamily\fontsize{15.000000}{18.000000}\selectfont\catcode`\^=\active\def^{\ifmmode\sp\else\^{}\fi}\catcode`\%=\active\def%{\%}$\mathdefault{10^{1}}$}}%
\end{pgfscope}%
\begin{pgfscope}%
\pgfsetbuttcap%
\pgfsetroundjoin%
\definecolor{currentfill}{rgb}{0.000000,0.000000,0.000000}%
\pgfsetfillcolor{currentfill}%
\pgfsetlinewidth{0.803000pt}%
\definecolor{currentstroke}{rgb}{0.000000,0.000000,0.000000}%
\pgfsetstrokecolor{currentstroke}%
\pgfsetdash{}{0pt}%
\pgfsys@defobject{currentmarker}{\pgfqpoint{-0.048611in}{0.000000in}}{\pgfqpoint{-0.000000in}{0.000000in}}{%
\pgfpathmoveto{\pgfqpoint{-0.000000in}{0.000000in}}%
\pgfpathlineto{\pgfqpoint{-0.048611in}{0.000000in}}%
\pgfusepath{stroke,fill}%
}%
\begin{pgfscope}%
\pgfsys@transformshift{0.922838in}{10.422116in}%
\pgfsys@useobject{currentmarker}{}%
\end{pgfscope}%
\end{pgfscope}%
\begin{pgfscope}%
\definecolor{textcolor}{rgb}{0.000000,0.000000,0.000000}%
\pgfsetstrokecolor{textcolor}%
\pgfsetfillcolor{textcolor}%
\pgftext[x=0.546798in, y=10.351267in, left, base]{\color{textcolor}{\rmfamily\fontsize{15.000000}{18.000000}\selectfont\catcode`\^=\active\def^{\ifmmode\sp\else\^{}\fi}\catcode`\%=\active\def%{\%}$\mathdefault{10^{3}}$}}%
\end{pgfscope}%
\begin{pgfscope}%
\pgfsetbuttcap%
\pgfsetroundjoin%
\definecolor{currentfill}{rgb}{0.000000,0.000000,0.000000}%
\pgfsetfillcolor{currentfill}%
\pgfsetlinewidth{0.803000pt}%
\definecolor{currentstroke}{rgb}{0.000000,0.000000,0.000000}%
\pgfsetstrokecolor{currentstroke}%
\pgfsetdash{}{0pt}%
\pgfsys@defobject{currentmarker}{\pgfqpoint{-0.048611in}{0.000000in}}{\pgfqpoint{-0.000000in}{0.000000in}}{%
\pgfpathmoveto{\pgfqpoint{-0.000000in}{0.000000in}}%
\pgfpathlineto{\pgfqpoint{-0.048611in}{0.000000in}}%
\pgfusepath{stroke,fill}%
}%
\begin{pgfscope}%
\pgfsys@transformshift{0.922838in}{11.030633in}%
\pgfsys@useobject{currentmarker}{}%
\end{pgfscope}%
\end{pgfscope}%
\begin{pgfscope}%
\definecolor{textcolor}{rgb}{0.000000,0.000000,0.000000}%
\pgfsetstrokecolor{textcolor}%
\pgfsetfillcolor{textcolor}%
\pgftext[x=0.546798in, y=10.959784in, left, base]{\color{textcolor}{\rmfamily\fontsize{15.000000}{18.000000}\selectfont\catcode`\^=\active\def^{\ifmmode\sp\else\^{}\fi}\catcode`\%=\active\def%{\%}$\mathdefault{10^{5}}$}}%
\end{pgfscope}%
\begin{pgfscope}%
\pgfsetbuttcap%
\pgfsetroundjoin%
\definecolor{currentfill}{rgb}{0.000000,0.000000,0.000000}%
\pgfsetfillcolor{currentfill}%
\pgfsetlinewidth{0.803000pt}%
\definecolor{currentstroke}{rgb}{0.000000,0.000000,0.000000}%
\pgfsetstrokecolor{currentstroke}%
\pgfsetdash{}{0pt}%
\pgfsys@defobject{currentmarker}{\pgfqpoint{-0.048611in}{0.000000in}}{\pgfqpoint{-0.000000in}{0.000000in}}{%
\pgfpathmoveto{\pgfqpoint{-0.000000in}{0.000000in}}%
\pgfpathlineto{\pgfqpoint{-0.048611in}{0.000000in}}%
\pgfusepath{stroke,fill}%
}%
\begin{pgfscope}%
\pgfsys@transformshift{0.922838in}{11.639151in}%
\pgfsys@useobject{currentmarker}{}%
\end{pgfscope}%
\end{pgfscope}%
\begin{pgfscope}%
\definecolor{textcolor}{rgb}{0.000000,0.000000,0.000000}%
\pgfsetstrokecolor{textcolor}%
\pgfsetfillcolor{textcolor}%
\pgftext[x=0.546798in, y=11.568301in, left, base]{\color{textcolor}{\rmfamily\fontsize{15.000000}{18.000000}\selectfont\catcode`\^=\active\def^{\ifmmode\sp\else\^{}\fi}\catcode`\%=\active\def%{\%}$\mathdefault{10^{7}}$}}%
\end{pgfscope}%
\begin{pgfscope}%
\definecolor{textcolor}{rgb}{0.000000,0.000000,0.000000}%
\pgfsetstrokecolor{textcolor}%
\pgfsetfillcolor{textcolor}%
\pgftext[x=0.372956in,y=10.422116in,,bottom,rotate=90.000000]{\color{textcolor}{\rmfamily\fontsize{16.000000}{19.200000}\selectfont\catcode`\^=\active\def^{\ifmmode\sp\else\^{}\fi}\catcode`\%=\active\def%{\%}$g^{(2)}(0)$}}%
\end{pgfscope}%
\begin{pgfscope}%
\pgfpathrectangle{\pgfqpoint{0.922838in}{9.205081in}}{\pgfqpoint{5.660427in}{2.434069in}}%
\pgfusepath{clip}%
\pgfsetrectcap%
\pgfsetroundjoin%
\pgfsetlinewidth{1.606000pt}%
\definecolor{currentstroke}{rgb}{0.000000,0.000000,1.000000}%
\pgfsetstrokecolor{currentstroke}%
\pgfsetdash{}{0pt}%
\pgfpathmoveto{\pgfqpoint{0.922838in}{10.332284in}}%
\pgfpathlineto{\pgfqpoint{0.934159in}{10.379753in}}%
\pgfpathlineto{\pgfqpoint{0.945479in}{10.374420in}}%
\pgfpathlineto{\pgfqpoint{0.979442in}{10.338983in}}%
\pgfpathlineto{\pgfqpoint{1.013405in}{10.308802in}}%
\pgfpathlineto{\pgfqpoint{1.070009in}{10.259521in}}%
\pgfpathlineto{\pgfqpoint{1.103971in}{10.227449in}}%
\pgfpathlineto{\pgfqpoint{1.137934in}{10.192223in}}%
\pgfpathlineto{\pgfqpoint{1.171897in}{10.152812in}}%
\pgfpathlineto{\pgfqpoint{1.194538in}{10.123588in}}%
\pgfpathlineto{\pgfqpoint{1.217180in}{10.091439in}}%
\pgfpathlineto{\pgfqpoint{1.239822in}{10.055756in}}%
\pgfpathlineto{\pgfqpoint{1.262463in}{10.015826in}}%
\pgfpathlineto{\pgfqpoint{1.285105in}{9.970977in}}%
\pgfpathlineto{\pgfqpoint{1.330388in}{9.871794in}}%
\pgfpathlineto{\pgfqpoint{1.341709in}{9.853940in}}%
\pgfpathlineto{\pgfqpoint{1.353030in}{9.851573in}}%
\pgfpathlineto{\pgfqpoint{1.364351in}{9.822734in}}%
\pgfpathlineto{\pgfqpoint{1.375672in}{9.762500in}}%
\pgfpathlineto{\pgfqpoint{1.386993in}{9.652109in}}%
\pgfpathlineto{\pgfqpoint{1.398314in}{9.755317in}}%
\pgfpathlineto{\pgfqpoint{1.409634in}{9.737974in}}%
\pgfpathlineto{\pgfqpoint{1.420955in}{9.754782in}}%
\pgfpathlineto{\pgfqpoint{1.443597in}{9.818680in}}%
\pgfpathlineto{\pgfqpoint{1.466239in}{9.881411in}}%
\pgfpathlineto{\pgfqpoint{1.488880in}{9.935134in}}%
\pgfpathlineto{\pgfqpoint{1.511522in}{9.980669in}}%
\pgfpathlineto{\pgfqpoint{1.534164in}{10.019659in}}%
\pgfpathlineto{\pgfqpoint{1.556806in}{10.053474in}}%
\pgfpathlineto{\pgfqpoint{1.579447in}{10.083146in}}%
\pgfpathlineto{\pgfqpoint{1.602089in}{10.109439in}}%
\pgfpathlineto{\pgfqpoint{1.624731in}{10.132919in}}%
\pgfpathlineto{\pgfqpoint{1.647372in}{10.154012in}}%
\pgfpathlineto{\pgfqpoint{1.681335in}{10.181865in}}%
\pgfpathlineto{\pgfqpoint{1.715297in}{10.205854in}}%
\pgfpathlineto{\pgfqpoint{1.749260in}{10.226521in}}%
\pgfpathlineto{\pgfqpoint{1.783223in}{10.244266in}}%
\pgfpathlineto{\pgfqpoint{1.828506in}{10.265145in}}%
\pgfpathlineto{\pgfqpoint{1.839827in}{10.273758in}}%
\pgfpathlineto{\pgfqpoint{1.851148in}{10.360936in}}%
\pgfpathlineto{\pgfqpoint{1.862469in}{10.294089in}}%
\pgfpathlineto{\pgfqpoint{1.873789in}{10.282165in}}%
\pgfpathlineto{\pgfqpoint{1.885110in}{10.282634in}}%
\pgfpathlineto{\pgfqpoint{1.941715in}{10.290705in}}%
\pgfpathlineto{\pgfqpoint{1.975677in}{10.292343in}}%
\pgfpathlineto{\pgfqpoint{1.998319in}{10.291610in}}%
\pgfpathlineto{\pgfqpoint{2.020961in}{10.289243in}}%
\pgfpathlineto{\pgfqpoint{2.043602in}{10.285063in}}%
\pgfpathlineto{\pgfqpoint{2.066244in}{10.278846in}}%
\pgfpathlineto{\pgfqpoint{2.088886in}{10.270309in}}%
\pgfpathlineto{\pgfqpoint{2.111527in}{10.259083in}}%
\pgfpathlineto{\pgfqpoint{2.134169in}{10.244676in}}%
\pgfpathlineto{\pgfqpoint{2.156811in}{10.226413in}}%
\pgfpathlineto{\pgfqpoint{2.179452in}{10.203342in}}%
\pgfpathlineto{\pgfqpoint{2.190773in}{10.189586in}}%
\pgfpathlineto{\pgfqpoint{2.202094in}{10.174045in}}%
\pgfpathlineto{\pgfqpoint{2.213415in}{10.156413in}}%
\pgfpathlineto{\pgfqpoint{2.224736in}{10.136292in}}%
\pgfpathlineto{\pgfqpoint{2.236057in}{10.113155in}}%
\pgfpathlineto{\pgfqpoint{2.247378in}{10.086276in}}%
\pgfpathlineto{\pgfqpoint{2.258698in}{10.054629in}}%
\pgfpathlineto{\pgfqpoint{2.270019in}{10.016686in}}%
\pgfpathlineto{\pgfqpoint{2.281340in}{9.970043in}}%
\pgfpathlineto{\pgfqpoint{2.292661in}{9.910636in}}%
\pgfpathlineto{\pgfqpoint{2.303982in}{9.830988in}}%
\pgfpathlineto{\pgfqpoint{2.315303in}{9.719227in}}%
\pgfpathlineto{\pgfqpoint{2.326624in}{9.524286in}}%
\pgfpathlineto{\pgfqpoint{2.337944in}{9.546037in}}%
\pgfpathlineto{\pgfqpoint{2.349265in}{9.743267in}}%
\pgfpathlineto{\pgfqpoint{2.360586in}{9.867512in}}%
\pgfpathlineto{\pgfqpoint{2.371907in}{9.956298in}}%
\pgfpathlineto{\pgfqpoint{2.383228in}{10.025420in}}%
\pgfpathlineto{\pgfqpoint{2.394549in}{10.082215in}}%
\pgfpathlineto{\pgfqpoint{2.405870in}{10.130594in}}%
\pgfpathlineto{\pgfqpoint{2.417190in}{10.172867in}}%
\pgfpathlineto{\pgfqpoint{2.439832in}{10.244519in}}%
\pgfpathlineto{\pgfqpoint{2.462474in}{10.304272in}}%
\pgfpathlineto{\pgfqpoint{2.485115in}{10.355855in}}%
\pgfpathlineto{\pgfqpoint{2.507757in}{10.401475in}}%
\pgfpathlineto{\pgfqpoint{2.530399in}{10.442545in}}%
\pgfpathlineto{\pgfqpoint{2.553041in}{10.480032in}}%
\pgfpathlineto{\pgfqpoint{2.587003in}{10.531008in}}%
\pgfpathlineto{\pgfqpoint{2.620966in}{10.577084in}}%
\pgfpathlineto{\pgfqpoint{2.654928in}{10.619391in}}%
\pgfpathlineto{\pgfqpoint{2.700212in}{10.671325in}}%
\pgfpathlineto{\pgfqpoint{2.745495in}{10.719425in}}%
\pgfpathlineto{\pgfqpoint{2.802099in}{10.775755in}}%
\pgfpathlineto{\pgfqpoint{2.870025in}{10.839735in}}%
\pgfpathlineto{\pgfqpoint{2.960591in}{10.921707in}}%
\pgfpathlineto{\pgfqpoint{3.028516in}{10.980739in}}%
\pgfpathlineto{\pgfqpoint{3.062479in}{11.008157in}}%
\pgfpathlineto{\pgfqpoint{3.085121in}{11.024674in}}%
\pgfpathlineto{\pgfqpoint{3.107762in}{11.038743in}}%
\pgfpathlineto{\pgfqpoint{3.130404in}{11.048856in}}%
\pgfpathlineto{\pgfqpoint{3.141725in}{11.051717in}}%
\pgfpathlineto{\pgfqpoint{3.153046in}{11.052571in}}%
\pgfpathlineto{\pgfqpoint{3.164367in}{11.050862in}}%
\pgfpathlineto{\pgfqpoint{3.175688in}{11.045871in}}%
\pgfpathlineto{\pgfqpoint{3.187008in}{11.036649in}}%
\pgfpathlineto{\pgfqpoint{3.198329in}{11.021919in}}%
\pgfpathlineto{\pgfqpoint{3.209650in}{10.999910in}}%
\pgfpathlineto{\pgfqpoint{3.220971in}{10.968021in}}%
\pgfpathlineto{\pgfqpoint{3.232292in}{10.922111in}}%
\pgfpathlineto{\pgfqpoint{3.243613in}{10.854673in}}%
\pgfpathlineto{\pgfqpoint{3.254934in}{10.749007in}}%
\pgfpathlineto{\pgfqpoint{3.266254in}{10.551832in}}%
\pgfpathlineto{\pgfqpoint{3.277575in}{10.327128in}}%
\pgfpathlineto{\pgfqpoint{3.288896in}{10.657730in}}%
\pgfpathlineto{\pgfqpoint{3.300217in}{10.807342in}}%
\pgfpathlineto{\pgfqpoint{3.311538in}{10.897032in}}%
\pgfpathlineto{\pgfqpoint{3.322859in}{10.957568in}}%
\pgfpathlineto{\pgfqpoint{3.334179in}{11.000503in}}%
\pgfpathlineto{\pgfqpoint{3.345500in}{11.031572in}}%
\pgfpathlineto{\pgfqpoint{3.356821in}{11.054134in}}%
\pgfpathlineto{\pgfqpoint{3.368142in}{11.070377in}}%
\pgfpathlineto{\pgfqpoint{3.379463in}{11.081831in}}%
\pgfpathlineto{\pgfqpoint{3.390784in}{11.089607in}}%
\pgfpathlineto{\pgfqpoint{3.402105in}{11.094540in}}%
\pgfpathlineto{\pgfqpoint{3.413425in}{11.097270in}}%
\pgfpathlineto{\pgfqpoint{3.424746in}{11.098289in}}%
\pgfpathlineto{\pgfqpoint{3.447388in}{11.096651in}}%
\pgfpathlineto{\pgfqpoint{3.470030in}{11.091819in}}%
\pgfpathlineto{\pgfqpoint{3.503992in}{11.081437in}}%
\pgfpathlineto{\pgfqpoint{3.639843in}{11.035675in}}%
\pgfpathlineto{\pgfqpoint{3.685126in}{11.023567in}}%
\pgfpathlineto{\pgfqpoint{3.730409in}{11.013714in}}%
\pgfpathlineto{\pgfqpoint{3.775693in}{11.006215in}}%
\pgfpathlineto{\pgfqpoint{3.820976in}{11.001181in}}%
\pgfpathlineto{\pgfqpoint{3.866260in}{10.998826in}}%
\pgfpathlineto{\pgfqpoint{3.900222in}{10.999071in}}%
\pgfpathlineto{\pgfqpoint{3.934185in}{11.001395in}}%
\pgfpathlineto{\pgfqpoint{3.968147in}{11.006351in}}%
\pgfpathlineto{\pgfqpoint{3.990789in}{11.011550in}}%
\pgfpathlineto{\pgfqpoint{4.013431in}{11.018730in}}%
\pgfpathlineto{\pgfqpoint{4.036072in}{11.028494in}}%
\pgfpathlineto{\pgfqpoint{4.058714in}{11.041717in}}%
\pgfpathlineto{\pgfqpoint{4.070035in}{11.050011in}}%
\pgfpathlineto{\pgfqpoint{4.081356in}{11.059705in}}%
\pgfpathlineto{\pgfqpoint{4.092677in}{11.071068in}}%
\pgfpathlineto{\pgfqpoint{4.103997in}{11.084434in}}%
\pgfpathlineto{\pgfqpoint{4.115318in}{11.100218in}}%
\pgfpathlineto{\pgfqpoint{4.126639in}{11.118940in}}%
\pgfpathlineto{\pgfqpoint{4.137960in}{11.141239in}}%
\pgfpathlineto{\pgfqpoint{4.149281in}{11.167897in}}%
\pgfpathlineto{\pgfqpoint{4.160602in}{11.199794in}}%
\pgfpathlineto{\pgfqpoint{4.171923in}{11.237678in}}%
\pgfpathlineto{\pgfqpoint{4.194564in}{11.323224in}}%
\pgfpathlineto{\pgfqpoint{4.205885in}{11.326074in}}%
\pgfpathlineto{\pgfqpoint{4.217206in}{11.158801in}}%
\pgfpathlineto{\pgfqpoint{4.228527in}{11.307863in}}%
\pgfpathlineto{\pgfqpoint{4.239848in}{11.331532in}}%
\pgfpathlineto{\pgfqpoint{4.251169in}{11.290688in}}%
\pgfpathlineto{\pgfqpoint{4.262489in}{11.244152in}}%
\pgfpathlineto{\pgfqpoint{4.273810in}{11.202782in}}%
\pgfpathlineto{\pgfqpoint{4.285131in}{11.167849in}}%
\pgfpathlineto{\pgfqpoint{4.296452in}{11.138710in}}%
\pgfpathlineto{\pgfqpoint{4.307773in}{11.114438in}}%
\pgfpathlineto{\pgfqpoint{4.319094in}{11.094180in}}%
\pgfpathlineto{\pgfqpoint{4.330415in}{11.077216in}}%
\pgfpathlineto{\pgfqpoint{4.341735in}{11.062962in}}%
\pgfpathlineto{\pgfqpoint{4.353056in}{11.050942in}}%
\pgfpathlineto{\pgfqpoint{4.364377in}{11.040770in}}%
\pgfpathlineto{\pgfqpoint{4.387019in}{11.024782in}}%
\pgfpathlineto{\pgfqpoint{4.409661in}{11.013129in}}%
\pgfpathlineto{\pgfqpoint{4.432302in}{11.004560in}}%
\pgfpathlineto{\pgfqpoint{4.454944in}{10.998224in}}%
\pgfpathlineto{\pgfqpoint{4.488906in}{10.991670in}}%
\pgfpathlineto{\pgfqpoint{4.522869in}{10.987555in}}%
\pgfpathlineto{\pgfqpoint{4.568152in}{10.984534in}}%
\pgfpathlineto{\pgfqpoint{4.624757in}{10.983252in}}%
\pgfpathlineto{\pgfqpoint{4.704003in}{10.984159in}}%
\pgfpathlineto{\pgfqpoint{4.805890in}{10.987843in}}%
\pgfpathlineto{\pgfqpoint{4.941741in}{10.995170in}}%
\pgfpathlineto{\pgfqpoint{5.032307in}{11.001835in}}%
\pgfpathlineto{\pgfqpoint{5.066270in}{11.005648in}}%
\pgfpathlineto{\pgfqpoint{5.088912in}{11.009707in}}%
\pgfpathlineto{\pgfqpoint{5.100233in}{11.012881in}}%
\pgfpathlineto{\pgfqpoint{5.111553in}{11.017730in}}%
\pgfpathlineto{\pgfqpoint{5.122874in}{11.026131in}}%
\pgfpathlineto{\pgfqpoint{5.134195in}{11.043332in}}%
\pgfpathlineto{\pgfqpoint{5.145516in}{11.087187in}}%
\pgfpathlineto{\pgfqpoint{5.156837in}{11.215752in}}%
\pgfpathlineto{\pgfqpoint{5.168158in}{11.200428in}}%
\pgfpathlineto{\pgfqpoint{5.179479in}{11.080581in}}%
\pgfpathlineto{\pgfqpoint{5.190799in}{11.041892in}}%
\pgfpathlineto{\pgfqpoint{5.202120in}{11.027658in}}%
\pgfpathlineto{\pgfqpoint{5.213441in}{11.021467in}}%
\pgfpathlineto{\pgfqpoint{5.224762in}{11.018502in}}%
\pgfpathlineto{\pgfqpoint{5.247404in}{11.016382in}}%
\pgfpathlineto{\pgfqpoint{5.281366in}{11.016475in}}%
\pgfpathlineto{\pgfqpoint{5.349291in}{11.019487in}}%
\pgfpathlineto{\pgfqpoint{5.575708in}{11.032921in}}%
\pgfpathlineto{\pgfqpoint{6.096468in}{11.064725in}}%
\pgfpathlineto{\pgfqpoint{6.107788in}{11.066781in}}%
\pgfpathlineto{\pgfqpoint{6.119109in}{11.065452in}}%
\pgfpathlineto{\pgfqpoint{6.164393in}{11.067371in}}%
\pgfpathlineto{\pgfqpoint{6.583264in}{11.089733in}}%
\pgfpathlineto{\pgfqpoint{6.583264in}{11.089733in}}%
\pgfusepath{stroke}%
\end{pgfscope}%
\begin{pgfscope}%
\pgfpathrectangle{\pgfqpoint{0.922838in}{9.205081in}}{\pgfqpoint{5.660427in}{2.434069in}}%
\pgfusepath{clip}%
\pgfsetbuttcap%
\pgfsetroundjoin%
\pgfsetlinewidth{0.903375pt}%
\definecolor{currentstroke}{rgb}{0.000000,0.000000,1.000000}%
\pgfsetstrokecolor{currentstroke}%
\pgfsetstrokeopacity{0.800000}%
\pgfsetdash{{3.330000pt}{1.440000pt}}{0.000000pt}%
\pgfpathmoveto{\pgfqpoint{3.281349in}{9.205081in}}%
\pgfpathlineto{\pgfqpoint{3.281349in}{10.327128in}}%
\pgfusepath{stroke}%
\end{pgfscope}%
\begin{pgfscope}%
\pgfsetrectcap%
\pgfsetmiterjoin%
\pgfsetlinewidth{0.803000pt}%
\definecolor{currentstroke}{rgb}{0.000000,0.000000,0.000000}%
\pgfsetstrokecolor{currentstroke}%
\pgfsetdash{}{0pt}%
\pgfpathmoveto{\pgfqpoint{0.922838in}{9.205081in}}%
\pgfpathlineto{\pgfqpoint{0.922838in}{11.639151in}}%
\pgfusepath{stroke}%
\end{pgfscope}%
\begin{pgfscope}%
\pgfsetrectcap%
\pgfsetmiterjoin%
\pgfsetlinewidth{0.803000pt}%
\definecolor{currentstroke}{rgb}{0.000000,0.000000,0.000000}%
\pgfsetstrokecolor{currentstroke}%
\pgfsetdash{}{0pt}%
\pgfpathmoveto{\pgfqpoint{6.583264in}{9.205081in}}%
\pgfpathlineto{\pgfqpoint{6.583264in}{11.639151in}}%
\pgfusepath{stroke}%
\end{pgfscope}%
\begin{pgfscope}%
\pgfsetrectcap%
\pgfsetmiterjoin%
\pgfsetlinewidth{0.803000pt}%
\definecolor{currentstroke}{rgb}{0.000000,0.000000,0.000000}%
\pgfsetstrokecolor{currentstroke}%
\pgfsetdash{}{0pt}%
\pgfpathmoveto{\pgfqpoint{0.922838in}{9.205081in}}%
\pgfpathlineto{\pgfqpoint{6.583264in}{9.205081in}}%
\pgfusepath{stroke}%
\end{pgfscope}%
\begin{pgfscope}%
\pgfsetrectcap%
\pgfsetmiterjoin%
\pgfsetlinewidth{0.803000pt}%
\definecolor{currentstroke}{rgb}{0.000000,0.000000,0.000000}%
\pgfsetstrokecolor{currentstroke}%
\pgfsetdash{}{0pt}%
\pgfpathmoveto{\pgfqpoint{0.922838in}{11.639151in}}%
\pgfpathlineto{\pgfqpoint{6.583264in}{11.639151in}}%
\pgfusepath{stroke}%
\end{pgfscope}%
\begin{pgfscope}%
\definecolor{textcolor}{rgb}{0.000000,0.000000,1.000000}%
\pgfsetstrokecolor{textcolor}%
\pgfsetfillcolor{textcolor}%
\pgftext[x=3.281349in,y=11.175802in,,bottom]{\color{textcolor}{\rmfamily\fontsize{16.000000}{19.200000}\selectfont\catcode`\^=\active\def^{\ifmmode\sp\else\^{}\fi}\catcode`\%=\active\def%{\%}$\Delta \approx -2.5\,\omega_b$}}%
\end{pgfscope}%
\begin{pgfscope}%
\definecolor{textcolor}{rgb}{0.000000,0.000000,0.000000}%
\pgfsetstrokecolor{textcolor}%
\pgfsetfillcolor{textcolor}%
\pgftext[x=0.979442in,y=11.517447in,left,top]{\color{textcolor}{\rmfamily\fontsize{16.000000}{19.200000}\selectfont\catcode`\^=\active\def^{\ifmmode\sp\else\^{}\fi}\catcode`\%=\active\def%{\%}$(a)$}}%
\end{pgfscope}%
\begin{pgfscope}%
\pgfsetbuttcap%
\pgfsetmiterjoin%
\definecolor{currentfill}{rgb}{1.000000,1.000000,1.000000}%
\pgfsetfillcolor{currentfill}%
\pgfsetlinewidth{0.000000pt}%
\definecolor{currentstroke}{rgb}{0.000000,0.000000,0.000000}%
\pgfsetstrokecolor{currentstroke}%
\pgfsetstrokeopacity{0.000000}%
\pgfsetdash{}{0pt}%
\pgfpathmoveto{\pgfqpoint{0.922838in}{6.395424in}}%
\pgfpathlineto{\pgfqpoint{6.583264in}{6.395424in}}%
\pgfpathlineto{\pgfqpoint{6.583264in}{8.829494in}}%
\pgfpathlineto{\pgfqpoint{0.922838in}{8.829494in}}%
\pgfpathlineto{\pgfqpoint{0.922838in}{6.395424in}}%
\pgfpathclose%
\pgfusepath{fill}%
\end{pgfscope}%
\begin{pgfscope}%
\pgfsetbuttcap%
\pgfsetroundjoin%
\definecolor{currentfill}{rgb}{0.000000,0.000000,0.000000}%
\pgfsetfillcolor{currentfill}%
\pgfsetlinewidth{0.803000pt}%
\definecolor{currentstroke}{rgb}{0.000000,0.000000,0.000000}%
\pgfsetstrokecolor{currentstroke}%
\pgfsetdash{}{0pt}%
\pgfsys@defobject{currentmarker}{\pgfqpoint{0.000000in}{-0.048611in}}{\pgfqpoint{0.000000in}{0.000000in}}{%
\pgfpathmoveto{\pgfqpoint{0.000000in}{0.000000in}}%
\pgfpathlineto{\pgfqpoint{0.000000in}{-0.048611in}}%
\pgfusepath{stroke,fill}%
}%
\begin{pgfscope}%
\pgfsys@transformshift{6.583264in}{6.395424in}%
\pgfsys@useobject{currentmarker}{}%
\end{pgfscope}%
\end{pgfscope}%
\begin{pgfscope}%
\pgfsetbuttcap%
\pgfsetroundjoin%
\definecolor{currentfill}{rgb}{0.000000,0.000000,0.000000}%
\pgfsetfillcolor{currentfill}%
\pgfsetlinewidth{0.803000pt}%
\definecolor{currentstroke}{rgb}{0.000000,0.000000,0.000000}%
\pgfsetstrokecolor{currentstroke}%
\pgfsetdash{}{0pt}%
\pgfsys@defobject{currentmarker}{\pgfqpoint{0.000000in}{-0.048611in}}{\pgfqpoint{0.000000in}{0.000000in}}{%
\pgfpathmoveto{\pgfqpoint{0.000000in}{0.000000in}}%
\pgfpathlineto{\pgfqpoint{0.000000in}{-0.048611in}}%
\pgfusepath{stroke,fill}%
}%
\begin{pgfscope}%
\pgfsys@transformshift{5.639860in}{6.395424in}%
\pgfsys@useobject{currentmarker}{}%
\end{pgfscope}%
\end{pgfscope}%
\begin{pgfscope}%
\pgfsetbuttcap%
\pgfsetroundjoin%
\definecolor{currentfill}{rgb}{0.000000,0.000000,0.000000}%
\pgfsetfillcolor{currentfill}%
\pgfsetlinewidth{0.803000pt}%
\definecolor{currentstroke}{rgb}{0.000000,0.000000,0.000000}%
\pgfsetstrokecolor{currentstroke}%
\pgfsetdash{}{0pt}%
\pgfsys@defobject{currentmarker}{\pgfqpoint{0.000000in}{-0.048611in}}{\pgfqpoint{0.000000in}{0.000000in}}{%
\pgfpathmoveto{\pgfqpoint{0.000000in}{0.000000in}}%
\pgfpathlineto{\pgfqpoint{0.000000in}{-0.048611in}}%
\pgfusepath{stroke,fill}%
}%
\begin{pgfscope}%
\pgfsys@transformshift{4.696455in}{6.395424in}%
\pgfsys@useobject{currentmarker}{}%
\end{pgfscope}%
\end{pgfscope}%
\begin{pgfscope}%
\pgfsetbuttcap%
\pgfsetroundjoin%
\definecolor{currentfill}{rgb}{0.000000,0.000000,0.000000}%
\pgfsetfillcolor{currentfill}%
\pgfsetlinewidth{0.803000pt}%
\definecolor{currentstroke}{rgb}{0.000000,0.000000,0.000000}%
\pgfsetstrokecolor{currentstroke}%
\pgfsetdash{}{0pt}%
\pgfsys@defobject{currentmarker}{\pgfqpoint{0.000000in}{-0.048611in}}{\pgfqpoint{0.000000in}{0.000000in}}{%
\pgfpathmoveto{\pgfqpoint{0.000000in}{0.000000in}}%
\pgfpathlineto{\pgfqpoint{0.000000in}{-0.048611in}}%
\pgfusepath{stroke,fill}%
}%
\begin{pgfscope}%
\pgfsys@transformshift{3.753051in}{6.395424in}%
\pgfsys@useobject{currentmarker}{}%
\end{pgfscope}%
\end{pgfscope}%
\begin{pgfscope}%
\pgfsetbuttcap%
\pgfsetroundjoin%
\definecolor{currentfill}{rgb}{0.000000,0.000000,0.000000}%
\pgfsetfillcolor{currentfill}%
\pgfsetlinewidth{0.803000pt}%
\definecolor{currentstroke}{rgb}{0.000000,0.000000,0.000000}%
\pgfsetstrokecolor{currentstroke}%
\pgfsetdash{}{0pt}%
\pgfsys@defobject{currentmarker}{\pgfqpoint{0.000000in}{-0.048611in}}{\pgfqpoint{0.000000in}{0.000000in}}{%
\pgfpathmoveto{\pgfqpoint{0.000000in}{0.000000in}}%
\pgfpathlineto{\pgfqpoint{0.000000in}{-0.048611in}}%
\pgfusepath{stroke,fill}%
}%
\begin{pgfscope}%
\pgfsys@transformshift{2.809647in}{6.395424in}%
\pgfsys@useobject{currentmarker}{}%
\end{pgfscope}%
\end{pgfscope}%
\begin{pgfscope}%
\pgfsetbuttcap%
\pgfsetroundjoin%
\definecolor{currentfill}{rgb}{0.000000,0.000000,0.000000}%
\pgfsetfillcolor{currentfill}%
\pgfsetlinewidth{0.803000pt}%
\definecolor{currentstroke}{rgb}{0.000000,0.000000,0.000000}%
\pgfsetstrokecolor{currentstroke}%
\pgfsetdash{}{0pt}%
\pgfsys@defobject{currentmarker}{\pgfqpoint{0.000000in}{-0.048611in}}{\pgfqpoint{0.000000in}{0.000000in}}{%
\pgfpathmoveto{\pgfqpoint{0.000000in}{0.000000in}}%
\pgfpathlineto{\pgfqpoint{0.000000in}{-0.048611in}}%
\pgfusepath{stroke,fill}%
}%
\begin{pgfscope}%
\pgfsys@transformshift{1.866242in}{6.395424in}%
\pgfsys@useobject{currentmarker}{}%
\end{pgfscope}%
\end{pgfscope}%
\begin{pgfscope}%
\pgfsetbuttcap%
\pgfsetroundjoin%
\definecolor{currentfill}{rgb}{0.000000,0.000000,0.000000}%
\pgfsetfillcolor{currentfill}%
\pgfsetlinewidth{0.803000pt}%
\definecolor{currentstroke}{rgb}{0.000000,0.000000,0.000000}%
\pgfsetstrokecolor{currentstroke}%
\pgfsetdash{}{0pt}%
\pgfsys@defobject{currentmarker}{\pgfqpoint{0.000000in}{-0.048611in}}{\pgfqpoint{0.000000in}{0.000000in}}{%
\pgfpathmoveto{\pgfqpoint{0.000000in}{0.000000in}}%
\pgfpathlineto{\pgfqpoint{0.000000in}{-0.048611in}}%
\pgfusepath{stroke,fill}%
}%
\begin{pgfscope}%
\pgfsys@transformshift{0.922838in}{6.395424in}%
\pgfsys@useobject{currentmarker}{}%
\end{pgfscope}%
\end{pgfscope}%
\begin{pgfscope}%
\pgfsetbuttcap%
\pgfsetroundjoin%
\definecolor{currentfill}{rgb}{0.000000,0.000000,0.000000}%
\pgfsetfillcolor{currentfill}%
\pgfsetlinewidth{0.803000pt}%
\definecolor{currentstroke}{rgb}{0.000000,0.000000,0.000000}%
\pgfsetstrokecolor{currentstroke}%
\pgfsetdash{}{0pt}%
\pgfsys@defobject{currentmarker}{\pgfqpoint{-0.048611in}{0.000000in}}{\pgfqpoint{-0.000000in}{0.000000in}}{%
\pgfpathmoveto{\pgfqpoint{-0.000000in}{0.000000in}}%
\pgfpathlineto{\pgfqpoint{-0.048611in}{0.000000in}}%
\pgfusepath{stroke,fill}%
}%
\begin{pgfscope}%
\pgfsys@transformshift{0.922838in}{6.395424in}%
\pgfsys@useobject{currentmarker}{}%
\end{pgfscope}%
\end{pgfscope}%
\begin{pgfscope}%
\definecolor{textcolor}{rgb}{0.000000,0.000000,0.000000}%
\pgfsetstrokecolor{textcolor}%
\pgfsetfillcolor{textcolor}%
\pgftext[x=0.428511in, y=6.324575in, left, base]{\color{textcolor}{\rmfamily\fontsize{15.000000}{18.000000}\selectfont\catcode`\^=\active\def^{\ifmmode\sp\else\^{}\fi}\catcode`\%=\active\def%{\%}$\mathdefault{10^{-1}}$}}%
\end{pgfscope}%
\begin{pgfscope}%
\pgfsetbuttcap%
\pgfsetroundjoin%
\definecolor{currentfill}{rgb}{0.000000,0.000000,0.000000}%
\pgfsetfillcolor{currentfill}%
\pgfsetlinewidth{0.803000pt}%
\definecolor{currentstroke}{rgb}{0.000000,0.000000,0.000000}%
\pgfsetstrokecolor{currentstroke}%
\pgfsetdash{}{0pt}%
\pgfsys@defobject{currentmarker}{\pgfqpoint{-0.048611in}{0.000000in}}{\pgfqpoint{-0.000000in}{0.000000in}}{%
\pgfpathmoveto{\pgfqpoint{-0.000000in}{0.000000in}}%
\pgfpathlineto{\pgfqpoint{-0.048611in}{0.000000in}}%
\pgfusepath{stroke,fill}%
}%
\begin{pgfscope}%
\pgfsys@transformshift{0.922838in}{6.968147in}%
\pgfsys@useobject{currentmarker}{}%
\end{pgfscope}%
\end{pgfscope}%
\begin{pgfscope}%
\definecolor{textcolor}{rgb}{0.000000,0.000000,0.000000}%
\pgfsetstrokecolor{textcolor}%
\pgfsetfillcolor{textcolor}%
\pgftext[x=0.546798in, y=6.897297in, left, base]{\color{textcolor}{\rmfamily\fontsize{15.000000}{18.000000}\selectfont\catcode`\^=\active\def^{\ifmmode\sp\else\^{}\fi}\catcode`\%=\active\def%{\%}$\mathdefault{10^{3}}$}}%
\end{pgfscope}%
\begin{pgfscope}%
\pgfsetbuttcap%
\pgfsetroundjoin%
\definecolor{currentfill}{rgb}{0.000000,0.000000,0.000000}%
\pgfsetfillcolor{currentfill}%
\pgfsetlinewidth{0.803000pt}%
\definecolor{currentstroke}{rgb}{0.000000,0.000000,0.000000}%
\pgfsetstrokecolor{currentstroke}%
\pgfsetdash{}{0pt}%
\pgfsys@defobject{currentmarker}{\pgfqpoint{-0.048611in}{0.000000in}}{\pgfqpoint{-0.000000in}{0.000000in}}{%
\pgfpathmoveto{\pgfqpoint{-0.000000in}{0.000000in}}%
\pgfpathlineto{\pgfqpoint{-0.048611in}{0.000000in}}%
\pgfusepath{stroke,fill}%
}%
\begin{pgfscope}%
\pgfsys@transformshift{0.922838in}{7.540869in}%
\pgfsys@useobject{currentmarker}{}%
\end{pgfscope}%
\end{pgfscope}%
\begin{pgfscope}%
\definecolor{textcolor}{rgb}{0.000000,0.000000,0.000000}%
\pgfsetstrokecolor{textcolor}%
\pgfsetfillcolor{textcolor}%
\pgftext[x=0.546798in, y=7.470020in, left, base]{\color{textcolor}{\rmfamily\fontsize{15.000000}{18.000000}\selectfont\catcode`\^=\active\def^{\ifmmode\sp\else\^{}\fi}\catcode`\%=\active\def%{\%}$\mathdefault{10^{7}}$}}%
\end{pgfscope}%
\begin{pgfscope}%
\pgfsetbuttcap%
\pgfsetroundjoin%
\definecolor{currentfill}{rgb}{0.000000,0.000000,0.000000}%
\pgfsetfillcolor{currentfill}%
\pgfsetlinewidth{0.803000pt}%
\definecolor{currentstroke}{rgb}{0.000000,0.000000,0.000000}%
\pgfsetstrokecolor{currentstroke}%
\pgfsetdash{}{0pt}%
\pgfsys@defobject{currentmarker}{\pgfqpoint{-0.048611in}{0.000000in}}{\pgfqpoint{-0.000000in}{0.000000in}}{%
\pgfpathmoveto{\pgfqpoint{-0.000000in}{0.000000in}}%
\pgfpathlineto{\pgfqpoint{-0.048611in}{0.000000in}}%
\pgfusepath{stroke,fill}%
}%
\begin{pgfscope}%
\pgfsys@transformshift{0.922838in}{8.113591in}%
\pgfsys@useobject{currentmarker}{}%
\end{pgfscope}%
\end{pgfscope}%
\begin{pgfscope}%
\definecolor{textcolor}{rgb}{0.000000,0.000000,0.000000}%
\pgfsetstrokecolor{textcolor}%
\pgfsetfillcolor{textcolor}%
\pgftext[x=0.470757in, y=8.042742in, left, base]{\color{textcolor}{\rmfamily\fontsize{15.000000}{18.000000}\selectfont\catcode`\^=\active\def^{\ifmmode\sp\else\^{}\fi}\catcode`\%=\active\def%{\%}$\mathdefault{10^{11}}$}}%
\end{pgfscope}%
\begin{pgfscope}%
\pgfsetbuttcap%
\pgfsetroundjoin%
\definecolor{currentfill}{rgb}{0.000000,0.000000,0.000000}%
\pgfsetfillcolor{currentfill}%
\pgfsetlinewidth{0.803000pt}%
\definecolor{currentstroke}{rgb}{0.000000,0.000000,0.000000}%
\pgfsetstrokecolor{currentstroke}%
\pgfsetdash{}{0pt}%
\pgfsys@defobject{currentmarker}{\pgfqpoint{-0.048611in}{0.000000in}}{\pgfqpoint{-0.000000in}{0.000000in}}{%
\pgfpathmoveto{\pgfqpoint{-0.000000in}{0.000000in}}%
\pgfpathlineto{\pgfqpoint{-0.048611in}{0.000000in}}%
\pgfusepath{stroke,fill}%
}%
\begin{pgfscope}%
\pgfsys@transformshift{0.922838in}{8.686313in}%
\pgfsys@useobject{currentmarker}{}%
\end{pgfscope}%
\end{pgfscope}%
\begin{pgfscope}%
\definecolor{textcolor}{rgb}{0.000000,0.000000,0.000000}%
\pgfsetstrokecolor{textcolor}%
\pgfsetfillcolor{textcolor}%
\pgftext[x=0.470757in, y=8.615464in, left, base]{\color{textcolor}{\rmfamily\fontsize{15.000000}{18.000000}\selectfont\catcode`\^=\active\def^{\ifmmode\sp\else\^{}\fi}\catcode`\%=\active\def%{\%}$\mathdefault{10^{15}}$}}%
\end{pgfscope}%
\begin{pgfscope}%
\definecolor{textcolor}{rgb}{0.000000,0.000000,0.000000}%
\pgfsetstrokecolor{textcolor}%
\pgfsetfillcolor{textcolor}%
\pgftext[x=0.372956in,y=7.612459in,,bottom,rotate=90.000000]{\color{textcolor}{\rmfamily\fontsize{16.000000}{19.200000}\selectfont\catcode`\^=\active\def^{\ifmmode\sp\else\^{}\fi}\catcode`\%=\active\def%{\%}$g^{(3)}(0)$}}%
\end{pgfscope}%
\begin{pgfscope}%
\pgfpathrectangle{\pgfqpoint{0.922838in}{6.395424in}}{\pgfqpoint{5.660427in}{2.434069in}}%
\pgfusepath{clip}%
\pgfsetrectcap%
\pgfsetroundjoin%
\pgfsetlinewidth{1.606000pt}%
\definecolor{currentstroke}{rgb}{0.000000,0.501961,0.000000}%
\pgfsetstrokecolor{currentstroke}%
\pgfsetdash{}{0pt}%
\pgfpathmoveto{\pgfqpoint{0.922838in}{7.334961in}}%
\pgfpathlineto{\pgfqpoint{0.934159in}{7.389740in}}%
\pgfpathlineto{\pgfqpoint{0.945479in}{7.387794in}}%
\pgfpathlineto{\pgfqpoint{1.002084in}{7.325611in}}%
\pgfpathlineto{\pgfqpoint{1.036046in}{7.294787in}}%
\pgfpathlineto{\pgfqpoint{1.103971in}{7.234673in}}%
\pgfpathlineto{\pgfqpoint{1.137934in}{7.202196in}}%
\pgfpathlineto{\pgfqpoint{1.171897in}{7.166816in}}%
\pgfpathlineto{\pgfqpoint{1.205859in}{7.127636in}}%
\pgfpathlineto{\pgfqpoint{1.239822in}{7.083674in}}%
\pgfpathlineto{\pgfqpoint{1.262463in}{7.051208in}}%
\pgfpathlineto{\pgfqpoint{1.285105in}{7.015983in}}%
\pgfpathlineto{\pgfqpoint{1.341709in}{6.923246in}}%
\pgfpathlineto{\pgfqpoint{1.353030in}{6.908954in}}%
\pgfpathlineto{\pgfqpoint{1.364351in}{6.842318in}}%
\pgfpathlineto{\pgfqpoint{1.375672in}{6.783343in}}%
\pgfpathlineto{\pgfqpoint{1.386993in}{6.664159in}}%
\pgfpathlineto{\pgfqpoint{1.398314in}{6.801795in}}%
\pgfpathlineto{\pgfqpoint{1.409634in}{6.811998in}}%
\pgfpathlineto{\pgfqpoint{1.420955in}{6.829970in}}%
\pgfpathlineto{\pgfqpoint{1.466239in}{6.924322in}}%
\pgfpathlineto{\pgfqpoint{1.488880in}{6.965074in}}%
\pgfpathlineto{\pgfqpoint{1.511522in}{7.000988in}}%
\pgfpathlineto{\pgfqpoint{1.534164in}{7.032778in}}%
\pgfpathlineto{\pgfqpoint{1.556806in}{7.061067in}}%
\pgfpathlineto{\pgfqpoint{1.579447in}{7.086370in}}%
\pgfpathlineto{\pgfqpoint{1.602089in}{7.109103in}}%
\pgfpathlineto{\pgfqpoint{1.624731in}{7.129603in}}%
\pgfpathlineto{\pgfqpoint{1.658693in}{7.156740in}}%
\pgfpathlineto{\pgfqpoint{1.692656in}{7.180156in}}%
\pgfpathlineto{\pgfqpoint{1.726618in}{7.200389in}}%
\pgfpathlineto{\pgfqpoint{1.760581in}{7.217976in}}%
\pgfpathlineto{\pgfqpoint{1.805864in}{7.240047in}}%
\pgfpathlineto{\pgfqpoint{1.817185in}{7.247516in}}%
\pgfpathlineto{\pgfqpoint{1.828506in}{7.260007in}}%
\pgfpathlineto{\pgfqpoint{1.839827in}{7.292443in}}%
\pgfpathlineto{\pgfqpoint{1.851148in}{7.441725in}}%
\pgfpathlineto{\pgfqpoint{1.862469in}{7.345497in}}%
\pgfpathlineto{\pgfqpoint{1.873789in}{7.289305in}}%
\pgfpathlineto{\pgfqpoint{1.885110in}{7.274963in}}%
\pgfpathlineto{\pgfqpoint{1.896431in}{7.270236in}}%
\pgfpathlineto{\pgfqpoint{1.919073in}{7.267977in}}%
\pgfpathlineto{\pgfqpoint{1.975677in}{7.266402in}}%
\pgfpathlineto{\pgfqpoint{2.009640in}{7.262543in}}%
\pgfpathlineto{\pgfqpoint{2.032281in}{7.258097in}}%
\pgfpathlineto{\pgfqpoint{2.054923in}{7.251892in}}%
\pgfpathlineto{\pgfqpoint{2.077565in}{7.243677in}}%
\pgfpathlineto{\pgfqpoint{2.100206in}{7.233143in}}%
\pgfpathlineto{\pgfqpoint{2.122848in}{7.219893in}}%
\pgfpathlineto{\pgfqpoint{2.145490in}{7.203398in}}%
\pgfpathlineto{\pgfqpoint{2.168132in}{7.182928in}}%
\pgfpathlineto{\pgfqpoint{2.190773in}{7.157432in}}%
\pgfpathlineto{\pgfqpoint{2.202094in}{7.142335in}}%
\pgfpathlineto{\pgfqpoint{2.213415in}{7.125327in}}%
\pgfpathlineto{\pgfqpoint{2.224736in}{7.106059in}}%
\pgfpathlineto{\pgfqpoint{2.236057in}{7.084071in}}%
\pgfpathlineto{\pgfqpoint{2.247378in}{7.058746in}}%
\pgfpathlineto{\pgfqpoint{2.258698in}{7.029231in}}%
\pgfpathlineto{\pgfqpoint{2.270019in}{6.994312in}}%
\pgfpathlineto{\pgfqpoint{2.281340in}{6.952211in}}%
\pgfpathlineto{\pgfqpoint{2.292661in}{6.900308in}}%
\pgfpathlineto{\pgfqpoint{2.303982in}{6.835335in}}%
\pgfpathlineto{\pgfqpoint{2.315303in}{6.762307in}}%
\pgfpathlineto{\pgfqpoint{2.326624in}{6.608666in}}%
\pgfpathlineto{\pgfqpoint{2.337944in}{6.584221in}}%
\pgfpathlineto{\pgfqpoint{2.349265in}{6.749812in}}%
\pgfpathlineto{\pgfqpoint{2.360586in}{6.854463in}}%
\pgfpathlineto{\pgfqpoint{2.371907in}{6.932299in}}%
\pgfpathlineto{\pgfqpoint{2.383228in}{6.994242in}}%
\pgfpathlineto{\pgfqpoint{2.394549in}{7.045720in}}%
\pgfpathlineto{\pgfqpoint{2.405870in}{7.089829in}}%
\pgfpathlineto{\pgfqpoint{2.417190in}{7.128488in}}%
\pgfpathlineto{\pgfqpoint{2.439832in}{7.194120in}}%
\pgfpathlineto{\pgfqpoint{2.462474in}{7.248839in}}%
\pgfpathlineto{\pgfqpoint{2.485115in}{7.295999in}}%
\pgfpathlineto{\pgfqpoint{2.507757in}{7.337609in}}%
\pgfpathlineto{\pgfqpoint{2.530399in}{7.374968in}}%
\pgfpathlineto{\pgfqpoint{2.553041in}{7.408964in}}%
\pgfpathlineto{\pgfqpoint{2.587003in}{7.455002in}}%
\pgfpathlineto{\pgfqpoint{2.620966in}{7.496383in}}%
\pgfpathlineto{\pgfqpoint{2.654928in}{7.534136in}}%
\pgfpathlineto{\pgfqpoint{2.688891in}{7.568996in}}%
\pgfpathlineto{\pgfqpoint{2.734174in}{7.611901in}}%
\pgfpathlineto{\pgfqpoint{2.779458in}{7.651597in}}%
\pgfpathlineto{\pgfqpoint{2.824741in}{7.688719in}}%
\pgfpathlineto{\pgfqpoint{2.881345in}{7.732244in}}%
\pgfpathlineto{\pgfqpoint{2.937950in}{7.773015in}}%
\pgfpathlineto{\pgfqpoint{2.994554in}{7.810791in}}%
\pgfpathlineto{\pgfqpoint{3.028516in}{7.831421in}}%
\pgfpathlineto{\pgfqpoint{3.062479in}{7.849553in}}%
\pgfpathlineto{\pgfqpoint{3.085121in}{7.859475in}}%
\pgfpathlineto{\pgfqpoint{3.107762in}{7.866727in}}%
\pgfpathlineto{\pgfqpoint{3.119083in}{7.868964in}}%
\pgfpathlineto{\pgfqpoint{3.130404in}{7.869985in}}%
\pgfpathlineto{\pgfqpoint{3.141725in}{7.869499in}}%
\pgfpathlineto{\pgfqpoint{3.153046in}{7.867134in}}%
\pgfpathlineto{\pgfqpoint{3.164367in}{7.862410in}}%
\pgfpathlineto{\pgfqpoint{3.175688in}{7.854703in}}%
\pgfpathlineto{\pgfqpoint{3.187008in}{7.843187in}}%
\pgfpathlineto{\pgfqpoint{3.198329in}{7.826740in}}%
\pgfpathlineto{\pgfqpoint{3.209650in}{7.803782in}}%
\pgfpathlineto{\pgfqpoint{3.220971in}{7.771967in}}%
\pgfpathlineto{\pgfqpoint{3.232292in}{7.727502in}}%
\pgfpathlineto{\pgfqpoint{3.243613in}{7.663429in}}%
\pgfpathlineto{\pgfqpoint{3.254934in}{7.564156in}}%
\pgfpathlineto{\pgfqpoint{3.266254in}{7.379824in}}%
\pgfpathlineto{\pgfqpoint{3.277575in}{7.170514in}}%
\pgfpathlineto{\pgfqpoint{3.288896in}{7.483513in}}%
\pgfpathlineto{\pgfqpoint{3.300217in}{7.627576in}}%
\pgfpathlineto{\pgfqpoint{3.311538in}{7.716086in}}%
\pgfpathlineto{\pgfqpoint{3.322859in}{7.777842in}}%
\pgfpathlineto{\pgfqpoint{3.334179in}{7.823605in}}%
\pgfpathlineto{\pgfqpoint{3.345500in}{7.858664in}}%
\pgfpathlineto{\pgfqpoint{3.356821in}{7.886076in}}%
\pgfpathlineto{\pgfqpoint{3.368142in}{7.907807in}}%
\pgfpathlineto{\pgfqpoint{3.379463in}{7.925210in}}%
\pgfpathlineto{\pgfqpoint{3.390784in}{7.939262in}}%
\pgfpathlineto{\pgfqpoint{3.402105in}{7.950689in}}%
\pgfpathlineto{\pgfqpoint{3.413425in}{7.960044in}}%
\pgfpathlineto{\pgfqpoint{3.436067in}{7.974147in}}%
\pgfpathlineto{\pgfqpoint{3.458709in}{7.983988in}}%
\pgfpathlineto{\pgfqpoint{3.481351in}{7.991080in}}%
\pgfpathlineto{\pgfqpoint{3.515313in}{7.998588in}}%
\pgfpathlineto{\pgfqpoint{3.560597in}{8.005560in}}%
\pgfpathlineto{\pgfqpoint{3.753051in}{8.031273in}}%
\pgfpathlineto{\pgfqpoint{3.809655in}{8.041965in}}%
\pgfpathlineto{\pgfqpoint{3.854939in}{8.052370in}}%
\pgfpathlineto{\pgfqpoint{3.900222in}{8.064867in}}%
\pgfpathlineto{\pgfqpoint{3.945506in}{8.080040in}}%
\pgfpathlineto{\pgfqpoint{3.979468in}{8.093685in}}%
\pgfpathlineto{\pgfqpoint{4.013431in}{8.109868in}}%
\pgfpathlineto{\pgfqpoint{4.036072in}{8.122453in}}%
\pgfpathlineto{\pgfqpoint{4.058714in}{8.136864in}}%
\pgfpathlineto{\pgfqpoint{4.081356in}{8.153577in}}%
\pgfpathlineto{\pgfqpoint{4.103997in}{8.173266in}}%
\pgfpathlineto{\pgfqpoint{4.126639in}{8.196893in}}%
\pgfpathlineto{\pgfqpoint{4.149281in}{8.225734in}}%
\pgfpathlineto{\pgfqpoint{4.171923in}{8.260403in}}%
\pgfpathlineto{\pgfqpoint{4.183243in}{8.277869in}}%
\pgfpathlineto{\pgfqpoint{4.194564in}{8.287280in}}%
\pgfpathlineto{\pgfqpoint{4.205885in}{8.254566in}}%
\pgfpathlineto{\pgfqpoint{4.217206in}{8.073524in}}%
\pgfpathlineto{\pgfqpoint{4.228527in}{8.229184in}}%
\pgfpathlineto{\pgfqpoint{4.239848in}{8.287453in}}%
\pgfpathlineto{\pgfqpoint{4.251169in}{8.282154in}}%
\pgfpathlineto{\pgfqpoint{4.296452in}{8.210559in}}%
\pgfpathlineto{\pgfqpoint{4.319094in}{8.182129in}}%
\pgfpathlineto{\pgfqpoint{4.341735in}{8.158584in}}%
\pgfpathlineto{\pgfqpoint{4.364377in}{8.138723in}}%
\pgfpathlineto{\pgfqpoint{4.387019in}{8.121691in}}%
\pgfpathlineto{\pgfqpoint{4.409661in}{8.106899in}}%
\pgfpathlineto{\pgfqpoint{4.443623in}{8.088030in}}%
\pgfpathlineto{\pgfqpoint{4.477586in}{8.072322in}}%
\pgfpathlineto{\pgfqpoint{4.511548in}{8.059145in}}%
\pgfpathlineto{\pgfqpoint{4.545511in}{8.048063in}}%
\pgfpathlineto{\pgfqpoint{4.590794in}{8.036011in}}%
\pgfpathlineto{\pgfqpoint{4.636078in}{8.026599in}}%
\pgfpathlineto{\pgfqpoint{4.681361in}{8.019443in}}%
\pgfpathlineto{\pgfqpoint{4.726644in}{8.014286in}}%
\pgfpathlineto{\pgfqpoint{4.783249in}{8.010393in}}%
\pgfpathlineto{\pgfqpoint{4.828532in}{8.009266in}}%
\pgfpathlineto{\pgfqpoint{4.873815in}{8.010066in}}%
\pgfpathlineto{\pgfqpoint{4.919099in}{8.013196in}}%
\pgfpathlineto{\pgfqpoint{4.953061in}{8.017626in}}%
\pgfpathlineto{\pgfqpoint{4.987024in}{8.024698in}}%
\pgfpathlineto{\pgfqpoint{5.009666in}{8.031555in}}%
\pgfpathlineto{\pgfqpoint{5.032307in}{8.040991in}}%
\pgfpathlineto{\pgfqpoint{5.054949in}{8.054205in}}%
\pgfpathlineto{\pgfqpoint{5.066270in}{8.062828in}}%
\pgfpathlineto{\pgfqpoint{5.077591in}{8.073297in}}%
\pgfpathlineto{\pgfqpoint{5.088912in}{8.086158in}}%
\pgfpathlineto{\pgfqpoint{5.100233in}{8.102226in}}%
\pgfpathlineto{\pgfqpoint{5.111553in}{8.122747in}}%
\pgfpathlineto{\pgfqpoint{5.122874in}{8.149812in}}%
\pgfpathlineto{\pgfqpoint{5.134195in}{8.187351in}}%
\pgfpathlineto{\pgfqpoint{5.145516in}{8.244001in}}%
\pgfpathlineto{\pgfqpoint{5.156837in}{8.336456in}}%
\pgfpathlineto{\pgfqpoint{5.168158in}{8.327364in}}%
\pgfpathlineto{\pgfqpoint{5.179479in}{8.235732in}}%
\pgfpathlineto{\pgfqpoint{5.190799in}{8.179303in}}%
\pgfpathlineto{\pgfqpoint{5.202120in}{8.141252in}}%
\pgfpathlineto{\pgfqpoint{5.213441in}{8.113537in}}%
\pgfpathlineto{\pgfqpoint{5.224762in}{8.092452in}}%
\pgfpathlineto{\pgfqpoint{5.236083in}{8.075985in}}%
\pgfpathlineto{\pgfqpoint{5.247404in}{8.062919in}}%
\pgfpathlineto{\pgfqpoint{5.258724in}{8.052441in}}%
\pgfpathlineto{\pgfqpoint{5.270045in}{8.043955in}}%
\pgfpathlineto{\pgfqpoint{5.292687in}{8.031418in}}%
\pgfpathlineto{\pgfqpoint{5.315329in}{8.022992in}}%
\pgfpathlineto{\pgfqpoint{5.337970in}{8.017243in}}%
\pgfpathlineto{\pgfqpoint{5.371933in}{8.011878in}}%
\pgfpathlineto{\pgfqpoint{5.405896in}{8.008907in}}%
\pgfpathlineto{\pgfqpoint{5.451179in}{8.007137in}}%
\pgfpathlineto{\pgfqpoint{5.519104in}{8.007010in}}%
\pgfpathlineto{\pgfqpoint{5.620992in}{8.009450in}}%
\pgfpathlineto{\pgfqpoint{5.802125in}{8.016682in}}%
\pgfpathlineto{\pgfqpoint{6.005901in}{8.027037in}}%
\pgfpathlineto{\pgfqpoint{6.039863in}{8.030060in}}%
\pgfpathlineto{\pgfqpoint{6.062505in}{8.034182in}}%
\pgfpathlineto{\pgfqpoint{6.073826in}{8.038760in}}%
\pgfpathlineto{\pgfqpoint{6.085147in}{8.049347in}}%
\pgfpathlineto{\pgfqpoint{6.107788in}{8.112794in}}%
\pgfpathlineto{\pgfqpoint{6.119109in}{8.063187in}}%
\pgfpathlineto{\pgfqpoint{6.130430in}{8.044661in}}%
\pgfpathlineto{\pgfqpoint{6.141751in}{8.038673in}}%
\pgfpathlineto{\pgfqpoint{6.153072in}{8.036481in}}%
\pgfpathlineto{\pgfqpoint{6.175714in}{8.035420in}}%
\pgfpathlineto{\pgfqpoint{6.220997in}{8.036374in}}%
\pgfpathlineto{\pgfqpoint{6.379489in}{8.043188in}}%
\pgfpathlineto{\pgfqpoint{6.583264in}{8.052448in}}%
\pgfpathlineto{\pgfqpoint{6.583264in}{8.052448in}}%
\pgfusepath{stroke}%
\end{pgfscope}%
\begin{pgfscope}%
\pgfpathrectangle{\pgfqpoint{0.922838in}{6.395424in}}{\pgfqpoint{5.660427in}{2.434069in}}%
\pgfusepath{clip}%
\pgfsetbuttcap%
\pgfsetroundjoin%
\pgfsetlinewidth{0.903375pt}%
\definecolor{currentstroke}{rgb}{0.000000,0.501961,0.000000}%
\pgfsetstrokecolor{currentstroke}%
\pgfsetstrokeopacity{0.800000}%
\pgfsetdash{{3.330000pt}{1.440000pt}}{0.000000pt}%
\pgfpathmoveto{\pgfqpoint{4.224753in}{6.395424in}}%
\pgfpathlineto{\pgfqpoint{4.224753in}{8.229184in}}%
\pgfusepath{stroke}%
\end{pgfscope}%
\begin{pgfscope}%
\pgfsetrectcap%
\pgfsetmiterjoin%
\pgfsetlinewidth{0.803000pt}%
\definecolor{currentstroke}{rgb}{0.000000,0.000000,0.000000}%
\pgfsetstrokecolor{currentstroke}%
\pgfsetdash{}{0pt}%
\pgfpathmoveto{\pgfqpoint{0.922838in}{6.395424in}}%
\pgfpathlineto{\pgfqpoint{0.922838in}{8.829494in}}%
\pgfusepath{stroke}%
\end{pgfscope}%
\begin{pgfscope}%
\pgfsetrectcap%
\pgfsetmiterjoin%
\pgfsetlinewidth{0.803000pt}%
\definecolor{currentstroke}{rgb}{0.000000,0.000000,0.000000}%
\pgfsetstrokecolor{currentstroke}%
\pgfsetdash{}{0pt}%
\pgfpathmoveto{\pgfqpoint{6.583264in}{6.395424in}}%
\pgfpathlineto{\pgfqpoint{6.583264in}{8.829494in}}%
\pgfusepath{stroke}%
\end{pgfscope}%
\begin{pgfscope}%
\pgfsetrectcap%
\pgfsetmiterjoin%
\pgfsetlinewidth{0.803000pt}%
\definecolor{currentstroke}{rgb}{0.000000,0.000000,0.000000}%
\pgfsetstrokecolor{currentstroke}%
\pgfsetdash{}{0pt}%
\pgfpathmoveto{\pgfqpoint{0.922838in}{6.395424in}}%
\pgfpathlineto{\pgfqpoint{6.583264in}{6.395424in}}%
\pgfusepath{stroke}%
\end{pgfscope}%
\begin{pgfscope}%
\pgfsetrectcap%
\pgfsetmiterjoin%
\pgfsetlinewidth{0.803000pt}%
\definecolor{currentstroke}{rgb}{0.000000,0.000000,0.000000}%
\pgfsetstrokecolor{currentstroke}%
\pgfsetdash{}{0pt}%
\pgfpathmoveto{\pgfqpoint{0.922838in}{8.829494in}}%
\pgfpathlineto{\pgfqpoint{6.583264in}{8.829494in}}%
\pgfusepath{stroke}%
\end{pgfscope}%
\begin{pgfscope}%
\definecolor{textcolor}{rgb}{0.000000,0.501961,0.000000}%
\pgfsetstrokecolor{textcolor}%
\pgfsetfillcolor{textcolor}%
\pgftext[x=4.224753in,y=8.399952in,,bottom]{\color{textcolor}{\rmfamily\fontsize{16.000000}{19.200000}\selectfont\catcode`\^=\active\def^{\ifmmode\sp\else\^{}\fi}\catcode`\%=\active\def%{\%}$\Delta \approx -3.5\,\omega_b$}}%
\end{pgfscope}%
\begin{pgfscope}%
\definecolor{textcolor}{rgb}{0.000000,0.000000,0.000000}%
\pgfsetstrokecolor{textcolor}%
\pgfsetfillcolor{textcolor}%
\pgftext[x=0.979442in,y=8.707790in,left,top]{\color{textcolor}{\rmfamily\fontsize{16.000000}{19.200000}\selectfont\catcode`\^=\active\def^{\ifmmode\sp\else\^{}\fi}\catcode`\%=\active\def%{\%}$(b)$}}%
\end{pgfscope}%
\begin{pgfscope}%
\pgfsetbuttcap%
\pgfsetmiterjoin%
\definecolor{currentfill}{rgb}{1.000000,1.000000,1.000000}%
\pgfsetfillcolor{currentfill}%
\pgfsetlinewidth{0.000000pt}%
\definecolor{currentstroke}{rgb}{0.000000,0.000000,0.000000}%
\pgfsetstrokecolor{currentstroke}%
\pgfsetstrokeopacity{0.000000}%
\pgfsetdash{}{0pt}%
\pgfpathmoveto{\pgfqpoint{0.922838in}{3.585768in}}%
\pgfpathlineto{\pgfqpoint{6.583264in}{3.585768in}}%
\pgfpathlineto{\pgfqpoint{6.583264in}{6.019837in}}%
\pgfpathlineto{\pgfqpoint{0.922838in}{6.019837in}}%
\pgfpathlineto{\pgfqpoint{0.922838in}{3.585768in}}%
\pgfpathclose%
\pgfusepath{fill}%
\end{pgfscope}%
\begin{pgfscope}%
\pgfsetbuttcap%
\pgfsetroundjoin%
\definecolor{currentfill}{rgb}{0.000000,0.000000,0.000000}%
\pgfsetfillcolor{currentfill}%
\pgfsetlinewidth{0.803000pt}%
\definecolor{currentstroke}{rgb}{0.000000,0.000000,0.000000}%
\pgfsetstrokecolor{currentstroke}%
\pgfsetdash{}{0pt}%
\pgfsys@defobject{currentmarker}{\pgfqpoint{0.000000in}{-0.048611in}}{\pgfqpoint{0.000000in}{0.000000in}}{%
\pgfpathmoveto{\pgfqpoint{0.000000in}{0.000000in}}%
\pgfpathlineto{\pgfqpoint{0.000000in}{-0.048611in}}%
\pgfusepath{stroke,fill}%
}%
\begin{pgfscope}%
\pgfsys@transformshift{6.583264in}{3.585768in}%
\pgfsys@useobject{currentmarker}{}%
\end{pgfscope}%
\end{pgfscope}%
\begin{pgfscope}%
\pgfsetbuttcap%
\pgfsetroundjoin%
\definecolor{currentfill}{rgb}{0.000000,0.000000,0.000000}%
\pgfsetfillcolor{currentfill}%
\pgfsetlinewidth{0.803000pt}%
\definecolor{currentstroke}{rgb}{0.000000,0.000000,0.000000}%
\pgfsetstrokecolor{currentstroke}%
\pgfsetdash{}{0pt}%
\pgfsys@defobject{currentmarker}{\pgfqpoint{0.000000in}{-0.048611in}}{\pgfqpoint{0.000000in}{0.000000in}}{%
\pgfpathmoveto{\pgfqpoint{0.000000in}{0.000000in}}%
\pgfpathlineto{\pgfqpoint{0.000000in}{-0.048611in}}%
\pgfusepath{stroke,fill}%
}%
\begin{pgfscope}%
\pgfsys@transformshift{5.639860in}{3.585768in}%
\pgfsys@useobject{currentmarker}{}%
\end{pgfscope}%
\end{pgfscope}%
\begin{pgfscope}%
\pgfsetbuttcap%
\pgfsetroundjoin%
\definecolor{currentfill}{rgb}{0.000000,0.000000,0.000000}%
\pgfsetfillcolor{currentfill}%
\pgfsetlinewidth{0.803000pt}%
\definecolor{currentstroke}{rgb}{0.000000,0.000000,0.000000}%
\pgfsetstrokecolor{currentstroke}%
\pgfsetdash{}{0pt}%
\pgfsys@defobject{currentmarker}{\pgfqpoint{0.000000in}{-0.048611in}}{\pgfqpoint{0.000000in}{0.000000in}}{%
\pgfpathmoveto{\pgfqpoint{0.000000in}{0.000000in}}%
\pgfpathlineto{\pgfqpoint{0.000000in}{-0.048611in}}%
\pgfusepath{stroke,fill}%
}%
\begin{pgfscope}%
\pgfsys@transformshift{4.696455in}{3.585768in}%
\pgfsys@useobject{currentmarker}{}%
\end{pgfscope}%
\end{pgfscope}%
\begin{pgfscope}%
\pgfsetbuttcap%
\pgfsetroundjoin%
\definecolor{currentfill}{rgb}{0.000000,0.000000,0.000000}%
\pgfsetfillcolor{currentfill}%
\pgfsetlinewidth{0.803000pt}%
\definecolor{currentstroke}{rgb}{0.000000,0.000000,0.000000}%
\pgfsetstrokecolor{currentstroke}%
\pgfsetdash{}{0pt}%
\pgfsys@defobject{currentmarker}{\pgfqpoint{0.000000in}{-0.048611in}}{\pgfqpoint{0.000000in}{0.000000in}}{%
\pgfpathmoveto{\pgfqpoint{0.000000in}{0.000000in}}%
\pgfpathlineto{\pgfqpoint{0.000000in}{-0.048611in}}%
\pgfusepath{stroke,fill}%
}%
\begin{pgfscope}%
\pgfsys@transformshift{3.753051in}{3.585768in}%
\pgfsys@useobject{currentmarker}{}%
\end{pgfscope}%
\end{pgfscope}%
\begin{pgfscope}%
\pgfsetbuttcap%
\pgfsetroundjoin%
\definecolor{currentfill}{rgb}{0.000000,0.000000,0.000000}%
\pgfsetfillcolor{currentfill}%
\pgfsetlinewidth{0.803000pt}%
\definecolor{currentstroke}{rgb}{0.000000,0.000000,0.000000}%
\pgfsetstrokecolor{currentstroke}%
\pgfsetdash{}{0pt}%
\pgfsys@defobject{currentmarker}{\pgfqpoint{0.000000in}{-0.048611in}}{\pgfqpoint{0.000000in}{0.000000in}}{%
\pgfpathmoveto{\pgfqpoint{0.000000in}{0.000000in}}%
\pgfpathlineto{\pgfqpoint{0.000000in}{-0.048611in}}%
\pgfusepath{stroke,fill}%
}%
\begin{pgfscope}%
\pgfsys@transformshift{2.809647in}{3.585768in}%
\pgfsys@useobject{currentmarker}{}%
\end{pgfscope}%
\end{pgfscope}%
\begin{pgfscope}%
\pgfsetbuttcap%
\pgfsetroundjoin%
\definecolor{currentfill}{rgb}{0.000000,0.000000,0.000000}%
\pgfsetfillcolor{currentfill}%
\pgfsetlinewidth{0.803000pt}%
\definecolor{currentstroke}{rgb}{0.000000,0.000000,0.000000}%
\pgfsetstrokecolor{currentstroke}%
\pgfsetdash{}{0pt}%
\pgfsys@defobject{currentmarker}{\pgfqpoint{0.000000in}{-0.048611in}}{\pgfqpoint{0.000000in}{0.000000in}}{%
\pgfpathmoveto{\pgfqpoint{0.000000in}{0.000000in}}%
\pgfpathlineto{\pgfqpoint{0.000000in}{-0.048611in}}%
\pgfusepath{stroke,fill}%
}%
\begin{pgfscope}%
\pgfsys@transformshift{1.866242in}{3.585768in}%
\pgfsys@useobject{currentmarker}{}%
\end{pgfscope}%
\end{pgfscope}%
\begin{pgfscope}%
\pgfsetbuttcap%
\pgfsetroundjoin%
\definecolor{currentfill}{rgb}{0.000000,0.000000,0.000000}%
\pgfsetfillcolor{currentfill}%
\pgfsetlinewidth{0.803000pt}%
\definecolor{currentstroke}{rgb}{0.000000,0.000000,0.000000}%
\pgfsetstrokecolor{currentstroke}%
\pgfsetdash{}{0pt}%
\pgfsys@defobject{currentmarker}{\pgfqpoint{0.000000in}{-0.048611in}}{\pgfqpoint{0.000000in}{0.000000in}}{%
\pgfpathmoveto{\pgfqpoint{0.000000in}{0.000000in}}%
\pgfpathlineto{\pgfqpoint{0.000000in}{-0.048611in}}%
\pgfusepath{stroke,fill}%
}%
\begin{pgfscope}%
\pgfsys@transformshift{0.922838in}{3.585768in}%
\pgfsys@useobject{currentmarker}{}%
\end{pgfscope}%
\end{pgfscope}%
\begin{pgfscope}%
\pgfsetbuttcap%
\pgfsetroundjoin%
\definecolor{currentfill}{rgb}{0.000000,0.000000,0.000000}%
\pgfsetfillcolor{currentfill}%
\pgfsetlinewidth{0.803000pt}%
\definecolor{currentstroke}{rgb}{0.000000,0.000000,0.000000}%
\pgfsetstrokecolor{currentstroke}%
\pgfsetdash{}{0pt}%
\pgfsys@defobject{currentmarker}{\pgfqpoint{-0.048611in}{0.000000in}}{\pgfqpoint{-0.000000in}{0.000000in}}{%
\pgfpathmoveto{\pgfqpoint{-0.000000in}{0.000000in}}%
\pgfpathlineto{\pgfqpoint{-0.048611in}{0.000000in}}%
\pgfusepath{stroke,fill}%
}%
\begin{pgfscope}%
\pgfsys@transformshift{0.922838in}{3.585768in}%
\pgfsys@useobject{currentmarker}{}%
\end{pgfscope}%
\end{pgfscope}%
\begin{pgfscope}%
\definecolor{textcolor}{rgb}{0.000000,0.000000,0.000000}%
\pgfsetstrokecolor{textcolor}%
\pgfsetfillcolor{textcolor}%
\pgftext[x=0.428511in, y=3.514918in, left, base]{\color{textcolor}{\rmfamily\fontsize{15.000000}{18.000000}\selectfont\catcode`\^=\active\def^{\ifmmode\sp\else\^{}\fi}\catcode`\%=\active\def%{\%}$\mathdefault{10^{-1}}$}}%
\end{pgfscope}%
\begin{pgfscope}%
\pgfsetbuttcap%
\pgfsetroundjoin%
\definecolor{currentfill}{rgb}{0.000000,0.000000,0.000000}%
\pgfsetfillcolor{currentfill}%
\pgfsetlinewidth{0.803000pt}%
\definecolor{currentstroke}{rgb}{0.000000,0.000000,0.000000}%
\pgfsetstrokecolor{currentstroke}%
\pgfsetdash{}{0pt}%
\pgfsys@defobject{currentmarker}{\pgfqpoint{-0.048611in}{0.000000in}}{\pgfqpoint{-0.000000in}{0.000000in}}{%
\pgfpathmoveto{\pgfqpoint{-0.000000in}{0.000000in}}%
\pgfpathlineto{\pgfqpoint{-0.048611in}{0.000000in}}%
\pgfusepath{stroke,fill}%
}%
\begin{pgfscope}%
\pgfsys@transformshift{0.922838in}{4.092865in}%
\pgfsys@useobject{currentmarker}{}%
\end{pgfscope}%
\end{pgfscope}%
\begin{pgfscope}%
\definecolor{textcolor}{rgb}{0.000000,0.000000,0.000000}%
\pgfsetstrokecolor{textcolor}%
\pgfsetfillcolor{textcolor}%
\pgftext[x=0.546798in, y=4.022016in, left, base]{\color{textcolor}{\rmfamily\fontsize{15.000000}{18.000000}\selectfont\catcode`\^=\active\def^{\ifmmode\sp\else\^{}\fi}\catcode`\%=\active\def%{\%}$\mathdefault{10^{4}}$}}%
\end{pgfscope}%
\begin{pgfscope}%
\pgfsetbuttcap%
\pgfsetroundjoin%
\definecolor{currentfill}{rgb}{0.000000,0.000000,0.000000}%
\pgfsetfillcolor{currentfill}%
\pgfsetlinewidth{0.803000pt}%
\definecolor{currentstroke}{rgb}{0.000000,0.000000,0.000000}%
\pgfsetstrokecolor{currentstroke}%
\pgfsetdash{}{0pt}%
\pgfsys@defobject{currentmarker}{\pgfqpoint{-0.048611in}{0.000000in}}{\pgfqpoint{-0.000000in}{0.000000in}}{%
\pgfpathmoveto{\pgfqpoint{-0.000000in}{0.000000in}}%
\pgfpathlineto{\pgfqpoint{-0.048611in}{0.000000in}}%
\pgfusepath{stroke,fill}%
}%
\begin{pgfscope}%
\pgfsys@transformshift{0.922838in}{4.599963in}%
\pgfsys@useobject{currentmarker}{}%
\end{pgfscope}%
\end{pgfscope}%
\begin{pgfscope}%
\definecolor{textcolor}{rgb}{0.000000,0.000000,0.000000}%
\pgfsetstrokecolor{textcolor}%
\pgfsetfillcolor{textcolor}%
\pgftext[x=0.546798in, y=4.529114in, left, base]{\color{textcolor}{\rmfamily\fontsize{15.000000}{18.000000}\selectfont\catcode`\^=\active\def^{\ifmmode\sp\else\^{}\fi}\catcode`\%=\active\def%{\%}$\mathdefault{10^{9}}$}}%
\end{pgfscope}%
\begin{pgfscope}%
\pgfsetbuttcap%
\pgfsetroundjoin%
\definecolor{currentfill}{rgb}{0.000000,0.000000,0.000000}%
\pgfsetfillcolor{currentfill}%
\pgfsetlinewidth{0.803000pt}%
\definecolor{currentstroke}{rgb}{0.000000,0.000000,0.000000}%
\pgfsetstrokecolor{currentstroke}%
\pgfsetdash{}{0pt}%
\pgfsys@defobject{currentmarker}{\pgfqpoint{-0.048611in}{0.000000in}}{\pgfqpoint{-0.000000in}{0.000000in}}{%
\pgfpathmoveto{\pgfqpoint{-0.000000in}{0.000000in}}%
\pgfpathlineto{\pgfqpoint{-0.048611in}{0.000000in}}%
\pgfusepath{stroke,fill}%
}%
\begin{pgfscope}%
\pgfsys@transformshift{0.922838in}{5.107061in}%
\pgfsys@useobject{currentmarker}{}%
\end{pgfscope}%
\end{pgfscope}%
\begin{pgfscope}%
\definecolor{textcolor}{rgb}{0.000000,0.000000,0.000000}%
\pgfsetstrokecolor{textcolor}%
\pgfsetfillcolor{textcolor}%
\pgftext[x=0.470757in, y=5.036212in, left, base]{\color{textcolor}{\rmfamily\fontsize{15.000000}{18.000000}\selectfont\catcode`\^=\active\def^{\ifmmode\sp\else\^{}\fi}\catcode`\%=\active\def%{\%}$\mathdefault{10^{14}}$}}%
\end{pgfscope}%
\begin{pgfscope}%
\pgfsetbuttcap%
\pgfsetroundjoin%
\definecolor{currentfill}{rgb}{0.000000,0.000000,0.000000}%
\pgfsetfillcolor{currentfill}%
\pgfsetlinewidth{0.803000pt}%
\definecolor{currentstroke}{rgb}{0.000000,0.000000,0.000000}%
\pgfsetstrokecolor{currentstroke}%
\pgfsetdash{}{0pt}%
\pgfsys@defobject{currentmarker}{\pgfqpoint{-0.048611in}{0.000000in}}{\pgfqpoint{-0.000000in}{0.000000in}}{%
\pgfpathmoveto{\pgfqpoint{-0.000000in}{0.000000in}}%
\pgfpathlineto{\pgfqpoint{-0.048611in}{0.000000in}}%
\pgfusepath{stroke,fill}%
}%
\begin{pgfscope}%
\pgfsys@transformshift{0.922838in}{5.614159in}%
\pgfsys@useobject{currentmarker}{}%
\end{pgfscope}%
\end{pgfscope}%
\begin{pgfscope}%
\definecolor{textcolor}{rgb}{0.000000,0.000000,0.000000}%
\pgfsetstrokecolor{textcolor}%
\pgfsetfillcolor{textcolor}%
\pgftext[x=0.470757in, y=5.543309in, left, base]{\color{textcolor}{\rmfamily\fontsize{15.000000}{18.000000}\selectfont\catcode`\^=\active\def^{\ifmmode\sp\else\^{}\fi}\catcode`\%=\active\def%{\%}$\mathdefault{10^{19}}$}}%
\end{pgfscope}%
\begin{pgfscope}%
\definecolor{textcolor}{rgb}{0.000000,0.000000,0.000000}%
\pgfsetstrokecolor{textcolor}%
\pgfsetfillcolor{textcolor}%
\pgftext[x=0.372956in,y=4.802802in,,bottom,rotate=90.000000]{\color{textcolor}{\rmfamily\fontsize{16.000000}{19.200000}\selectfont\catcode`\^=\active\def^{\ifmmode\sp\else\^{}\fi}\catcode`\%=\active\def%{\%}$g^{(4)}(0)$}}%
\end{pgfscope}%
\begin{pgfscope}%
\pgfpathrectangle{\pgfqpoint{0.922838in}{3.585768in}}{\pgfqpoint{5.660427in}{2.434069in}}%
\pgfusepath{clip}%
\pgfsetrectcap%
\pgfsetroundjoin%
\pgfsetlinewidth{1.606000pt}%
\definecolor{currentstroke}{rgb}{1.000000,0.647059,0.000000}%
\pgfsetstrokecolor{currentstroke}%
\pgfsetdash{}{0pt}%
\pgfpathmoveto{\pgfqpoint{0.922838in}{4.546444in}}%
\pgfpathlineto{\pgfqpoint{0.934159in}{4.613280in}}%
\pgfpathlineto{\pgfqpoint{0.945479in}{4.617256in}}%
\pgfpathlineto{\pgfqpoint{0.956800in}{4.607287in}}%
\pgfpathlineto{\pgfqpoint{1.002084in}{4.556224in}}%
\pgfpathlineto{\pgfqpoint{1.036046in}{4.522759in}}%
\pgfpathlineto{\pgfqpoint{1.149255in}{4.415796in}}%
\pgfpathlineto{\pgfqpoint{1.183217in}{4.379746in}}%
\pgfpathlineto{\pgfqpoint{1.217180in}{4.339865in}}%
\pgfpathlineto{\pgfqpoint{1.239822in}{4.310434in}}%
\pgfpathlineto{\pgfqpoint{1.262463in}{4.277982in}}%
\pgfpathlineto{\pgfqpoint{1.285105in}{4.241711in}}%
\pgfpathlineto{\pgfqpoint{1.307747in}{4.200650in}}%
\pgfpathlineto{\pgfqpoint{1.330388in}{4.154408in}}%
\pgfpathlineto{\pgfqpoint{1.353030in}{4.103942in}}%
\pgfpathlineto{\pgfqpoint{1.364351in}{4.014980in}}%
\pgfpathlineto{\pgfqpoint{1.375672in}{3.949254in}}%
\pgfpathlineto{\pgfqpoint{1.386993in}{3.816045in}}%
\pgfpathlineto{\pgfqpoint{1.398314in}{3.981098in}}%
\pgfpathlineto{\pgfqpoint{1.454918in}{4.122501in}}%
\pgfpathlineto{\pgfqpoint{1.477560in}{4.166411in}}%
\pgfpathlineto{\pgfqpoint{1.500201in}{4.202489in}}%
\pgfpathlineto{\pgfqpoint{1.522843in}{4.232998in}}%
\pgfpathlineto{\pgfqpoint{1.545485in}{4.259446in}}%
\pgfpathlineto{\pgfqpoint{1.568126in}{4.282824in}}%
\pgfpathlineto{\pgfqpoint{1.590768in}{4.303784in}}%
\pgfpathlineto{\pgfqpoint{1.624731in}{4.331642in}}%
\pgfpathlineto{\pgfqpoint{1.658693in}{4.355885in}}%
\pgfpathlineto{\pgfqpoint{1.692656in}{4.377326in}}%
\pgfpathlineto{\pgfqpoint{1.737939in}{4.402867in}}%
\pgfpathlineto{\pgfqpoint{1.771902in}{4.422175in}}%
\pgfpathlineto{\pgfqpoint{1.783223in}{4.429864in}}%
\pgfpathlineto{\pgfqpoint{1.794543in}{4.439209in}}%
\pgfpathlineto{\pgfqpoint{1.805864in}{4.451666in}}%
\pgfpathlineto{\pgfqpoint{1.817185in}{4.469831in}}%
\pgfpathlineto{\pgfqpoint{1.828506in}{4.498621in}}%
\pgfpathlineto{\pgfqpoint{1.839827in}{4.549879in}}%
\pgfpathlineto{\pgfqpoint{1.851148in}{4.679266in}}%
\pgfpathlineto{\pgfqpoint{1.862469in}{4.604576in}}%
\pgfpathlineto{\pgfqpoint{1.873789in}{4.542862in}}%
\pgfpathlineto{\pgfqpoint{1.885110in}{4.515096in}}%
\pgfpathlineto{\pgfqpoint{1.896431in}{4.499633in}}%
\pgfpathlineto{\pgfqpoint{1.907752in}{4.490134in}}%
\pgfpathlineto{\pgfqpoint{1.919073in}{4.483795in}}%
\pgfpathlineto{\pgfqpoint{1.941715in}{4.475740in}}%
\pgfpathlineto{\pgfqpoint{2.043602in}{4.449047in}}%
\pgfpathlineto{\pgfqpoint{2.066244in}{4.440067in}}%
\pgfpathlineto{\pgfqpoint{2.088886in}{4.429111in}}%
\pgfpathlineto{\pgfqpoint{2.111527in}{4.415777in}}%
\pgfpathlineto{\pgfqpoint{2.134169in}{4.399601in}}%
\pgfpathlineto{\pgfqpoint{2.156811in}{4.379988in}}%
\pgfpathlineto{\pgfqpoint{2.179452in}{4.356155in}}%
\pgfpathlineto{\pgfqpoint{2.202094in}{4.326984in}}%
\pgfpathlineto{\pgfqpoint{2.224736in}{4.290840in}}%
\pgfpathlineto{\pgfqpoint{2.236057in}{4.269409in}}%
\pgfpathlineto{\pgfqpoint{2.247378in}{4.245169in}}%
\pgfpathlineto{\pgfqpoint{2.258698in}{4.217517in}}%
\pgfpathlineto{\pgfqpoint{2.270019in}{4.185622in}}%
\pgfpathlineto{\pgfqpoint{2.281340in}{4.148284in}}%
\pgfpathlineto{\pgfqpoint{2.292661in}{4.103680in}}%
\pgfpathlineto{\pgfqpoint{2.303982in}{4.049120in}}%
\pgfpathlineto{\pgfqpoint{2.315303in}{3.982391in}}%
\pgfpathlineto{\pgfqpoint{2.326624in}{3.817350in}}%
\pgfpathlineto{\pgfqpoint{2.337944in}{3.766364in}}%
\pgfpathlineto{\pgfqpoint{2.349265in}{3.942737in}}%
\pgfpathlineto{\pgfqpoint{2.360586in}{4.041641in}}%
\pgfpathlineto{\pgfqpoint{2.371907in}{4.112975in}}%
\pgfpathlineto{\pgfqpoint{2.383228in}{4.170329in}}%
\pgfpathlineto{\pgfqpoint{2.394549in}{4.218977in}}%
\pgfpathlineto{\pgfqpoint{2.405870in}{4.261502in}}%
\pgfpathlineto{\pgfqpoint{2.428511in}{4.333628in}}%
\pgfpathlineto{\pgfqpoint{2.451153in}{4.393646in}}%
\pgfpathlineto{\pgfqpoint{2.473795in}{4.445194in}}%
\pgfpathlineto{\pgfqpoint{2.496436in}{4.490475in}}%
\pgfpathlineto{\pgfqpoint{2.519078in}{4.530937in}}%
\pgfpathlineto{\pgfqpoint{2.541720in}{4.567577in}}%
\pgfpathlineto{\pgfqpoint{2.564361in}{4.601116in}}%
\pgfpathlineto{\pgfqpoint{2.598324in}{4.646741in}}%
\pgfpathlineto{\pgfqpoint{2.632287in}{4.687870in}}%
\pgfpathlineto{\pgfqpoint{2.666249in}{4.725423in}}%
\pgfpathlineto{\pgfqpoint{2.700212in}{4.760067in}}%
\pgfpathlineto{\pgfqpoint{2.745495in}{4.802627in}}%
\pgfpathlineto{\pgfqpoint{2.779458in}{4.833092in}}%
\pgfpathlineto{\pgfqpoint{2.802099in}{4.862146in}}%
\pgfpathlineto{\pgfqpoint{2.813420in}{4.861685in}}%
\pgfpathlineto{\pgfqpoint{2.847383in}{4.886689in}}%
\pgfpathlineto{\pgfqpoint{2.903987in}{4.927644in}}%
\pgfpathlineto{\pgfqpoint{2.949270in}{4.957907in}}%
\pgfpathlineto{\pgfqpoint{2.994554in}{4.985512in}}%
\pgfpathlineto{\pgfqpoint{3.028516in}{5.003864in}}%
\pgfpathlineto{\pgfqpoint{3.062479in}{5.019196in}}%
\pgfpathlineto{\pgfqpoint{3.085121in}{5.026915in}}%
\pgfpathlineto{\pgfqpoint{3.107762in}{5.031654in}}%
\pgfpathlineto{\pgfqpoint{3.119083in}{5.032506in}}%
\pgfpathlineto{\pgfqpoint{3.130404in}{5.032045in}}%
\pgfpathlineto{\pgfqpoint{3.141725in}{5.029974in}}%
\pgfpathlineto{\pgfqpoint{3.153046in}{5.025910in}}%
\pgfpathlineto{\pgfqpoint{3.164367in}{5.019360in}}%
\pgfpathlineto{\pgfqpoint{3.175688in}{5.009684in}}%
\pgfpathlineto{\pgfqpoint{3.187008in}{4.996029in}}%
\pgfpathlineto{\pgfqpoint{3.198329in}{4.977237in}}%
\pgfpathlineto{\pgfqpoint{3.209650in}{4.951676in}}%
\pgfpathlineto{\pgfqpoint{3.220971in}{4.916909in}}%
\pgfpathlineto{\pgfqpoint{3.232292in}{4.869000in}}%
\pgfpathlineto{\pgfqpoint{3.243613in}{4.800763in}}%
\pgfpathlineto{\pgfqpoint{3.254934in}{4.696358in}}%
\pgfpathlineto{\pgfqpoint{3.266254in}{4.506403in}}%
\pgfpathlineto{\pgfqpoint{3.277575in}{4.286074in}}%
\pgfpathlineto{\pgfqpoint{3.288896in}{4.611777in}}%
\pgfpathlineto{\pgfqpoint{3.300217in}{4.763446in}}%
\pgfpathlineto{\pgfqpoint{3.311538in}{4.857852in}}%
\pgfpathlineto{\pgfqpoint{3.322859in}{4.924455in}}%
\pgfpathlineto{\pgfqpoint{3.334179in}{4.974401in}}%
\pgfpathlineto{\pgfqpoint{3.345500in}{5.013195in}}%
\pgfpathlineto{\pgfqpoint{3.356821in}{5.044022in}}%
\pgfpathlineto{\pgfqpoint{3.368142in}{5.068925in}}%
\pgfpathlineto{\pgfqpoint{3.379463in}{5.089304in}}%
\pgfpathlineto{\pgfqpoint{3.390784in}{5.106170in}}%
\pgfpathlineto{\pgfqpoint{3.402105in}{5.120272in}}%
\pgfpathlineto{\pgfqpoint{3.413425in}{5.132178in}}%
\pgfpathlineto{\pgfqpoint{3.436067in}{5.151061in}}%
\pgfpathlineto{\pgfqpoint{3.458709in}{5.165310in}}%
\pgfpathlineto{\pgfqpoint{3.481351in}{5.176489in}}%
\pgfpathlineto{\pgfqpoint{3.515313in}{5.189614in}}%
\pgfpathlineto{\pgfqpoint{3.549276in}{5.200123in}}%
\pgfpathlineto{\pgfqpoint{3.605880in}{5.214788in}}%
\pgfpathlineto{\pgfqpoint{3.798334in}{5.261821in}}%
\pgfpathlineto{\pgfqpoint{3.866260in}{5.281668in}}%
\pgfpathlineto{\pgfqpoint{3.922864in}{5.300803in}}%
\pgfpathlineto{\pgfqpoint{3.968147in}{5.318474in}}%
\pgfpathlineto{\pgfqpoint{4.002110in}{5.333583in}}%
\pgfpathlineto{\pgfqpoint{4.036072in}{5.350804in}}%
\pgfpathlineto{\pgfqpoint{4.070035in}{5.370867in}}%
\pgfpathlineto{\pgfqpoint{4.092677in}{5.386361in}}%
\pgfpathlineto{\pgfqpoint{4.115318in}{5.404097in}}%
\pgfpathlineto{\pgfqpoint{4.137960in}{5.424663in}}%
\pgfpathlineto{\pgfqpoint{4.183243in}{5.470514in}}%
\pgfpathlineto{\pgfqpoint{4.194564in}{5.469917in}}%
\pgfpathlineto{\pgfqpoint{4.205885in}{5.423185in}}%
\pgfpathlineto{\pgfqpoint{4.217206in}{5.223510in}}%
\pgfpathlineto{\pgfqpoint{4.228527in}{5.395880in}}%
\pgfpathlineto{\pgfqpoint{4.239848in}{5.472583in}}%
\pgfpathlineto{\pgfqpoint{4.251169in}{5.481231in}}%
\pgfpathlineto{\pgfqpoint{4.262489in}{5.474516in}}%
\pgfpathlineto{\pgfqpoint{4.296452in}{5.444749in}}%
\pgfpathlineto{\pgfqpoint{4.319094in}{5.428155in}}%
\pgfpathlineto{\pgfqpoint{4.341735in}{5.414575in}}%
\pgfpathlineto{\pgfqpoint{4.364377in}{5.403382in}}%
\pgfpathlineto{\pgfqpoint{4.398340in}{5.389973in}}%
\pgfpathlineto{\pgfqpoint{4.432302in}{5.379597in}}%
\pgfpathlineto{\pgfqpoint{4.466265in}{5.371502in}}%
\pgfpathlineto{\pgfqpoint{4.511548in}{5.363434in}}%
\pgfpathlineto{\pgfqpoint{4.556832in}{5.357840in}}%
\pgfpathlineto{\pgfqpoint{4.602115in}{5.354270in}}%
\pgfpathlineto{\pgfqpoint{4.658719in}{5.352272in}}%
\pgfpathlineto{\pgfqpoint{4.715324in}{5.352799in}}%
\pgfpathlineto{\pgfqpoint{4.771928in}{5.355829in}}%
\pgfpathlineto{\pgfqpoint{4.817211in}{5.360210in}}%
\pgfpathlineto{\pgfqpoint{4.862495in}{5.366595in}}%
\pgfpathlineto{\pgfqpoint{4.907778in}{5.375436in}}%
\pgfpathlineto{\pgfqpoint{4.941741in}{5.384126in}}%
\pgfpathlineto{\pgfqpoint{4.975703in}{5.395140in}}%
\pgfpathlineto{\pgfqpoint{4.998345in}{5.404177in}}%
\pgfpathlineto{\pgfqpoint{5.020987in}{5.415002in}}%
\pgfpathlineto{\pgfqpoint{5.043628in}{5.428212in}}%
\pgfpathlineto{\pgfqpoint{5.066270in}{5.444775in}}%
\pgfpathlineto{\pgfqpoint{5.077591in}{5.454809in}}%
\pgfpathlineto{\pgfqpoint{5.088912in}{5.466436in}}%
\pgfpathlineto{\pgfqpoint{5.100233in}{5.480177in}}%
\pgfpathlineto{\pgfqpoint{5.111553in}{5.496845in}}%
\pgfpathlineto{\pgfqpoint{5.122874in}{5.517845in}}%
\pgfpathlineto{\pgfqpoint{5.134195in}{5.545865in}}%
\pgfpathlineto{\pgfqpoint{5.145516in}{5.586864in}}%
\pgfpathlineto{\pgfqpoint{5.156837in}{5.651909in}}%
\pgfpathlineto{\pgfqpoint{5.168158in}{5.645581in}}%
\pgfpathlineto{\pgfqpoint{5.179479in}{5.580817in}}%
\pgfpathlineto{\pgfqpoint{5.190799in}{5.539700in}}%
\pgfpathlineto{\pgfqpoint{5.202120in}{5.510913in}}%
\pgfpathlineto{\pgfqpoint{5.213441in}{5.488876in}}%
\pgfpathlineto{\pgfqpoint{5.224762in}{5.471045in}}%
\pgfpathlineto{\pgfqpoint{5.236083in}{5.456080in}}%
\pgfpathlineto{\pgfqpoint{5.258724in}{5.431894in}}%
\pgfpathlineto{\pgfqpoint{5.281366in}{5.412803in}}%
\pgfpathlineto{\pgfqpoint{5.304008in}{5.397117in}}%
\pgfpathlineto{\pgfqpoint{5.326650in}{5.383898in}}%
\pgfpathlineto{\pgfqpoint{5.360612in}{5.367519in}}%
\pgfpathlineto{\pgfqpoint{5.394575in}{5.354289in}}%
\pgfpathlineto{\pgfqpoint{5.428537in}{5.343467in}}%
\pgfpathlineto{\pgfqpoint{5.462500in}{5.334552in}}%
\pgfpathlineto{\pgfqpoint{5.507783in}{5.325297in}}%
\pgfpathlineto{\pgfqpoint{5.553067in}{5.318248in}}%
\pgfpathlineto{\pgfqpoint{5.609671in}{5.312181in}}%
\pgfpathlineto{\pgfqpoint{5.677596in}{5.307977in}}%
\pgfpathlineto{\pgfqpoint{5.745521in}{5.306699in}}%
\pgfpathlineto{\pgfqpoint{5.802125in}{5.307345in}}%
\pgfpathlineto{\pgfqpoint{5.881371in}{5.311734in}}%
\pgfpathlineto{\pgfqpoint{5.937976in}{5.318878in}}%
\pgfpathlineto{\pgfqpoint{5.960617in}{5.323573in}}%
\pgfpathlineto{\pgfqpoint{5.983259in}{5.329894in}}%
\pgfpathlineto{\pgfqpoint{6.005901in}{5.338991in}}%
\pgfpathlineto{\pgfqpoint{6.017222in}{5.344944in}}%
\pgfpathlineto{\pgfqpoint{6.028543in}{5.352468in}}%
\pgfpathlineto{\pgfqpoint{6.039863in}{5.362012in}}%
\pgfpathlineto{\pgfqpoint{6.051184in}{5.374132in}}%
\pgfpathlineto{\pgfqpoint{6.062505in}{5.390372in}}%
\pgfpathlineto{\pgfqpoint{6.073826in}{5.412750in}}%
\pgfpathlineto{\pgfqpoint{6.085147in}{5.445667in}}%
\pgfpathlineto{\pgfqpoint{6.096468in}{5.497789in}}%
\pgfpathlineto{\pgfqpoint{6.107788in}{5.532239in}}%
\pgfpathlineto{\pgfqpoint{6.119109in}{5.471465in}}%
\pgfpathlineto{\pgfqpoint{6.130430in}{5.426817in}}%
\pgfpathlineto{\pgfqpoint{6.141751in}{5.397604in}}%
\pgfpathlineto{\pgfqpoint{6.153072in}{5.377195in}}%
\pgfpathlineto{\pgfqpoint{6.164393in}{5.362445in}}%
\pgfpathlineto{\pgfqpoint{6.175714in}{5.351271in}}%
\pgfpathlineto{\pgfqpoint{6.187034in}{5.342859in}}%
\pgfpathlineto{\pgfqpoint{6.198355in}{5.336413in}}%
\pgfpathlineto{\pgfqpoint{6.220997in}{5.327694in}}%
\pgfpathlineto{\pgfqpoint{6.232318in}{5.324286in}}%
\pgfpathlineto{\pgfqpoint{6.254960in}{5.320520in}}%
\pgfpathlineto{\pgfqpoint{6.300243in}{5.315857in}}%
\pgfpathlineto{\pgfqpoint{6.402131in}{5.315480in}}%
\pgfpathlineto{\pgfqpoint{6.413452in}{5.316336in}}%
\pgfpathlineto{\pgfqpoint{6.424772in}{5.315875in}}%
\pgfpathlineto{\pgfqpoint{6.458735in}{5.317480in}}%
\pgfpathlineto{\pgfqpoint{6.549302in}{5.320121in}}%
\pgfpathlineto{\pgfqpoint{6.560623in}{5.320984in}}%
\pgfpathlineto{\pgfqpoint{6.571943in}{5.320535in}}%
\pgfpathlineto{\pgfqpoint{6.583264in}{5.321342in}}%
\pgfpathlineto{\pgfqpoint{6.583264in}{5.321342in}}%
\pgfusepath{stroke}%
\end{pgfscope}%
\begin{pgfscope}%
\pgfpathrectangle{\pgfqpoint{0.922838in}{3.585768in}}{\pgfqpoint{5.660427in}{2.434069in}}%
\pgfusepath{clip}%
\pgfsetbuttcap%
\pgfsetroundjoin%
\pgfsetlinewidth{0.903375pt}%
\definecolor{currentstroke}{rgb}{1.000000,0.647059,0.000000}%
\pgfsetstrokecolor{currentstroke}%
\pgfsetstrokeopacity{0.800000}%
\pgfsetdash{{3.330000pt}{1.440000pt}}{0.000000pt}%
\pgfpathmoveto{\pgfqpoint{5.168158in}{3.585768in}}%
\pgfpathlineto{\pgfqpoint{5.168158in}{5.645581in}}%
\pgfusepath{stroke}%
\end{pgfscope}%
\begin{pgfscope}%
\pgfsetrectcap%
\pgfsetmiterjoin%
\pgfsetlinewidth{0.803000pt}%
\definecolor{currentstroke}{rgb}{0.000000,0.000000,0.000000}%
\pgfsetstrokecolor{currentstroke}%
\pgfsetdash{}{0pt}%
\pgfpathmoveto{\pgfqpoint{0.922838in}{3.585768in}}%
\pgfpathlineto{\pgfqpoint{0.922838in}{6.019837in}}%
\pgfusepath{stroke}%
\end{pgfscope}%
\begin{pgfscope}%
\pgfsetrectcap%
\pgfsetmiterjoin%
\pgfsetlinewidth{0.803000pt}%
\definecolor{currentstroke}{rgb}{0.000000,0.000000,0.000000}%
\pgfsetstrokecolor{currentstroke}%
\pgfsetdash{}{0pt}%
\pgfpathmoveto{\pgfqpoint{6.583264in}{3.585768in}}%
\pgfpathlineto{\pgfqpoint{6.583264in}{6.019837in}}%
\pgfusepath{stroke}%
\end{pgfscope}%
\begin{pgfscope}%
\pgfsetrectcap%
\pgfsetmiterjoin%
\pgfsetlinewidth{0.803000pt}%
\definecolor{currentstroke}{rgb}{0.000000,0.000000,0.000000}%
\pgfsetstrokecolor{currentstroke}%
\pgfsetdash{}{0pt}%
\pgfpathmoveto{\pgfqpoint{0.922838in}{3.585768in}}%
\pgfpathlineto{\pgfqpoint{6.583264in}{3.585768in}}%
\pgfusepath{stroke}%
\end{pgfscope}%
\begin{pgfscope}%
\pgfsetrectcap%
\pgfsetmiterjoin%
\pgfsetlinewidth{0.803000pt}%
\definecolor{currentstroke}{rgb}{0.000000,0.000000,0.000000}%
\pgfsetstrokecolor{currentstroke}%
\pgfsetdash{}{0pt}%
\pgfpathmoveto{\pgfqpoint{0.922838in}{6.019837in}}%
\pgfpathlineto{\pgfqpoint{6.583264in}{6.019837in}}%
\pgfusepath{stroke}%
\end{pgfscope}%
\begin{pgfscope}%
\definecolor{textcolor}{rgb}{1.000000,0.647059,0.000000}%
\pgfsetstrokecolor{textcolor}%
\pgfsetfillcolor{textcolor}%
\pgftext[x=5.168158in,y=5.715578in,,bottom]{\color{textcolor}{\rmfamily\fontsize{16.000000}{19.200000}\selectfont\catcode`\^=\active\def^{\ifmmode\sp\else\^{}\fi}\catcode`\%=\active\def%{\%}$\Delta \approx -4.5\,\omega_b$}}%
\end{pgfscope}%
\begin{pgfscope}%
\definecolor{textcolor}{rgb}{0.000000,0.000000,0.000000}%
\pgfsetstrokecolor{textcolor}%
\pgfsetfillcolor{textcolor}%
\pgftext[x=0.979442in,y=5.898134in,left,top]{\color{textcolor}{\rmfamily\fontsize{16.000000}{19.200000}\selectfont\catcode`\^=\active\def^{\ifmmode\sp\else\^{}\fi}\catcode`\%=\active\def%{\%}$(c)$}}%
\end{pgfscope}%
\begin{pgfscope}%
\pgfsetbuttcap%
\pgfsetmiterjoin%
\definecolor{currentfill}{rgb}{1.000000,1.000000,1.000000}%
\pgfsetfillcolor{currentfill}%
\pgfsetlinewidth{0.000000pt}%
\definecolor{currentstroke}{rgb}{0.000000,0.000000,0.000000}%
\pgfsetstrokecolor{currentstroke}%
\pgfsetstrokeopacity{0.000000}%
\pgfsetdash{}{0pt}%
\pgfpathmoveto{\pgfqpoint{0.922838in}{0.776111in}}%
\pgfpathlineto{\pgfqpoint{6.583264in}{0.776111in}}%
\pgfpathlineto{\pgfqpoint{6.583264in}{3.210180in}}%
\pgfpathlineto{\pgfqpoint{0.922838in}{3.210180in}}%
\pgfpathlineto{\pgfqpoint{0.922838in}{0.776111in}}%
\pgfpathclose%
\pgfusepath{fill}%
\end{pgfscope}%
\begin{pgfscope}%
\pgfsetbuttcap%
\pgfsetroundjoin%
\definecolor{currentfill}{rgb}{0.000000,0.000000,0.000000}%
\pgfsetfillcolor{currentfill}%
\pgfsetlinewidth{0.803000pt}%
\definecolor{currentstroke}{rgb}{0.000000,0.000000,0.000000}%
\pgfsetstrokecolor{currentstroke}%
\pgfsetdash{}{0pt}%
\pgfsys@defobject{currentmarker}{\pgfqpoint{0.000000in}{-0.048611in}}{\pgfqpoint{0.000000in}{0.000000in}}{%
\pgfpathmoveto{\pgfqpoint{0.000000in}{0.000000in}}%
\pgfpathlineto{\pgfqpoint{0.000000in}{-0.048611in}}%
\pgfusepath{stroke,fill}%
}%
\begin{pgfscope}%
\pgfsys@transformshift{6.583264in}{0.776111in}%
\pgfsys@useobject{currentmarker}{}%
\end{pgfscope}%
\end{pgfscope}%
\begin{pgfscope}%
\definecolor{textcolor}{rgb}{0.000000,0.000000,0.000000}%
\pgfsetstrokecolor{textcolor}%
\pgfsetfillcolor{textcolor}%
\pgftext[x=6.583264in,y=0.678888in,,top]{\color{textcolor}{\rmfamily\fontsize{15.000000}{18.000000}\selectfont\catcode`\^=\active\def^{\ifmmode\sp\else\^{}\fi}\catcode`\%=\active\def%{\%}\ensuremath{-}6}}%
\end{pgfscope}%
\begin{pgfscope}%
\pgfsetbuttcap%
\pgfsetroundjoin%
\definecolor{currentfill}{rgb}{0.000000,0.000000,0.000000}%
\pgfsetfillcolor{currentfill}%
\pgfsetlinewidth{0.803000pt}%
\definecolor{currentstroke}{rgb}{0.000000,0.000000,0.000000}%
\pgfsetstrokecolor{currentstroke}%
\pgfsetdash{}{0pt}%
\pgfsys@defobject{currentmarker}{\pgfqpoint{0.000000in}{-0.048611in}}{\pgfqpoint{0.000000in}{0.000000in}}{%
\pgfpathmoveto{\pgfqpoint{0.000000in}{0.000000in}}%
\pgfpathlineto{\pgfqpoint{0.000000in}{-0.048611in}}%
\pgfusepath{stroke,fill}%
}%
\begin{pgfscope}%
\pgfsys@transformshift{5.639860in}{0.776111in}%
\pgfsys@useobject{currentmarker}{}%
\end{pgfscope}%
\end{pgfscope}%
\begin{pgfscope}%
\definecolor{textcolor}{rgb}{0.000000,0.000000,0.000000}%
\pgfsetstrokecolor{textcolor}%
\pgfsetfillcolor{textcolor}%
\pgftext[x=5.639860in,y=0.678888in,,top]{\color{textcolor}{\rmfamily\fontsize{15.000000}{18.000000}\selectfont\catcode`\^=\active\def^{\ifmmode\sp\else\^{}\fi}\catcode`\%=\active\def%{\%}\ensuremath{-}5}}%
\end{pgfscope}%
\begin{pgfscope}%
\pgfsetbuttcap%
\pgfsetroundjoin%
\definecolor{currentfill}{rgb}{0.000000,0.000000,0.000000}%
\pgfsetfillcolor{currentfill}%
\pgfsetlinewidth{0.803000pt}%
\definecolor{currentstroke}{rgb}{0.000000,0.000000,0.000000}%
\pgfsetstrokecolor{currentstroke}%
\pgfsetdash{}{0pt}%
\pgfsys@defobject{currentmarker}{\pgfqpoint{0.000000in}{-0.048611in}}{\pgfqpoint{0.000000in}{0.000000in}}{%
\pgfpathmoveto{\pgfqpoint{0.000000in}{0.000000in}}%
\pgfpathlineto{\pgfqpoint{0.000000in}{-0.048611in}}%
\pgfusepath{stroke,fill}%
}%
\begin{pgfscope}%
\pgfsys@transformshift{4.696455in}{0.776111in}%
\pgfsys@useobject{currentmarker}{}%
\end{pgfscope}%
\end{pgfscope}%
\begin{pgfscope}%
\definecolor{textcolor}{rgb}{0.000000,0.000000,0.000000}%
\pgfsetstrokecolor{textcolor}%
\pgfsetfillcolor{textcolor}%
\pgftext[x=4.696455in,y=0.678888in,,top]{\color{textcolor}{\rmfamily\fontsize{15.000000}{18.000000}\selectfont\catcode`\^=\active\def^{\ifmmode\sp\else\^{}\fi}\catcode`\%=\active\def%{\%}\ensuremath{-}4}}%
\end{pgfscope}%
\begin{pgfscope}%
\pgfsetbuttcap%
\pgfsetroundjoin%
\definecolor{currentfill}{rgb}{0.000000,0.000000,0.000000}%
\pgfsetfillcolor{currentfill}%
\pgfsetlinewidth{0.803000pt}%
\definecolor{currentstroke}{rgb}{0.000000,0.000000,0.000000}%
\pgfsetstrokecolor{currentstroke}%
\pgfsetdash{}{0pt}%
\pgfsys@defobject{currentmarker}{\pgfqpoint{0.000000in}{-0.048611in}}{\pgfqpoint{0.000000in}{0.000000in}}{%
\pgfpathmoveto{\pgfqpoint{0.000000in}{0.000000in}}%
\pgfpathlineto{\pgfqpoint{0.000000in}{-0.048611in}}%
\pgfusepath{stroke,fill}%
}%
\begin{pgfscope}%
\pgfsys@transformshift{3.753051in}{0.776111in}%
\pgfsys@useobject{currentmarker}{}%
\end{pgfscope}%
\end{pgfscope}%
\begin{pgfscope}%
\definecolor{textcolor}{rgb}{0.000000,0.000000,0.000000}%
\pgfsetstrokecolor{textcolor}%
\pgfsetfillcolor{textcolor}%
\pgftext[x=3.753051in,y=0.678888in,,top]{\color{textcolor}{\rmfamily\fontsize{15.000000}{18.000000}\selectfont\catcode`\^=\active\def^{\ifmmode\sp\else\^{}\fi}\catcode`\%=\active\def%{\%}\ensuremath{-}3}}%
\end{pgfscope}%
\begin{pgfscope}%
\pgfsetbuttcap%
\pgfsetroundjoin%
\definecolor{currentfill}{rgb}{0.000000,0.000000,0.000000}%
\pgfsetfillcolor{currentfill}%
\pgfsetlinewidth{0.803000pt}%
\definecolor{currentstroke}{rgb}{0.000000,0.000000,0.000000}%
\pgfsetstrokecolor{currentstroke}%
\pgfsetdash{}{0pt}%
\pgfsys@defobject{currentmarker}{\pgfqpoint{0.000000in}{-0.048611in}}{\pgfqpoint{0.000000in}{0.000000in}}{%
\pgfpathmoveto{\pgfqpoint{0.000000in}{0.000000in}}%
\pgfpathlineto{\pgfqpoint{0.000000in}{-0.048611in}}%
\pgfusepath{stroke,fill}%
}%
\begin{pgfscope}%
\pgfsys@transformshift{2.809647in}{0.776111in}%
\pgfsys@useobject{currentmarker}{}%
\end{pgfscope}%
\end{pgfscope}%
\begin{pgfscope}%
\definecolor{textcolor}{rgb}{0.000000,0.000000,0.000000}%
\pgfsetstrokecolor{textcolor}%
\pgfsetfillcolor{textcolor}%
\pgftext[x=2.809647in,y=0.678888in,,top]{\color{textcolor}{\rmfamily\fontsize{15.000000}{18.000000}\selectfont\catcode`\^=\active\def^{\ifmmode\sp\else\^{}\fi}\catcode`\%=\active\def%{\%}\ensuremath{-}2}}%
\end{pgfscope}%
\begin{pgfscope}%
\pgfsetbuttcap%
\pgfsetroundjoin%
\definecolor{currentfill}{rgb}{0.000000,0.000000,0.000000}%
\pgfsetfillcolor{currentfill}%
\pgfsetlinewidth{0.803000pt}%
\definecolor{currentstroke}{rgb}{0.000000,0.000000,0.000000}%
\pgfsetstrokecolor{currentstroke}%
\pgfsetdash{}{0pt}%
\pgfsys@defobject{currentmarker}{\pgfqpoint{0.000000in}{-0.048611in}}{\pgfqpoint{0.000000in}{0.000000in}}{%
\pgfpathmoveto{\pgfqpoint{0.000000in}{0.000000in}}%
\pgfpathlineto{\pgfqpoint{0.000000in}{-0.048611in}}%
\pgfusepath{stroke,fill}%
}%
\begin{pgfscope}%
\pgfsys@transformshift{1.866242in}{0.776111in}%
\pgfsys@useobject{currentmarker}{}%
\end{pgfscope}%
\end{pgfscope}%
\begin{pgfscope}%
\definecolor{textcolor}{rgb}{0.000000,0.000000,0.000000}%
\pgfsetstrokecolor{textcolor}%
\pgfsetfillcolor{textcolor}%
\pgftext[x=1.866242in,y=0.678888in,,top]{\color{textcolor}{\rmfamily\fontsize{15.000000}{18.000000}\selectfont\catcode`\^=\active\def^{\ifmmode\sp\else\^{}\fi}\catcode`\%=\active\def%{\%}\ensuremath{-}1}}%
\end{pgfscope}%
\begin{pgfscope}%
\pgfsetbuttcap%
\pgfsetroundjoin%
\definecolor{currentfill}{rgb}{0.000000,0.000000,0.000000}%
\pgfsetfillcolor{currentfill}%
\pgfsetlinewidth{0.803000pt}%
\definecolor{currentstroke}{rgb}{0.000000,0.000000,0.000000}%
\pgfsetstrokecolor{currentstroke}%
\pgfsetdash{}{0pt}%
\pgfsys@defobject{currentmarker}{\pgfqpoint{0.000000in}{-0.048611in}}{\pgfqpoint{0.000000in}{0.000000in}}{%
\pgfpathmoveto{\pgfqpoint{0.000000in}{0.000000in}}%
\pgfpathlineto{\pgfqpoint{0.000000in}{-0.048611in}}%
\pgfusepath{stroke,fill}%
}%
\begin{pgfscope}%
\pgfsys@transformshift{0.922838in}{0.776111in}%
\pgfsys@useobject{currentmarker}{}%
\end{pgfscope}%
\end{pgfscope}%
\begin{pgfscope}%
\definecolor{textcolor}{rgb}{0.000000,0.000000,0.000000}%
\pgfsetstrokecolor{textcolor}%
\pgfsetfillcolor{textcolor}%
\pgftext[x=0.922838in,y=0.678888in,,top]{\color{textcolor}{\rmfamily\fontsize{15.000000}{18.000000}\selectfont\catcode`\^=\active\def^{\ifmmode\sp\else\^{}\fi}\catcode`\%=\active\def%{\%}0}}%
\end{pgfscope}%
\begin{pgfscope}%
\definecolor{textcolor}{rgb}{0.000000,0.000000,0.000000}%
\pgfsetstrokecolor{textcolor}%
\pgfsetfillcolor{textcolor}%
\pgftext[x=3.753051in,y=0.445556in,,top]{\color{textcolor}{\rmfamily\fontsize{24.000000}{28.800000}\selectfont\catcode`\^=\active\def^{\ifmmode\sp\else\^{}\fi}\catcode`\%=\active\def%{\%}$\Delta/\omega_b$}}%
\end{pgfscope}%
\begin{pgfscope}%
\pgfsetbuttcap%
\pgfsetroundjoin%
\definecolor{currentfill}{rgb}{0.000000,0.000000,0.000000}%
\pgfsetfillcolor{currentfill}%
\pgfsetlinewidth{0.803000pt}%
\definecolor{currentstroke}{rgb}{0.000000,0.000000,0.000000}%
\pgfsetstrokecolor{currentstroke}%
\pgfsetdash{}{0pt}%
\pgfsys@defobject{currentmarker}{\pgfqpoint{-0.048611in}{0.000000in}}{\pgfqpoint{-0.000000in}{0.000000in}}{%
\pgfpathmoveto{\pgfqpoint{-0.000000in}{0.000000in}}%
\pgfpathlineto{\pgfqpoint{-0.048611in}{0.000000in}}%
\pgfusepath{stroke,fill}%
}%
\begin{pgfscope}%
\pgfsys@transformshift{0.922838in}{0.776111in}%
\pgfsys@useobject{currentmarker}{}%
\end{pgfscope}%
\end{pgfscope}%
\begin{pgfscope}%
\definecolor{textcolor}{rgb}{0.000000,0.000000,0.000000}%
\pgfsetstrokecolor{textcolor}%
\pgfsetfillcolor{textcolor}%
\pgftext[x=0.428511in, y=0.705261in, left, base]{\color{textcolor}{\rmfamily\fontsize{15.000000}{18.000000}\selectfont\catcode`\^=\active\def^{\ifmmode\sp\else\^{}\fi}\catcode`\%=\active\def%{\%}$\mathdefault{10^{-1}}$}}%
\end{pgfscope}%
\begin{pgfscope}%
\pgfsetbuttcap%
\pgfsetroundjoin%
\definecolor{currentfill}{rgb}{0.000000,0.000000,0.000000}%
\pgfsetfillcolor{currentfill}%
\pgfsetlinewidth{0.803000pt}%
\definecolor{currentstroke}{rgb}{0.000000,0.000000,0.000000}%
\pgfsetstrokecolor{currentstroke}%
\pgfsetdash{}{0pt}%
\pgfsys@defobject{currentmarker}{\pgfqpoint{-0.048611in}{0.000000in}}{\pgfqpoint{-0.000000in}{0.000000in}}{%
\pgfpathmoveto{\pgfqpoint{-0.000000in}{0.000000in}}%
\pgfpathlineto{\pgfqpoint{-0.048611in}{0.000000in}}%
\pgfusepath{stroke,fill}%
}%
\begin{pgfscope}%
\pgfsys@transformshift{0.922838in}{1.262925in}%
\pgfsys@useobject{currentmarker}{}%
\end{pgfscope}%
\end{pgfscope}%
\begin{pgfscope}%
\definecolor{textcolor}{rgb}{0.000000,0.000000,0.000000}%
\pgfsetstrokecolor{textcolor}%
\pgfsetfillcolor{textcolor}%
\pgftext[x=0.546798in, y=1.192075in, left, base]{\color{textcolor}{\rmfamily\fontsize{15.000000}{18.000000}\selectfont\catcode`\^=\active\def^{\ifmmode\sp\else\^{}\fi}\catcode`\%=\active\def%{\%}$\mathdefault{10^{5}}$}}%
\end{pgfscope}%
\begin{pgfscope}%
\pgfsetbuttcap%
\pgfsetroundjoin%
\definecolor{currentfill}{rgb}{0.000000,0.000000,0.000000}%
\pgfsetfillcolor{currentfill}%
\pgfsetlinewidth{0.803000pt}%
\definecolor{currentstroke}{rgb}{0.000000,0.000000,0.000000}%
\pgfsetstrokecolor{currentstroke}%
\pgfsetdash{}{0pt}%
\pgfsys@defobject{currentmarker}{\pgfqpoint{-0.048611in}{0.000000in}}{\pgfqpoint{-0.000000in}{0.000000in}}{%
\pgfpathmoveto{\pgfqpoint{-0.000000in}{0.000000in}}%
\pgfpathlineto{\pgfqpoint{-0.048611in}{0.000000in}}%
\pgfusepath{stroke,fill}%
}%
\begin{pgfscope}%
\pgfsys@transformshift{0.922838in}{1.749738in}%
\pgfsys@useobject{currentmarker}{}%
\end{pgfscope}%
\end{pgfscope}%
\begin{pgfscope}%
\definecolor{textcolor}{rgb}{0.000000,0.000000,0.000000}%
\pgfsetstrokecolor{textcolor}%
\pgfsetfillcolor{textcolor}%
\pgftext[x=0.470757in, y=1.678889in, left, base]{\color{textcolor}{\rmfamily\fontsize{15.000000}{18.000000}\selectfont\catcode`\^=\active\def^{\ifmmode\sp\else\^{}\fi}\catcode`\%=\active\def%{\%}$\mathdefault{10^{11}}$}}%
\end{pgfscope}%
\begin{pgfscope}%
\pgfsetbuttcap%
\pgfsetroundjoin%
\definecolor{currentfill}{rgb}{0.000000,0.000000,0.000000}%
\pgfsetfillcolor{currentfill}%
\pgfsetlinewidth{0.803000pt}%
\definecolor{currentstroke}{rgb}{0.000000,0.000000,0.000000}%
\pgfsetstrokecolor{currentstroke}%
\pgfsetdash{}{0pt}%
\pgfsys@defobject{currentmarker}{\pgfqpoint{-0.048611in}{0.000000in}}{\pgfqpoint{-0.000000in}{0.000000in}}{%
\pgfpathmoveto{\pgfqpoint{-0.000000in}{0.000000in}}%
\pgfpathlineto{\pgfqpoint{-0.048611in}{0.000000in}}%
\pgfusepath{stroke,fill}%
}%
\begin{pgfscope}%
\pgfsys@transformshift{0.922838in}{2.236552in}%
\pgfsys@useobject{currentmarker}{}%
\end{pgfscope}%
\end{pgfscope}%
\begin{pgfscope}%
\definecolor{textcolor}{rgb}{0.000000,0.000000,0.000000}%
\pgfsetstrokecolor{textcolor}%
\pgfsetfillcolor{textcolor}%
\pgftext[x=0.470757in, y=2.165703in, left, base]{\color{textcolor}{\rmfamily\fontsize{15.000000}{18.000000}\selectfont\catcode`\^=\active\def^{\ifmmode\sp\else\^{}\fi}\catcode`\%=\active\def%{\%}$\mathdefault{10^{17}}$}}%
\end{pgfscope}%
\begin{pgfscope}%
\pgfsetbuttcap%
\pgfsetroundjoin%
\definecolor{currentfill}{rgb}{0.000000,0.000000,0.000000}%
\pgfsetfillcolor{currentfill}%
\pgfsetlinewidth{0.803000pt}%
\definecolor{currentstroke}{rgb}{0.000000,0.000000,0.000000}%
\pgfsetstrokecolor{currentstroke}%
\pgfsetdash{}{0pt}%
\pgfsys@defobject{currentmarker}{\pgfqpoint{-0.048611in}{0.000000in}}{\pgfqpoint{-0.000000in}{0.000000in}}{%
\pgfpathmoveto{\pgfqpoint{-0.000000in}{0.000000in}}%
\pgfpathlineto{\pgfqpoint{-0.048611in}{0.000000in}}%
\pgfusepath{stroke,fill}%
}%
\begin{pgfscope}%
\pgfsys@transformshift{0.922838in}{2.723366in}%
\pgfsys@useobject{currentmarker}{}%
\end{pgfscope}%
\end{pgfscope}%
\begin{pgfscope}%
\definecolor{textcolor}{rgb}{0.000000,0.000000,0.000000}%
\pgfsetstrokecolor{textcolor}%
\pgfsetfillcolor{textcolor}%
\pgftext[x=0.470757in, y=2.652517in, left, base]{\color{textcolor}{\rmfamily\fontsize{15.000000}{18.000000}\selectfont\catcode`\^=\active\def^{\ifmmode\sp\else\^{}\fi}\catcode`\%=\active\def%{\%}$\mathdefault{10^{23}}$}}%
\end{pgfscope}%
\begin{pgfscope}%
\pgfsetbuttcap%
\pgfsetroundjoin%
\definecolor{currentfill}{rgb}{0.000000,0.000000,0.000000}%
\pgfsetfillcolor{currentfill}%
\pgfsetlinewidth{0.803000pt}%
\definecolor{currentstroke}{rgb}{0.000000,0.000000,0.000000}%
\pgfsetstrokecolor{currentstroke}%
\pgfsetdash{}{0pt}%
\pgfsys@defobject{currentmarker}{\pgfqpoint{-0.048611in}{0.000000in}}{\pgfqpoint{-0.000000in}{0.000000in}}{%
\pgfpathmoveto{\pgfqpoint{-0.000000in}{0.000000in}}%
\pgfpathlineto{\pgfqpoint{-0.048611in}{0.000000in}}%
\pgfusepath{stroke,fill}%
}%
\begin{pgfscope}%
\pgfsys@transformshift{0.922838in}{3.210180in}%
\pgfsys@useobject{currentmarker}{}%
\end{pgfscope}%
\end{pgfscope}%
\begin{pgfscope}%
\definecolor{textcolor}{rgb}{0.000000,0.000000,0.000000}%
\pgfsetstrokecolor{textcolor}%
\pgfsetfillcolor{textcolor}%
\pgftext[x=0.470757in, y=3.139331in, left, base]{\color{textcolor}{\rmfamily\fontsize{15.000000}{18.000000}\selectfont\catcode`\^=\active\def^{\ifmmode\sp\else\^{}\fi}\catcode`\%=\active\def%{\%}$\mathdefault{10^{29}}$}}%
\end{pgfscope}%
\begin{pgfscope}%
\definecolor{textcolor}{rgb}{0.000000,0.000000,0.000000}%
\pgfsetstrokecolor{textcolor}%
\pgfsetfillcolor{textcolor}%
\pgftext[x=0.372956in,y=1.993145in,,bottom,rotate=90.000000]{\color{textcolor}{\rmfamily\fontsize{16.000000}{19.200000}\selectfont\catcode`\^=\active\def^{\ifmmode\sp\else\^{}\fi}\catcode`\%=\active\def%{\%}$g^{(5)}(0)$}}%
\end{pgfscope}%
\begin{pgfscope}%
\pgfpathrectangle{\pgfqpoint{0.922838in}{0.776111in}}{\pgfqpoint{5.660427in}{2.434069in}}%
\pgfusepath{clip}%
\pgfsetrectcap%
\pgfsetroundjoin%
\pgfsetlinewidth{1.606000pt}%
\definecolor{currentstroke}{rgb}{1.000000,0.000000,0.000000}%
\pgfsetstrokecolor{currentstroke}%
\pgfsetdash{}{0pt}%
\pgfpathmoveto{\pgfqpoint{0.922838in}{1.782201in}}%
\pgfpathlineto{\pgfqpoint{0.934159in}{1.859483in}}%
\pgfpathlineto{\pgfqpoint{0.945479in}{1.868825in}}%
\pgfpathlineto{\pgfqpoint{0.956800in}{1.862014in}}%
\pgfpathlineto{\pgfqpoint{0.979442in}{1.838757in}}%
\pgfpathlineto{\pgfqpoint{1.013405in}{1.803541in}}%
\pgfpathlineto{\pgfqpoint{1.070009in}{1.749563in}}%
\pgfpathlineto{\pgfqpoint{1.126613in}{1.694949in}}%
\pgfpathlineto{\pgfqpoint{1.160576in}{1.659159in}}%
\pgfpathlineto{\pgfqpoint{1.183217in}{1.633682in}}%
\pgfpathlineto{\pgfqpoint{1.217180in}{1.590993in}}%
\pgfpathlineto{\pgfqpoint{1.239822in}{1.559032in}}%
\pgfpathlineto{\pgfqpoint{1.262463in}{1.523146in}}%
\pgfpathlineto{\pgfqpoint{1.285105in}{1.482345in}}%
\pgfpathlineto{\pgfqpoint{1.307747in}{1.435184in}}%
\pgfpathlineto{\pgfqpoint{1.330388in}{1.380664in}}%
\pgfpathlineto{\pgfqpoint{1.341709in}{1.350946in}}%
\pgfpathlineto{\pgfqpoint{1.353030in}{1.317923in}}%
\pgfpathlineto{\pgfqpoint{1.364351in}{1.212849in}}%
\pgfpathlineto{\pgfqpoint{1.375672in}{1.141011in}}%
\pgfpathlineto{\pgfqpoint{1.386993in}{0.994917in}}%
\pgfpathlineto{\pgfqpoint{1.398314in}{1.181606in}}%
\pgfpathlineto{\pgfqpoint{1.409634in}{1.222062in}}%
\pgfpathlineto{\pgfqpoint{1.443597in}{1.327283in}}%
\pgfpathlineto{\pgfqpoint{1.454918in}{1.357161in}}%
\pgfpathlineto{\pgfqpoint{1.466239in}{1.383745in}}%
\pgfpathlineto{\pgfqpoint{1.488880in}{1.428687in}}%
\pgfpathlineto{\pgfqpoint{1.511522in}{1.465167in}}%
\pgfpathlineto{\pgfqpoint{1.534164in}{1.495508in}}%
\pgfpathlineto{\pgfqpoint{1.556806in}{1.521147in}}%
\pgfpathlineto{\pgfqpoint{1.579447in}{1.543148in}}%
\pgfpathlineto{\pgfqpoint{1.602089in}{1.562257in}}%
\pgfpathlineto{\pgfqpoint{1.624731in}{1.579244in}}%
\pgfpathlineto{\pgfqpoint{1.658693in}{1.600492in}}%
\pgfpathlineto{\pgfqpoint{1.692656in}{1.619190in}}%
\pgfpathlineto{\pgfqpoint{1.726618in}{1.637296in}}%
\pgfpathlineto{\pgfqpoint{1.760581in}{1.659512in}}%
\pgfpathlineto{\pgfqpoint{1.771902in}{1.669223in}}%
\pgfpathlineto{\pgfqpoint{1.783223in}{1.681129in}}%
\pgfpathlineto{\pgfqpoint{1.794543in}{1.695732in}}%
\pgfpathlineto{\pgfqpoint{1.805864in}{1.714208in}}%
\pgfpathlineto{\pgfqpoint{1.817185in}{1.737847in}}%
\pgfpathlineto{\pgfqpoint{1.828506in}{1.769136in}}%
\pgfpathlineto{\pgfqpoint{1.839827in}{1.816281in}}%
\pgfpathlineto{\pgfqpoint{1.851148in}{1.922206in}}%
\pgfpathlineto{\pgfqpoint{1.862469in}{1.863112in}}%
\pgfpathlineto{\pgfqpoint{1.873789in}{1.811788in}}%
\pgfpathlineto{\pgfqpoint{1.885110in}{1.786461in}}%
\pgfpathlineto{\pgfqpoint{1.896431in}{1.770544in}}%
\pgfpathlineto{\pgfqpoint{1.907752in}{1.759496in}}%
\pgfpathlineto{\pgfqpoint{1.919073in}{1.751116in}}%
\pgfpathlineto{\pgfqpoint{1.941715in}{1.738728in}}%
\pgfpathlineto{\pgfqpoint{1.975677in}{1.725137in}}%
\pgfpathlineto{\pgfqpoint{2.032281in}{1.703809in}}%
\pgfpathlineto{\pgfqpoint{2.066244in}{1.688331in}}%
\pgfpathlineto{\pgfqpoint{2.088886in}{1.676084in}}%
\pgfpathlineto{\pgfqpoint{2.111527in}{1.661793in}}%
\pgfpathlineto{\pgfqpoint{2.134169in}{1.644995in}}%
\pgfpathlineto{\pgfqpoint{2.156811in}{1.625102in}}%
\pgfpathlineto{\pgfqpoint{2.179452in}{1.601402in}}%
\pgfpathlineto{\pgfqpoint{2.202094in}{1.572796in}}%
\pgfpathlineto{\pgfqpoint{2.224736in}{1.537620in}}%
\pgfpathlineto{\pgfqpoint{2.236057in}{1.516772in}}%
\pgfpathlineto{\pgfqpoint{2.247378in}{1.493101in}}%
\pgfpathlineto{\pgfqpoint{2.258698in}{1.465878in}}%
\pgfpathlineto{\pgfqpoint{2.270019in}{1.434039in}}%
\pgfpathlineto{\pgfqpoint{2.281340in}{1.395969in}}%
\pgfpathlineto{\pgfqpoint{2.292661in}{1.349093in}}%
\pgfpathlineto{\pgfqpoint{2.303982in}{1.289228in}}%
\pgfpathlineto{\pgfqpoint{2.315303in}{1.208111in}}%
\pgfpathlineto{\pgfqpoint{2.326624in}{1.016332in}}%
\pgfpathlineto{\pgfqpoint{2.337944in}{0.975418in}}%
\pgfpathlineto{\pgfqpoint{2.349265in}{1.169965in}}%
\pgfpathlineto{\pgfqpoint{2.360586in}{1.277167in}}%
\pgfpathlineto{\pgfqpoint{2.371907in}{1.351165in}}%
\pgfpathlineto{\pgfqpoint{2.383228in}{1.408123in}}%
\pgfpathlineto{\pgfqpoint{2.394549in}{1.454838in}}%
\pgfpathlineto{\pgfqpoint{2.405870in}{1.494778in}}%
\pgfpathlineto{\pgfqpoint{2.428511in}{1.561553in}}%
\pgfpathlineto{\pgfqpoint{2.451153in}{1.617154in}}%
\pgfpathlineto{\pgfqpoint{2.473795in}{1.665532in}}%
\pgfpathlineto{\pgfqpoint{2.496436in}{1.708741in}}%
\pgfpathlineto{\pgfqpoint{2.519078in}{1.747976in}}%
\pgfpathlineto{\pgfqpoint{2.541720in}{1.784003in}}%
\pgfpathlineto{\pgfqpoint{2.575682in}{1.833133in}}%
\pgfpathlineto{\pgfqpoint{2.609645in}{1.877473in}}%
\pgfpathlineto{\pgfqpoint{2.643607in}{1.917932in}}%
\pgfpathlineto{\pgfqpoint{2.677570in}{1.955190in}}%
\pgfpathlineto{\pgfqpoint{2.722853in}{2.001139in}}%
\pgfpathlineto{\pgfqpoint{2.756816in}{2.035507in}}%
\pgfpathlineto{\pgfqpoint{2.768137in}{2.049552in}}%
\pgfpathlineto{\pgfqpoint{2.779458in}{2.071088in}}%
\pgfpathlineto{\pgfqpoint{2.790779in}{2.125188in}}%
\pgfpathlineto{\pgfqpoint{2.802099in}{2.155757in}}%
\pgfpathlineto{\pgfqpoint{2.813420in}{2.110637in}}%
\pgfpathlineto{\pgfqpoint{2.824741in}{2.104464in}}%
\pgfpathlineto{\pgfqpoint{2.836062in}{2.107429in}}%
\pgfpathlineto{\pgfqpoint{2.858704in}{2.120192in}}%
\pgfpathlineto{\pgfqpoint{2.971912in}{2.192895in}}%
\pgfpathlineto{\pgfqpoint{3.005875in}{2.211703in}}%
\pgfpathlineto{\pgfqpoint{3.039837in}{2.227938in}}%
\pgfpathlineto{\pgfqpoint{3.062479in}{2.236743in}}%
\pgfpathlineto{\pgfqpoint{3.085121in}{2.243263in}}%
\pgfpathlineto{\pgfqpoint{3.107762in}{2.246577in}}%
\pgfpathlineto{\pgfqpoint{3.119083in}{2.246620in}}%
\pgfpathlineto{\pgfqpoint{3.130404in}{2.245275in}}%
\pgfpathlineto{\pgfqpoint{3.141725in}{2.242237in}}%
\pgfpathlineto{\pgfqpoint{3.153046in}{2.237116in}}%
\pgfpathlineto{\pgfqpoint{3.164367in}{2.229411in}}%
\pgfpathlineto{\pgfqpoint{3.175688in}{2.218470in}}%
\pgfpathlineto{\pgfqpoint{3.187008in}{2.203442in}}%
\pgfpathlineto{\pgfqpoint{3.198329in}{2.183173in}}%
\pgfpathlineto{\pgfqpoint{3.209650in}{2.156067in}}%
\pgfpathlineto{\pgfqpoint{3.220971in}{2.119802in}}%
\pgfpathlineto{\pgfqpoint{3.232292in}{2.070771in}}%
\pgfpathlineto{\pgfqpoint{3.243613in}{2.002802in}}%
\pgfpathlineto{\pgfqpoint{3.254934in}{1.903659in}}%
\pgfpathlineto{\pgfqpoint{3.266254in}{1.732360in}}%
\pgfpathlineto{\pgfqpoint{3.277575in}{1.483007in}}%
\pgfpathlineto{\pgfqpoint{3.288896in}{1.810717in}}%
\pgfpathlineto{\pgfqpoint{3.300217in}{1.960582in}}%
\pgfpathlineto{\pgfqpoint{3.311538in}{2.056262in}}%
\pgfpathlineto{\pgfqpoint{3.322859in}{2.125254in}}%
\pgfpathlineto{\pgfqpoint{3.334179in}{2.177801in}}%
\pgfpathlineto{\pgfqpoint{3.345500in}{2.219118in}}%
\pgfpathlineto{\pgfqpoint{3.356821in}{2.252309in}}%
\pgfpathlineto{\pgfqpoint{3.368142in}{2.279401in}}%
\pgfpathlineto{\pgfqpoint{3.379463in}{2.301806in}}%
\pgfpathlineto{\pgfqpoint{3.390784in}{2.320548in}}%
\pgfpathlineto{\pgfqpoint{3.402105in}{2.336395in}}%
\pgfpathlineto{\pgfqpoint{3.413425in}{2.349928in}}%
\pgfpathlineto{\pgfqpoint{3.436067in}{2.371770in}}%
\pgfpathlineto{\pgfqpoint{3.458709in}{2.388649in}}%
\pgfpathlineto{\pgfqpoint{3.481351in}{2.402195in}}%
\pgfpathlineto{\pgfqpoint{3.503992in}{2.413463in}}%
\pgfpathlineto{\pgfqpoint{3.537955in}{2.427565in}}%
\pgfpathlineto{\pgfqpoint{3.583238in}{2.443260in}}%
\pgfpathlineto{\pgfqpoint{3.651163in}{2.463715in}}%
\pgfpathlineto{\pgfqpoint{3.832297in}{2.516683in}}%
\pgfpathlineto{\pgfqpoint{3.900222in}{2.539262in}}%
\pgfpathlineto{\pgfqpoint{3.956826in}{2.560431in}}%
\pgfpathlineto{\pgfqpoint{4.002110in}{2.579613in}}%
\pgfpathlineto{\pgfqpoint{4.036072in}{2.595834in}}%
\pgfpathlineto{\pgfqpoint{4.070035in}{2.614222in}}%
\pgfpathlineto{\pgfqpoint{4.103997in}{2.635597in}}%
\pgfpathlineto{\pgfqpoint{4.126639in}{2.652038in}}%
\pgfpathlineto{\pgfqpoint{4.171923in}{2.689075in}}%
\pgfpathlineto{\pgfqpoint{4.183243in}{2.694849in}}%
\pgfpathlineto{\pgfqpoint{4.194564in}{2.688221in}}%
\pgfpathlineto{\pgfqpoint{4.205885in}{2.631671in}}%
\pgfpathlineto{\pgfqpoint{4.217206in}{2.414499in}}%
\pgfpathlineto{\pgfqpoint{4.228527in}{2.601687in}}%
\pgfpathlineto{\pgfqpoint{4.239848in}{2.691009in}}%
\pgfpathlineto{\pgfqpoint{4.251169in}{2.707497in}}%
\pgfpathlineto{\pgfqpoint{4.262489in}{2.706507in}}%
\pgfpathlineto{\pgfqpoint{4.285131in}{2.694639in}}%
\pgfpathlineto{\pgfqpoint{4.307773in}{2.682576in}}%
\pgfpathlineto{\pgfqpoint{4.330415in}{2.672577in}}%
\pgfpathlineto{\pgfqpoint{4.353056in}{2.664523in}}%
\pgfpathlineto{\pgfqpoint{4.387019in}{2.655319in}}%
\pgfpathlineto{\pgfqpoint{4.420981in}{2.648718in}}%
\pgfpathlineto{\pgfqpoint{4.454944in}{2.644070in}}%
\pgfpathlineto{\pgfqpoint{4.500227in}{2.640193in}}%
\pgfpathlineto{\pgfqpoint{4.545511in}{2.638425in}}%
\pgfpathlineto{\pgfqpoint{4.602115in}{2.638596in}}%
\pgfpathlineto{\pgfqpoint{4.658719in}{2.641051in}}%
\pgfpathlineto{\pgfqpoint{4.715324in}{2.645628in}}%
\pgfpathlineto{\pgfqpoint{4.771928in}{2.652327in}}%
\pgfpathlineto{\pgfqpoint{4.828532in}{2.661365in}}%
\pgfpathlineto{\pgfqpoint{4.873815in}{2.670578in}}%
\pgfpathlineto{\pgfqpoint{4.919099in}{2.681967in}}%
\pgfpathlineto{\pgfqpoint{4.953061in}{2.692356in}}%
\pgfpathlineto{\pgfqpoint{4.987024in}{2.704835in}}%
\pgfpathlineto{\pgfqpoint{5.020987in}{2.720213in}}%
\pgfpathlineto{\pgfqpoint{5.043628in}{2.732762in}}%
\pgfpathlineto{\pgfqpoint{5.066270in}{2.747997in}}%
\pgfpathlineto{\pgfqpoint{5.088912in}{2.767309in}}%
\pgfpathlineto{\pgfqpoint{5.100233in}{2.779288in}}%
\pgfpathlineto{\pgfqpoint{5.111553in}{2.793605in}}%
\pgfpathlineto{\pgfqpoint{5.122874in}{2.811375in}}%
\pgfpathlineto{\pgfqpoint{5.134195in}{2.834736in}}%
\pgfpathlineto{\pgfqpoint{5.145516in}{2.868387in}}%
\pgfpathlineto{\pgfqpoint{5.156837in}{2.920631in}}%
\pgfpathlineto{\pgfqpoint{5.168158in}{2.916674in}}%
\pgfpathlineto{\pgfqpoint{5.179479in}{2.866468in}}%
\pgfpathlineto{\pgfqpoint{5.190799in}{2.834630in}}%
\pgfpathlineto{\pgfqpoint{5.202120in}{2.812570in}}%
\pgfpathlineto{\pgfqpoint{5.213441in}{2.795879in}}%
\pgfpathlineto{\pgfqpoint{5.224762in}{2.782533in}}%
\pgfpathlineto{\pgfqpoint{5.236083in}{2.771465in}}%
\pgfpathlineto{\pgfqpoint{5.258724in}{2.753870in}}%
\pgfpathlineto{\pgfqpoint{5.281366in}{2.740283in}}%
\pgfpathlineto{\pgfqpoint{5.304008in}{2.729345in}}%
\pgfpathlineto{\pgfqpoint{5.337970in}{2.716350in}}%
\pgfpathlineto{\pgfqpoint{5.371933in}{2.706215in}}%
\pgfpathlineto{\pgfqpoint{5.405896in}{2.698192in}}%
\pgfpathlineto{\pgfqpoint{5.451179in}{2.689877in}}%
\pgfpathlineto{\pgfqpoint{5.496462in}{2.683778in}}%
\pgfpathlineto{\pgfqpoint{5.553067in}{2.678543in}}%
\pgfpathlineto{\pgfqpoint{5.620992in}{2.675348in}}%
\pgfpathlineto{\pgfqpoint{5.666275in}{2.674852in}}%
\pgfpathlineto{\pgfqpoint{5.711559in}{2.675647in}}%
\pgfpathlineto{\pgfqpoint{5.779484in}{2.679611in}}%
\pgfpathlineto{\pgfqpoint{5.813446in}{2.683058in}}%
\pgfpathlineto{\pgfqpoint{5.847409in}{2.687400in}}%
\pgfpathlineto{\pgfqpoint{5.892692in}{2.695836in}}%
\pgfpathlineto{\pgfqpoint{5.926655in}{2.704296in}}%
\pgfpathlineto{\pgfqpoint{5.949297in}{2.711321in}}%
\pgfpathlineto{\pgfqpoint{5.971938in}{2.720035in}}%
\pgfpathlineto{\pgfqpoint{5.994580in}{2.730832in}}%
\pgfpathlineto{\pgfqpoint{6.017222in}{2.744667in}}%
\pgfpathlineto{\pgfqpoint{6.028543in}{2.753235in}}%
\pgfpathlineto{\pgfqpoint{6.039863in}{2.763343in}}%
\pgfpathlineto{\pgfqpoint{6.051184in}{2.775498in}}%
\pgfpathlineto{\pgfqpoint{6.062505in}{2.790656in}}%
\pgfpathlineto{\pgfqpoint{6.073826in}{2.810437in}}%
\pgfpathlineto{\pgfqpoint{6.085147in}{2.838208in}}%
\pgfpathlineto{\pgfqpoint{6.096468in}{2.880763in}}%
\pgfpathlineto{\pgfqpoint{6.107788in}{2.908499in}}%
\pgfpathlineto{\pgfqpoint{6.119109in}{2.859306in}}%
\pgfpathlineto{\pgfqpoint{6.130430in}{2.822211in}}%
\pgfpathlineto{\pgfqpoint{6.141751in}{2.796744in}}%
\pgfpathlineto{\pgfqpoint{6.153072in}{2.777604in}}%
\pgfpathlineto{\pgfqpoint{6.164393in}{2.762356in}}%
\pgfpathlineto{\pgfqpoint{6.175714in}{2.749624in}}%
\pgfpathlineto{\pgfqpoint{6.198355in}{2.729329in}}%
\pgfpathlineto{\pgfqpoint{6.220997in}{2.713551in}}%
\pgfpathlineto{\pgfqpoint{6.243639in}{2.700584in}}%
\pgfpathlineto{\pgfqpoint{6.277601in}{2.685120in}}%
\pgfpathlineto{\pgfqpoint{6.311564in}{2.673215in}}%
\pgfpathlineto{\pgfqpoint{6.334206in}{2.666795in}}%
\pgfpathlineto{\pgfqpoint{6.379489in}{2.656550in}}%
\pgfpathlineto{\pgfqpoint{6.402131in}{2.652227in}}%
\pgfpathlineto{\pgfqpoint{6.413452in}{2.651971in}}%
\pgfpathlineto{\pgfqpoint{6.424772in}{2.648691in}}%
\pgfpathlineto{\pgfqpoint{6.436093in}{2.648688in}}%
\pgfpathlineto{\pgfqpoint{6.447414in}{2.646209in}}%
\pgfpathlineto{\pgfqpoint{6.458735in}{2.646384in}}%
\pgfpathlineto{\pgfqpoint{6.481377in}{2.642562in}}%
\pgfpathlineto{\pgfqpoint{6.492697in}{2.643193in}}%
\pgfpathlineto{\pgfqpoint{6.515339in}{2.639699in}}%
\pgfpathlineto{\pgfqpoint{6.526660in}{2.638662in}}%
\pgfpathlineto{\pgfqpoint{6.537981in}{2.640373in}}%
\pgfpathlineto{\pgfqpoint{6.549302in}{2.639544in}}%
\pgfpathlineto{\pgfqpoint{6.560623in}{2.640656in}}%
\pgfpathlineto{\pgfqpoint{6.571943in}{2.637297in}}%
\pgfpathlineto{\pgfqpoint{6.583264in}{2.637821in}}%
\pgfpathlineto{\pgfqpoint{6.583264in}{2.637821in}}%
\pgfusepath{stroke}%
\end{pgfscope}%
\begin{pgfscope}%
\pgfpathrectangle{\pgfqpoint{0.922838in}{0.776111in}}{\pgfqpoint{5.660427in}{2.434069in}}%
\pgfusepath{clip}%
\pgfsetbuttcap%
\pgfsetroundjoin%
\pgfsetlinewidth{0.903375pt}%
\definecolor{currentstroke}{rgb}{1.000000,0.000000,0.000000}%
\pgfsetstrokecolor{currentstroke}%
\pgfsetstrokeopacity{0.800000}%
\pgfsetdash{{3.330000pt}{1.440000pt}}{0.000000pt}%
\pgfpathmoveto{\pgfqpoint{6.111562in}{0.776111in}}%
\pgfpathlineto{\pgfqpoint{6.111562in}{2.908499in}}%
\pgfusepath{stroke}%
\end{pgfscope}%
\begin{pgfscope}%
\pgfsetrectcap%
\pgfsetmiterjoin%
\pgfsetlinewidth{0.803000pt}%
\definecolor{currentstroke}{rgb}{0.000000,0.000000,0.000000}%
\pgfsetstrokecolor{currentstroke}%
\pgfsetdash{}{0pt}%
\pgfpathmoveto{\pgfqpoint{0.922838in}{0.776111in}}%
\pgfpathlineto{\pgfqpoint{0.922838in}{3.210180in}}%
\pgfusepath{stroke}%
\end{pgfscope}%
\begin{pgfscope}%
\pgfsetrectcap%
\pgfsetmiterjoin%
\pgfsetlinewidth{0.803000pt}%
\definecolor{currentstroke}{rgb}{0.000000,0.000000,0.000000}%
\pgfsetstrokecolor{currentstroke}%
\pgfsetdash{}{0pt}%
\pgfpathmoveto{\pgfqpoint{6.583264in}{0.776111in}}%
\pgfpathlineto{\pgfqpoint{6.583264in}{3.210180in}}%
\pgfusepath{stroke}%
\end{pgfscope}%
\begin{pgfscope}%
\pgfsetrectcap%
\pgfsetmiterjoin%
\pgfsetlinewidth{0.803000pt}%
\definecolor{currentstroke}{rgb}{0.000000,0.000000,0.000000}%
\pgfsetstrokecolor{currentstroke}%
\pgfsetdash{}{0pt}%
\pgfpathmoveto{\pgfqpoint{0.922838in}{0.776111in}}%
\pgfpathlineto{\pgfqpoint{6.583264in}{0.776111in}}%
\pgfusepath{stroke}%
\end{pgfscope}%
\begin{pgfscope}%
\pgfsetrectcap%
\pgfsetmiterjoin%
\pgfsetlinewidth{0.803000pt}%
\definecolor{currentstroke}{rgb}{0.000000,0.000000,0.000000}%
\pgfsetstrokecolor{currentstroke}%
\pgfsetdash{}{0pt}%
\pgfpathmoveto{\pgfqpoint{0.922838in}{3.210180in}}%
\pgfpathlineto{\pgfqpoint{6.583264in}{3.210180in}}%
\pgfusepath{stroke}%
\end{pgfscope}%
\begin{pgfscope}%
\definecolor{textcolor}{rgb}{1.000000,0.000000,0.000000}%
\pgfsetstrokecolor{textcolor}%
\pgfsetfillcolor{textcolor}%
\pgftext[x=5.828541in,y=2.934486in,,bottom]{\color{textcolor}{\rmfamily\fontsize{16.000000}{19.200000}\selectfont\catcode`\^=\active\def^{\ifmmode\sp\else\^{}\fi}\catcode`\%=\active\def%{\%}$\Delta \approx -5.5\,\omega_b$}}%
\end{pgfscope}%
\begin{pgfscope}%
\definecolor{textcolor}{rgb}{0.000000,0.000000,0.000000}%
\pgfsetstrokecolor{textcolor}%
\pgfsetfillcolor{textcolor}%
\pgftext[x=0.979442in,y=3.088477in,left,top]{\color{textcolor}{\rmfamily\fontsize{16.000000}{19.200000}\selectfont\catcode`\^=\active\def^{\ifmmode\sp\else\^{}\fi}\catcode`\%=\active\def%{\%}$(d)$}}%
\end{pgfscope}%
\end{pgfpicture}%
\makeatother%
\endgroup%

	}
	\caption{
		Dinámica coherente tipo Rabi para la emisión de paquetes de $n$ fonones
		en una molécula excitónica acoplada a un modo acústico común.}
	\label{fig:rabi_all_regimes}
\end{figure}
\include{Kap4/Kap4}
\begin{appendix}
\chapter{Derivación de las tasas de transición de n-fonones}\label{AnexoA}

Consideramos dos puntos cuánticos (QDs) idénticos acoplados a un modo fonónico común.
En el laboratorio ($\hbar=1$), el Hamiltoniano dependiente del tiempo es
\begin{equation}
H_{\mathrm{lab}}(t)= \omega_b\, b^\dagger b
+ \omega_\sigma\sum_{j=1}^{2}\sigma_j^\dagger\sigma_j
+ \lambda\sum_{j=1}^{2}\sigma_j^\dagger\sigma_j\,(b^\dagger + b)
+ \Omega\!\left(e^{i\omega_L t}\sum_{j=1}^{2}\sigma_j
+ e^{-i\omega_L t}\sum_{j=1}^{2}\sigma_j^\dagger\right)
+ J\!\left(\sigma_1^\dagger\sigma_2 + \sigma_2^\dagger\sigma_1\right).
\label{eq:H_lab}
\end{equation}
donde $b$ ($b^\dagger$) aniquila (crea) fonones a frecuencia $\omega_b$, $\sigma_j=\ket{\vv_j}\bra{\cc_j}$ es el operador de bajada del QD $j$ con frecuencia de transición $\omega_\sigma$, $\lambda$ es el acoplamiento electrón--fonón, $\Omega$ es la amplitud del bombeo (frecuencia de Rabi), y $J$ es el acoplamiento de F\"orster.

\noindent
En el marco rotante a la frecuencia del láser $\omega_L$ (y tras aplicar la aproximación de onda rotante al bombeo), se obtiene un Hamiltoniano efectivo independiente del tiempo:
\begin{equation}
H = \omega_b\, b^\dagger b
+ \Delta\sum_{j=1}^{2}\sigma_j^\dagger\sigma_j
+ \lambda\sum_{j=1}^{2}\sigma_j^\dagger\sigma_j\,(b^\dagger + b)
+ \Omega\sum_{j=1}^{2}(\sigma_j + \sigma_j^\dagger)
+ J\!\left(\sigma_1^\dagger\sigma_2 + \sigma_2^\dagger\sigma_1\right),
\label{eq:H_rot}
\end{equation}
donde $\Delta=\omega_\sigma-\omega_L$ es la desintonía del láser. La base de estados producto es $\ket{m,s_1s_2}$, con $m\in\mathbb{N}_0$ el número de fonones y $s_j\in\{\vv,\cc\}$ el estado electrónico del QD $j$.

\subsection{Régimen de bajo bombeo y acoplamiento débil}

El término de F\"orster acopla los estados electrónicos de una excitación $\ket{\cc\vv}$ y $\ket{\vv\cc}$.
Definimos los estados simétrico y antisimétrico:
\begin{align}
\ket{S} &= \frac{1}{\sqrt{2}}\!\left(\ket{\cc\vv} + \ket{\vv\cc}\right), \label{eq:S_def}\\[2pt]
\ket{A} &= \frac{1}{\sqrt{2}}\!\left(\ket{\cc\vv} - \ket{\vv\cc}\right). \label{eq:A_def}
\end{align}
En este subespacio, las energías electrónicas son
\begin{align}
E_S &= \Delta + J, \label{eq:E_S}\\
E_A &= \Delta - J. \label{eq:E_A}
\end{align}
y para los estados compuestos fonón--electrón se tiene
\begin{align}
E(\ket{m,S}) &= m\omega_b + \Delta + J, \label{eq:E_mS}\\
E(\ket{m,A}) &= m\omega_b + \Delta - J, \label{eq:E_mA}\\
E(\ket{m,\vv\vv}) &= m\omega_b. \label{eq:E_mvv}
\end{align}

El operador de bombeo $H_{\mathrm{drive}}=\Omega(\sigma_1+\sigma_2+\sigma_1^\dagger+\sigma_2^\dagger)$ cumple
\begin{equation}
(\sigma_1^\dagger+\sigma_2^\dagger)\ket{\vv\vv}
=\ket{\cc\vv}+\ket{\vv\cc}
=\sqrt{2}\,\ket{S}.
\label{eq:drive_vv}
\end{equation}
Por lo tanto, para cualquier $m$,
\begin{align}
\mel{m,S}{H_{\mathrm{drive}}}{m,\vv\vv} &= \sqrt{2}\,\Omega, \label{eq:mel_drive_S}\\
\mel{m,A}{H_{\mathrm{drive}}}{m,\vv\vv} &= 0. \label{eq:mel_drive_A}
\end{align}
La ecuación \eqref{eq:mel_drive_A} muestra que el estado antisimétrico $\ket{A}$ es oscuro respecto al bombeo cuando el drive es perfectamente simétrico.

La interacción se escribe como $H_{\mathrm{int}}=\lambda\,\hat n_{\mathrm{exc}}(b^\dagger+b)$ con
$\hat n_{\mathrm{exc}}=\sigma_1^\dagger\sigma_1+\sigma_2^\dagger\sigma_2$.
En el sector de una excitación, $\hat n_{\mathrm{exc}}\ket{S}=\ket{S}$ y $\hat n_{\mathrm{exc}}\ket{A}=\ket{A}$, de modo que
\begin{align}
\mel{k,S}{H_{\mathrm{int}}}{k-1,S} &= \lambda\sqrt{k}, \label{eq:mel_int_SS}\\
\mel{k,A}{H_{\mathrm{int}}}{k-1,A} &= \lambda\sqrt{k}, \label{eq:mel_int_AA}\\
\mel{k,S}{H_{\mathrm{int}}}{k-1,A} &= 0. \label{eq:mel_int_SA}
\end{align}
Por \eqref{eq:mel_drive_A} y \eqref{eq:mel_int_SA}, el subespacio antisimétrico permanece desacoplado de
$\ket{0,\vv\vv}$ bajo las simetrías asumidas (QDs idénticos y bombeo en fase).

Suponemos el régimen $\Omega,\lambda,J\ll \omega_b$. En estas condiciones, el proceso relevante conecta
\[
\ket{0,\vv\vv}\ \longleftrightarrow\ \ket{n,S},
\]
y la condición de resonancia (degeneración en el marco rotante) es
\begin{equation}
E(\ket{n,S})=n\omega_b+\Delta+J=0=E(\ket{0,vv}),
\label{eq:resonancia_cond}
\end{equation}
de donde
\begin{equation}
\boxed{\Delta=\omega_\sigma-\omega_L=-n\omega_b-J,.}
\label{eq:resonancia_2QD}
\end{equation}

\medskip
\noindent
Para obtener una tasa de transición efectiva (acoplamiento coherente efectivo) en el subespacio resonante
$\{\ket{0,\vv\vv},\ket{n,S}\}$, consideramos una eliminación adiabática de estados intermedios en un truncamiento mínimo.
Introducimos el bloque de cuatro estados
\[
\{\ket{0,\vv\vv},\ \ket{n,S},\ \ket{n-1,S},\ \ket{n,\vv\vv}\},
\]
donde los elementos de matriz dominantes son:
\begin{itemize}
\item acoplamiento efectivo $\Omega_{\mathrm{eff}}^{(n-1)}$ entre $\ket{0,vv}$ y $\ket{n-1,S}$;
\item acoplamiento electrón--fonón $\sqrt{n}\,\lambda$ entre $\ket{n-1,S}$ y $\ket{n,S}$;
\item bombeo $\sqrt{2}\,\Omega$ entre $\ket{n,\vv\vv}$ y $\ket{n,S}$.
\end{itemize}
En esta base, el Hamiltoniano reducido es
\begin{equation}
H^{(n)}=
\begin{pmatrix}
0 & 0 & \Omega_{\mathrm{eff}}^{(n-1)} & 0 \\[6pt]
0 & \Delta+n\omega_b+J & \sqrt{n}\,\lambda & \sqrt{2}\,\Omega \\[6pt]
\Omega_{\mathrm{eff}}^{(n-1)} & \sqrt{n}\,\lambda & \Delta+(n-1)\omega_b+J & 0 \\[6pt]
0 & \sqrt{2}\,\Omega & 0 & n\omega_b
\end{pmatrix}.
\label{eq:Hn_4x4}
\end{equation}

\medskip
\noindent
Particionamos $H^{(n)}$ en bloques como
\begin{equation}
H^{(n)}=
\begin{pmatrix}
\mathcal{H}^{(n)} & \mathcal{X}^{(n)}\\[4pt]
\mathcal{X}^{(n)\mathsf{T}} & \mathcal{R}^{(n)}
\end{pmatrix},
\label{eq:Hn_bloques}
\end{equation}
donde el subespacio resonante es $\{\ket{0,\vv\vv},\ket{n,S}\}$ y el subespacio intermedio es $\{\ket{n-1,S},\ket{n,\vv\vv}\}$.
En resonancia \eqref{eq:resonancia_2QD}, ambos estados resonantes tienen energía cero, de modo que
\begin{equation}
\mathcal{H}^{(n)}=
\begin{pmatrix}
0 & 0\\
0 & 0
\end{pmatrix}.
\label{eq:Hn_sub}
\end{equation}
Los bloques $\mathcal{X}^{(n)}$ y $\mathcal{R}^{(n)}$ (aquí de tamaño $2\times 2$) son
\begin{equation}
\mathcal{X}^{(n)}=
\begin{pmatrix}
\Omega_{\mathrm{eff}}^{(n-1)} & 0 \\[4pt]
\sqrt{n}\,\lambda & \sqrt{2}\,\Omega
\end{pmatrix},
\qquad
\mathcal{R}^{(n)}=
\begin{pmatrix}
\Delta+(n-1)\omega_b+J & 0 \\[4pt]
0 & n\omega_b
\end{pmatrix}.
\label{eq:Xn_Rn}
\end{equation}

Aplicando la partición de Löwdin (a segundo orden en los acoplamientos entre subespacios),
el Hamiltoniano efectivo en el subespacio resonante es
\begin{equation}
\mathcal{H}_{\mathrm{eff}}^{(n)}=\mathcal{H}^{(n)}+\mathcal{X}^{(n)}\left(-\mathcal{R}^{(n)}\right)^{-1}\mathcal{X}^{(n)\mathsf{T}}.
\label{eq:Lowdin}
\end{equation}
Como $\mathcal{R}^{(n)}$ es diagonal,
\begin{equation}
\left(-\mathcal{R}^{(n)}\right)^{-1}=
\begin{pmatrix}
\dfrac{-1}{\Delta+(n-1)\omega_b+J} & 0 \\[10pt]
0 & \dfrac{-1}{n\omega_b}
\end{pmatrix}.
\label{eq:Rn_inv}
\end{equation}


\medskip
\noindent
El acoplamiento efectivo buscado es el elemento fuera de la diagonal
\begin{equation}
\Omega_{\mathrm{eff}}^{(n)}\equiv \mathcal{H}*{\mathrm{eff}}^{(n)}(1,2)=\mathcal{H}*{\mathrm{eff}}^{(n)}(2,1).
\end{equation}
Usando \eqref{eq:Lowdin}--\eqref{eq:Rn_inv}, se obtiene directamente
\begin{equation}
\boxed{
\Oeff^{(n)}
=-
\frac{\sqrt{n}\,\lambda}{\Delta+(n-1)\ob+J}\,\Oeff^{(n-1)}\;.
}
\label{eq:recurrencia}
\end{equation}

Para $n=1$ consideramos el truncamiento $\{\ket{0,\vv\vv},\ket{1,S},\ket{0,S},\ket{1,\vv\vv}\}$.
El bloque intermedio relevante para conectar $\ket{0,\vv\vv}\leftrightarrow\ket{1,S}$ incluye
$\ket{0,S}$ (energía $\Delta+J$) y $\ket{1,vv}$ (energía $\omega_b$), con acoplamientos $\sqrt{2},\Omega$ y $\lambda$.
Aplicando \eqref{eq:Lowdin}, se obtiene
\begin{equation}
\boxed{
\Oeff^{(1)}=
-\frac{\sqrt{2}\,\Omega\,\lambda}{\Delta+J}.
}
\label{eq:Oeff1}
\end{equation}

\medskip
\noindent
Iterando \eqref{eq:recurrencia} desde $n$ hasta $1$ y sustituyendo \eqref{eq:Oeff1}, resulta
\begin{equation}
\Omega_{\mathrm{eff}}^{(n)}
=
(-1)^n,
\frac{\sqrt{2},\Omega,\sqrt{n!},\lambda^n}{\displaystyle\prod_{k=0}^{n-1}\big(\Delta+k\omega_b+J\big)}.
\label{eq:Oeff_general}
\end{equation}
En la resonancia de $n$ fonones \eqref{eq:resonancia_2QD}, cada factor del denominador evalúa como
\begin{equation}
\Delta+k\omega_b+J=-n\omega_b-J+k\omega_b+J=-(n-k)\omega_b,
\qquad k=0,1,\ldots,n-1,
\end{equation}
y por tanto
\begin{equation}
\prod_{k=0}^{n-1}\big(\Delta+k\omega_b+J\big)=(-1)^n,\omega_b^n,n!.
\label{eq:prod_denom}
\end{equation}
Sustituyendo \eqref{eq:prod_denom} en \eqref{eq:Oeff_general} se obtiene finalmente
\begin{equation}
\boxed{
\Omega_{\mathrm{eff}}^{(n)}=
\sqrt{2}\left(\frac{\lambda}{\omega_b}\right)^n\frac{\Omega}{\sqrt{n!}}.
}
\label{eq:Oeff_final}
\end{equation}

\bigskip
\noindent
Las oscilaciones super-Rabi ocurren a la tasa $\Omega_{\mathrm{eff}}^{(n)}$ entre los estados $\ket{0,vv}$ y $\ket{n,S}=(\ket{n,cv}+\ket{n,vc})/\sqrt{2}$ bajo la condición de resonancia $\Delta=-n\omega_b-J$. El factor $\sqrt{2}$ refleja la interferencia constructiva asociada al bombeo simétrico de dos emisores. El estado antisimétrico $\ket{n,A}=(\ket{n,cv}-\ket{n,vc})/\sqrt{2}$ permanece oscuro bajo las simetrías asumidas; su condición resonante $\Delta=-n\omega_b+J$ no se manifiesta en la dinámica iniciada desde $\ket{0,vv}$ con bombeo simétrico.



\end{appendix}
\addcontentsline{toc}{chapter}{\numberline{}Bibliograf\'{\i}a}
\bibliographystyle{unsrt}
\bibliography{Bibli}
\end{document}