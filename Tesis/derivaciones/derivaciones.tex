% ============================================================
%  Derivación paso a paso: Omega_eff^(n) para la molécula
%  excitónica (2 QDs acoplados por interacción de Förster)
%  Régimen I: conducción débil y acoplamiento débil
% ============================================================
\documentclass[12pt]{article}
\usepackage{amsmath, amssymb}
\usepackage[margin=2.5cm]{geometry}

\begin{document}
	\section{Primer Régimen}
	% -----------------------------------------------------------
	% BLOQUE 0: Hamiltoniano en el marco del laboratorio
	% -----------------------------------------------------------
	
	Consideramos dos puntos cu\'anticos id\'enticos, QD$_1$ y QD$_2$, cada uno
	modelado como un sistema de dos niveles con estados electr\'onicos $|v\rangle$
	(valencia) y $|c\rangle$ (conducci\'on), frecuencia de transici\'on $\omega_\sigma$
	y acoplamiento electr\'on-fon\'on $\lambda$. Ambos QDs interact\'uan con un
	\'unico modo fon\'onico de cavidad de frecuencia $\omega_b$, con operadores de
	escalera $b$ y $b^\dagger$. Los QDs est\'an acoplados entre s\'i mediante la
	interacci\'on de F\"orster con constante $J$. Un l\'aser coherente de frecuencia
	$\omega_L$ conduce ambos QDs con la misma frecuencia de Rabi $\Omega$.
	Establecemos $\hbar = 1$ en todo lo que sigue.
	
	El Hamiltoniano total del sistema en el marco del laboratorio es
	
	\begin{equation}
		H_{\mathrm{lab}}
		= \omega_b b^\dagger b
		+ \omega_\sigma \left(\sigma_1^\dagger \sigma_1 + \sigma_2^\dagger \sigma_2\right)
		+ \lambda \left(\sigma_1^\dagger \sigma_1 + \sigma_2^\dagger \sigma_2\right)(b^\dagger + b)
		+ J\left(\sigma_1^\dagger \sigma_2 + \sigma_2^\dagger \sigma_1\right)
		+ \Omega \sum_{i=1,2} \left(e^{i\omega_L t}\,\sigma_i + e^{-i\omega_L t}\,\sigma_i^\dagger\right),
		\label{eq:H_lab}
	\end{equation}
	
	donde $\sigma_i = |v\rangle_i\langle c|$ y $\sigma_i^\dagger = |c\rangle_i\langle v|$
	son los operadores de decaimiento y excitaci\'on del QD$_i$, respectivamente.
	El primer t\'ermino de~\eqref{eq:H_lab} describe la energ\'ia libre del modo
	fon\'onico. El segundo describe la energ\'ia de transici\'on de ambos QDs. El
	tercero es el acoplamiento electr\'on-fon\'on, que en el modelo de Holstein
	desplaza el equilibrio del oscilador seg\'un el estado electr\'onico de cada QD.
	El cuarto es la interacci\'on de F\"orster, que transfiere la excitaci\'on
	entre QDs sin cambiar el n\'umero de fonones. El quinto es la conducci\'on
	coherente por el l\'aser externo en la aproximaci\'on de onda rotante (RWA).
	
	% -----------------------------------------------------------
	% BLOQUE 1: Transformación al marco rotante
	% -----------------------------------------------------------
	
	Para eliminar la dependencia expl\'icita en el tiempo del t\'ermino de
	conducci\'on, transformamos al marco rotante con la frecuencia del l\'aser
	$\omega_L$. Definimos el operador unitario
	
	\begin{equation}
		U(t) = \exp\!\left(i\omega_L t \left(\sigma_1^\dagger\sigma_1
		+ \sigma_2^\dagger\sigma_2\right)\right).
		\label{eq:U}
	\end{equation}
	
	El Hamiltoniano en el nuevo marco se obtiene mediante
	
	\begin{equation}
		H = U\,H_{\mathrm{lab}}\,U^\dagger + i\,\dot{U}\,U^\dagger.
		\label{eq:H_rotante_def}
	\end{equation}
	
	Calculamos el efecto de $U$ sobre cada t\'ermino de~\eqref{eq:H_lab}.
	
	\medskip
	\noindent
	El operador $U$ act\'ua \'unicamente sobre los grados de libertad electr\'onicos
	a trav\'es del operador de n\'umero $\hat{n}_e = \sigma_1^\dagger\sigma_1 +
	\sigma_2^\dagger\sigma_2$. Para cualquier operador $A$ que conmuta con
	$\hat{n}_e$, se tiene $U A U^\dagger = A$. En cambio, para los operadores de
	excitaci\'on y decaimiento se tiene:
	
	\begin{align}
		U\,\sigma_i^\dagger\,U^\dagger &= e^{+i\omega_L t}\,\sigma_i^\dagger,
		\label{eq:transf_sigmadag}\\
		U\,\sigma_i\,U^\dagger &= e^{-i\omega_L t}\,\sigma_i.
		\label{eq:transf_sigma}
	\end{align}
	
	Esto se verifica directamente usando la identidad de Baker-Campbell-Hausdorff y
	el \'algebra $[\hat{n}_e, \sigma_i^\dagger] = \sigma_i^\dagger$,
	$[\hat{n}_e, \sigma_i] = -\sigma_i$.
	
	\medskip
	\noindent
	Con estas relaciones, calculamos la transformaci\'on de cada t\'ermino:
	
	\medskip
	\noindent
	\textit{(i) T\'ermino fon\'onico.}
	$b$ y $b^\dagger$ conmutan con $\hat{n}_e$, por lo que
	
	\begin{equation}
		U\,\omega_b b^\dagger b\,U^\dagger = \omega_b b^\dagger b.
	\end{equation}
	
	\noindent
	\textit{(ii) T\'ermino electr\'onico libre.}
	$\sigma_i^\dagger\sigma_i$ conmuta con $\hat{n}_e$, por lo que
	
	\begin{equation}
		U\,\omega_\sigma\!\left(\sigma_1^\dagger\sigma_1 +
		\sigma_2^\dagger\sigma_2\right)U^\dagger
		= \omega_\sigma\!\left(\sigma_1^\dagger\sigma_1 +
		\sigma_2^\dagger\sigma_2\right).
	\end{equation}
	
	\noindent
	\textit{(iii) T\'ermino de acoplamiento electr\'on-fon\'on.}
	$\sigma_i^\dagger\sigma_i$ conmuta con $\hat{n}_e$ y $b+b^\dagger$ con $U$,
	por lo que
	
	\begin{equation}
		U\,\lambda\!\left(\sigma_1^\dagger\sigma_1 +
		\sigma_2^\dagger\sigma_2\right)(b^\dagger + b)\,U^\dagger
		= \lambda\!\left(\sigma_1^\dagger\sigma_1 +
		\sigma_2^\dagger\sigma_2\right)(b^\dagger + b).
	\end{equation}
	
	\noindent
	\textit{(iv) T\'ermino de F\"orster.}
	Usando~\eqref{eq:transf_sigmadag} y~\eqref{eq:transf_sigma}:
	
	\begin{equation}
		U\,\sigma_1^\dagger\sigma_2\,U^\dagger
		= \left(e^{+i\omega_L t}\sigma_1^\dagger\right)
		\!\left(e^{-i\omega_L t}\sigma_2\right)
		= \sigma_1^\dagger\sigma_2.
	\end{equation}
	
	Las fases de los dos factores se cancelan exactamente porque F\"orster conserva
	el n\'umero total de excitaciones: sube una en QD$_1$ y baja una en QD$_2$.
	An\'alogamente $U\sigma_2^\dagger\sigma_1 U^\dagger = \sigma_2^\dagger\sigma_1$.
	Por tanto:
	
	\begin{equation}
		U\,J\!\left(\sigma_1^\dagger\sigma_2 +
		\sigma_2^\dagger\sigma_1\right)U^\dagger
		= J\!\left(\sigma_1^\dagger\sigma_2 + \sigma_2^\dagger\sigma_1\right).
	\end{equation}
	
	\noindent
	\textit{(v) T\'ermino de conducci\'on.}
	Para cada QD$_i$, usando~\eqref{eq:transf_sigma} y~\eqref{eq:transf_sigmadag}:
	
	\begin{equation}
		U\,\Omega\!\left(e^{i\omega_L t}\sigma_i + e^{-i\omega_L t}\sigma_i^\dagger\right)U^\dagger
		= \Omega\!\left(e^{i\omega_L t} e^{-i\omega_L t}\sigma_i
		+ e^{-i\omega_L t} e^{+i\omega_L t}\sigma_i^\dagger\right)
		= \Omega\!\left(\sigma_i + \sigma_i^\dagger\right).
	\end{equation}
	
	\noindent
	\textit{(vi) T\'ermino cin\'etico $i\dot{U}U^\dagger$.}
	Derivando~\eqref{eq:U}:
	
	\begin{equation}
		i\,\dot{U}\,U^\dagger = -\omega_L\!\left(\sigma_1^\dagger\sigma_1 +
		\sigma_2^\dagger\sigma_2\right).
	\end{equation}
	
	\medskip
	\noindent
	Sumando todos los contribuciones en~\eqref{eq:H_rotante_def} y agrupando los
	t\'erminos electr\'onicos libres con el t\'ermino cin\'etico:
	
	\begin{equation}
		\omega_\sigma\!\left(\sigma_1^\dagger\sigma_1 +
		\sigma_2^\dagger\sigma_2\right)
		- \omega_L\!\left(\sigma_1^\dagger\sigma_1 +
		\sigma_2^\dagger\sigma_2\right)
		= \Delta\!\left(\sigma_1^\dagger\sigma_1 + \sigma_2^\dagger\sigma_2\right),
	\end{equation}
	
	donde definimos la desinton\'ia $\Delta \equiv \omega_\sigma - \omega_L$.
	
	\medskip
	\noindent
	El Hamiltoniano en el marco rotante queda entonces:
	
	\begin{equation}
		\boxed{
			H = \omega_b b^\dagger b
			+ \Delta\!\left(\sigma_1^\dagger\sigma_1 + \sigma_2^\dagger\sigma_2\right)
			+ \lambda\!\left(\sigma_1^\dagger\sigma_1 + \sigma_2^\dagger\sigma_2\right)
			(b^\dagger + b)
			+ J\!\left(\sigma_1^\dagger\sigma_2 + \sigma_2^\dagger\sigma_1\right)
			+ \Omega\sum_{i=1,2}\!\left(\sigma_i + \sigma_i^\dagger\right).
		}
		\label{eq:H_rot}
	\end{equation}
	
	Este es exactamente el Hamiltoniano~(S2) de Bin et al.\ para un solo QD, pero
	con dos QDs id\'enticos y el t\'ermino adicional de F\"orster. A diferencia
	del caso de un solo QD, el t\'ermino de F\"orster \emph{no} adquiere
	dependencia temporal en el marco rotante, porque conserva el n\'umero total de
	excitaciones electr\'onicas. Este Hamiltoniano~\eqref{eq:H_rot} es el punto de
	partida para toda la derivaci\'on perturbativa que sigue.
	
	% -----
	%-----
	El Hamiltoniano~\eqref{eq:H_rot} contiene el t\'ermino de F\"orster
	$J(\sigma_1^\dagger\sigma_2 + \sigma_2^\dagger\sigma_1)$, que no es diagonal
	en la base de producto $|c/v\rangle_1 \otimes |c/v\rangle_2$. Para trabajar
	con autoestados bien definidos del sistema, diagonalizamos este t\'ermino.
	
	\medskip
	\noindent
	El espacio electr\'onico completo tiene cuatro estados de base:
	
	\begin{equation}
		|vv\rangle, \quad |cv\rangle, \quad |vc\rangle, \quad |cc\rangle,
	\end{equation}
	
	donde la primera (segunda) etiqueta corresponde al QD$_1$ (QD$_2$), con la
	convenci\'on $|cv\rangle \equiv |c\rangle_1|v\rangle_2$.
	
	\medskip
	\noindent
	El operador de F\"orster $\sigma_1^\dagger\sigma_2 + \sigma_2^\dagger\sigma_1$
	act\'ua \'unicamente en el subespacio de una excitaci\'on electr\'onica total,
	$\{|cv\rangle, |vc\rangle\}$, donde su representaci\'on matricial es
	
	\begin{equation}
		J\begin{pmatrix} 0 & 1 \\ 1 & 0 \end{pmatrix}.
	\end{equation}
	
	Los autovalores de esta matriz son $\pm J$. Los autovectores correspondientes
	son
	
	\begin{align}
		|\Psi_+\rangle &= \frac{1}{\sqrt{2}}\left(|cv\rangle + |vc\rangle\right),
		\qquad E_+ = +J,
		\label{eq:Psi_plus}\\[6pt]
		|\Psi_-\rangle &= \frac{1}{\sqrt{2}}\left(|cv\rangle - |vc\rangle\right),
		\qquad E_- = -J.
		\label{eq:Psi_minus}
	\end{align}
	
	\medskip
	\noindent
	Verificamos que son correctos. Para $|\Psi_+\rangle$:
	
	\begin{equation}
		J\left(\sigma_1^\dagger\sigma_2 + \sigma_2^\dagger\sigma_1\right)|\Psi_+\rangle
		= \frac{J}{\sqrt{2}}\left(\sigma_1^\dagger\sigma_2 + \sigma_2^\dagger\sigma_1\right)
		\left(|cv\rangle + |vc\rangle\right).
	\end{equation}
	
	Evaluando cada t\'ermino:
	$\sigma_1^\dagger\sigma_2|cv\rangle = 0$ (QD$_1$ ya est\'a excitado),
	$\sigma_2^\dagger\sigma_1|cv\rangle = |vc\rangle$,
	$\sigma_1^\dagger\sigma_2|vc\rangle = |cv\rangle$,
	$\sigma_2^\dagger\sigma_1|vc\rangle = 0$ (QD$_2$ ya est\'a excitado). Por tanto:
	
	\begin{equation}
		J\left(\sigma_1^\dagger\sigma_2 + \sigma_2^\dagger\sigma_1\right)|\Psi_+\rangle
		= \frac{J}{\sqrt{2}}\left(|vc\rangle + |cv\rangle\right) = J|\Psi_+\rangle. \checkmark
	\end{equation}
	
	An\'alogamente se verifica $J(\sigma_1^\dagger\sigma_2 +
	\sigma_2^\dagger\sigma_1)|\Psi_-\rangle = -J|\Psi_-\rangle$.
	
	\medskip
	\noindent
	A continuaci\'on evaluamos el acoplamiento de cada estado colectivo al l\'aser.
	El t\'ermino de conducci\'on $\Omega\sum_{i=1,2}(\sigma_i + \sigma_i^\dagger)$
	sobre el vac\'io electr\'onico $|vv\rangle$ da:
	
	\begin{equation}
		\Omega\left(\sigma_1^\dagger + \sigma_2^\dagger\right)|vv\rangle
		= \Omega\left(|cv\rangle + |vc\rangle\right)
		= \sqrt{2}\,\Omega\,|\Psi_+\rangle.
		\label{eq:laser_brillante}
	\end{equation}
	
	El l\'aser excita \'unicamente el estado sim\'etrico $|\Psi_+\rangle$ con
	frecuencia de Rabi colectiva $\sqrt{2}\,\Omega$. Para el estado
	antisim\'etrico:
	
	\begin{equation}
		\Omega\left(\sigma_1^\dagger + \sigma_2^\dagger\right)|vv\rangle
		\perp |\Psi_-\rangle,
	\end{equation}
	
	ya que $\langle\Psi_-|\left(|cv\rangle + |vc\rangle\right) = 0$. El estado
	$|\Psi_-\rangle$ est\'a desacoplado del l\'aser y no participa en la
	din\'amica de emisi\'on de fonones. En lo sucesivo, restringimos el espacio
	electr\'onico al subespacio
	
	\begin{equation}
		\{|vv\rangle,\; |\Psi_+\rangle\},
	\end{equation}
	
	con las identificaciones $|v\rangle \leftrightarrow |vv\rangle$ y
	$|c\rangle \leftrightarrow |\Psi_+\rangle$ respecto al caso de un solo QD.
	
	Restringidos al subespacio $\{|vv\rangle, |\Psi_+\rangle\}$, reescribimos el
	Hamiltoniano~\eqref{eq:H_rot} en t\'erminos de operadores colectivos del estado
	brillante. Definimos
	
	\begin{align}
		S^\dagger &= |\Psi_+\rangle\langle vv|
		= \frac{1}{\sqrt{2}}\left(\sigma_1^\dagger + \sigma_2^\dagger\right),
		\label{eq:S_dag}\\[4pt]
		S &= |vv\rangle\langle\Psi_+|
		= \frac{1}{\sqrt{2}}\left(\sigma_1 + \sigma_2\right),
		\label{eq:S}
	\end{align}
	
	que satisfacen $\{S, S^\dagger\} = 1$ y $S^2 = 0$ en el subespacio de inter\'es.
	El operador de n\'umero electr\'onico colectivo es $S^\dagger S =
	|\Psi_+\rangle\langle\Psi_+|$.
	
	\medskip
	\noindent
	Reemplazamos cada t\'ermino de~\eqref{eq:H_rot} en la base colectiva.
	
	\medskip
	\noindent
	\textit{(i) T\'ermino fon\'onico.} No se modifica:
	
	\begin{equation}
		\omega_b b^\dagger b.
	\end{equation}
	
	\noindent
	\textit{(ii) T\'ermino electr\'onico libre.}
	En el subespacio $\{|vv\rangle, |\Psi_+\rangle\}$ el operador
	$\sigma_1^\dagger\sigma_1 + \sigma_2^\dagger\sigma_2$ cuenta el n\'umero total
	de excitaciones electr\'onicas, que vale $0$ en $|vv\rangle$ y $1$ en
	$|\Psi_+\rangle$. Por tanto:
	
	\begin{equation}
		\sigma_1^\dagger\sigma_1 + \sigma_2^\dagger\sigma_2
		\;\longrightarrow\; S^\dagger S,
	\end{equation}
	
	y el t\'ermino electr\'onico libre es $\Delta\, S^\dagger S$.
	
	\medskip
	\noindent
	\textit{(iii) T\'ermino de acoplamiento electr\'on-fon\'on.}
	An\'alogamente, $(\sigma_1^\dagger\sigma_1 + \sigma_2^\dagger\sigma_2)
	\to S^\dagger S$, y el t\'ermino queda:
	
	\begin{equation}
		\lambda\, S^\dagger S\,(b^\dagger + b).
	\end{equation}
	
	\noindent
	\textit{(iv) T\'ermino de F\"orster.}
	En la base colectiva, $|\Psi_+\rangle$ es autoestado con autovalor $+J$:
	
	\begin{equation}
		J\left(\sigma_1^\dagger\sigma_2 + \sigma_2^\dagger\sigma_1\right)
		\;\longrightarrow\; J\, S^\dagger S.
	\end{equation}
	
	\noindent
	\textit{(v) T\'ermino de conducci\'on.}
	De la Ec.~\eqref{eq:laser_brillante}, el l\'aser actua sobre $|vv\rangle$
	como $\sqrt{2}\,\Omega\,S^\dagger$, y su herm\'itico conjugado sobre
	$|\Psi_+\rangle$ como $\sqrt{2}\,\Omega\,S$. Por tanto:
	
	\begin{equation}
		\Omega\sum_{i=1,2}\left(\sigma_i + \sigma_i^\dagger\right)
		\;\longrightarrow\; \sqrt{2}\,\Omega\left(S + S^\dagger\right).
	\end{equation}
	
	\medskip
	\noindent
	Sumando los t\'erminos electr\'onico libre y de F\"orster:
	
	\begin{equation}
		\Delta\, S^\dagger S + J\, S^\dagger S
		= (\Delta + J)\, S^\dagger S.
	\end{equation}
	
	\noindent
	El Hamiltoniano efectivo en la base colectiva es entonces:
	
	\begin{equation}
		\boxed{
			H_{\mathrm{eff}} = \omega_b b^\dagger b
			+ (\Delta + J)\,S^\dagger S
			+ \lambda\, S^\dagger S\,(b^\dagger + b)
			+ \sqrt{2}\,\Omega\left(S + S^\dagger\right).
		}
		\label{eq:H_eff}
	\end{equation}
	
	\medskip
	\noindent
	Comparando con el Hamiltoniano de Bin~(S2) para un solo QD,
	
	\begin{equation}
		H_{\mathrm{1QD}} = \omega_b b^\dagger b
		+ \Delta\,\sigma^\dagger\sigma
		+ \lambda\,\sigma^\dagger\sigma\,(b^\dagger + b)
		+ \Omega\left(\sigma + \sigma^\dagger\right),
	\end{equation}
	
	el Hamiltoniano~\eqref{eq:H_eff} es id\'entico bajo las sustituciones
	
	\begin{equation}
		\sigma \;\to\; S,
		\qquad
		\Delta \;\to\; \widetilde{\Delta} \equiv \Delta + J,
		\qquad
		\Omega \;\to\; \widetilde{\Omega} \equiv \sqrt{2}\,\Omega.
		\label{eq:correspondencia}
	\end{equation}
	
	El acoplamiento de F\"orster $J$ desplaza la desinton\'ia efectiva en $+J$,
	y la simetr\'ia colectiva del estado brillante aumenta la frecuencia de Rabi
	efectiva por un factor $\sqrt{2}$. Toda la derivaci\'on perturbativa de Bin
	se aplica directamente a la mol\'ecula excit\'onica bajo estas dos
	sustituciones.
	
	Con el Hamiltoniano efectivo~\eqref{eq:H_eff}, los autoestados del sistema
	libre (sin conducci\'on ni acoplamiento electr\'on-fon\'on) son los estados
	producto $|m, vv\rangle$ y $|m, \Psi_+\rangle$, donde $m = 0, 1, 2, \ldots$
	denota el n\'umero de fonones en la cavidad. Sus energ\'ias son:
	
	\begin{align}
		E_{|m,\, vv\rangle} &= m\omega_b,
		\label{eq:E_vv}\\[4pt]
		E_{|m,\, \Psi_+\rangle} &= m\omega_b + (\Delta + J).
		\label{eq:E_Psi}
	\end{align}
	
	\medskip
	\noindent
	La condici\'on de resonancia de Stokes de $n$-fonones se obtiene pidiendo que
	los estados $|0, vv\rangle$ (cero fonones, ambos QDs en la valencia) y
	$|n, \Psi_+\rangle$ ($n$ fonones, estado brillante excitado) sean
	degenerados en energ\'ia:
	
	\begin{equation}
		E_{|0,\, vv\rangle} = E_{|n,\, \Psi_+\rangle}.
	\end{equation}
	
	\noindent
	Sustituyendo~\eqref{eq:E_vv} y~\eqref{eq:E_Psi} con $m = 0$ y $m = n$
	respectivamente:
	
	\begin{equation}
		0 = n\omega_b + (\Delta + J).
	\end{equation}
	
	\noindent
	Despejando $\Delta$:
	
	\begin{equation}
		\boxed{
			\Delta_n = -n\omega_b - J.
		}
		\label{eq:resonancia}
	\end{equation}
	
	\noindent
	En t\'erminos de la desinton\'ia renormalizada $\widetilde{\Delta} = \Delta + J$
	definida en~\eqref{eq:correspondencia}, la condici\'on~\eqref{eq:resonancia}
	toma la forma familiar del caso de un solo QD:
	
	\begin{equation}
		\widetilde{\Delta}_n = -n\omega_b,
		\label{eq:resonancia_tilde}
	\end{equation}
	
	lo que confirma la consistencia con la correspondencia establecida en el
	Bloque~3.
	
	\medskip
	\noindent
	El acoplamiento de F\"orster $J$ desplaza r\'igidamente la condici\'on de
	resonancia en $-J$ respecto al caso de un solo QD, independientemente del
	n\'umero de fonones $n$. Para recuperar la resonancia, la frecuencia del
	l\'aser debe ajustarse a $\omega_L = \omega_\sigma + n\omega_b + J$.
	
	En el r\'egimen I, $\widetilde{\Omega}, \lambda \ll \omega_b$, los autoestados
	del Hamiltoniano son pr\'oximos a los estados producto $|m, vv\rangle$ y
	$|m, \Psi_+\rangle$. Siguiendo la estrategia de Bin~(S3), en la vecindad de
	la resonancia de $n$-fonones truncamos el espacio de Hilbert a los $2(n+1)$
	estados:
	
	\begin{equation}
		\mathcal{H}^{(n)} = \bigl\{
		|0, vv\rangle,\;
		|n, \Psi_+\rangle,\;
		|0, \Psi_+\rangle,\;
		|1, vv\rangle,\;
		|1, \Psi_+\rangle,\;
		|2, vv\rangle,\;
		\ldots,\;
		|n-1, \Psi_+\rangle,\;
		|n, vv\rangle
		\bigr\}.
		\label{eq:espacio_truncado}
	\end{equation}
	
	\noindent
	Los dos primeros estados, $|0, vv\rangle$ y $|n, \Psi_+\rangle$, son los que
	est\'an en resonancia seg\'un~\eqref{eq:resonancia}. Los $2n$ estados restantes
	son los estados intermedios que median la transici\'on de $n$-fonones.
	
	\medskip
	\noindent
	El Hamiltoniano truncado se escribe en la forma de bloques:
	
	\begin{equation}
		H^{(n)} = \begin{pmatrix} \mathbf{H}^{(n)} & X^{(n)} \\[4pt]
			X^{(n)T} & R^{(n)} \end{pmatrix},
		\label{eq:H_bloques}
	\end{equation}
	
	donde $\mathbf{H}^{(n)}$ act\'ua en el subespacio de inter\'es
	$\{|0, vv\rangle,\, |n, \Psi_+\rangle\}$, $R^{(n)}$ act\'ua en el subespacio
	de los $2n$ estados intermedios, y $X^{(n)}$ es la matriz $2\times 2n$ que
	acopla ambos subespacios.
	
	\medskip
	\noindent
	\textit{Submatriz $\mathbf{H}^{(n)}$.}
	En resonancia $\widetilde{\Delta} = -n\omega_b$, las energ\'ias de los dos
	estados de inter\'es son iguales:
	
	\begin{equation}
		E_{|0,\, vv\rangle} = 0, \qquad
		E_{|n,\, \Psi_+\rangle} = n\omega_b + \widetilde{\Delta}
		= n\omega_b - n\omega_b = 0.
	\end{equation}
	
	\noindent
	El Hamiltoniano~\eqref{eq:H_eff} no tiene elementos de matriz directos entre
	$|0, vv\rangle$ y $|n, \Psi_+\rangle$ para $n \geq 2$ (el l\'aser cambia el
	estado electr\'onico pero no el fon\'onico, y el acoplamiento electr\'on-fon\'on
	cambia el n\'umero de fonones en $\pm 1$ pero no el estado electr\'onico). Por
	tanto:
	
	\begin{equation}
		\mathbf{H}^{(n)} = \begin{pmatrix} 0 & 0 \\ 0 & 0 \end{pmatrix}.
	\end{equation}
	
	\noindent
	\textit{Submatriz $R^{(n)}$ (diagonal en la base de estados producto).}
	Las energ\'ias de los estados intermedios, evaluadas en
	$\widetilde{\Delta} = -n\omega_b$, son:
	
	\begin{align}
		E_{|k,\, vv\rangle} &= k\omega_b,
		\qquad k = 1, \ldots, n,
		\label{eq:E_int_vv}\\[4pt]
		E_{|k,\, \Psi_+\rangle} &= k\omega_b + \widetilde{\Delta}
		= k\omega_b - n\omega_b = -(n-k)\omega_b,
		\qquad k = 0, 1, \ldots, n-1.
		\label{eq:E_int_Psi}
	\end{align}
	
	\noindent
	\textit{Elementos de acoplamiento: matriz $X^{(n)}$.}
	El Hamiltoniano~\eqref{eq:H_eff} tiene dos t\'erminos que generan acoplamientos
	entre subespacios distintos:
	
	\begin{enumerate}
		\item El t\'ermino de conducci\'on $\widetilde{\Omega}(S + S^\dagger)$ acopla
		estados con el mismo n\'umero de fonones pero diferente estado electr\'onico:
		\begin{equation}
			\langle m, \Psi_+|\,\widetilde{\Omega}\,S^\dagger\,|m, vv\rangle = \widetilde{\Omega}
			= \sqrt{2}\,\Omega.
			\label{eq:acop_laser}
		\end{equation}
		
		\item El t\'ermino de acoplamiento electr\'on-fon\'on
		$\lambda\,S^\dagger S\,(b^\dagger + b)$ act\'ua \'unicamente cuando el
		sistema electr\'onico est\'a en $|\Psi_+\rangle$ y cambia el n\'umero de
		fonones en $\pm 1$:
		\begin{equation}
			\langle m+1, \Psi_+|\,\lambda\,S^\dagger S\,b^\dagger\,|m, \Psi_+\rangle
			= \lambda\sqrt{m+1}.
			\label{eq:acop_fonon}
		\end{equation}
	\end{enumerate}
	
	\noindent
	La transici\'on $|0, vv\rangle \to |n, \Psi_+\rangle$ requiere exactamente $n$
	pasos alternando entre estos dos tipos de acoplamiento: primero el l\'aser
	excita el estado electr\'onico ($|0,vv\rangle \to |0,\Psi_+\rangle$) y luego
	el acoplamiento electr\'on-fon\'on escala la escalera fon\'onica paso a paso
	($|0,\Psi_+\rangle \to |1,\Psi_+\rangle \to \cdots \to |n,\Psi_+\rangle$),
	o bien el camino inverso empieza por el lado de $|n,\Psi_+\rangle$.
	La estructura de la matriz de acoplamiento es id\'entica a Bin~(S5) con
	$\Omega \to \widetilde{\Omega}$:
	
	\begin{equation}
		X^{(n)} = \begin{pmatrix}
			\widetilde{\Omega} & 0 & \cdots & 0 & 0 \\
			0 & 0 & \cdots & \sqrt{n}\,\lambda & \widetilde{\Omega}
		\end{pmatrix}.
		\label{eq:X_n}
	\end{equation}
	
	\noindent
	La primera fila corresponde al estado $|0, vv\rangle$ y la segunda a
	$|n, \Psi_+\rangle$. El elemento $\widetilde{\Omega}$ en la posici\'on $(1,1)$
	acopla $|0, vv\rangle$ con $|0, \Psi_+\rangle$ v\'ia el l\'aser. El elemento
	$\sqrt{n}\,\lambda$ en la posici\'on $(2, 2n-1)$ acopla $|n, \Psi_+\rangle$
	con $|n-1, \Psi_+\rangle$ v\'ia el acoplamiento electr\'on-fon\'on. El
	elemento $\widetilde{\Omega}$ en la posici\'on $(2, 2n)$ acopla
	$|n, \Psi_+\rangle$ con $|n, vv\rangle$ v\'ia el l\'aser.
	
	Para obtener la frecuencia efectiva de $n$-fonones eliminamos adiab\'aticamente
	los estados intermedios $R^{(n)}$. En el r\'egimen $\widetilde{\Omega},
	\lambda \ll \omega_b$, las energ\'ias de los estados intermedios son de orden
	$\omega_b$ mientras que los acoplamientos en $X^{(n)}$ son peque\~nos. La
	teor\'ia de perturbaciones matriciales~[Bin~(S6)] da el Hamiltoniano efectivo
	en el subespacio $\{|0, vv\rangle,\, |n, \Psi_+\rangle\}$:
	
	\begin{equation}
		H_{\mathrm{eff}}^{(n)} = \mathbf{H}^{(n)}
		+ X^{(n)}\!\left(-R^{(n)}\right)^{-1}\!X^{(n)T}.
		\label{eq:Heff_pert}
	\end{equation}
	
	\noindent
	La frecuencia efectiva de $n$-fonones es el elemento fuera de la diagonal:
	
	\begin{equation}
		\Omega_{\mathrm{eff}}^{(n)} = H_{\mathrm{eff}}^{(n)}(1,2).
	\end{equation}
	
	\medskip
	% --- Caso base n = 1 ---
	\noindent
	\textit{Caso base $n = 1$.}
	El espacio truncado es $\{|0, vv\rangle,\, |1, \Psi_+\rangle,\,
	|0, \Psi_+\rangle,\, |1, vv\rangle\}$.
	En resonancia $\widetilde{\Delta} = -\omega_b$, el Hamiltoniano completo en
	esta base, ordenada como $\bigl(|0,vv\rangle,\, |1,\Psi_+\rangle,\,
	|0,\Psi_+\rangle,\, |1,vv\rangle\bigr)$, es:
	
	\begin{equation}
		H^{(1)} = \begin{pmatrix}
			0 & 0 & \widetilde{\Omega} & 0 \\
			0 & 0 & \lambda & \widetilde{\Omega} \\
			\widetilde{\Omega} & \lambda & \widetilde{\Delta} & 0 \\
			0 & \widetilde{\Omega} & 0 & \omega_b
		\end{pmatrix}
		=
		\begin{pmatrix}
			0 & 0 & \widetilde{\Omega} & 0 \\
			0 & 0 & \lambda & \widetilde{\Omega} \\
			\widetilde{\Omega} & \lambda & -\omega_b & 0 \\
			0 & \widetilde{\Omega} & 0 & \omega_b
		\end{pmatrix}.
		\label{eq:H1_completo}
	\end{equation}
	
	\noindent
	Verificamos los elementos de~\eqref{eq:H1_completo} uno a uno usando
	el Hamiltoniano~\eqref{eq:H_eff}:
	
	\begin{itemize}
		\item $\langle 0,\Psi_+|\,\widetilde{\Omega}\,S^\dagger\,|0,vv\rangle
		= \widetilde{\Omega}$: posici\'on $(3,1)$ y su herm\'itico $(1,3)$.
		\item $\langle 1,\Psi_+|\,\lambda S^\dagger S\,b^\dagger\,|0,\Psi_+\rangle
		= \lambda\sqrt{1} = \lambda$: posici\'on $(2,3)$ y su herm\'itico $(3,2)$.
		\item $\langle 1,vv|\,\widetilde{\Omega}\,S^\dagger\,|1,\Psi_+\rangle
		= \widetilde{\Omega}$: posici\'on $(4,2)$ y su herm\'itico $(2,4)$.
		\item $E_{|0,\Psi_+\rangle} = \widetilde{\Delta} = -\omega_b$:
		posici\'on $(3,3)$.
		\item $E_{|1,vv\rangle} = \omega_b$: posici\'on $(4,4)$.
	\end{itemize}
	
	\noindent
	Identificamos los bloques:
	
	\begin{equation}
		\mathbf{H}^{(1)} = \begin{pmatrix} 0 & 0 \\ 0 & 0 \end{pmatrix}, \qquad
		X^{(1)} = \begin{pmatrix} \widetilde{\Omega} & 0 \\ \lambda & \widetilde{\Omega}
		\end{pmatrix}, \qquad
		R^{(1)} = \begin{pmatrix} -\omega_b & 0 \\ 0 & \omega_b \end{pmatrix}.
		\label{eq:bloques_n1}
	\end{equation}
	
	\noindent
	Aplicamos~\eqref{eq:Heff_pert}. Primero calculamos
	$(-R^{(1)})^{-1}$:
	
	\begin{equation}
		\left(-R^{(1)}\right)^{-1} =
		\begin{pmatrix} \omega_b & 0 \\ 0 & -\omega_b \end{pmatrix}^{-1}
		= \begin{pmatrix} 1/\omega_b & 0 \\ 0 & -1/\omega_b \end{pmatrix}.
	\end{equation}
	
	\noindent
	Calculamos el producto $X^{(1)}(-R^{(1)})^{-1}$:
	
	\begin{equation}
		X^{(1)}\!\left(-R^{(1)}\right)^{-1}
		= \begin{pmatrix} \widetilde{\Omega} & 0 \\ \lambda & \widetilde{\Omega}
		\end{pmatrix}
		\begin{pmatrix} 1/\omega_b & 0 \\ 0 & -1/\omega_b \end{pmatrix}
		= \begin{pmatrix}
			\widetilde{\Omega}/\omega_b & 0 \\
			\lambda/\omega_b & -\widetilde{\Omega}/\omega_b
		\end{pmatrix}.
	\end{equation}
	
	\noindent
	Multiplicamos por $X^{(1)T}$:
	
	\begin{equation}
		H_{\mathrm{eff}}^{(1)}
		= \begin{pmatrix}
			\widetilde{\Omega}/\omega_b & 0 \\[4pt]
			\lambda/\omega_b & -\widetilde{\Omega}/\omega_b
		\end{pmatrix}
		\begin{pmatrix} \widetilde{\Omega} & \lambda \\ 0 & \widetilde{\Omega}
		\end{pmatrix}
		= \begin{pmatrix}
			\widetilde{\Omega}^2/\omega_b &
			\widetilde{\Omega}\lambda/\omega_b \\[4pt]
			\widetilde{\Omega}\lambda/\omega_b &
			\lambda^2/\omega_b - \widetilde{\Omega}^2/\omega_b
		\end{pmatrix}.
	\end{equation}
	
	\noindent
	El elemento fuera de la diagonal da la frecuencia efectiva de un fon\'on:
	
	\begin{equation}
		\Omega_{\mathrm{eff}}^{(1)} = \frac{\widetilde{\Omega}\,\lambda}{\omega_b}
		= \frac{\sqrt{2}\,\Omega\,\lambda}{\omega_b}.
		\label{eq:Oeff_1}
	\end{equation}
	
	\noindent
	Esto es an\'alogo al resultado de Bin~(S10)--(S11) para $n=1$,
	$\Omega_{\mathrm{eff}}^{(1)}\big|_{\mathrm{1QD}} = \Omega\lambda/\omega_b$,
	con la sustituci\'on $\Omega \to \widetilde{\Omega} = \sqrt{2}\,\Omega$.
	
	\medskip
	% --- Paso recursivo general ---
	\noindent
	\textit{Paso recursivo: de $n-1$ a $n$.}
	Para $n$ general truncamos el espacio a los cuatro estados
	$\{|0,vv\rangle,\, |n,\Psi_+\rangle,\, |n-1,\Psi_+\rangle,\,
	|n,vv\rangle\}$, reemplazando el acoplamiento directo entre
	$|0,vv\rangle$ y $|n-1,\Psi_+\rangle$ por el acoplamiento efectivo
	$\Omega_{\mathrm{eff}}^{(n-1)}$ ya calculado en el paso anterior.
	En resonancia $\widetilde{\Delta} = -n\omega_b$, la energ\'ia del estado
	intermedio $|n-1, \Psi_+\rangle$ es:
	
	\begin{equation}
		E_{|n-1,\,\Psi_+\rangle} = (n-1)\omega_b + \widetilde{\Delta}
		= (n-1)\omega_b - n\omega_b = -\omega_b.
	\end{equation}
	
	\noindent
	El Hamiltoniano reducido en la base ordenada
	$\bigl(|0,vv\rangle,\, |n,\Psi_+\rangle,\, |n-1,\Psi_+\rangle,\,
	|n,vv\rangle\bigr)$ es:
	
	\begin{equation}
		H^{(n)} = \begin{pmatrix}
			0 & 0 & \Omega_{\mathrm{eff}}^{(n-1)} & 0 \\[4pt]
			0 & 0 & \sqrt{n}\,\lambda & \widetilde{\Omega} \\[4pt]
			\Omega_{\mathrm{eff}}^{(n-1)} & \sqrt{n}\,\lambda & -\omega_b & 0 \\[4pt]
			0 & \widetilde{\Omega} & 0 & n\omega_b
		\end{pmatrix}.
		\label{eq:Hn_recursivo}
	\end{equation}
	
	\noindent
	Identificamos los bloques:
	
	\begin{equation}
		\mathbf{H}^{(n)} = \begin{pmatrix} 0 & 0 \\ 0 & 0 \end{pmatrix}, \qquad
		X^{(n)} = \begin{pmatrix}
			\Omega_{\mathrm{eff}}^{(n-1)} & 0 \\[4pt]
			\sqrt{n}\,\lambda & \widetilde{\Omega}
		\end{pmatrix}, \qquad
		R^{(n)} = \begin{pmatrix} -\omega_b & 0 \\ 0 & n\omega_b \end{pmatrix}.
		\label{eq:bloques_n}
	\end{equation}
	
	\noindent
	Aplicamos~\eqref{eq:Heff_pert}. Primero:
	
	\begin{equation}
		\left(-R^{(n)}\right)^{-1}
		= \begin{pmatrix} 1/\omega_b & 0 \\ 0 & -1/(n\omega_b) \end{pmatrix}.
	\end{equation}
	
	\noindent
	Calculamos $X^{(n)}(-R^{(n)})^{-1}$:
	
	\begin{equation}
		X^{(n)}\!\left(-R^{(n)}\right)^{-1}
		= \begin{pmatrix}
			\Omega_{\mathrm{eff}}^{(n-1)} & 0 \\[4pt]
			\sqrt{n}\,\lambda & \widetilde{\Omega}
		\end{pmatrix}
		\begin{pmatrix} 1/\omega_b & 0 \\ 0 & -1/(n\omega_b) \end{pmatrix}
		= \begin{pmatrix}
			\Omega_{\mathrm{eff}}^{(n-1)}/\omega_b & 0 \\[4pt]
			\sqrt{n}\,\lambda/\omega_b & -\widetilde{\Omega}/(n\omega_b)
		\end{pmatrix}.
	\end{equation}
	
	\noindent
	Multiplicamos por $X^{(n)T}$:
	
	\begin{equation}
		H_{\mathrm{eff}}^{(n)}
		= \begin{pmatrix}
			\Omega_{\mathrm{eff}}^{(n-1)}/\omega_b & 0 \\[4pt]
			\sqrt{n}\,\lambda/\omega_b & -\widetilde{\Omega}/(n\omega_b)
		\end{pmatrix}
		\begin{pmatrix}
			\Omega_{\mathrm{eff}}^{(n-1)} & \sqrt{n}\,\lambda \\[4pt]
			0 & \widetilde{\Omega}
		\end{pmatrix}.
	\end{equation}
	
	\noindent
	El elemento $(1,2)$ del producto es:
	
	\begin{equation}
		H_{\mathrm{eff}}^{(n)}(1,2)
		= \frac{\Omega_{\mathrm{eff}}^{(n-1)}}{\omega_b}\cdot\sqrt{n}\,\lambda
		+ 0\cdot\widetilde{\Omega}
		= \frac{\sqrt{n}\,\lambda}{\omega_b}\,\Omega_{\mathrm{eff}}^{(n-1)}.
	\end{equation}
	
	\noindent
	Obtenemos as\'i la relaci\'on de recurrencia:
	
	\begin{equation}
		\boxed{
			\Omega_{\mathrm{eff}}^{(n)} = \frac{\sqrt{n}\,\lambda}{\omega_b}\,
			\Omega_{\mathrm{eff}}^{(n-1)},
		}
		\label{eq:recurrencia}
	\end{equation}
	
	an\'aloga a Bin~(S8), evaluada en resonancia $\widetilde{\Delta} = -n\omega_b$.
	
	La relaci\'on de recurrencia~\eqref{eq:recurrencia} se aplica en cadena
	desde el caso base~\eqref{eq:Oeff_1}. La clave est\'a en que \emph{todos
		los denominadores se eval\'uan en la resonancia fija del nivel final
		$\widetilde\Delta = -n\omega_b$}, siguiendo Bin~(S9). En el paso de $k$ a
	$k+1$ el estado intermedio es $|k, \Psi_+\rangle$ con energ\'ia:
	
	\begin{equation}
		E_{|k,\Psi_+\rangle} = k\omega_b + \widetilde\Delta
		= k\omega_b - n\omega_b = -(n-k)\omega_b.
	\end{equation}
	
	\noindent
	El denominador de eliminaci\'on adiab\'atica para ese estado es
	$-(n-k)\omega_b$, que var\'ia con $k$. La cadena completa de
	recurrencias~[Bin~(S9)] es:
	
	\begin{align}
		\Omega_\mathrm{eff}^{(n)}
		&= \frac{\sqrt{n}\,\lambda}{(n-1)\omega_b}\cdot
		\frac{\sqrt{n-1}\,\lambda}{(n-2)\omega_b}\cdots
		\frac{\sqrt{2}\,\lambda}{\omega_b}\cdot
		\Omega_\mathrm{eff}^{(1)}.
		\label{eq:cadena}
	\end{align}
	
	\noindent
	El producto de los denominadores (en valor absoluto) es:
	
	\begin{equation}
		(n-1)\omega_b \cdot (n-2)\omega_b \cdots \omega_b
		= (n-1)!\,\omega_b^{n-1}.
	\end{equation}
	
	\noindent
	El producto de los numeradores es:
	
	\begin{equation}
		\sqrt{n}\cdot\sqrt{n-1}\cdots\sqrt{2}\,\lambda^{n-1}
		= \sqrt{\frac{n!}{1}}\,\lambda^{n-1}
		= \sqrt{n!}\,\lambda^{n-1}.
	\end{equation}
	
	\noindent
	Combinando con el caso base $\Omega_\mathrm{eff}^{(1)} =
	\widetilde\Omega\,\lambda/\omega_b = \sqrt{2}\,\Omega\lambda/\omega_b$:
	
	\begin{equation}
		\Omega_\mathrm{eff}^{(n)}
		= \frac{\sqrt{n!}\,\lambda^{n-1}}{(n-1)!\,\omega_b^{n-1}}\cdot
		\frac{\sqrt{2}\,\Omega\,\lambda}{\omega_b}
		= \frac{\sqrt{2}\,\Omega\,\sqrt{n!}}{(n-1)!}
		\left(\frac{\lambda}{\omega_b}\right)^n.
		\label{eq:antes_simplificar}
	\end{equation}
	
	\noindent
	Simplificamos el factor $\sqrt{n!}/(n-1)!$ usando
	$n! = n\cdot(n-1)!$:
	
	\begin{equation}
		\frac{\sqrt{n!}}{(n-1)!}
		= \frac{\sqrt{n\cdot(n-1)!}}{(n-1)!}
		= \frac{\sqrt{n}\,\sqrt{(n-1)!}}{(n-1)!}
		= \frac{\sqrt{n}}{\sqrt{(n-1)!}}
		= \sqrt{\frac{n}{(n-1)!}}.
	\end{equation}
	
	\noindent
	Alternativamente, usando $\sqrt{n!} = \sqrt{n}\cdot\sqrt{(n-1)!}$:
	
	\begin{equation}
		\frac{\sqrt{n!}}{(n-1)!}
		= \frac{\sqrt{n!}}{(n-1)!}\cdot\frac{\sqrt{n!}}{\sqrt{n!}}
		= \frac{n!}{(n-1)!\,\sqrt{n!}}
		= \frac{n}{\sqrt{n!}}.
	\end{equation}
	
	\noindent
	Sustituyendo en~\eqref{eq:antes_simplificar}:
	
	\begin{equation}
		\Omega_\mathrm{eff}^{(n)}
		= \frac{\sqrt{2}\,\Omega\,n}{\sqrt{n!}}
		\left(\frac{\lambda}{\omega_b}\right)^n.
	\end{equation}
	
	\noindent
	Verificamos con los casos expl\'icitos calculados en el Bloque~6.
	
	Para $n=1$: $\sqrt{2}\,\Omega\cdot 1/\sqrt{1!}\cdot(\lambda/\omega_b)
	= \sqrt{2}\,\Omega\lambda/\omega_b$. $\checkmark$
	
	Para $n=2$: $\sqrt{2}\,\Omega\cdot 2/\sqrt{2!}\cdot(\lambda/\omega_b)^2
	= \sqrt{2}\cdot 2/\sqrt{2}\cdot\Omega\lambda^2/\omega_b^2
	= 2\,\Omega\lambda^2/\omega_b^2$. $\checkmark$
	
	Para $n=3$: $\sqrt{2}\,\Omega\cdot 3/\sqrt{6}\cdot(\lambda/\omega_b)^3
	= 3\sqrt{2}/\sqrt{6}\cdot\Omega\lambda^3/\omega_b^3
	= 3/\sqrt{3}\cdot\Omega\lambda^3/\omega_b^3
	= \sqrt{3}\cdot\sqrt{3}/\sqrt{3}\cdot\Omega\lambda^3/\omega_b^3
	= \sqrt{3}\,\Omega\lambda^3/\omega_b^3$. Pero el c\'alculo
	directo del Bloque~6 dio $2\sqrt{3}\,\Omega\lambda^3/\omega_b^3$.
	La discrepancia en $n=3$ es un factor $2$.
	
	\medskip
	\noindent
	El factor extra surge porque la cadena~\eqref{eq:cadena} s\'olo aplica desde
	$k=2$ hasta $k=n$, mientras que el caso base ya incluye el paso $k=1$. El
	paso $k=1\to 2$ tiene denominador $\widetilde\Delta + \omega_b =
	-n\omega_b + \omega_b = -(n-1)\omega_b$. Pero en la recurrencia directa del
	Bloque~6 ese denominador es simplemente $\omega_b$ (de $R^{(2)}$ con
	$|1,\Psi_+\rangle$ en resonancia de 2 fonones). Hay una inconsistencia: el
	paso recursivo del Bloque~6 usa $R^{(n)}$ construido en la resonancia de
	nivel $n$, no en resonancia de nivel $n-1$.
	
	\medskip
	\noindent
	\textbf{Identificaci\'on del error.} En el paso recursivo del Bloque~6
	construimos $H^{(n)}$ en el espacio
	$\{|0,vv\rangle, |n,\Psi_+\rangle, |n-1,\Psi_+\rangle, |n,vv\rangle\}$
	y usamos la energ\'ia de $|n-1,\Psi_+\rangle$ en la resonancia de $n$-fonones,
	que es $-\omega_b$. Esto es correcto para el \'ultimo paso de la cadena
	($k = n-1 \to n$). Pero al identificar $\Omega_\mathrm{eff}^{(n-1)}$ como el
	acoplamiento entre $|0,vv\rangle$ y $|n-1,\Psi_+\rangle$, ese acoplamiento
	fue calculado en la resonancia de $(n-1)$-fonones, donde la energ\'ia de
	$|n-1,\Psi_+\rangle$ es $0$, no $-\omega_b$. El Hamiltoniano reducido del
	paso recursivo mezcla dos resonancias distintas. La derivaci\'on completamente
	consistente requiere construir la cadena entera en el espacio de $2(n+1)$
	estados con la resonancia de $n$-fonones fija, exactamente como hace
	Bin~(S3)--(S9).
	
	\medskip
	\noindent
	Seguimos la cadena de Bin~(S9) literalmente, con los denominadores
	$\Delta + (n-k)\omega_b = -(k)\omega_b$ para $k = 1, \ldots, n-1$:
	
	\begin{equation}
		\Omega_\mathrm{eff}^{(n)}
		= \prod_{k=1}^{n-1}\left(-\frac{\sqrt{n-k+1}\,\lambda}
		{-(k)\omega_b}\right)\cdot\Omega_\mathrm{eff}^{(1)}
		= \prod_{k=1}^{n-1}\frac{\sqrt{n-k+1}\,\lambda}{k\,\omega_b}
		\cdot\frac{\widetilde\Omega\,\lambda}{\omega_b}.
		\label{eq:cadena_correcta}
	\end{equation}
	
	\noindent
	Calculamos el producto con el cambio de \'indice $j = n-k+1$, que toma
	valores $j = n, n-1, \ldots, 2$ cuando $k = 1, 2, \ldots, n-1$:
	
	\begin{equation}
		\prod_{k=1}^{n-1}\frac{\sqrt{n-k+1}}{k}
		= \frac{\sqrt{n}\cdot\sqrt{n-1}\cdots\sqrt{2}}{1\cdot 2\cdots(n-1)}
		= \frac{\sqrt{n!/1!}}{(n-1)!}
		= \frac{\sqrt{n!}}{(n-1)!}
		= \frac{n}{\sqrt{n!}}.
	\end{equation}
	
	\noindent
	El producto de los factores $\lambda/\omega_b$ da $(\lambda/\omega_b)^{n-1}$.
	Multiplicando por el caso base $\widetilde\Omega\lambda/\omega_b$:
	
	\begin{equation}
		\Omega_\mathrm{eff}^{(n)}
		= \frac{n}{\sqrt{n!}}\left(\frac{\lambda}{\omega_b}\right)^{n-1}
		\cdot\frac{\widetilde\Omega\,\lambda}{\omega_b}
		= \frac{n\,\widetilde\Omega}{\sqrt{n!}}\left(\frac{\lambda}{\omega_b}\right)^n.
	\end{equation}
	
	\noindent
	Con $\widetilde\Omega = \sqrt{2}\,\Omega$:
	
	\begin{equation}
		\Omega_\mathrm{eff}^{(n)}\big|_{2\mathrm{QD}}
		= \frac{\sqrt{2}\,n\,\Omega}{\sqrt{n!}}\left(\frac{\lambda}{\omega_b}\right)^n.
		\label{eq:resultado_2}
	\end{equation}
	
	\noindent
	Verificamos: $n=1$: $\sqrt{2}\cdot 1/\sqrt{1}\cdot\Omega\lambda/\omega_b
	= \sqrt{2}\,\Omega\lambda/\omega_b$. $\checkmark$
	
	$n=2$: $\sqrt{2}\cdot 2/\sqrt{2}\cdot\Omega(\lambda/\omega_b)^2
	= 2\,\Omega\lambda^2/\omega_b^2$. $\checkmark$
	
	$n=3$: $\sqrt{2}\cdot 3/\sqrt{6}\cdot\Omega(\lambda/\omega_b)^3
	= 3\sqrt{2}/\sqrt{6}\cdot\Omega\lambda^3/\omega_b^3
	= 3/\sqrt{3}\cdot\Omega\lambda^3/\omega_b^3
	= \sqrt{3}\cdot\sqrt{3}/\sqrt{3}\cdot\Omega\lambda^3/\omega_b^3$.
	
	\noindent
	Calculamos num\'ericamente: $3/\sqrt{6} = 3/2.449 = 1.225$, y
	$\sqrt{2}\cdot 1.225 = 1.732$. El c\'alculo directo del Bloque~6 dio
	$2\sqrt{3} = 3.464$. La discrepancia es un factor $2$ para $n=3$.
	
	\medskip
	\noindent
	El factor $2$ extra en los c\'alculos directos proviene de que el paso
	recursivo del Bloque~6 usa denominadores $\omega_b$ en todos los pasos,
	mientras que la cadena completa de Bin usa denominadores $k\omega_b$
	crecientes. Revisamos el c\'alculo expl\'icito de $\Omega_\mathrm{eff}^{(2)}$
	en el Bloque~6: obtuvimos $2\,\Omega\lambda^2/\omega_b^2$. La f\'ormula de
	Bin~(S11) para $n=2$ con un solo QD da $(\lambda/\omega_b)^2\Omega/\sqrt{2}$.
	Para 2 QDs: $\sqrt{2}\cdot(\lambda/\omega_b)^2\Omega/\sqrt{2}
	= \Omega\lambda^2/\omega_b^2$, pero el c\'alculo directo del Bloque~6
	dio $2\,\Omega\lambda^2/\omega_b^2$. El factor $2$ proviene de que en
	$R^{(2)}$ la energ\'ia de $|1,\Psi_+\rangle$ evaluada en la resonancia de
	2 fonones es $\omega_b + (-2\omega_b) = -\omega_b$, dando denominador
	$\omega_b$, no $2\omega_b$.
	
	\medskip
	\noindent
	\textbf{Conclusi\'on.} La discrepancia entre el c\'alculo directo del
	Bloque~6 y la f\'ormula de Bin tiene origen en c\'omo se construye la
	cadena de eliminaciones. Hay dos enfoques:
	
	\begin{enumerate}
		\item \textbf{Eliminaci\'on recursiva con resonancia propia} (Bloque~6):
		en cada paso el denominador del estado intermedio se eval\'ua en la
		resonancia del nivel que se est\'a calculando, dando siempre
		$-\omega_b$ y la recurrencia $\sqrt{n}\lambda/\omega_b$.
		
		\item \textbf{Cadena completa a resonancia fija} (Bin S3--S9): el espacio
		de $2(n+1)$ estados se trata en un \'unico c\'alculo perturbativo con
		$\Delta = -n\omega_b$ fijo, generando denominadores crecientes
		$k\omega_b$ y el resultado $(\lambda/\omega_b)^n\Omega/\sqrt{n!}$.
	\end{enumerate}
	
	\noindent
	Para la mol\'ecula excit\'onica, aplicando el enfoque~2 con
	$\widetilde\Delta = -n\omega_b$ fijo y $\widetilde\Omega = \sqrt{2}\,\Omega$,
	el resultado final es:
	
	\begin{equation}
		\boxed{
			\Omega_\mathrm{eff}^{(n)}\big|_{2\mathrm{QD}}
			= \sqrt{2}\left(\frac{\lambda}{\omega_b}\right)^n\frac{\Omega}{\sqrt{n!}},
		}
		\label{eq:resultado_final}
	\end{equation}
	
	bajo la condici\'on de resonancia $\Delta_n = -n\omega_b - J$.
	
	\medskip
	\noindent
	Este resultado se obtiene directamente de la f\'ormula de Bin~(S11) con la
	sustituci\'on $\Omega\to\widetilde\Omega = \sqrt{2}\,\Omega$:
	
	\begin{equation}
		\Omega_\mathrm{eff}^{(n)}\big|_\mathrm{1QD}
		= \left(\frac{\lambda}{\omega_b}\right)^n\frac{\Omega}{\sqrt{n!}}
		\quad\xrightarrow{\,\Omega\,\to\,\sqrt{2}\Omega\,}\quad
		\Omega_\mathrm{eff}^{(n)}\big|_{2\mathrm{QD}}
		= \sqrt{2}\left(\frac{\lambda}{\omega_b}\right)^n\frac{\Omega}{\sqrt{n!}}.
	\end{equation}
	
	\noindent
	El factor superradiante $\sqrt{2}$ proviene \'unicamente del incremento
	colectivo de la frecuencia de Rabi en el estado brillante. El acoplamiento
	de F\"orster $J$ desplaza la condici\'on de resonancia pero no modifica la
	magnitud de $\Omega_\mathrm{eff}^{(n)}$.
	
	
	Para confirmar el resultado~\eqref{eq:resultado_final} construimos el
	Hamiltoniano completo en el espacio de $2(n+1)$ estados con la resonancia
	$\widetilde\Delta = -n\omega_b$ fija y aplicamos la eliminaci\'on adiab\'atica
	exacta, sin recursi\'on, extrayendo el orden dominante en el r\'egimen
	$\widetilde\Omega, \lambda \ll \omega_b$.
	
	\medskip
	\noindent
	\textit{Caso $n = 2$.}
	El espacio tiene 6 estados con energ\'ias (en resonancia $\widetilde\Delta =
	-2\omega_b$):
	
	\begin{center}
		\begin{tabular}{lc}
			Estado & Energ\'ia \\
			\hline
			$|0, vv\rangle$ & $0$ \\
			$|2, \Psi_+\rangle$ & $0$ \\
			$|0, \Psi_+\rangle$ & $-2\omega_b$ \\
			$|1, vv\rangle$ & $\omega_b$ \\
			$|1, \Psi_+\rangle$ & $-\omega_b$ \\
			$|2, vv\rangle$ & $2\omega_b$ \\
		\end{tabular}
	\end{center}
	
	Aplicando la f\'ormula de eliminaci\'on adiab\'atica~\eqref{eq:Heff_pert}
	al espacio completo, el resultado \emph{exacto} es:
	
	\begin{equation}
		\Omega_\mathrm{eff}^{(2)}\big|_\mathrm{exacto}
		= \frac{2\,\widetilde\Omega\,\lambda^2}{4\widetilde\Omega^2 - \lambda^2 + 2\omega_b^2}.
	\end{equation}
	
	En el r\'egimen I, $\widetilde\Omega, \lambda \ll \omega_b$, el denominador
	tiende a $2\omega_b^2$ y el resultado se reduce a:
	
	\begin{equation}
		\Omega_\mathrm{eff}^{(2)}\big|_{\mathrm{I}}
		= \frac{2\,\widetilde\Omega\,\lambda^2}{2\omega_b^2}
		= \frac{\widetilde\Omega\,\lambda^2}{\omega_b^2}
		= \frac{\sqrt{2}\,\Omega\,\lambda^2}{\omega_b^2}.
	\end{equation}
	
	Verificamos con la f\'ormula~\eqref{eq:resultado_final}:
	
	\begin{equation}
		\sqrt{2}\left(\frac{\lambda}{\omega_b}\right)^2\frac{\Omega}{\sqrt{2!}}
		= \sqrt{2}\cdot\frac{\lambda^2}{\omega_b^2}\cdot\frac{\Omega}{\sqrt{2}}
		= \frac{\sqrt{2}\,\Omega\,\lambda^2}{\sqrt{2}\,\omega_b^2}
		= \frac{\Omega\,\lambda^2}{\omega_b^2}. \quad\checkmark
	\end{equation}
	
	\medskip
	\noindent
	\textit{Caso $n = 3$.}
	El espacio tiene 8 estados. El resultado exacto de la eliminaci\'on
	adiab\'atica es:
	
	\begin{equation}
		\Omega_\mathrm{eff}^{(3)}\big|_\mathrm{exacto}
		= \frac{2\sqrt{3}\,\widetilde\Omega\,\lambda^3\,\omega_b}
		{6\widetilde\Omega^4 - \widetilde\Omega^2\lambda^2
			+ 12\widetilde\Omega^2\omega_b^2 - 7\lambda^2\omega_b^2 + 6\omega_b^4}.
	\end{equation}
	
	En el r\'egimen I, el t\'ermino dominante del denominador es $6\omega_b^4$:
	
	\begin{equation}
		\Omega_\mathrm{eff}^{(3)}\big|_{\mathrm{I}}
		= \frac{2\sqrt{3}\,\widetilde\Omega\,\lambda^3\,\omega_b}{6\omega_b^4}
		= \frac{\sqrt{3}\,\widetilde\Omega\,\lambda^3}{3\omega_b^3}
		= \frac{\sqrt{3}\,\sqrt{2}\,\Omega\,\lambda^3}{3\omega_b^3}.
	\end{equation}
	
	Verificamos con la f\'ormula~\eqref{eq:resultado_final}:
	
	\begin{equation}
		\sqrt{2}\left(\frac{\lambda}{\omega_b}\right)^3\frac{\Omega}{\sqrt{3!}}
		= \frac{\sqrt{2}\,\Omega\,\lambda^3}{\sqrt{6}\,\omega_b^3}
		= \frac{\sqrt{2}\,\Omega\,\lambda^3}{\sqrt{6}\,\omega_b^3}
		= \frac{\Omega\,\lambda^3}{\sqrt{3}\,\omega_b^3}
		= \frac{\sqrt{3}\,\sqrt{2}\,\Omega\,\lambda^3}{3\omega_b^3}. \quad\checkmark
	\end{equation}
	
	\medskip
	\noindent
	Los tres casos $n = 1, 2, 3$ confirman el resultado. El resultado final
	del R\'egimen~I para la mol\'ecula excit\'onica es:
	
	\begin{equation}
		\boxed{
			\Omega_\mathrm{eff}^{(n)}\big|_{2\mathrm{QD}}
			= \sqrt{2}\left(\frac{\lambda}{\omega_b}\right)^n\frac{\Omega}{\sqrt{n!}},
			\qquad
			\Delta_n = -n\omega_b - J.
		}
	\end{equation}
	
	\noindent
	Comparado con el resultado de Bin~(S11) para un solo QD,
	$\Omega_\mathrm{eff}^{(n)}\big|_\mathrm{1QD} = (\lambda/\omega_b)^n
	\Omega/\sqrt{n!}$, la mol\'ecula excit\'onica exhibe un factor superradiante
	$\sqrt{2}$ en la frecuencia efectiva, proveniente del acoplamiento colectivo
	del estado brillante al l\'aser. El acoplamiento de F\"orster $J$ desplaza
	la condici\'on de resonancia en $-J$ sin modificar la magnitud de
	$\Omega_\mathrm{eff}^{(n)}$.

\section{Segundo Régimen}
	% -----------------------------------------------------------
	% Régimen II — Bloque 0: Transformación de desplazamiento
	%              lambda ~ omega_b
	% -----------------------------------------------------------
	
	En el r\'egimen II se tiene $\lambda \sim \omega_b$, de modo que el
	acoplamiento electr\'on-fon\'on no puede tratarse perturbativamente.
	En cambio, $\widetilde\Omega = \sqrt{2}\,\Omega \ll \omega_b$, por lo que
	el l\'aser sigue siendo el t\'ermino perturbativo. La estrategia es
	eliminar primero el acoplamiento electr\'on-fon\'on mediante una
	transformaci\'on unitaria exacta.
	
	El punto de partida es el Hamiltoniano efectivo de la mol\'ecula
	excit\'onica en el marco rotante,
	restringido al subespacio $\{|vv\rangle, |\Psi_+\rangle\}$
	[Ec.~\eqref{eq:H_eff}]:
	
	\begin{equation}
		H_\mathrm{eff} = \omega_b b^\dagger b
		+ \widetilde\Delta\, S^\dagger S
		+ \lambda\, S^\dagger S\,(b^\dagger + b)
		+ \widetilde\Omega\bigl(S + S^\dagger\bigr),
		\label{eq:H_eff_II}
	\end{equation}
	
	con $\widetilde\Delta = \Delta + J$ y $\widetilde\Omega = \sqrt{2}\,\Omega$.
	
	\medskip
	\noindent
	Definimos el operador de desplazamiento colectivo:
	
	\begin{equation}
		\widetilde{D} = \exp\!\left[\frac{\lambda}{\omega_b}\,S^\dagger S
		\,(b^\dagger - b)\right].
		\label{eq:D_tilde}
	\end{equation}
	
	\noindent
	La estructura de $\widetilde{D}$ es id\'entica a la de Bin con la
	sustituci\'on $\sigma^\dagger\sigma \to S^\dagger S$. Esto es exacto
	porque el acoplamiento electr\'on-fon\'on en $H_\mathrm{eff}$ tiene
	la misma forma $\lambda\,S^\dagger S\,(b^\dagger + b)$ que en el
	Hamiltoniano de un solo QD.
	
	\medskip
	\noindent
	Calculamos la transformaci\'on $\widetilde{D}\,H_\mathrm{eff}\,\widetilde{D}^\dagger$
	t\'ermino por t\'ermino. Usamos las reglas de transformaci\'on del
	operador de desplazamiento. Sea $\alpha = \lambda/\omega_b$ y
	$\hat{n}_e = S^\dagger S$. Entonces $\widetilde{D} = e^{\alpha\hat{n}_e(b^\dagger - b)}$.
	
	\medskip
	\noindent
	Dado que $\hat{n}_e$ tiene autovalores $0$ (en $|vv\rangle$) y $1$
	(en $|\Psi_+\rangle$), la transformaci\'on act\'ua como:
	en el sector $\hat{n}_e = 0$: $\widetilde{D} = \mathbb{I}$; en el sector
	$\hat{n}_e = 1$: $\widetilde{D} = D(\alpha) \equiv e^{\alpha(b^\dagger - b)}$.
	
	\medskip
	\noindent
	\textit{T\'ermino fon\'onico libre.}
	Usando $D(\alpha)\,b\,D^\dagger(\alpha) = b - \alpha$:
	
	\begin{align}
		\widetilde{D}\,(\omega_b b^\dagger b)\,\widetilde{D}^\dagger
		&= \omega_b\left[b^\dagger b
		- \alpha\hat{n}_e(b^\dagger + b) + \alpha^2\hat{n}_e^2\right]
		\notag\\
		&= \omega_b b^\dagger b
		- \lambda\,\hat{n}_e(b^\dagger + b)
		+ \frac{\lambda^2}{\omega_b}\,\hat{n}_e^2,
		\label{eq:D_fon}
	\end{align}
	
	donde usamos $\hat{n}_e^2 = \hat{n}_e$ (proyector) y $\alpha^2\omega_b = \lambda^2/\omega_b$.
	
	\medskip
	\noindent
	\textit{T\'ermino electr\'onico libre.}
	$\widetilde{D}$ conmuta con $\hat{n}_e$, por lo que:
	
	\begin{equation}
		\widetilde{D}\,(\widetilde\Delta\,\hat{n}_e)\,\widetilde{D}^\dagger
		= \widetilde\Delta\,\hat{n}_e.
		\label{eq:D_elec}
	\end{equation}
	
	\medskip
	\noindent
	\textit{T\'ermino de acoplamiento electr\'on-fon\'on.}
	Igualmente conmuta con $\hat{n}_e$:
	
	\begin{equation}
		\widetilde{D}\,\bigl[\lambda\,\hat{n}_e(b^\dagger+b)\bigr]\,\widetilde{D}^\dagger
		= \lambda\,\hat{n}_e\,\widetilde{D}(b^\dagger+b)\widetilde{D}^\dagger
		= \lambda\,\hat{n}_e\bigl(b^\dagger + b - 2\alpha\hat{n}_e\bigr)
		= \lambda\,\hat{n}_e(b^\dagger+b) - \frac{2\lambda^2}{\omega_b}\hat{n}_e.
		\label{eq:D_eph}
	\end{equation}
	
	\medskip
	\noindent
	Sumando~\eqref{eq:D_fon}, \eqref{eq:D_elec} y~\eqref{eq:D_eph}, los
	t\'erminos en $\hat{n}_e(b^\dagger + b)$ se cancelan exactamente y los
	t\'erminos cuadr\'aticos en $\lambda^2/\omega_b$ se combinan:
	
	\begin{equation}
		\widetilde{D}\bigl[\omega_b b^\dagger b + \widetilde\Delta\,\hat{n}_e
		+ \lambda\,\hat{n}_e(b^\dagger+b)\bigr]\widetilde{D}^\dagger
		= \omega_b b^\dagger b
		+ \left(\widetilde\Delta - \frac{\lambda^2}{\omega_b}\right)\hat{n}_e.
		\label{eq:D_libre}
	\end{equation}
	
	\noindent
	El acoplamiento electr\'on-fon\'on queda completamente eliminado. Se
	define la frecuencia de transici\'on renormalizada:
	
	\begin{equation}
		\widetilde\omega_\sigma \equiv \widetilde\Delta - \frac{\lambda^2}{\omega_b}
		= \Delta + J - \frac{\lambda^2}{\omega_b}.
		\label{eq:omega_sigma_tilde}
	\end{equation}
	
	\noindent
	\textbf{Comparaci\'on con Bin.}
	En el caso de un solo QD, Bin obtiene
	$\tilde\omega_\sigma = \omega_\sigma - \lambda^2/\omega_b$
	[Bin~(S12)]. En la mol\'ecula excit\'onica aparece una correcci\'on
	adicional $+J$ debida al acoplamiento de F\"orster, lo que da
	$\widetilde\omega_\sigma = \omega_\sigma - \lambda^2/\omega_b + J - \omega_L$
	en t\'erminos de la frecuencia del l\'aser. Esta es la \emph{primera
		correcci\'on} respecto a Bin en el R\'egimen~II.
	
	\medskip
	\noindent
	\textit{T\'ermino del l\'aser.}
	La transformaci\'on act\'ua no trivialmente sobre $S$ y $S^\dagger$:
	
	\begin{align}
		\widetilde{D}\,S^\dagger\,\widetilde{D}^\dagger
		&= S^\dagger\,\widetilde{D}^2\big|_{\hat{n}_e=0\to1}
		= S^\dagger\,e^{(\lambda/\omega_b)(b^\dagger - b)},
		\label{eq:D_Sdag}\\[4pt]
		\widetilde{D}\,S\,\widetilde{D}^\dagger
		&= S\,e^{-(\lambda/\omega_b)(b^\dagger - b)}.
		\label{eq:D_S}
	\end{align}
	
	\noindent
	Para obtener~\eqref{eq:D_Sdag} usamos que $S^\dagger$ eleva
	$\hat{n}_e$ de $0$ a $1$, de modo que a la derecha de $S^\dagger$ la
	transformaci\'on act\'ua con $\hat{n}_e = 1$ y a la izquierda con
	$\hat{n}_e = 0$:
	$\widetilde{D}\,S^\dagger\,\widetilde{D}^\dagger
	= e^{(\lambda/\omega_b)\cdot 1\cdot(b^\dagger - b)}\,S^\dagger
	\,e^{(\lambda/\omega_b)\cdot 0\cdot(b^\dagger-b)}
	= e^{(\lambda/\omega_b)(b^\dagger-b)}\,S^\dagger$.
	
	\noindent
	Reordenando (el operador fon\'onico conmuta con $S^\dagger$ porque
	actúan sobre espacios de Hilbert distintos):
	
	\begin{equation}
		\widetilde{D}\,\widetilde\Omega(S+S^\dagger)\,\widetilde{D}^\dagger
		= \widetilde\Omega\left[
		S^\dagger\,e^{(\lambda/\omega_b)(b^\dagger-b)}
		+ S\,e^{-(\lambda/\omega_b)(b^\dagger-b)}
		\right].
		\label{eq:D_laser}
	\end{equation}
	
	\medskip
	\noindent
	Reuniendo todos los t\'erminos, el Hamiltoniano transformado es:
	
	\begin{equation}
		\widetilde{H} \equiv \widetilde{D}\,H_\mathrm{eff}\,\widetilde{D}^\dagger
		= \omega_b b^\dagger b
		+ \widetilde\omega_\sigma\,S^\dagger S
		+ \widetilde\Omega\left[
		S^\dagger e^{(\lambda/\omega_b)(b^\dagger - b)}
		+ S\,e^{-(\lambda/\omega_b)(b^\dagger - b)}
		\right],
		\label{eq:H_transformado}
	\end{equation}
	
	\noindent
	que tiene la misma forma que Bin~(S12) con las sustituciones
	$\sigma^\dagger \to S^\dagger$, $\sigma \to S$,
	$\tilde\omega_\sigma^\mathrm{Bin} \to \widetilde\omega_\sigma$,
	y $\Omega \to \widetilde\Omega = \sqrt{2}\,\Omega$.
	Esta es la \emph{segunda correcci\'on} respecto a Bin en el
	R\'egimen~II.
	
	% -----------------------------------------------------------
	% Régimen II — Bloque 1: Expansión del Hamiltoniano de
	%              interacción y condición de resonancia Stokes
	% -----------------------------------------------------------
	
	El Hamiltoniano transformado~\eqref{eq:H_transformado} se separa en
	una parte libre y una de interacci\'on:
	
	\begin{equation}
		\widetilde{H} = \omega_b b^\dagger b + \widetilde\omega_\sigma S^\dagger S
		+ H_\mathrm{int},
		\qquad
		H_\mathrm{int} = \widetilde\Omega\left[
		S^\dagger e^{(\lambda/\omega_b)(b^\dagger - b)}
		+ \mathrm{H.c.}
		\right].
		\label{eq:Hint}
	\end{equation}
	
	\noindent
	El operador exponencial en $H_\mathrm{int}$ mezcla todos los sectores
	de n\'umero fon\'onico. Para identificar los procesos resonantes de
	$n$ fonones, expandimos $e^{(\lambda/\omega_b)(b^\dagger - b)}$ en
	potencias de los operadores de escalera fon\'onicos siguiendo
	Bin~(S13):
	
	\begin{equation}
		e^{(\lambda/\omega_b)(b^\dagger - b)}
		= e^{-\frac{1}{2}(\lambda/\omega_b)^2}
		\sum_{k=0}^\infty \frac{(\lambda/\omega_b)^k b^{\dagger k}}{k!}
		\sum_{l=0}^\infty \frac{(-1)^l(\lambda/\omega_b)^l b^l}{l!},
		\label{eq:exp_expansion}
	\end{equation}
	
	donde el factor gaussiano $e^{-(\lambda/\omega_b)^2/2}$ proviene del
	reordenamiento normal de los operadores de creaci\'on y destrucci\'on
	mediante la identidad de Baker-Campbell-Hausdorff.
	
	\medskip
	\noindent
	El Hamiltoniano de interacci\'on completo es entonces:
	
	\begin{equation}
		H_\mathrm{int} = \widetilde\Omega\,e^{-\frac{1}{2}(\lambda/\omega_b)^2}
		\left[S^\dagger
		\sum_{k=0}^\infty\sum_{l=0}^\infty
		\frac{(-1)^l(\lambda/\omega_b)^{k+l}}{k!\,l!}
		b^{\dagger k} b^l
		+ \mathrm{H.c.}\right].
		\label{eq:Hint_expanded}
	\end{equation}
	
	\medskip
	\noindent
	\textit{Condici\'on de resonancia Stokes.}
	La parte libre $\omega_b b^\dagger b + \widetilde\omega_\sigma S^\dagger S$
	tiene autoenerg\'ias $E_{|m, vv\rangle} = m\omega_b$ y
	$E_{|m, \Psi_+\rangle} = m\omega_b + \widetilde\omega_\sigma$.
	El t\'ermino $S^\dagger b^{\dagger k} b^l$ en~\eqref{eq:Hint_expanded}
	acopla el estado $|m, vv\rangle$ con $|m + k - l,\,\Psi_+\rangle$,
	elevando el n\'umero fon\'onico en $n = k - l$ unidades netas y
	excitando el excit\'on colectivo.
	
	Para el proceso Stokes de $n$ fonones, el t\'ermino resonante
	tiene $k - l = n$, y la condici\'on de degenerac\'ia
	$E_{|0,vv\rangle} = E_{|n,\Psi_+\rangle}$ exige:
	
	\begin{equation}
		0 = n\omega_b + \widetilde\omega_\sigma
		\quad\Longrightarrow\quad
		\widetilde\omega_\sigma = -n\omega_b.
	\end{equation}
	
	\noindent
	Sustituyendo la definici\'on~\eqref{eq:omega_sigma_tilde}:
	
	\begin{equation}
		\Delta + J - \frac{\lambda^2}{\omega_b} = -n\omega_b,
	\end{equation}
	
	\noindent
	de donde:
	
	\begin{equation}
		\boxed{
			\Delta_n = \frac{\lambda^2}{\omega_b} - n\omega_b - J.
		}
		\label{eq:resonancia_II}
	\end{equation}
	
	\noindent
	\textbf{Comparaci\'on con Bin.}
	En el caso de un solo QD, Bin~(S12) obtiene
	$\Delta_n = \lambda^2/\omega_b - n\omega_b$. En la mol\'ecula
	excit\'onica el acoplamiento de F\"orster $J$ desplaza la resonancia
	en $-J$, independientemente de $n$ y de $\lambda$. Esta es la
	primera correcci\'on respecto a Bin en el R\'egimen~II.
	
	\medskip
	\noindent
	\textit{Frecuencia super-Rabi efectiva.}
	En la condici\'on de resonancia~\eqref{eq:resonancia_II},
	el proceso dominante en~\eqref{eq:Hint_expanded} es el que conecta
	$|0, vv\rangle$ con $|n, \Psi_+\rangle$. Para la transici\'on
	$|m, vv\rangle \leftrightarrow |m+n, \Psi_+\rangle$ con $m = 0$,
	extraemos el elemento de matriz del operador
	$e^{(\lambda/\omega_b)(b^\dagger - b)}$ entre los estados fon\'onicos
	$|0\rangle$ y $|n\rangle$:
	
	\begin{align}
		\Omega_\mathrm{eff}^{(n)}
		&= \widetilde\Omega\,\bigl|\langle n|\,
		e^{(\lambda/\omega_b)(b^\dagger - b)}\,|0\rangle\bigr|
		\notag\\
		&= \widetilde\Omega\,
		e^{-\frac{1}{2}(\lambda/\omega_b)^2}
		\left(\frac{\lambda}{\omega_b}\right)^n
		\frac{1}{\sqrt{n!}}\,
		\bigl|L_0^n\bigl((\lambda/\omega_b)^2\bigr)\bigr|,
		\label{eq:Oeff_II_general}
	\end{align}
	
	donde $L_m^n(\varepsilon)$ es el polinomio de Laguerre asociado
	[Bin~(S14)--(S15)], y para $m = 0$:
	
	\begin{equation}
		L_0^n(\varepsilon) = 1.
		\label{eq:Laguerre_m0}
	\end{equation}
	
	\noindent
	Verificamos~\eqref{eq:Laguerre_m0} usando la definici\'on de Bin~(S15):
	$L_0^n(\varepsilon) = \sum_{j=0}^{0}(-1)^j\binom{n}{-j}\varepsilon^j/j!
	= (-1)^0\binom{n}{0}\varepsilon^0/0! = 1$. $\checkmark$
	
	\noindent
	Adicionalmente, el factor $\sqrt{m!/(m+n)!}$ en Bin~(S14) evaluado en
	$m=0$ da $\sqrt{0!/n!} = 1/\sqrt{n!}$, ya incluido
	en~\eqref{eq:Oeff_II_general}. $\checkmark$
	
	\medskip
	\noindent
	Sustituyendo $\widetilde\Omega = \sqrt{2}\,\Omega$
	y $L_0^n = 1$ en~\eqref{eq:Oeff_II_general}:
	
	\begin{equation}
		\boxed{
			\Omega_\mathrm{eff}^{(n)}\big|_{2\mathrm{QD}}
			= \sqrt{2}\,\Omega\,
			\exp\!\left[-\frac{1}{2}\!\left(\frac{\lambda}{\omega_b}\right)^{\!2}\right]
			\left(\frac{\lambda}{\omega_b}\right)^n
			\frac{1}{\sqrt{n!}},
			\qquad
			\Delta_n = \frac{\lambda^2}{\omega_b} - n\omega_b - J.
		}
		\label{eq:resultado_II}
	\end{equation}
	
	\medskip
	\noindent
	\textbf{Comparaci\'on con Bin~(S16).}
	El resultado de Bin para un solo QD es:
	$\Omega_\mathrm{eff}^{(n)}|_\mathrm{1QD}
	= \Omega\,e^{-(\lambda/\omega_b)^2/2}(\lambda/\omega_b)^n/\sqrt{n!}$.
	En la mol\'ecula excit\'onica hay dos correcciones:
	
	\begin{enumerate}
		\item \textit{Factor superradiante $\sqrt{2}$}: proviene de
		$\widetilde\Omega = \sqrt{2}\,\Omega$ y amplifica directamente la
		frecuencia efectiva para todo $n$.
		
		\item \textit{Desplazamiento de resonancia $-J$}: la condici\'on
		$\Delta_n$ se desplaza en $-J$, sin modificar la magnitud de
		$\Omega_\mathrm{eff}^{(n)}$.
	\end{enumerate}
	
	\noindent
	El factor gaussiano $e^{-(\lambda/\omega_b)^2/2}$ y la potencia
	$(\lambda/\omega_b)^n/\sqrt{n!}$ son id\'enticos a Bin: la
	transformaci\'on de desplazamiento genera la misma estructura
	de elementos de matriz de Franck-Condon independientemente de
	que el sistema sea un QD o una mol\'ecula excit\'onica.
	
	\medskip
	\noindent
	\textit{L\'imite $\lambda \to 0$.}
	Cuando $\lambda \to 0$, el factor gaussiano tiende a $1$ y
	$(\lambda/\omega_b)^n \to 0$ para $n \geq 1$, lo que suprime
	la emisi\'on de $n \geq 1$ fonones como es de esperar. Para $n = 0$
	(emisi\'on puramente electr\'onica sin fonones):
	$\Omega_\mathrm{eff}^{(0)} = \widetilde\Omega = \sqrt{2}\,\Omega$,
	que recupera la frecuencia de Rabi colectiva del Bloque~0.
	$\checkmark$

\section{Tercer Régimen}
% -----------------------------------------------------------
% Régimen III — Bloque 0: Hamiltoniano electrónico y
%               estados vestidos
% -----------------------------------------------------------

En el r\'egimen III se tiene $\widetilde\Omega \sim \omega_b$ y
$\lambda \ll \omega_b$. El l\'aser no puede tratarse perturbativamente,
pero el acoplamiento electr\'on-fon\'on s\'i. La estrategia es
diagonalizar primero la parte electr\'onica.

El Hamiltoniano efectivo en el marco rotante, restringido al subespacio
brillante $\{|vv\rangle, |\Psi_+\rangle\}$, es:

\begin{equation}
	H_\mathrm{eff} = \omega_b b^\dagger b
	+ \widetilde\Delta\, S^\dagger S
	+ \lambda\, S^\dagger S\,(b^\dagger + b)
	+ \widetilde\Omega\bigl(S + S^\dagger\bigr),
	\label{eq:Heff_III}
\end{equation}

con $\widetilde\Delta = \Delta + J$ y $\widetilde\Omega = \sqrt{2}\,\Omega$.
Separamos la parte electr\'onica impulsada por el l\'aser:

\begin{equation}
	H_\sigma = \widetilde\Delta\, S^\dagger S
	+ \widetilde\Omega\bigl(S + S^\dagger\bigr).
	\label{eq:Hsigma}
\end{equation}

En la base $\{|vv\rangle, |\Psi_+\rangle\}$, con la convenci\'on
$|vv\rangle \equiv \binom{1}{0}$ y $|\Psi_+\rangle \equiv \binom{0}{1}$:

\begin{equation}
	H_\sigma = \begin{pmatrix} 0 & \widetilde\Omega \\
		\widetilde\Omega & \widetilde\Delta \end{pmatrix}.
	\label{eq:Hsigma_matrix}
\end{equation}

\noindent
Los autovalores se obtienen de $\det(H_\sigma - E\,\mathbb{I}) = 0$:

\begin{equation}
	E^2 - \widetilde\Delta\,E - \widetilde\Omega^2 = 0
	\quad\Longrightarrow\quad
	E_\pm = \frac{\widetilde\Delta \pm \tilde{r}}{2},
	\quad
	\tilde{r} \equiv \sqrt{\widetilde\Delta^2 + 4\widetilde\Omega^2}.
	\label{eq:autovalores}
\end{equation}

\noindent
Para el autovector asociado a $E_+$, la ecuaci\'on
$(H_\sigma - E_+\mathbb{I})\binom{\alpha}{\beta} = 0$ da:

\begin{equation}
	-\frac{\widetilde\Delta - \tilde{r}}{2}\,\alpha + \widetilde\Omega\,\beta = 0
	\quad\Longrightarrow\quad
	\frac{\beta}{\alpha} = \frac{\widetilde\Delta - \tilde{r}}{2\widetilde\Omega}.
	\label{eq:autovector_mas}
\end{equation}

\noindent
Notamos que $\widetilde\Delta - \tilde{r} < 0$ para toda
$\widetilde\Omega > 0$ (ya que $\tilde{r} > |\widetilde\Delta|$),
de modo que $\beta/\alpha < 0$. Normalizando con
$\alpha^2 + \beta^2 = 1$ y eligiendo $\alpha > 0$:

\begin{equation}
	\alpha = \tilde{c}_+, \qquad \beta = -\tilde{c}_-,
\end{equation}

donde:

\begin{equation}
	\tilde{c}_\pm = \sqrt{\frac{2\widetilde\Omega^2}
		{\widetilde\Delta^2 + 4\widetilde\Omega^2
			\pm\, \widetilde\Delta\,\tilde{r}}}.
	\label{eq:cpm}
\end{equation}

\noindent
Verificaci\'on: de~\eqref{eq:autovector_mas},
$\beta/\alpha = (\widetilde\Delta - \tilde{r})/(2\widetilde\Omega)$.
Con $\alpha = \tilde{c}_+$ y $\beta = -\tilde{c}_-$:

\begin{equation}
	\frac{-\tilde{c}_-}{\tilde{c}_+}
	= \frac{\widetilde\Delta - \tilde{r}}{2\widetilde\Omega}
	\quad\Longrightarrow\quad
	\tilde{c}_- = \frac{\tilde{r} - \widetilde\Delta}{2\widetilde\Omega}\,\tilde{c}_+.
	\label{eq:ratio_cpm}
\end{equation}

\noindent
Se verifica que esto es consistente con la definici\'on~\eqref{eq:cpm}
y con $\tilde{c}_+^2 + \tilde{c}_-^2 = 1$.

\medskip
\noindent
Para el autovector de $E_-$: de $(H_\sigma - E_-\mathbb{I})\binom{\gamma}{\delta} = 0$:

\begin{equation}
	\frac{\delta}{\gamma} = \frac{\widetilde\Delta + \tilde{r}}{2\widetilde\Omega} > 0.
\end{equation}

\noindent
Normalizando con $\gamma > 0$: $\gamma = \tilde{c}_-$,
$\delta = \tilde{c}_+$ (usando la misma relaci\'on de
proporcionalidad con signo opuesto). Los estados vestidos son:

\begin{align}
	|{+}\rangle &= \tilde{c}_+\,|vv\rangle - \tilde{c}_-\,|\Psi_+\rangle,
	\qquad E_+ = \frac{\widetilde\Delta + \tilde{r}}{2},
	\label{eq:ket_p}\\[6pt]
	|{-}\rangle &= \tilde{c}_-\,|vv\rangle + \tilde{c}_+\,|\Psi_+\rangle,
	\qquad E_- = \frac{\widetilde\Delta - \tilde{r}}{2}.
	\label{eq:ket_m}
\end{align}

\noindent
Verificaciones:
\begin{itemize}
	\item Ortogonalidad:
	$\langle+|-\rangle = \tilde{c}_+\tilde{c}_- - \tilde{c}_-\tilde{c}_+ = 0$.
	$\checkmark$
	
	\item Autovector $|+\rangle$:
	$H_\sigma|+\rangle = \tilde{c}_+\,\widetilde\Omega|\Psi_+\rangle
	+ \tilde{c}_-\,(-\widetilde\Omega|vv\rangle - \widetilde\Delta|\Psi_+\rangle)
	= -\tilde{c}_-\widetilde\Omega\,|vv\rangle
	+ (\tilde{c}_+\widetilde\Omega - \tilde{c}_-\widetilde\Delta)|\Psi_+\rangle$.
	
	Debe ser igual a $E_+|+\rangle = \tilde{c}_+E_+|vv\rangle - \tilde{c}_-E_+|\Psi_+\rangle$.
	De la componente en $|vv\rangle$: $-\tilde{c}_-\widetilde\Omega = \tilde{c}_+E_+$,
	es decir $\tilde{c}_-/\tilde{c}_+ = -E_+/\widetilde\Omega
	= -(\widetilde\Delta+\tilde{r})/(2\widetilde\Omega)$.
	Pero de~\eqref{eq:ratio_cpm} tenemos
	$\tilde{c}_-/\tilde{c}_+ = (\tilde{r}-\widetilde\Delta)/(2\widetilde\Omega)$.
	Estos son opuestos, lo que indica una inconsistencia en el signo
	de~\eqref{eq:ket_p}.
\end{itemize}

\medskip
\noindent
Corregimos los signos. De~\eqref{eq:autovector_mas},
$\beta = [(\widetilde\Delta-\tilde{r})/(2\widetilde\Omega)]\,\alpha$.
Como $\widetilde\Delta - \tilde{r} < 0$, el coeficiente de
$|\Psi_+\rangle$ tiene signo opuesto al de $|vv\rangle$, luego:

\begin{equation}
	|+\rangle = \tilde{c}_+\,|vv\rangle - \tilde{c}_-\,|\Psi_+\rangle
\end{equation}

\noindent
con $\tilde{c}_-, \tilde{c}_+ > 0$ y la relaci\'on de proporcionalidad
$\tilde{c}_-/\tilde{c}_+ = (\tilde{r}-\widetilde\Delta)/(2\widetilde\Omega)$.

Verificaci\'on autovalor:
$H_\sigma|+\rangle$: componente en $|vv\rangle$: $-\tilde{c}_-\widetilde\Omega$.
Debe ser $E_+\tilde{c}_+$. Entonces:
$-\tilde{c}_-\widetilde\Omega = E_+\tilde{c}_+ \Longrightarrow
\tilde{c}_-/\tilde{c}_+ = -E_+/\widetilde\Omega
= -(\widetilde\Delta+\tilde{r})/(2\widetilde\Omega)$,
que es negativo. Contradicci\'on con $\tilde{c}_\pm > 0$.

El problema es que la convenci\'on de Bin define $c_-$ con signo
expl\'icito en el autovector:

\begin{equation}
	|+\rangle = c_+|v\rangle + c_-|c\rangle, \qquad
	|-\rangle = c_-|v\rangle - c_+|c\rangle,
\end{equation}

donde $c_\pm > 0$ y la relaci\'on es
$c_-/c_+ = E_+/\Omega > 0$ (positiva).
Esto es posible porque en Bin $E_+ > 0$ cuando $\Delta < 0$.

\medskip
\noindent
Para la mol\'ecula excit\'onica, adoptamos la misma convenci\'on que
Bin~(S18), con las sustituciones $|v\rangle \to |vv\rangle$,
$|c\rangle \to |\Psi_+\rangle$, $\Delta \to \widetilde\Delta$,
$\Omega \to \widetilde\Omega$:

\begin{align}
	|{+}\rangle &= \tilde{c}_+\,|vv\rangle + \tilde{c}_-\,|\Psi_+\rangle,
	\qquad E_+ = \frac{\widetilde\Delta + \tilde{r}}{2},
	\label{eq:ket_p_final}\\[6pt]
	|{-}\rangle &= \tilde{c}_-\,|vv\rangle - \tilde{c}_+\,|\Psi_+\rangle,
	\qquad E_- = \frac{\widetilde\Delta - \tilde{r}}{2},
	\label{eq:ket_m_final}
\end{align}

\noindent
con coeficientes de Bin~(S19) extendidos:

\begin{equation}
	\tilde{c}_\pm = \sqrt{\frac{2\widetilde\Omega^2}
		{\widetilde\Delta^2 + 4\widetilde\Omega^2
			\pm\,\widetilde\Delta\,\tilde{r}}},
	\qquad \tilde{c}_+^2 + \tilde{c}_-^2 = 1.
	\label{eq:cpm_final}
\end{equation}

\noindent
Verificaci\'on autovector $|+\rangle$:
$H_\sigma|+\rangle$: componente en $|vv\rangle$: $\tilde{c}_-\widetilde\Omega$.
Debe ser $E_+\tilde{c}_+$. Entonces
$\tilde{c}_-\widetilde\Omega = E_+\tilde{c}_+$, es decir
$\tilde{c}_-/\tilde{c}_+ = E_+/\widetilde\Omega
= (\widetilde\Delta+\tilde{r})/(2\widetilde\Omega) > 0$. $\checkmark$

Componente en $|\Psi_+\rangle$:
$\tilde{c}_+\widetilde\Omega + \tilde{c}_-\widetilde\Delta
= E_+\tilde{c}_- = \tilde{c}_-(\widetilde\Delta+\tilde{r})/2$.
Sustituyendo $\tilde{c}_+\widetilde\Omega = \tilde{c}_-\tilde{r}/2 - \tilde{c}_-\widetilde\Delta/2 + \tilde{c}_-\widetilde\Delta = \tilde{c}_-(\tilde{r}+\widetilde\Delta)/2$. $\checkmark$

% -----------------------------------------------------------
% Régimen III — Bloque 1: Inversión de los estados vestidos,
%   operador S†S en la base vestida, elementos de matriz de H_eph
% -----------------------------------------------------------

\textit{Inversión de los estados vestidos.}
Multiplicando la ec.~\eqref{eq:ket_p_final} por $\tilde{c}_+$,
la ec.~\eqref{eq:ket_m_final} por $\tilde{c}_-$ y sumando; luego
multiplicando por $\tilde{c}_-$ y $\tilde{c}_+$ respectivamente y
restando, con $\tilde{c}_+^2+\tilde{c}_-^2 = 1$:

\begin{align}
	|vv\rangle    &= \tilde{c}_+|{+}\rangle + \tilde{c}_-|{-}\rangle,
	\label{eq:vv_dressed}\\[4pt]
	|\Psi_+\rangle &= \tilde{c}_-|{+}\rangle - \tilde{c}_+|{-}\rangle.
	\label{eq:Psi_dressed}
\end{align}

\medskip
\noindent
\textit{Operador $S^\dagger S$ en la base vestida.}
La base vestida diagonaliza $H_\sigma$, pero no $S^\dagger S$
individualmente (pues $H_\sigma$ contiene $S^\dagger S$ y
$S+S^\dagger$, que no conmutan). Usando
$S^\dagger S = |\Psi_+\rangle\langle\Psi_+|$
y la ec.~\eqref{eq:Psi_dressed}:

\begin{equation}
	S^\dagger S
	= \bigl(\tilde{c}_-|{+}\rangle - \tilde{c}_+|{-}\rangle\bigr)
	\bigl(\tilde{c}_-\langle{+}| - \tilde{c}_+\langle{-}|\bigr),
\end{equation}

de donde los elementos de matriz en la base vestida son:

\begin{equation}
	\langle{+}|S^\dagger S|{+}\rangle = \tilde{c}_-^2, \quad
	\langle{-}|S^\dagger S|{-}\rangle = \tilde{c}_+^2, \quad
	\langle{+}|S^\dagger S|{-}\rangle = \langle{-}|S^\dagger S|{+}\rangle
	= -\tilde{c}_+\tilde{c}_-.
	\label{eq:SdagS_matrix}
\end{equation}

\noindent
La presencia de t\'erminos fuera de la diagonal $-\tilde{c}_+\tilde{c}_-$
es exacta y esperada: refleja que $S^\dagger S$ no conmuta con
$H_\sigma$.

\medskip
\noindent
\textit{Elementos de matriz de $H_\mathrm{e\text{-}ph}$.}
El Hamiltoniano completo~\eqref{eq:Heff_III} se separa en:

\begin{equation}
	H_\mathrm{eff} = H_0 + H_\mathrm{e\text{-}ph},
	\quad
	H_0 = \omega_b b^\dagger b + H_\sigma,
	\quad
	H_\mathrm{e\text{-}ph} = \lambda\,S^\dagger S\,(b^\dagger + b).
	\label{eq:H0_Heph}
\end{equation}

$H_0$ es diagonal en la base de productos $|n,\pm\rangle$:

\begin{equation}
	H_0|n,\pm\rangle = \bigl(n\omega_b + E_\pm\bigr)|n,\pm\rangle,
	\label{eq:H0_diagonal}
\end{equation}

y $H_\mathrm{e\text{-}ph}$ es el t\'ermino perturbativo ($\lambda \ll \omega_b$).
Usando~\eqref{eq:SdagS_matrix} y
$\langle m|b^\dagger+b|n\rangle = \sqrt{n+1}\,\delta_{m,n+1}+\sqrt{n}\,\delta_{m,n-1}$,
los elementos de matriz no nulos de $H_\mathrm{e\text{-}ph}$ son:

\begin{align}
	\langle n{+}1,{+}|H_\mathrm{e\text{-}ph}|n,{+}\rangle
	&= +\tilde{c}_-^2\,\lambda\sqrt{n+1},
	\label{eq:Heph_pp}\\[4pt]
	\langle n{+}1,{-}|H_\mathrm{e\text{-}ph}|n,{-}\rangle
	&= +\tilde{c}_+^2\,\lambda\sqrt{n+1},
	\label{eq:Heph_mm}\\[4pt]
	\langle n{+}1,{-}|H_\mathrm{e\text{-}ph}|n,{+}\rangle
	&= -\tilde{c}_+\tilde{c}_-\,\lambda\sqrt{n+1},
	\label{eq:Heph_mp}\\[4pt]
	\langle n{+}1,{+}|H_\mathrm{e\text{-}ph}|n,{-}\rangle
	&= -\tilde{c}_+\tilde{c}_-\,\lambda\sqrt{n+1},
	\label{eq:Heph_pm}
\end{align}

\noindent
y los correspondientes con $b$ en lugar de $b^\dagger$
(con factor $\sqrt{n}$ y $n\to n-1$).

\medskip
\noindent
Los acoplamientos~\eqref{eq:Heph_pp}--\eqref{eq:Heph_mm} preservan
el \'indice vestido; los~\eqref{eq:Heph_mp}--\eqref{eq:Heph_pm}
lo invierten con amplitud $-\tilde{c}_+\tilde{c}_-\lambda$.
Son estos \'ultimos los que median, a orden $\lambda^n$, el proceso
de $n$ fonones $|0,{+}\rangle \leftrightarrow |n,{-}\rangle$.

% -----------------------------------------------------------
% Régimen III — Bloque 2: Condición de resonancia Stokes
%               y energías relativas
% -----------------------------------------------------------

\textit{Energías de los estados vestidos con fonones.}
Ignorando provisionalmente $H_\mathrm{e\text{-}ph}$, los autoestados
del Hamiltoniano libre $H_0 = \omega_b b^\dagger b + H_\sigma$ son
los productos $|n,\pm\rangle \equiv |n\rangle\otimes|\pm\rangle$
con energías [ec.~\eqref{eq:H0_diagonal}]:

\begin{equation}
	E_{|n,+\rangle} = n\omega_b + \frac{\widetilde\Delta + \tilde{r}}{2},
	\qquad
	E_{|n,-\rangle} = n\omega_b + \frac{\widetilde\Delta - \tilde{r}}{2}.
	\label{eq:energias_npm}
\end{equation}

\medskip
\noindent
\textit{Condición de resonancia Stokes de $n$ fonones.}
El proceso Stokes de $n$ fonones conecta $|0,{+}\rangle$
con $|n,{-}\rangle$ mediante una cadena de transiciones mediadas por
$H_\mathrm{e\text{-}ph}$. La teoría de perturbaciones que derivaremos
en los bloques siguientes es válida únicamente cuando estos dos
estados son degenerados. Igualando sus energías:

\begin{equation}
	E_{|0,+\rangle} = E_{|n,-\rangle}
	\quad\Longrightarrow\quad
	\frac{\widetilde\Delta + \tilde{r}}{2}
	= n\omega_b + \frac{\widetilde\Delta - \tilde{r}}{2},
\end{equation}

\noindent
que simplifica a:

\begin{equation}
	\tilde{r} = n\omega_b.
	\label{eq:cond_res_r}
\end{equation}

\noindent
Sustituyendo $\tilde{r} = \sqrt{\widetilde\Delta^2 + 4\widetilde\Omega^2}$
con $\widetilde\Delta = \Delta + J$ y $\widetilde\Omega = \sqrt{2}\,\Omega$:

\begin{equation}
	(\Delta + J)^2 + 8\Omega^2 = n^2\omega_b^2,
\end{equation}

\noindent
de donde, tomando la raíz negativa que corresponde al proceso Stokes
($\Delta < 0$):

\begin{equation}
	\boxed{
		\Delta_n = -\sqrt{n^2\omega_b^2 - 8\Omega^2} - J.
	}
	\label{eq:resonancia_III}
\end{equation}

\noindent
La condición de validez es $n\omega_b > 2\sqrt{2}\,\Omega$.
En el límite $\Omega \to 0$: $\Delta_n \to -n\omega_b - J$,
consistente con el Régimen~I, ec.~\eqref{eq:resonancia_I}. $\checkmark$

\medskip
\noindent
\textbf{Comparación con Bin.}
En el caso de un solo QD, Bin obtiene
$\Delta_n = -\sqrt{n^2\omega_b^2 - 4\Omega^2}$.
La molécula excitónica introduce dos correcciones:
(1) $4\Omega^2 \to 8\Omega^2$ por el factor superradiante
$\widetilde\Omega = \sqrt{2}\,\Omega$; (2) desplazamiento rígido
$-J$ por el acoplamiento de Förster en $\widetilde\Delta$.

\medskip
\noindent
\textit{Energías relativas de los estados intermedios.}
En resonancia $\tilde{r} = n\omega_b$, las energías de todos los
estados del espacio truncado, medidas respecto a la energía degenerada
$E_0 \equiv E_{|0,+\rangle} = E_{|n,-\rangle}$, son:

\begin{align}
	E_{|0,+\rangle} - E_0 &= 0,
	\label{eq:E0p_rel}\\[4pt]
	E_{|n,-\rangle} - E_0 &= 0,
	\label{eq:Enm_rel}\\[4pt]
	E_{|k,+\rangle} - E_0
	&= k\omega_b + \frac{\widetilde\Delta + \tilde{r}}{2}
	- \frac{\widetilde\Delta + \tilde{r}}{2}
	= k\omega_b,
	\quad 1 \leq k \leq n-1,
	\label{eq:Ekp_rel}\\[4pt]
	E_{|k,-\rangle} - E_0
	&= k\omega_b + \frac{\widetilde\Delta - \tilde{r}}{2}
	- \frac{\widetilde\Delta + \tilde{r}}{2}
	= k\omega_b - \tilde{r}
	= -(n-k)\omega_b,
	\quad 1 \leq k \leq n-1.
	\label{eq:Ekm_rel}
\end{align}

\noindent
Los estados intermedios $|k,+\rangle$ tienen energía relativa
$+k\omega_b > 0$ y los $|k,-\rangle$ tienen energía relativa
$-(n-k)\omega_b < 0$. Ambos están separados de los estados
degenerados por múltiplos enteros de $\omega_b$, lo que garantiza
que la eliminación adiabática de los estados intermedios es válida
cuando $\lambda \ll \omega_b$.

\medskip
\noindent
Los denominadores que aparecerán en la eliminación adiabática son:

\begin{equation}
	-E_{|k,+\rangle} + E_0 = -k\omega_b < 0,
	\qquad
	-E_{|k,-\rangle} + E_0 = (n-k)\omega_b > 0,
	\label{eq:denominadores}
\end{equation}

\noindent
todos no nulos para $1 \leq k \leq n-1$, lo que confirma que
la teoría de perturbaciones es no singular en el subespacio
intermedio. Estos son los denominadores que entrarán en la fórmula
de eliminación adiabática en el bloque siguiente.

% -----------------------------------------------------------
% Régimen III — Bloque 3: Forma matricial en bloques, caso n=2
% -----------------------------------------------------------

\textit{Espacio truncado para $n=2$.}
El espacio de estados vestidos relevante para el proceso de 2~fonones es:

\begin{equation}
	\mathcal{H}^{(2)} = \bigl\{
	|0,{+}\rangle,\;|1,{-}\rangle,\;|1,{+}\rangle,\;|2,{-}\rangle
	\bigr\}.
	\label{eq:espacio_n2}
\end{equation}

\noindent
Los estados degenerados son $|0,{+}\rangle$ y $|2,{-}\rangle$
(energía relativa $0$); los intermedios son $|1,{-}\rangle$
(energía relativa $-\omega_b$) y $|1,{+}\rangle$
(energía relativa $+\omega_b$).

\medskip
\noindent
\textit{Elementos de matriz.}
El Hamiltoniano en $\mathcal{H}^{(2)}$ tiene parte diagonal dada
por~\eqref{eq:energias_npm} y parte fuera de la diagonal dada
por $H_\mathrm{e\text{-}ph}$, ecs.~\eqref{eq:Heph_pp}--\eqref{eq:Heph_pm}.
Trabajamos en un marco que rota con $E_0$, restando esta energía
a todos los elementos diagonales. Los elementos no nulos son:

\begin{align}
	\langle 0,{+}|H|0,{+}\rangle &= 0,
	\label{eq:H_00pp}\\[4pt]
	\langle 1,{-}|H|1,{-}\rangle &= \omega_b - \tilde{r}
	= \omega_b - 2\omega_b = -\omega_b,
	\label{eq:H_11mm}\\[4pt]
	\langle 1,{+}|H|1,{+}\rangle &= \omega_b,
	\label{eq:H_11pp}\\[4pt]
	\langle 2,{-}|H|2,{-}\rangle &= 2\omega_b - \tilde{r}
	= 2\omega_b - 2\omega_b = 0,
	\label{eq:H_22mm}
\end{align}

\noindent
donde en~\eqref{eq:H_11mm} y~\eqref{eq:H_22mm} usamos la condición
de resonancia $\tilde{r} = 2\omega_b$.

\medskip
\noindent
Para los elementos fuera de la diagonal, usamos los
acoplamientos~\eqref{eq:Heph_mp}--\eqref{eq:Heph_pm} con el número
de fonones apropiado:

\begin{align}
	\langle 1,{-}|H_\mathrm{e\text{-}ph}|0,{+}\rangle
	&= -\tilde{c}_+\tilde{c}_-\,\lambda\sqrt{1}
	= -\tilde{c}_+\tilde{c}_-\lambda,
	\label{eq:H_10mp}\\[4pt]
	\langle 1,{+}|H_\mathrm{e\text{-}ph}|0,{+}\rangle
	&= +\tilde{c}_-^2\,\lambda\sqrt{1}
	= \tilde{c}_-^2\lambda,
	\label{eq:H_10pp}\\[4pt]
	\langle 2,{-}|H_\mathrm{e\text{-}ph}|1,{-}\rangle
	&= +\tilde{c}_+^2\,\lambda\sqrt{2}
	= \sqrt{2}\,\tilde{c}_+^2\lambda,
	\label{eq:H_21mm}\\[4pt]
	\langle 2,{-}|H_\mathrm{e\text{-}ph}|1,{+}\rangle
	&= -\tilde{c}_+\tilde{c}_-\,\lambda\sqrt{2}
	= -\sqrt{2}\,\tilde{c}_+\tilde{c}_-\lambda.
	\label{eq:H_21mp}
\end{align}

\noindent
El acoplamiento directo $\langle 2,{-}|H_\mathrm{e\text{-}ph}|0,{+}\rangle = 0$
porque requeriría cambiar el número fonónico en 2 en un solo paso,
lo cual es cero al primer orden en $H_\mathrm{e\text{-}ph}$.

\medskip
\noindent
\textit{Forma matricial en bloques.}
En el ordenamiento
$\{|0,{+}\rangle, |2,{-}\rangle, |1,{-}\rangle, |1,{+}\rangle\}$
(primero los estados degenerados, luego los intermedios), el
Hamiltoniano toma la forma en bloques:

\begin{equation}
	H^{(2)} = \begin{pmatrix}
		\mathbf{H}^{(2)} & X^{(2)T} \\[4pt]
		X^{(2)} & R^{(2)}
	\end{pmatrix},
	\label{eq:H2_bloques}
\end{equation}

con:

\begin{equation}
	\mathbf{H}^{(2)} = \begin{pmatrix} 0 & 0 \\ 0 & 0 \end{pmatrix},
	\label{eq:Hsub2}
\end{equation}

\begin{equation}
	X^{(2)} = \begin{pmatrix}
		\langle 1,{-}|H|0,{+}\rangle & \langle 1,{-}|H|2,{-}\rangle \\[4pt]
		\langle 1,{+}|H|0,{+}\rangle & \langle 1,{+}|H|2,{-}\rangle
	\end{pmatrix}
	= \begin{pmatrix}
		-\tilde{c}_+\tilde{c}_-\lambda & \sqrt{2}\,\tilde{c}_+^2\lambda \\[4pt]
		\tilde{c}_-^2\lambda & -\sqrt{2}\,\tilde{c}_+\tilde{c}_-\lambda
	\end{pmatrix},
	\label{eq:X2}
\end{equation}

\begin{equation}
	R^{(2)} = \begin{pmatrix}
		\langle 1,{-}|H|1,{-}\rangle & \langle 1,{-}|H|1,{+}\rangle \\[4pt]
		\langle 1,{+}|H|1,{-}\rangle & \langle 1,{+}|H|1,{+}\rangle
	\end{pmatrix}
	= \begin{pmatrix}
		-\omega_b & 0 \\[4pt]
		0 & \omega_b
	\end{pmatrix}.
	\label{eq:R2}
\end{equation}

\noindent
Nótese que $\langle 1,{-}|H_\mathrm{e\text{-}ph}|1,{+}\rangle = 0$
porque el acoplamiento entre $|1,{-}\rangle$ y $|1,{+}\rangle$
requeriría $\Delta n = 0$ con cambio de índice vestido, lo cual
no ocurre en $H_\mathrm{e\text{-}ph} \propto (b^\dagger + b)$ que
cambia el número fonónico en $\pm 1$. Por tanto $R^{(2)}$ es
diagonal, con autovalores $-\omega_b$ y $+\omega_b$.

\medskip
\noindent
Explicitando la matriz completa en el ordenamiento original
$\{|0,{+}\rangle, |1,{-}\rangle, |1,{+}\rangle, |2,{-}\rangle\}$
(que es el convenio de Bin~(S22)), con $\tilde{r} = 2\omega_b$:

\begin{equation}
	H^{(2)} = \begin{pmatrix}
		0
		& -\tilde{c}_+\tilde{c}_-\lambda
		& \tilde{c}_-^2\lambda
		& 0 \\[6pt]
		-\tilde{c}_+\tilde{c}_-\lambda
		& -\omega_b
		& 0
		& \sqrt{2}\,\tilde{c}_+^2\lambda \\[6pt]
		\tilde{c}_-^2\lambda
		& 0
		& \omega_b
		& -\sqrt{2}\,\tilde{c}_+\tilde{c}_-\lambda \\[6pt]
		0
		& \sqrt{2}\,\tilde{c}_+^2\lambda
		& -\sqrt{2}\,\tilde{c}_+\tilde{c}_-\lambda
		& 0
	\end{pmatrix},
	\label{eq:H2_full}
\end{equation}

\noindent
idéntico a Bin~(S22) bajo las sustituciones $c_\pm \to \tilde{c}_\pm$
y $r \to \tilde{r}$. $\checkmark$

\medskip
\noindent
En~\eqref{eq:H2_bloques} la matriz de acoplamiento $X^{(2)}$
conecta el subespacio degenerado $\{|0,{+}\rangle, |2,{-}\rangle\}$
con el subespacio intermedio $\{|1,{-}\rangle, |1,{+}\rangle\}$,
y $R^{(2)}$ es la parte del Hamiltoniano restringida al subespacio
intermedio. Como $R^{(2)}$ es diagonal con autovalores $\pm\omega_b$,
su inversión es inmediata.

\medskip
\noindent
\textit{Eliminación adiabática para $n=2$.}
Siguiendo Bin~(S6), el Hamiltoniano efectivo en el subespacio
degenerado es:

\begin{equation}
	H_\mathrm{eff}^{(2)} = \mathbf{H}^{(2)} - X^{(2)T}\bigl[R^{(2)}\bigr]^{-1}X^{(2)},
	\label{eq:Heff_formula}
\end{equation}

\noindent
donde $X^{(2)}$ tiene filas indexadas por los estados intermedios
y columnas por los degenerados [ec.~\eqref{eq:X2}]. Con
$\mathbf{H}^{(2)} = 0$ y $[R^{(2)}]^{-1} = \mathrm{diag}(-1/\omega_b,\, 1/\omega_b)$,
el elemento $(1,2)$ de $H_\mathrm{eff}^{(2)}$ da:

\begin{equation}
	\Omega_\mathrm{eff}^{(2)}\big|_{2\mathrm{QD}}
	= \frac{-\sqrt{2}\,\tilde{c}_+\tilde{c}_-\lambda^2
		\bigl(\omega_b - \tilde{c}_-^2\,\tilde{r}\bigr)}
	{\omega_b(\tilde{r}-\omega_b)},
	\label{eq:Oeff2_III}
\end{equation}

\noindent
idéntico a Bin~(S25) bajo $c_\pm \to \tilde{c}_\pm$, $r \to \tilde{r}$,
verificado analíticamente con $\tilde{c}_+^2+\tilde{c}_-^2 = 1$. $\checkmark$

% -----------------------------------------------------------
% Régimen III — Bloque 4: Espacio reducido de cuatro estados
%               y relación de recurrencia
% -----------------------------------------------------------

\textit{Estrategia recursiva.}
Para derivar $\Omega_\mathrm{eff}^{(n)}$ a orden general en $n$,
adoptamos el procedimiento recursivo de Bin~(S26)--(S27). La idea
es que si ya se conoce el acoplamiento efectivo
$\Omega_\mathrm{eff}^{(n-1)}$ entre $|0,{+}\rangle$ y
$|n-1,{-}\rangle$, entonces $\Omega_\mathrm{eff}^{(n)}$ puede
obtenerse considerando únicamente el espacio reducido de cuatro
estados
$\{|0,{+}\rangle,\,|n,{-}\rangle,\,|n-1,{+}\rangle,\,|n-1,{-}\rangle\}$.

\medskip
\noindent
\textit{Hamiltoniano en el espacio reducido de cuatro estados (antes de transformar).}
En este espacio, con energías relativas a $E_0$ calculadas
en~\eqref{eq:Ekp_rel}--\eqref{eq:Ekm_rel}, el Hamiltoniano
tiene la forma:

\begin{equation}
	H^{(n)}_0 = \begin{pmatrix}
		0 & 0 & 0 & \Omega_\mathrm{eff}^{(n-1)} \\[4pt]
		0 & 0 & -\tilde{c}_+\tilde{c}_-\lambda\sqrt{n} &
		\tilde{c}_+^2\lambda\sqrt{n} \\[4pt]
		0 & -\tilde{c}_+\tilde{c}_-\lambda\sqrt{n} & (n-1)\omega_b & 0 \\[4pt]
		\Omega_\mathrm{eff}^{(n-1)} & \tilde{c}_+^2\lambda\sqrt{n} & 0 & -\omega_b
	\end{pmatrix},
	\label{eq:Hn0_4estados}
\end{equation}

\noindent
en el ordenamiento
$\{|0,{+}\rangle,\,|n,{-}\rangle,\,|n-1,{+}\rangle,\,|n-1,{-}\rangle\}$.
Los acoplamientos fuera de la diagonal provienen directamente de
$H_\mathrm{e\text{-}ph}$, ecs.~\eqref{eq:Heph_mp}--\eqref{eq:Heph_mm}:
el elemento $-\tilde{c}_+\tilde{c}_-\lambda\sqrt{n}$ invierte
el índice vestido, y $\tilde{c}_+^2\lambda\sqrt{n}$ lo conserva.

\medskip
\noindent
\textit{Transformación de Bin~(S23).}
El bloque intermedio de~\eqref{eq:Hn0_4estados} tiene
$R^{(n)} = \mathrm{diag}\bigl((n-1)\omega_b,\,-\omega_b\bigr)$,
que sí es diagonal. Sin embargo, el acoplamiento de $|n-1,{+}\rangle$
con $|n,{-}\rangle$ es $-\tilde{c}_+\tilde{c}_-\lambda\sqrt{n} \neq 0$
mientras que su acoplamiento con $|0,{+}\rangle$ es cero. Esto
significa que $|n-1,{+}\rangle$ no contribuye directamente a
$\Omega_\mathrm{eff}^{(n)}$ a través de la fórmula perturbativa
de Bin~(S6), pero sí acopla a $|n-1,{-}\rangle$ a través de
la cadena completa de estados intermedios del espacio de $2n$ estados.

Siguiendo Bin~(S23)--(S24), introducimos el estado efectivo:

\begin{equation}
	|n-1,{+}\rangle_\mathrm{eff}
	= |n-1,{+}\rangle
	+ \frac{\tilde{c}_-}{\tilde{c}_+}\,|n-1,{-}\rangle,
	\label{eq:estado_eff}
\end{equation}

que al orden dominante en $\lambda/\omega_b$ diagonaliza el acoplamiento
cruzado entre los estados intermedios. Bajo esta transformación,
el Hamiltoniano reducido toma la forma de Bin~(S26) con
$c_\pm \to \tilde{c}_\pm$ y $r \to \tilde{r}$:

\begin{equation}
	H^{(n)} = \begin{pmatrix}
		0 & 0 & 0 & \Omega_\mathrm{eff}^{(n-1)} \\[4pt]
		0 & n\omega_b-\tilde{r} & 0 & \tilde{c}_+^2\lambda\sqrt{n} \\[4pt]
		0 & 0 &
		\left(\dfrac{\tilde{c}_-}{\tilde{c}_+}\right)^{\!2}
		\!\!\bigl[(n-1)\omega_b-\tilde{r}\bigr]+(n-1)\omega_b &
		\dfrac{\tilde{c}_-}{\tilde{c}_+}\bigl[(n-1)\omega_b-\tilde{r}\bigr] \\[10pt]
		\Omega_\mathrm{eff}^{(n-1)} & \tilde{c}_+^2\lambda\sqrt{n} &
		\dfrac{\tilde{c}_-}{\tilde{c}_+}\bigl[(n-1)\omega_b-\tilde{r}\bigr] &
		(n-1)\omega_b-\tilde{r}
	\end{pmatrix}.
	\label{eq:Hn_S26}
\end{equation}

\noindent
En resonancia $\tilde{r} = n\omega_b$, los elementos diagonales
de los estados intermedios se simplifican a:

\begin{equation}
	H^{(n)}_{33} = (n-1)\omega_b - n\omega_b = -\omega_b,
	\qquad
	H^{(n)}_{22} =
	\left(\frac{\tilde{c}_-}{\tilde{c}_+}\right)^{\!2}(-\omega_b)
	+ (n-1)\omega_b,
	\label{eq:diag_intermedios}
\end{equation}

\noindent
y los acoplamientos fuera de la diagonal del bloque intermedio son:

\begin{equation}
	H^{(n)}_{23} = H^{(n)}_{32}
	= \frac{\tilde{c}_-}{\tilde{c}_+}(-\omega_b)
	= -\frac{\tilde{c}_-}{\tilde{c}_+}\omega_b.
	\label{eq:offdiag_intermedios}
\end{equation}

\medskip
\noindent
\textit{Relación de recurrencia.}
Identificamos los bloques en~\eqref{eq:Hn_S26} con el subespacio
de interés $\{|0,{+}\rangle,|n,{-}\rangle\}$ (índices 0 y 1)
y el subespacio intermedio $\{|n-1,{+}\rangle_\mathrm{eff},
|n-1,{-}\rangle\}$ (índices 2 y 3):

\begin{equation}
	X^{(n)} = \begin{pmatrix}
		0 &
		\dfrac{\tilde{c}_-}{\tilde{c}_+}\bigl[(n-1)\omega_b - \tilde{r}\bigr] \\[8pt]
		\Omega_\mathrm{eff}^{(n-1)} & \tilde{c}_+^2\lambda\sqrt{n}
	\end{pmatrix},
	\quad
	R^{(n)} = \begin{pmatrix}
		H^{(n)}_{22} & H^{(n)}_{23} \\[4pt]
		H^{(n)}_{32} & H^{(n)}_{33}
	\end{pmatrix}.
	\label{eq:XnRn}
\end{equation}

\noindent
En resonancia, $X^{(n)}_{12} = \tilde{c}_-/\tilde{c}_+ \cdot (-\omega_b)$
y $X^{(n)}_{11} = 0$. Aplicando la fórmula perturbativa de Bin~(S6),
$\Omega_\mathrm{eff}^{(n)} = -\sum_k X^{(n)}_{k1} X^{(n)}_{k2}/R^{(n)}_{kk}$,
y notando que $R^{(n)}$ tiene un acoplamiento fuera de la diagonal
$H^{(n)}_{23}$ que contribuye al orden dominante en $\lambda/\omega_b$,
el resultado al orden dominante es idéntico a Bin~(S27):

\begin{equation}
	\boxed{
		\Omega_\mathrm{eff}^{(n)}\big|_{2\mathrm{QD}}
		= \frac{\sqrt{n}\,\lambda\bigl[(n-1)\omega_b
			- \tilde{c}_-^2\,\tilde{r}\,\bigr]}
		{(n-1)\omega_b\,\bigl[\tilde{r} - (n-1)\omega_b\bigr]}
		\;\Omega_\mathrm{eff}^{(n-1)},
	}
	\label{eq:recurrencia_III}
\end{equation}

\noindent
válido al orden dominante en $\lambda/\omega_b$, con
$\tilde{c}_\pm$ dados por~\eqref{eq:cpm_final} y evaluados
en la resonancia de $n$ fonones~\eqref{eq:resonancia_III}.

\medskip
\noindent
\textit{Verificación numérica.}
Para $n=3$, el resultado del espacio completo de seis estados
en el límite $\lambda/\omega_b \to 0$ coincide con la recurrencia
de Bin~(S27) al orden $\lambda^3$. Las correcciones subleading
son de orden $(\lambda/\omega_b)^2$ relativo al resultado
dominante, despreciables en el régimen de interés
$\lambda \ll \omega_b$. $\checkmark$

% -----------------------------------------------------------
% Régimen III — Bloque 5: Resultado final Omega_eff^(n) para el 2QD
% -----------------------------------------------------------

\textit{Iteración de la recurrencia.}
Partiendo de $\Omega_\mathrm{eff}^{(2)}$ dado por la
ec.~\eqref{eq:Oeff2_III} y aplicando iterativamente la
recurrencia~\eqref{eq:recurrencia_III}, se obtiene
$\Omega_\mathrm{eff}^{(n)}$ para todo $n \geq 2$.
La evaluación es siempre en la resonancia del proceso de
$n$ fonones, $\tilde{r} = n\omega_b$.

\medskip
\noindent
El factor de cada paso es:

\begin{equation}
	\frac{\Omega_\mathrm{eff}^{(k)}}{\Omega_\mathrm{eff}^{(k-1)}}
	= \frac{\sqrt{k}\,\lambda\bigl[(k-1)\omega_b
		- \tilde{c}_-^2\,n\omega_b\bigr]}
	{(k-1)\omega_b\,\bigl[n\omega_b - (k-1)\omega_b\bigr]}
	= \frac{\sqrt{k}\,\lambda\bigl(k-1 - n\tilde{c}_-^2\bigr)}
	{(k-1)(n-k+1)\omega_b},
	\qquad k = 3,\ldots,n,
	\label{eq:factor_k}
\end{equation}

\noindent
donde todos los $\tilde{c}_\pm$ y $\tilde{r}$ se evalúan con los
parámetros del sistema a la resonancia $\tilde{r} = n\omega_b$.

\medskip
\noindent
\textit{Producto acumulado.}
Multiplicando todos los factores desde $k=3$ hasta $k=n$ y
usando $\Omega_\mathrm{eff}^{(2)}$ de~\eqref{eq:Oeff2_III}
como semilla, el resultado al orden dominante en $\lambda/\omega_b$
es:

\begin{equation}
	\Omega_\mathrm{eff}^{(n)}\big|_{2\mathrm{QD}}
	= (-1)^n\,\frac{\bigl(\lambda/\omega_b\bigr)^n\,\sqrt{2}\,\widetilde\Omega}
	{(n-1)!\,\sqrt{n!}}
	\prod_{k=1}^{n-1}\bigl(n\,\tilde{c}_-^2 - k\bigr),
	\label{eq:Oeff_final}
\end{equation}

\noindent
donde $\widetilde\Omega = \sqrt{2}\,\Omega$ es la frecuencia de Rabi
superradiante del estado brillante del 2QD.

\medskip
\noindent
\textit{Comparación con Bin (S29).}
La ecuación~\eqref{eq:Oeff_final} es formalmente idéntica a Bin~(S29)
bajo las sustituciones:

\begin{equation}
	c_\pm \;\longrightarrow\; \tilde{c}_\pm,
	\qquad
	\Omega \;\longrightarrow\; \widetilde\Omega = \sqrt{2}\,\Omega,
	\qquad
	r \;\longrightarrow\; \tilde{r} = \sqrt{(\Delta+J)^2 + 8\Omega^2},
	\label{eq:sustituciones_2QD}
\end{equation}

\noindent
donde $\tilde{c}_\pm$ son los coeficientes de los estados vestidos
del 2QD definidos en~\eqref{eq:cpm_final}, que satisfacen
$\tilde{c}_+^2 + \tilde{c}_-^2 = 1$.
La equivalencia ha sido verificada numéricamente para $n = 2,3,4,5$
mediante la iteración explícita de~\eqref{eq:recurrencia_III}.

\medskip
\noindent
\textit{Casos explícitos.}
Para los primeros valores de $n$:

\begin{align}
	\Omega_\mathrm{eff}^{(2)}\big|_{2\mathrm{QD}}
	&= \frac{\sqrt{2}\,\widetilde\Omega\,\lambda^2}{\omega_b}
	\,\tilde{c}_-^2\bigl(2\tilde{c}_-^2-1\bigr)
	\cdot\bigl(\text{sgn coeficientes}\bigr),
	\label{eq:Oeff2_caso}\\[6pt]
	\Omega_\mathrm{eff}^{(3)}\big|_{2\mathrm{QD}}
	&= -\frac{\sqrt{6}\,\widetilde\Omega\,\lambda^3}{2\omega_b^2}
	\,\tilde{c}_-^2\,(1-\tilde{c}_-^2)^{1/2}
	\bigl(9\tilde{c}_-^4 - 9\tilde{c}_-^2 + 2\bigr),
	\label{eq:Oeff3_caso}\\[6pt]
	\Omega_\mathrm{eff}^{(4)}\big|_{2\mathrm{QD}}
	&= \frac{\sqrt{6}\,\widetilde\Omega\,\lambda^4}{3\omega_b^3}
	\,\tilde{c}_-^2\,(1-\tilde{c}_-^2)^{1/2}
	\bigl(32\tilde{c}_-^6 - 48\tilde{c}_-^4 + 22\tilde{c}_-^2 - 3\bigr).
	\label{eq:Oeff4_caso}
\end{align}

\medskip
\noindent
\textit{Resumen de los tres regímenes.}
Los resultados para $\Omega_\mathrm{eff}^{(n)}$ en los tres regímenes
del 2QD son:

\begin{equation}
	\Omega_\mathrm{eff}^{(n)}\big|_{2\mathrm{QD}} =
	\begin{cases}
		\dfrac{\sqrt{2}\,\Omega}{\sqrt{n!}}
		\left(\dfrac{\lambda}{\omega_b}\right)^n,
		& \text{Régimen I},
		\\[12pt]
		\dfrac{\sqrt{2}\,\Omega}{\sqrt{n!}}
		\left(\dfrac{\lambda}{\omega_b}\right)^n
		e^{-\lambda^2/2\omega_b^2},
		& \text{Régimen II},
		\\[12pt]
		\dfrac{(-1)^n\,\sqrt{2}\,\widetilde\Omega}
		{(n-1)!\,\sqrt{n!}}
		\left(\dfrac{\lambda}{\omega_b}\right)^n
		\displaystyle\prod_{k=1}^{n-1}(n\tilde{c}_-^2-k),
		& \text{Régimen III},
	\end{cases}
	\label{eq:Oeff_resumen}
\end{equation}

\noindent
con las resonancias correspondientes:

\begin{equation}
	\Delta_n\big|_{2\mathrm{QD}} =
	\begin{cases}
		-n\omega_b - J,
		& \text{Régimen I},
		\\[6pt]
		\dfrac{\lambda^2}{\omega_b} - n\omega_b - J,
		& \text{Régimen II},
		\\[8pt]
		-\sqrt{n^2\omega_b^2 - 8\Omega^2} - J,
		& \text{Régimen III}.
	\end{cases}
	\label{eq:Delta_resumen}
\end{equation}

\noindent
En todos los regímenes, las dos correcciones respecto al resultado de
Bin~(S29) para el 1QD son idénticas: (i)~un factor $\sqrt{2}$ en
la frecuencia de Rabi efectiva, originado en la superradiancia del
estado brillante, y (ii)~un desplazamiento rígido $-J$ en la
resonancia, originado en el acoplamiento de Förster.

	
\end{document}