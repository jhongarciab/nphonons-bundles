% ============================================================
%  Derivación paso a paso: Omega_eff^(n) para la molécula
%  excitónica (2 QDs acoplados por interacción de Förster)
%  Régimen I: conducción débil y acoplamiento débil
% ============================================================
\documentclass[12pt]{article}
\usepackage{amsmath, amssymb}
\usepackage[margin=2.5cm]{geometry}

\begin{document}
	
	% -----------------------------------------------------------
	% BLOQUE 0: Hamiltoniano en el marco del laboratorio
	% -----------------------------------------------------------
	
	Consideramos dos puntos cu\'anticos id\'enticos, QD$_1$ y QD$_2$, cada uno
	modelado como un sistema de dos niveles con estados electr\'onicos $|v\rangle$
	(valencia) y $|c\rangle$ (conducci\'on), frecuencia de transici\'on $\omega_\sigma$
	y acoplamiento electr\'on-fon\'on $\lambda$. Ambos QDs interact\'uan con un
	\'unico modo fon\'onico de cavidad de frecuencia $\omega_b$, con operadores de
	escalera $b$ y $b^\dagger$. Los QDs est\'an acoplados entre s\'i mediante la
	interacci\'on de F\"orster con constante $J$. Un l\'aser coherente de frecuencia
	$\omega_L$ conduce ambos QDs con la misma frecuencia de Rabi $\Omega$.
	Establecemos $\hbar = 1$ en todo lo que sigue.
	
	El Hamiltoniano total del sistema en el marco del laboratorio es
	
	\begin{equation}
		H_{\mathrm{lab}}
		= \omega_b b^\dagger b
		+ \omega_\sigma \left(\sigma_1^\dagger \sigma_1 + \sigma_2^\dagger \sigma_2\right)
		+ \lambda \left(\sigma_1^\dagger \sigma_1 + \sigma_2^\dagger \sigma_2\right)(b^\dagger + b)
		+ J\left(\sigma_1^\dagger \sigma_2 + \sigma_2^\dagger \sigma_1\right)
		+ \Omega \sum_{i=1,2} \left(e^{i\omega_L t}\,\sigma_i + e^{-i\omega_L t}\,\sigma_i^\dagger\right),
		\label{eq:H_lab}
	\end{equation}
	
	donde $\sigma_i = |v\rangle_i\langle c|$ y $\sigma_i^\dagger = |c\rangle_i\langle v|$
	son los operadores de decaimiento y excitaci\'on del QD$_i$, respectivamente.
	El primer t\'ermino de~\eqref{eq:H_lab} describe la energ\'ia libre del modo
	fon\'onico. El segundo describe la energ\'ia de transici\'on de ambos QDs. El
	tercero es el acoplamiento electr\'on-fon\'on, que en el modelo de Holstein
	desplaza el equilibrio del oscilador seg\'un el estado electr\'onico de cada QD.
	El cuarto es la interacci\'on de F\"orster, que transfiere la excitaci\'on
	entre QDs sin cambiar el n\'umero de fonones. El quinto es la conducci\'on
	coherente por el l\'aser externo en la aproximaci\'on de onda rotante (RWA).
	
	% -----------------------------------------------------------
	% BLOQUE 1: Transformación al marco rotante
	% -----------------------------------------------------------
	
	Para eliminar la dependencia expl\'icita en el tiempo del t\'ermino de
	conducci\'on, transformamos al marco rotante con la frecuencia del l\'aser
	$\omega_L$. Definimos el operador unitario
	
	\begin{equation}
		U(t) = \exp\!\left(i\omega_L t \left(\sigma_1^\dagger\sigma_1
		+ \sigma_2^\dagger\sigma_2\right)\right).
		\label{eq:U}
	\end{equation}
	
	El Hamiltoniano en el nuevo marco se obtiene mediante
	
	\begin{equation}
		H = U\,H_{\mathrm{lab}}\,U^\dagger + i\,\dot{U}\,U^\dagger.
		\label{eq:H_rotante_def}
	\end{equation}
	
	Calculamos el efecto de $U$ sobre cada t\'ermino de~\eqref{eq:H_lab}.
	
	\medskip
	\noindent
	El operador $U$ act\'ua \'unicamente sobre los grados de libertad electr\'onicos
	a trav\'es del operador de n\'umero $\hat{n}_e = \sigma_1^\dagger\sigma_1 +
	\sigma_2^\dagger\sigma_2$. Para cualquier operador $A$ que conmuta con
	$\hat{n}_e$, se tiene $U A U^\dagger = A$. En cambio, para los operadores de
	excitaci\'on y decaimiento se tiene:
	
	\begin{align}
		U\,\sigma_i^\dagger\,U^\dagger &= e^{+i\omega_L t}\,\sigma_i^\dagger,
		\label{eq:transf_sigmadag}\\
		U\,\sigma_i\,U^\dagger &= e^{-i\omega_L t}\,\sigma_i.
		\label{eq:transf_sigma}
	\end{align}
	
	Esto se verifica directamente usando la identidad de Baker-Campbell-Hausdorff y
	el \'algebra $[\hat{n}_e, \sigma_i^\dagger] = \sigma_i^\dagger$,
	$[\hat{n}_e, \sigma_i] = -\sigma_i$.
	
	\medskip
	\noindent
	Con estas relaciones, calculamos la transformaci\'on de cada t\'ermino:
	
	\medskip
	\noindent
	\textit{(i) T\'ermino fon\'onico.}
	$b$ y $b^\dagger$ conmutan con $\hat{n}_e$, por lo que
	
	\begin{equation}
		U\,\omega_b b^\dagger b\,U^\dagger = \omega_b b^\dagger b.
	\end{equation}
	
	\noindent
	\textit{(ii) T\'ermino electr\'onico libre.}
	$\sigma_i^\dagger\sigma_i$ conmuta con $\hat{n}_e$, por lo que
	
	\begin{equation}
		U\,\omega_\sigma\!\left(\sigma_1^\dagger\sigma_1 +
		\sigma_2^\dagger\sigma_2\right)U^\dagger
		= \omega_\sigma\!\left(\sigma_1^\dagger\sigma_1 +
		\sigma_2^\dagger\sigma_2\right).
	\end{equation}
	
	\noindent
	\textit{(iii) T\'ermino de acoplamiento electr\'on-fon\'on.}
	$\sigma_i^\dagger\sigma_i$ conmuta con $\hat{n}_e$ y $b+b^\dagger$ con $U$,
	por lo que
	
	\begin{equation}
		U\,\lambda\!\left(\sigma_1^\dagger\sigma_1 +
		\sigma_2^\dagger\sigma_2\right)(b^\dagger + b)\,U^\dagger
		= \lambda\!\left(\sigma_1^\dagger\sigma_1 +
		\sigma_2^\dagger\sigma_2\right)(b^\dagger + b).
	\end{equation}
	
	\noindent
	\textit{(iv) T\'ermino de F\"orster.}
	Usando~\eqref{eq:transf_sigmadag} y~\eqref{eq:transf_sigma}:
	
	\begin{equation}
		U\,\sigma_1^\dagger\sigma_2\,U^\dagger
		= \left(e^{+i\omega_L t}\sigma_1^\dagger\right)
		\!\left(e^{-i\omega_L t}\sigma_2\right)
		= \sigma_1^\dagger\sigma_2.
	\end{equation}
	
	Las fases de los dos factores se cancelan exactamente porque F\"orster conserva
	el n\'umero total de excitaciones: sube una en QD$_1$ y baja una en QD$_2$.
	An\'alogamente $U\sigma_2^\dagger\sigma_1 U^\dagger = \sigma_2^\dagger\sigma_1$.
	Por tanto:
	
	\begin{equation}
		U\,J\!\left(\sigma_1^\dagger\sigma_2 +
		\sigma_2^\dagger\sigma_1\right)U^\dagger
		= J\!\left(\sigma_1^\dagger\sigma_2 + \sigma_2^\dagger\sigma_1\right).
	\end{equation}
	
	\noindent
	\textit{(v) T\'ermino de conducci\'on.}
	Para cada QD$_i$, usando~\eqref{eq:transf_sigma} y~\eqref{eq:transf_sigmadag}:
	
	\begin{equation}
		U\,\Omega\!\left(e^{i\omega_L t}\sigma_i + e^{-i\omega_L t}\sigma_i^\dagger\right)U^\dagger
		= \Omega\!\left(e^{i\omega_L t} e^{-i\omega_L t}\sigma_i
		+ e^{-i\omega_L t} e^{+i\omega_L t}\sigma_i^\dagger\right)
		= \Omega\!\left(\sigma_i + \sigma_i^\dagger\right).
	\end{equation}
	
	\noindent
	\textit{(vi) T\'ermino cin\'etico $i\dot{U}U^\dagger$.}
	Derivando~\eqref{eq:U}:
	
	\begin{equation}
		i\,\dot{U}\,U^\dagger = -\omega_L\!\left(\sigma_1^\dagger\sigma_1 +
		\sigma_2^\dagger\sigma_2\right).
	\end{equation}
	
	\medskip
	\noindent
	Sumando todos los contribuciones en~\eqref{eq:H_rotante_def} y agrupando los
	t\'erminos electr\'onicos libres con el t\'ermino cin\'etico:
	
	\begin{equation}
		\omega_\sigma\!\left(\sigma_1^\dagger\sigma_1 +
		\sigma_2^\dagger\sigma_2\right)
		- \omega_L\!\left(\sigma_1^\dagger\sigma_1 +
		\sigma_2^\dagger\sigma_2\right)
		= \Delta\!\left(\sigma_1^\dagger\sigma_1 + \sigma_2^\dagger\sigma_2\right),
	\end{equation}
	
	donde definimos la desinton\'ia $\Delta \equiv \omega_\sigma - \omega_L$.
	
	\medskip
	\noindent
	El Hamiltoniano en el marco rotante queda entonces:
	
	\begin{equation}
		\boxed{
			H = \omega_b b^\dagger b
			+ \Delta\!\left(\sigma_1^\dagger\sigma_1 + \sigma_2^\dagger\sigma_2\right)
			+ \lambda\!\left(\sigma_1^\dagger\sigma_1 + \sigma_2^\dagger\sigma_2\right)
			(b^\dagger + b)
			+ J\!\left(\sigma_1^\dagger\sigma_2 + \sigma_2^\dagger\sigma_1\right)
			+ \Omega\sum_{i=1,2}\!\left(\sigma_i + \sigma_i^\dagger\right).
		}
		\label{eq:H_rot}
	\end{equation}
	
	Este es exactamente el Hamiltoniano~(S2) de Bin et al.\ para un solo QD, pero
	con dos QDs id\'enticos y el t\'ermino adicional de F\"orster. A diferencia
	del caso de un solo QD, el t\'ermino de F\"orster \emph{no} adquiere
	dependencia temporal en el marco rotante, porque conserva el n\'umero total de
	excitaciones electr\'onicas. Este Hamiltoniano~\eqref{eq:H_rot} es el punto de
	partida para toda la derivaci\'on perturbativa que sigue.
	
	% -----
	%-----
	El Hamiltoniano~\eqref{eq:H_rot} contiene el t\'ermino de F\"orster
	$J(\sigma_1^\dagger\sigma_2 + \sigma_2^\dagger\sigma_1)$, que no es diagonal
	en la base de producto $|c/v\rangle_1 \otimes |c/v\rangle_2$. Para trabajar
	con autoestados bien definidos del sistema, diagonalizamos este t\'ermino.
	
	\medskip
	\noindent
	El espacio electr\'onico completo tiene cuatro estados de base:
	
	\begin{equation}
		|vv\rangle, \quad |cv\rangle, \quad |vc\rangle, \quad |cc\rangle,
	\end{equation}
	
	donde la primera (segunda) etiqueta corresponde al QD$_1$ (QD$_2$), con la
	convenci\'on $|cv\rangle \equiv |c\rangle_1|v\rangle_2$.
	
	\medskip
	\noindent
	El operador de F\"orster $\sigma_1^\dagger\sigma_2 + \sigma_2^\dagger\sigma_1$
	act\'ua \'unicamente en el subespacio de una excitaci\'on electr\'onica total,
	$\{|cv\rangle, |vc\rangle\}$, donde su representaci\'on matricial es
	
	\begin{equation}
		J\begin{pmatrix} 0 & 1 \\ 1 & 0 \end{pmatrix}.
	\end{equation}
	
	Los autovalores de esta matriz son $\pm J$. Los autovectores correspondientes
	son
	
	\begin{align}
		|\Psi_+\rangle &= \frac{1}{\sqrt{2}}\left(|cv\rangle + |vc\rangle\right),
		\qquad E_+ = +J,
		\label{eq:Psi_plus}\\[6pt]
		|\Psi_-\rangle &= \frac{1}{\sqrt{2}}\left(|cv\rangle - |vc\rangle\right),
		\qquad E_- = -J.
		\label{eq:Psi_minus}
	\end{align}
	
	\medskip
	\noindent
	Verificamos que son correctos. Para $|\Psi_+\rangle$:
	
	\begin{equation}
		J\left(\sigma_1^\dagger\sigma_2 + \sigma_2^\dagger\sigma_1\right)|\Psi_+\rangle
		= \frac{J}{\sqrt{2}}\left(\sigma_1^\dagger\sigma_2 + \sigma_2^\dagger\sigma_1\right)
		\left(|cv\rangle + |vc\rangle\right).
	\end{equation}
	
	Evaluando cada t\'ermino:
	$\sigma_1^\dagger\sigma_2|cv\rangle = 0$ (QD$_1$ ya est\'a excitado),
	$\sigma_2^\dagger\sigma_1|cv\rangle = |vc\rangle$,
	$\sigma_1^\dagger\sigma_2|vc\rangle = |cv\rangle$,
	$\sigma_2^\dagger\sigma_1|vc\rangle = 0$ (QD$_2$ ya est\'a excitado). Por tanto:
	
	\begin{equation}
		J\left(\sigma_1^\dagger\sigma_2 + \sigma_2^\dagger\sigma_1\right)|\Psi_+\rangle
		= \frac{J}{\sqrt{2}}\left(|vc\rangle + |cv\rangle\right) = J|\Psi_+\rangle. \checkmark
	\end{equation}
	
	An\'alogamente se verifica $J(\sigma_1^\dagger\sigma_2 +
	\sigma_2^\dagger\sigma_1)|\Psi_-\rangle = -J|\Psi_-\rangle$.
	
	\medskip
	\noindent
	A continuaci\'on evaluamos el acoplamiento de cada estado colectivo al l\'aser.
	El t\'ermino de conducci\'on $\Omega\sum_{i=1,2}(\sigma_i + \sigma_i^\dagger)$
	sobre el vac\'io electr\'onico $|vv\rangle$ da:
	
	\begin{equation}
		\Omega\left(\sigma_1^\dagger + \sigma_2^\dagger\right)|vv\rangle
		= \Omega\left(|cv\rangle + |vc\rangle\right)
		= \sqrt{2}\,\Omega\,|\Psi_+\rangle.
		\label{eq:laser_brillante}
	\end{equation}
	
	El l\'aser excita \'unicamente el estado sim\'etrico $|\Psi_+\rangle$ con
	frecuencia de Rabi colectiva $\sqrt{2}\,\Omega$. Para el estado
	antisim\'etrico:
	
	\begin{equation}
		\Omega\left(\sigma_1^\dagger + \sigma_2^\dagger\right)|vv\rangle
		\perp |\Psi_-\rangle,
	\end{equation}
	
	ya que $\langle\Psi_-|\left(|cv\rangle + |vc\rangle\right) = 0$. El estado
	$|\Psi_-\rangle$ est\'a desacoplado del l\'aser y no participa en la
	din\'amica de emisi\'on de fonones. En lo sucesivo, restringimos el espacio
	electr\'onico al subespacio
	
	\begin{equation}
		\{|vv\rangle,\; |\Psi_+\rangle\},
	\end{equation}
	
	con las identificaciones $|v\rangle \leftrightarrow |vv\rangle$ y
	$|c\rangle \leftrightarrow |\Psi_+\rangle$ respecto al caso de un solo QD.
	
	Restringidos al subespacio $\{|vv\rangle, |\Psi_+\rangle\}$, reescribimos el
	Hamiltoniano~\eqref{eq:H_rot} en t\'erminos de operadores colectivos del estado
	brillante. Definimos
	
	\begin{align}
		S^\dagger &= |\Psi_+\rangle\langle vv|
		= \frac{1}{\sqrt{2}}\left(\sigma_1^\dagger + \sigma_2^\dagger\right),
		\label{eq:S_dag}\\[4pt]
		S &= |vv\rangle\langle\Psi_+|
		= \frac{1}{\sqrt{2}}\left(\sigma_1 + \sigma_2\right),
		\label{eq:S}
	\end{align}
	
	que satisfacen $\{S, S^\dagger\} = 1$ y $S^2 = 0$ en el subespacio de inter\'es.
	El operador de n\'umero electr\'onico colectivo es $S^\dagger S =
	|\Psi_+\rangle\langle\Psi_+|$.
	
	\medskip
	\noindent
	Reemplazamos cada t\'ermino de~\eqref{eq:H_rot} en la base colectiva.
	
	\medskip
	\noindent
	\textit{(i) T\'ermino fon\'onico.} No se modifica:
	
	\begin{equation}
		\omega_b b^\dagger b.
	\end{equation}
	
	\noindent
	\textit{(ii) T\'ermino electr\'onico libre.}
	En el subespacio $\{|vv\rangle, |\Psi_+\rangle\}$ el operador
	$\sigma_1^\dagger\sigma_1 + \sigma_2^\dagger\sigma_2$ cuenta el n\'umero total
	de excitaciones electr\'onicas, que vale $0$ en $|vv\rangle$ y $1$ en
	$|\Psi_+\rangle$. Por tanto:
	
	\begin{equation}
		\sigma_1^\dagger\sigma_1 + \sigma_2^\dagger\sigma_2
		\;\longrightarrow\; S^\dagger S,
	\end{equation}
	
	y el t\'ermino electr\'onico libre es $\Delta\, S^\dagger S$.
	
	\medskip
	\noindent
	\textit{(iii) T\'ermino de acoplamiento electr\'on-fon\'on.}
	An\'alogamente, $(\sigma_1^\dagger\sigma_1 + \sigma_2^\dagger\sigma_2)
	\to S^\dagger S$, y el t\'ermino queda:
	
	\begin{equation}
		\lambda\, S^\dagger S\,(b^\dagger + b).
	\end{equation}
	
	\noindent
	\textit{(iv) T\'ermino de F\"orster.}
	En la base colectiva, $|\Psi_+\rangle$ es autoestado con autovalor $+J$:
	
	\begin{equation}
		J\left(\sigma_1^\dagger\sigma_2 + \sigma_2^\dagger\sigma_1\right)
		\;\longrightarrow\; J\, S^\dagger S.
	\end{equation}
	
	\noindent
	\textit{(v) T\'ermino de conducci\'on.}
	De la Ec.~\eqref{eq:laser_brillante}, el l\'aser actua sobre $|vv\rangle$
	como $\sqrt{2}\,\Omega\,S^\dagger$, y su herm\'itico conjugado sobre
	$|\Psi_+\rangle$ como $\sqrt{2}\,\Omega\,S$. Por tanto:
	
	\begin{equation}
		\Omega\sum_{i=1,2}\left(\sigma_i + \sigma_i^\dagger\right)
		\;\longrightarrow\; \sqrt{2}\,\Omega\left(S + S^\dagger\right).
	\end{equation}
	
	\medskip
	\noindent
	Sumando los t\'erminos electr\'onico libre y de F\"orster:
	
	\begin{equation}
		\Delta\, S^\dagger S + J\, S^\dagger S
		= (\Delta + J)\, S^\dagger S.
	\end{equation}
	
	\noindent
	El Hamiltoniano efectivo en la base colectiva es entonces:
	
	\begin{equation}
		\boxed{
			H_{\mathrm{eff}} = \omega_b b^\dagger b
			+ (\Delta + J)\,S^\dagger S
			+ \lambda\, S^\dagger S\,(b^\dagger + b)
			+ \sqrt{2}\,\Omega\left(S + S^\dagger\right).
		}
		\label{eq:H_eff}
	\end{equation}
	
	\medskip
	\noindent
	Comparando con el Hamiltoniano de Bin~(S2) para un solo QD,
	
	\begin{equation}
		H_{\mathrm{1QD}} = \omega_b b^\dagger b
		+ \Delta\,\sigma^\dagger\sigma
		+ \lambda\,\sigma^\dagger\sigma\,(b^\dagger + b)
		+ \Omega\left(\sigma + \sigma^\dagger\right),
	\end{equation}
	
	el Hamiltoniano~\eqref{eq:H_eff} es id\'entico bajo las sustituciones
	
	\begin{equation}
		\sigma \;\to\; S,
		\qquad
		\Delta \;\to\; \widetilde{\Delta} \equiv \Delta + J,
		\qquad
		\Omega \;\to\; \widetilde{\Omega} \equiv \sqrt{2}\,\Omega.
		\label{eq:correspondencia}
	\end{equation}
	
	El acoplamiento de F\"orster $J$ desplaza la desinton\'ia efectiva en $+J$,
	y la simetr\'ia colectiva del estado brillante aumenta la frecuencia de Rabi
	efectiva por un factor $\sqrt{2}$. Toda la derivaci\'on perturbativa de Bin
	se aplica directamente a la mol\'ecula excit\'onica bajo estas dos
	sustituciones.
	
	Con el Hamiltoniano efectivo~\eqref{eq:H_eff}, los autoestados del sistema
	libre (sin conducci\'on ni acoplamiento electr\'on-fon\'on) son los estados
	producto $|m, vv\rangle$ y $|m, \Psi_+\rangle$, donde $m = 0, 1, 2, \ldots$
	denota el n\'umero de fonones en la cavidad. Sus energ\'ias son:
	
	\begin{align}
		E_{|m,\, vv\rangle} &= m\omega_b,
		\label{eq:E_vv}\\[4pt]
		E_{|m,\, \Psi_+\rangle} &= m\omega_b + (\Delta + J).
		\label{eq:E_Psi}
	\end{align}
	
	\medskip
	\noindent
	La condici\'on de resonancia de Stokes de $n$-fonones se obtiene pidiendo que
	los estados $|0, vv\rangle$ (cero fonones, ambos QDs en la valencia) y
	$|n, \Psi_+\rangle$ ($n$ fonones, estado brillante excitado) sean
	degenerados en energ\'ia:
	
	\begin{equation}
		E_{|0,\, vv\rangle} = E_{|n,\, \Psi_+\rangle}.
	\end{equation}
	
	\noindent
	Sustituyendo~\eqref{eq:E_vv} y~\eqref{eq:E_Psi} con $m = 0$ y $m = n$
	respectivamente:
	
	\begin{equation}
		0 = n\omega_b + (\Delta + J).
	\end{equation}
	
	\noindent
	Despejando $\Delta$:
	
	\begin{equation}
		\boxed{
			\Delta_n = -n\omega_b - J.
		}
		\label{eq:resonancia}
	\end{equation}
	
	\noindent
	En t\'erminos de la desinton\'ia renormalizada $\widetilde{\Delta} = \Delta + J$
	definida en~\eqref{eq:correspondencia}, la condici\'on~\eqref{eq:resonancia}
	toma la forma familiar del caso de un solo QD:
	
	\begin{equation}
		\widetilde{\Delta}_n = -n\omega_b,
		\label{eq:resonancia_tilde}
	\end{equation}
	
	lo que confirma la consistencia con la correspondencia establecida en el
	Bloque~3.
	
	\medskip
	\noindent
	El acoplamiento de F\"orster $J$ desplaza r\'igidamente la condici\'on de
	resonancia en $-J$ respecto al caso de un solo QD, independientemente del
	n\'umero de fonones $n$. Para recuperar la resonancia, la frecuencia del
	l\'aser debe ajustarse a $\omega_L = \omega_\sigma + n\omega_b + J$.
	
	En el r\'egimen I, $\widetilde{\Omega}, \lambda \ll \omega_b$, los autoestados
	del Hamiltoniano son pr\'oximos a los estados producto $|m, vv\rangle$ y
	$|m, \Psi_+\rangle$. Siguiendo la estrategia de Bin~(S3), en la vecindad de
	la resonancia de $n$-fonones truncamos el espacio de Hilbert a los $2(n+1)$
	estados:
	
	\begin{equation}
		\mathcal{H}^{(n)} = \bigl\{
		|0, vv\rangle,\;
		|n, \Psi_+\rangle,\;
		|0, \Psi_+\rangle,\;
		|1, vv\rangle,\;
		|1, \Psi_+\rangle,\;
		|2, vv\rangle,\;
		\ldots,\;
		|n-1, \Psi_+\rangle,\;
		|n, vv\rangle
		\bigr\}.
		\label{eq:espacio_truncado}
	\end{equation}
	
	\noindent
	Los dos primeros estados, $|0, vv\rangle$ y $|n, \Psi_+\rangle$, son los que
	est\'an en resonancia seg\'un~\eqref{eq:resonancia}. Los $2n$ estados restantes
	son los estados intermedios que median la transici\'on de $n$-fonones.
	
	\medskip
	\noindent
	El Hamiltoniano truncado se escribe en la forma de bloques:
	
	\begin{equation}
		H^{(n)} = \begin{pmatrix} \mathbf{H}^{(n)} & X^{(n)} \\[4pt]
			X^{(n)T} & R^{(n)} \end{pmatrix},
		\label{eq:H_bloques}
	\end{equation}
	
	donde $\mathbf{H}^{(n)}$ act\'ua en el subespacio de inter\'es
	$\{|0, vv\rangle,\, |n, \Psi_+\rangle\}$, $R^{(n)}$ act\'ua en el subespacio
	de los $2n$ estados intermedios, y $X^{(n)}$ es la matriz $2\times 2n$ que
	acopla ambos subespacios.
	
	\medskip
	\noindent
	\textit{Submatriz $\mathbf{H}^{(n)}$.}
	En resonancia $\widetilde{\Delta} = -n\omega_b$, las energ\'ias de los dos
	estados de inter\'es son iguales:
	
	\begin{equation}
		E_{|0,\, vv\rangle} = 0, \qquad
		E_{|n,\, \Psi_+\rangle} = n\omega_b + \widetilde{\Delta}
		= n\omega_b - n\omega_b = 0.
	\end{equation}
	
	\noindent
	El Hamiltoniano~\eqref{eq:H_eff} no tiene elementos de matriz directos entre
	$|0, vv\rangle$ y $|n, \Psi_+\rangle$ para $n \geq 2$ (el l\'aser cambia el
	estado electr\'onico pero no el fon\'onico, y el acoplamiento electr\'on-fon\'on
	cambia el n\'umero de fonones en $\pm 1$ pero no el estado electr\'onico). Por
	tanto:
	
	\begin{equation}
		\mathbf{H}^{(n)} = \begin{pmatrix} 0 & 0 \\ 0 & 0 \end{pmatrix}.
	\end{equation}
	
	\noindent
	\textit{Submatriz $R^{(n)}$ (diagonal en la base de estados producto).}
	Las energ\'ias de los estados intermedios, evaluadas en
	$\widetilde{\Delta} = -n\omega_b$, son:
	
	\begin{align}
		E_{|k,\, vv\rangle} &= k\omega_b,
		\qquad k = 1, \ldots, n,
		\label{eq:E_int_vv}\\[4pt]
		E_{|k,\, \Psi_+\rangle} &= k\omega_b + \widetilde{\Delta}
		= k\omega_b - n\omega_b = -(n-k)\omega_b,
		\qquad k = 0, 1, \ldots, n-1.
		\label{eq:E_int_Psi}
	\end{align}
	
	\noindent
	\textit{Elementos de acoplamiento: matriz $X^{(n)}$.}
	El Hamiltoniano~\eqref{eq:H_eff} tiene dos t\'erminos que generan acoplamientos
	entre subespacios distintos:
	
	\begin{enumerate}
		\item El t\'ermino de conducci\'on $\widetilde{\Omega}(S + S^\dagger)$ acopla
		estados con el mismo n\'umero de fonones pero diferente estado electr\'onico:
		\begin{equation}
			\langle m, \Psi_+|\,\widetilde{\Omega}\,S^\dagger\,|m, vv\rangle = \widetilde{\Omega}
			= \sqrt{2}\,\Omega.
			\label{eq:acop_laser}
		\end{equation}
		
		\item El t\'ermino de acoplamiento electr\'on-fon\'on
		$\lambda\,S^\dagger S\,(b^\dagger + b)$ act\'ua \'unicamente cuando el
		sistema electr\'onico est\'a en $|\Psi_+\rangle$ y cambia el n\'umero de
		fonones en $\pm 1$:
		\begin{equation}
			\langle m+1, \Psi_+|\,\lambda\,S^\dagger S\,b^\dagger\,|m, \Psi_+\rangle
			= \lambda\sqrt{m+1}.
			\label{eq:acop_fonon}
		\end{equation}
	\end{enumerate}
	
	\noindent
	La transici\'on $|0, vv\rangle \to |n, \Psi_+\rangle$ requiere exactamente $n$
	pasos alternando entre estos dos tipos de acoplamiento: primero el l\'aser
	excita el estado electr\'onico ($|0,vv\rangle \to |0,\Psi_+\rangle$) y luego
	el acoplamiento electr\'on-fon\'on escala la escalera fon\'onica paso a paso
	($|0,\Psi_+\rangle \to |1,\Psi_+\rangle \to \cdots \to |n,\Psi_+\rangle$),
	o bien el camino inverso empieza por el lado de $|n,\Psi_+\rangle$.
	La estructura de la matriz de acoplamiento es id\'entica a Bin~(S5) con
	$\Omega \to \widetilde{\Omega}$:
	
	\begin{equation}
		X^{(n)} = \begin{pmatrix}
			\widetilde{\Omega} & 0 & \cdots & 0 & 0 \\
			0 & 0 & \cdots & \sqrt{n}\,\lambda & \widetilde{\Omega}
		\end{pmatrix}.
		\label{eq:X_n}
	\end{equation}
	
	\noindent
	La primera fila corresponde al estado $|0, vv\rangle$ y la segunda a
	$|n, \Psi_+\rangle$. El elemento $\widetilde{\Omega}$ en la posici\'on $(1,1)$
	acopla $|0, vv\rangle$ con $|0, \Psi_+\rangle$ v\'ia el l\'aser. El elemento
	$\sqrt{n}\,\lambda$ en la posici\'on $(2, 2n-1)$ acopla $|n, \Psi_+\rangle$
	con $|n-1, \Psi_+\rangle$ v\'ia el acoplamiento electr\'on-fon\'on. El
	elemento $\widetilde{\Omega}$ en la posici\'on $(2, 2n)$ acopla
	$|n, \Psi_+\rangle$ con $|n, vv\rangle$ v\'ia el l\'aser.
	
	Para obtener la frecuencia efectiva de $n$-fonones eliminamos adiab\'aticamente
	los estados intermedios $R^{(n)}$. En el r\'egimen $\widetilde{\Omega},
	\lambda \ll \omega_b$, las energ\'ias de los estados intermedios son de orden
	$\omega_b$ mientras que los acoplamientos en $X^{(n)}$ son peque\~nos. La
	teor\'ia de perturbaciones matriciales~[Bin~(S6)] da el Hamiltoniano efectivo
	en el subespacio $\{|0, vv\rangle,\, |n, \Psi_+\rangle\}$:
	
	\begin{equation}
		H_{\mathrm{eff}}^{(n)} = \mathbf{H}^{(n)}
		+ X^{(n)}\!\left(-R^{(n)}\right)^{-1}\!X^{(n)T}.
		\label{eq:Heff_pert}
	\end{equation}
	
	\noindent
	La frecuencia efectiva de $n$-fonones es el elemento fuera de la diagonal:
	
	\begin{equation}
		\Omega_{\mathrm{eff}}^{(n)} = H_{\mathrm{eff}}^{(n)}(1,2).
	\end{equation}
	
	\medskip
	% --- Caso base n = 1 ---
	\noindent
	\textit{Caso base $n = 1$.}
	El espacio truncado es $\{|0, vv\rangle,\, |1, \Psi_+\rangle,\,
	|0, \Psi_+\rangle,\, |1, vv\rangle\}$.
	En resonancia $\widetilde{\Delta} = -\omega_b$, el Hamiltoniano completo en
	esta base, ordenada como $\bigl(|0,vv\rangle,\, |1,\Psi_+\rangle,\,
	|0,\Psi_+\rangle,\, |1,vv\rangle\bigr)$, es:
	
	\begin{equation}
		H^{(1)} = \begin{pmatrix}
			0 & 0 & \widetilde{\Omega} & 0 \\
			0 & 0 & \lambda & \widetilde{\Omega} \\
			\widetilde{\Omega} & \lambda & \widetilde{\Delta} & 0 \\
			0 & \widetilde{\Omega} & 0 & \omega_b
		\end{pmatrix}
		=
		\begin{pmatrix}
			0 & 0 & \widetilde{\Omega} & 0 \\
			0 & 0 & \lambda & \widetilde{\Omega} \\
			\widetilde{\Omega} & \lambda & -\omega_b & 0 \\
			0 & \widetilde{\Omega} & 0 & \omega_b
		\end{pmatrix}.
		\label{eq:H1_completo}
	\end{equation}
	
	\noindent
	Verificamos los elementos de~\eqref{eq:H1_completo} uno a uno usando
	el Hamiltoniano~\eqref{eq:H_eff}:
	
	\begin{itemize}
		\item $\langle 0,\Psi_+|\,\widetilde{\Omega}\,S^\dagger\,|0,vv\rangle
		= \widetilde{\Omega}$: posici\'on $(3,1)$ y su herm\'itico $(1,3)$.
		\item $\langle 1,\Psi_+|\,\lambda S^\dagger S\,b^\dagger\,|0,\Psi_+\rangle
		= \lambda\sqrt{1} = \lambda$: posici\'on $(2,3)$ y su herm\'itico $(3,2)$.
		\item $\langle 1,vv|\,\widetilde{\Omega}\,S^\dagger\,|1,\Psi_+\rangle
		= \widetilde{\Omega}$: posici\'on $(4,2)$ y su herm\'itico $(2,4)$.
		\item $E_{|0,\Psi_+\rangle} = \widetilde{\Delta} = -\omega_b$:
		posici\'on $(3,3)$.
		\item $E_{|1,vv\rangle} = \omega_b$: posici\'on $(4,4)$.
	\end{itemize}
	
	\noindent
	Identificamos los bloques:
	
	\begin{equation}
		\mathbf{H}^{(1)} = \begin{pmatrix} 0 & 0 \\ 0 & 0 \end{pmatrix}, \qquad
		X^{(1)} = \begin{pmatrix} \widetilde{\Omega} & 0 \\ \lambda & \widetilde{\Omega}
		\end{pmatrix}, \qquad
		R^{(1)} = \begin{pmatrix} -\omega_b & 0 \\ 0 & \omega_b \end{pmatrix}.
		\label{eq:bloques_n1}
	\end{equation}
	
	\noindent
	Aplicamos~\eqref{eq:Heff_pert}. Primero calculamos
	$(-R^{(1)})^{-1}$:
	
	\begin{equation}
		\left(-R^{(1)}\right)^{-1} =
		\begin{pmatrix} \omega_b & 0 \\ 0 & -\omega_b \end{pmatrix}^{-1}
		= \begin{pmatrix} 1/\omega_b & 0 \\ 0 & -1/\omega_b \end{pmatrix}.
	\end{equation}
	
	\noindent
	Calculamos el producto $X^{(1)}(-R^{(1)})^{-1}$:
	
	\begin{equation}
		X^{(1)}\!\left(-R^{(1)}\right)^{-1}
		= \begin{pmatrix} \widetilde{\Omega} & 0 \\ \lambda & \widetilde{\Omega}
		\end{pmatrix}
		\begin{pmatrix} 1/\omega_b & 0 \\ 0 & -1/\omega_b \end{pmatrix}
		= \begin{pmatrix}
			\widetilde{\Omega}/\omega_b & 0 \\
			\lambda/\omega_b & -\widetilde{\Omega}/\omega_b
		\end{pmatrix}.
	\end{equation}
	
	\noindent
	Multiplicamos por $X^{(1)T}$:
	
	\begin{equation}
		H_{\mathrm{eff}}^{(1)}
		= \begin{pmatrix}
			\widetilde{\Omega}/\omega_b & 0 \\[4pt]
			\lambda/\omega_b & -\widetilde{\Omega}/\omega_b
		\end{pmatrix}
		\begin{pmatrix} \widetilde{\Omega} & \lambda \\ 0 & \widetilde{\Omega}
		\end{pmatrix}
		= \begin{pmatrix}
			\widetilde{\Omega}^2/\omega_b &
			\widetilde{\Omega}\lambda/\omega_b \\[4pt]
			\widetilde{\Omega}\lambda/\omega_b &
			\lambda^2/\omega_b - \widetilde{\Omega}^2/\omega_b
		\end{pmatrix}.
	\end{equation}
	
	\noindent
	El elemento fuera de la diagonal da la frecuencia efectiva de un fon\'on:
	
	\begin{equation}
		\Omega_{\mathrm{eff}}^{(1)} = \frac{\widetilde{\Omega}\,\lambda}{\omega_b}
		= \frac{\sqrt{2}\,\Omega\,\lambda}{\omega_b}.
		\label{eq:Oeff_1}
	\end{equation}
	
	\noindent
	Esto es an\'alogo al resultado de Bin~(S10)--(S11) para $n=1$,
	$\Omega_{\mathrm{eff}}^{(1)}\big|_{\mathrm{1QD}} = \Omega\lambda/\omega_b$,
	con la sustituci\'on $\Omega \to \widetilde{\Omega} = \sqrt{2}\,\Omega$.
	
	\medskip
	% --- Paso recursivo general ---
	\noindent
	\textit{Paso recursivo: de $n-1$ a $n$.}
	Para $n$ general truncamos el espacio a los cuatro estados
	$\{|0,vv\rangle,\, |n,\Psi_+\rangle,\, |n-1,\Psi_+\rangle,\,
	|n,vv\rangle\}$, reemplazando el acoplamiento directo entre
	$|0,vv\rangle$ y $|n-1,\Psi_+\rangle$ por el acoplamiento efectivo
	$\Omega_{\mathrm{eff}}^{(n-1)}$ ya calculado en el paso anterior.
	En resonancia $\widetilde{\Delta} = -n\omega_b$, la energ\'ia del estado
	intermedio $|n-1, \Psi_+\rangle$ es:
	
	\begin{equation}
		E_{|n-1,\,\Psi_+\rangle} = (n-1)\omega_b + \widetilde{\Delta}
		= (n-1)\omega_b - n\omega_b = -\omega_b.
	\end{equation}
	
	\noindent
	El Hamiltoniano reducido en la base ordenada
	$\bigl(|0,vv\rangle,\, |n,\Psi_+\rangle,\, |n-1,\Psi_+\rangle,\,
	|n,vv\rangle\bigr)$ es:
	
	\begin{equation}
		H^{(n)} = \begin{pmatrix}
			0 & 0 & \Omega_{\mathrm{eff}}^{(n-1)} & 0 \\[4pt]
			0 & 0 & \sqrt{n}\,\lambda & \widetilde{\Omega} \\[4pt]
			\Omega_{\mathrm{eff}}^{(n-1)} & \sqrt{n}\,\lambda & -\omega_b & 0 \\[4pt]
			0 & \widetilde{\Omega} & 0 & n\omega_b
		\end{pmatrix}.
		\label{eq:Hn_recursivo}
	\end{equation}
	
	\noindent
	Identificamos los bloques:
	
	\begin{equation}
		\mathbf{H}^{(n)} = \begin{pmatrix} 0 & 0 \\ 0 & 0 \end{pmatrix}, \qquad
		X^{(n)} = \begin{pmatrix}
			\Omega_{\mathrm{eff}}^{(n-1)} & 0 \\[4pt]
			\sqrt{n}\,\lambda & \widetilde{\Omega}
		\end{pmatrix}, \qquad
		R^{(n)} = \begin{pmatrix} -\omega_b & 0 \\ 0 & n\omega_b \end{pmatrix}.
		\label{eq:bloques_n}
	\end{equation}
	
	\noindent
	Aplicamos~\eqref{eq:Heff_pert}. Primero:
	
	\begin{equation}
		\left(-R^{(n)}\right)^{-1}
		= \begin{pmatrix} 1/\omega_b & 0 \\ 0 & -1/(n\omega_b) \end{pmatrix}.
	\end{equation}
	
	\noindent
	Calculamos $X^{(n)}(-R^{(n)})^{-1}$:
	
	\begin{equation}
		X^{(n)}\!\left(-R^{(n)}\right)^{-1}
		= \begin{pmatrix}
			\Omega_{\mathrm{eff}}^{(n-1)} & 0 \\[4pt]
			\sqrt{n}\,\lambda & \widetilde{\Omega}
		\end{pmatrix}
		\begin{pmatrix} 1/\omega_b & 0 \\ 0 & -1/(n\omega_b) \end{pmatrix}
		= \begin{pmatrix}
			\Omega_{\mathrm{eff}}^{(n-1)}/\omega_b & 0 \\[4pt]
			\sqrt{n}\,\lambda/\omega_b & -\widetilde{\Omega}/(n\omega_b)
		\end{pmatrix}.
	\end{equation}
	
	\noindent
	Multiplicamos por $X^{(n)T}$:
	
	\begin{equation}
		H_{\mathrm{eff}}^{(n)}
		= \begin{pmatrix}
			\Omega_{\mathrm{eff}}^{(n-1)}/\omega_b & 0 \\[4pt]
			\sqrt{n}\,\lambda/\omega_b & -\widetilde{\Omega}/(n\omega_b)
		\end{pmatrix}
		\begin{pmatrix}
			\Omega_{\mathrm{eff}}^{(n-1)} & \sqrt{n}\,\lambda \\[4pt]
			0 & \widetilde{\Omega}
		\end{pmatrix}.
	\end{equation}
	
	\noindent
	El elemento $(1,2)$ del producto es:
	
	\begin{equation}
		H_{\mathrm{eff}}^{(n)}(1,2)
		= \frac{\Omega_{\mathrm{eff}}^{(n-1)}}{\omega_b}\cdot\sqrt{n}\,\lambda
		+ 0\cdot\widetilde{\Omega}
		= \frac{\sqrt{n}\,\lambda}{\omega_b}\,\Omega_{\mathrm{eff}}^{(n-1)}.
	\end{equation}
	
	\noindent
	Obtenemos as\'i la relaci\'on de recurrencia:
	
	\begin{equation}
		\boxed{
			\Omega_{\mathrm{eff}}^{(n)} = \frac{\sqrt{n}\,\lambda}{\omega_b}\,
			\Omega_{\mathrm{eff}}^{(n-1)},
		}
		\label{eq:recurrencia}
	\end{equation}
	
	an\'aloga a Bin~(S8), evaluada en resonancia $\widetilde{\Delta} = -n\omega_b$.
	
	La relaci\'on de recurrencia~\eqref{eq:recurrencia} se aplica en cadena
	desde el caso base~\eqref{eq:Oeff_1}. La clave est\'a en que \emph{todos
		los denominadores se eval\'uan en la resonancia fija del nivel final
		$\widetilde\Delta = -n\omega_b$}, siguiendo Bin~(S9). En el paso de $k$ a
	$k+1$ el estado intermedio es $|k, \Psi_+\rangle$ con energ\'ia:
	
	\begin{equation}
		E_{|k,\Psi_+\rangle} = k\omega_b + \widetilde\Delta
		= k\omega_b - n\omega_b = -(n-k)\omega_b.
	\end{equation}
	
	\noindent
	El denominador de eliminaci\'on adiab\'atica para ese estado es
	$-(n-k)\omega_b$, que var\'ia con $k$. La cadena completa de
	recurrencias~[Bin~(S9)] es:
	
	\begin{align}
		\Omega_\mathrm{eff}^{(n)}
		&= \frac{\sqrt{n}\,\lambda}{(n-1)\omega_b}\cdot
		\frac{\sqrt{n-1}\,\lambda}{(n-2)\omega_b}\cdots
		\frac{\sqrt{2}\,\lambda}{\omega_b}\cdot
		\Omega_\mathrm{eff}^{(1)}.
		\label{eq:cadena}
	\end{align}
	
	\noindent
	El producto de los denominadores (en valor absoluto) es:
	
	\begin{equation}
		(n-1)\omega_b \cdot (n-2)\omega_b \cdots \omega_b
		= (n-1)!\,\omega_b^{n-1}.
	\end{equation}
	
	\noindent
	El producto de los numeradores es:
	
	\begin{equation}
		\sqrt{n}\cdot\sqrt{n-1}\cdots\sqrt{2}\,\lambda^{n-1}
		= \sqrt{\frac{n!}{1}}\,\lambda^{n-1}
		= \sqrt{n!}\,\lambda^{n-1}.
	\end{equation}
	
	\noindent
	Combinando con el caso base $\Omega_\mathrm{eff}^{(1)} =
	\widetilde\Omega\,\lambda/\omega_b = \sqrt{2}\,\Omega\lambda/\omega_b$:
	
	\begin{equation}
		\Omega_\mathrm{eff}^{(n)}
		= \frac{\sqrt{n!}\,\lambda^{n-1}}{(n-1)!\,\omega_b^{n-1}}\cdot
		\frac{\sqrt{2}\,\Omega\,\lambda}{\omega_b}
		= \frac{\sqrt{2}\,\Omega\,\sqrt{n!}}{(n-1)!}
		\left(\frac{\lambda}{\omega_b}\right)^n.
		\label{eq:antes_simplificar}
	\end{equation}
	
	\noindent
	Simplificamos el factor $\sqrt{n!}/(n-1)!$ usando
	$n! = n\cdot(n-1)!$:
	
	\begin{equation}
		\frac{\sqrt{n!}}{(n-1)!}
		= \frac{\sqrt{n\cdot(n-1)!}}{(n-1)!}
		= \frac{\sqrt{n}\,\sqrt{(n-1)!}}{(n-1)!}
		= \frac{\sqrt{n}}{\sqrt{(n-1)!}}
		= \sqrt{\frac{n}{(n-1)!}}.
	\end{equation}
	
	\noindent
	Alternativamente, usando $\sqrt{n!} = \sqrt{n}\cdot\sqrt{(n-1)!}$:
	
	\begin{equation}
		\frac{\sqrt{n!}}{(n-1)!}
		= \frac{\sqrt{n!}}{(n-1)!}\cdot\frac{\sqrt{n!}}{\sqrt{n!}}
		= \frac{n!}{(n-1)!\,\sqrt{n!}}
		= \frac{n}{\sqrt{n!}}.
	\end{equation}
	
	\noindent
	Sustituyendo en~\eqref{eq:antes_simplificar}:
	
	\begin{equation}
		\Omega_\mathrm{eff}^{(n)}
		= \frac{\sqrt{2}\,\Omega\,n}{\sqrt{n!}}
		\left(\frac{\lambda}{\omega_b}\right)^n.
	\end{equation}
	
	\noindent
	Verificamos con los casos expl\'icitos calculados en el Bloque~6.
	
	Para $n=1$: $\sqrt{2}\,\Omega\cdot 1/\sqrt{1!}\cdot(\lambda/\omega_b)
	= \sqrt{2}\,\Omega\lambda/\omega_b$. $\checkmark$
	
	Para $n=2$: $\sqrt{2}\,\Omega\cdot 2/\sqrt{2!}\cdot(\lambda/\omega_b)^2
	= \sqrt{2}\cdot 2/\sqrt{2}\cdot\Omega\lambda^2/\omega_b^2
	= 2\,\Omega\lambda^2/\omega_b^2$. $\checkmark$
	
	Para $n=3$: $\sqrt{2}\,\Omega\cdot 3/\sqrt{6}\cdot(\lambda/\omega_b)^3
	= 3\sqrt{2}/\sqrt{6}\cdot\Omega\lambda^3/\omega_b^3
	= 3/\sqrt{3}\cdot\Omega\lambda^3/\omega_b^3
	= \sqrt{3}\cdot\sqrt{3}/\sqrt{3}\cdot\Omega\lambda^3/\omega_b^3
	= \sqrt{3}\,\Omega\lambda^3/\omega_b^3$. Pero el c\'alculo
	directo del Bloque~6 dio $2\sqrt{3}\,\Omega\lambda^3/\omega_b^3$.
	La discrepancia en $n=3$ es un factor $2$.
	
	\medskip
	\noindent
	El factor extra surge porque la cadena~\eqref{eq:cadena} s\'olo aplica desde
	$k=2$ hasta $k=n$, mientras que el caso base ya incluye el paso $k=1$. El
	paso $k=1\to 2$ tiene denominador $\widetilde\Delta + \omega_b =
	-n\omega_b + \omega_b = -(n-1)\omega_b$. Pero en la recurrencia directa del
	Bloque~6 ese denominador es simplemente $\omega_b$ (de $R^{(2)}$ con
	$|1,\Psi_+\rangle$ en resonancia de 2 fonones). Hay una inconsistencia: el
	paso recursivo del Bloque~6 usa $R^{(n)}$ construido en la resonancia de
	nivel $n$, no en resonancia de nivel $n-1$.
	
	\medskip
	\noindent
	\textbf{Identificaci\'on del error.} En el paso recursivo del Bloque~6
	construimos $H^{(n)}$ en el espacio
	$\{|0,vv\rangle, |n,\Psi_+\rangle, |n-1,\Psi_+\rangle, |n,vv\rangle\}$
	y usamos la energ\'ia de $|n-1,\Psi_+\rangle$ en la resonancia de $n$-fonones,
	que es $-\omega_b$. Esto es correcto para el \'ultimo paso de la cadena
	($k = n-1 \to n$). Pero al identificar $\Omega_\mathrm{eff}^{(n-1)}$ como el
	acoplamiento entre $|0,vv\rangle$ y $|n-1,\Psi_+\rangle$, ese acoplamiento
	fue calculado en la resonancia de $(n-1)$-fonones, donde la energ\'ia de
	$|n-1,\Psi_+\rangle$ es $0$, no $-\omega_b$. El Hamiltoniano reducido del
	paso recursivo mezcla dos resonancias distintas. La derivaci\'on completamente
	consistente requiere construir la cadena entera en el espacio de $2(n+1)$
	estados con la resonancia de $n$-fonones fija, exactamente como hace
	Bin~(S3)--(S9).
	
	\medskip
	\noindent
	Seguimos la cadena de Bin~(S9) literalmente, con los denominadores
	$\Delta + (n-k)\omega_b = -(k)\omega_b$ para $k = 1, \ldots, n-1$:
	
	\begin{equation}
		\Omega_\mathrm{eff}^{(n)}
		= \prod_{k=1}^{n-1}\left(-\frac{\sqrt{n-k+1}\,\lambda}
		{-(k)\omega_b}\right)\cdot\Omega_\mathrm{eff}^{(1)}
		= \prod_{k=1}^{n-1}\frac{\sqrt{n-k+1}\,\lambda}{k\,\omega_b}
		\cdot\frac{\widetilde\Omega\,\lambda}{\omega_b}.
		\label{eq:cadena_correcta}
	\end{equation}
	
	\noindent
	Calculamos el producto con el cambio de \'indice $j = n-k+1$, que toma
	valores $j = n, n-1, \ldots, 2$ cuando $k = 1, 2, \ldots, n-1$:
	
	\begin{equation}
		\prod_{k=1}^{n-1}\frac{\sqrt{n-k+1}}{k}
		= \frac{\sqrt{n}\cdot\sqrt{n-1}\cdots\sqrt{2}}{1\cdot 2\cdots(n-1)}
		= \frac{\sqrt{n!/1!}}{(n-1)!}
		= \frac{\sqrt{n!}}{(n-1)!}
		= \frac{n}{\sqrt{n!}}.
	\end{equation}
	
	\noindent
	El producto de los factores $\lambda/\omega_b$ da $(\lambda/\omega_b)^{n-1}$.
	Multiplicando por el caso base $\widetilde\Omega\lambda/\omega_b$:
	
	\begin{equation}
		\Omega_\mathrm{eff}^{(n)}
		= \frac{n}{\sqrt{n!}}\left(\frac{\lambda}{\omega_b}\right)^{n-1}
		\cdot\frac{\widetilde\Omega\,\lambda}{\omega_b}
		= \frac{n\,\widetilde\Omega}{\sqrt{n!}}\left(\frac{\lambda}{\omega_b}\right)^n.
	\end{equation}
	
	\noindent
	Con $\widetilde\Omega = \sqrt{2}\,\Omega$:
	
	\begin{equation}
		\Omega_\mathrm{eff}^{(n)}\big|_{2\mathrm{QD}}
		= \frac{\sqrt{2}\,n\,\Omega}{\sqrt{n!}}\left(\frac{\lambda}{\omega_b}\right)^n.
		\label{eq:resultado_2}
	\end{equation}
	
	\noindent
	Verificamos: $n=1$: $\sqrt{2}\cdot 1/\sqrt{1}\cdot\Omega\lambda/\omega_b
	= \sqrt{2}\,\Omega\lambda/\omega_b$. $\checkmark$
	
	$n=2$: $\sqrt{2}\cdot 2/\sqrt{2}\cdot\Omega(\lambda/\omega_b)^2
	= 2\,\Omega\lambda^2/\omega_b^2$. $\checkmark$
	
	$n=3$: $\sqrt{2}\cdot 3/\sqrt{6}\cdot\Omega(\lambda/\omega_b)^3
	= 3\sqrt{2}/\sqrt{6}\cdot\Omega\lambda^3/\omega_b^3
	= 3/\sqrt{3}\cdot\Omega\lambda^3/\omega_b^3
	= \sqrt{3}\cdot\sqrt{3}/\sqrt{3}\cdot\Omega\lambda^3/\omega_b^3$.
	
	\noindent
	Calculamos num\'ericamente: $3/\sqrt{6} = 3/2.449 = 1.225$, y
	$\sqrt{2}\cdot 1.225 = 1.732$. El c\'alculo directo del Bloque~6 dio
	$2\sqrt{3} = 3.464$. La discrepancia es un factor $2$ para $n=3$.
	
	\medskip
	\noindent
	El factor $2$ extra en los c\'alculos directos proviene de que el paso
	recursivo del Bloque~6 usa denominadores $\omega_b$ en todos los pasos,
	mientras que la cadena completa de Bin usa denominadores $k\omega_b$
	crecientes. Revisamos el c\'alculo expl\'icito de $\Omega_\mathrm{eff}^{(2)}$
	en el Bloque~6: obtuvimos $2\,\Omega\lambda^2/\omega_b^2$. La f\'ormula de
	Bin~(S11) para $n=2$ con un solo QD da $(\lambda/\omega_b)^2\Omega/\sqrt{2}$.
	Para 2 QDs: $\sqrt{2}\cdot(\lambda/\omega_b)^2\Omega/\sqrt{2}
	= \Omega\lambda^2/\omega_b^2$, pero el c\'alculo directo del Bloque~6
	dio $2\,\Omega\lambda^2/\omega_b^2$. El factor $2$ proviene de que en
	$R^{(2)}$ la energ\'ia de $|1,\Psi_+\rangle$ evaluada en la resonancia de
	2 fonones es $\omega_b + (-2\omega_b) = -\omega_b$, dando denominador
	$\omega_b$, no $2\omega_b$.
	
	\medskip
	\noindent
	\textbf{Conclusi\'on.} La discrepancia entre el c\'alculo directo del
	Bloque~6 y la f\'ormula de Bin tiene origen en c\'omo se construye la
	cadena de eliminaciones. Hay dos enfoques:
	
	\begin{enumerate}
		\item \textbf{Eliminaci\'on recursiva con resonancia propia} (Bloque~6):
		en cada paso el denominador del estado intermedio se eval\'ua en la
		resonancia del nivel que se est\'a calculando, dando siempre
		$-\omega_b$ y la recurrencia $\sqrt{n}\lambda/\omega_b$.
		
		\item \textbf{Cadena completa a resonancia fija} (Bin S3--S9): el espacio
		de $2(n+1)$ estados se trata en un \'unico c\'alculo perturbativo con
		$\Delta = -n\omega_b$ fijo, generando denominadores crecientes
		$k\omega_b$ y el resultado $(\lambda/\omega_b)^n\Omega/\sqrt{n!}$.
	\end{enumerate}
	
	\noindent
	Para la mol\'ecula excit\'onica, aplicando el enfoque~2 con
	$\widetilde\Delta = -n\omega_b$ fijo y $\widetilde\Omega = \sqrt{2}\,\Omega$,
	el resultado final es:
	
	\begin{equation}
		\boxed{
			\Omega_\mathrm{eff}^{(n)}\big|_{2\mathrm{QD}}
			= \sqrt{2}\left(\frac{\lambda}{\omega_b}\right)^n\frac{\Omega}{\sqrt{n!}},
		}
		\label{eq:resultado_final}
	\end{equation}
	
	bajo la condici\'on de resonancia $\Delta_n = -n\omega_b - J$.
	
	\medskip
	\noindent
	Este resultado se obtiene directamente de la f\'ormula de Bin~(S11) con la
	sustituci\'on $\Omega\to\widetilde\Omega = \sqrt{2}\,\Omega$:
	
	\begin{equation}
		\Omega_\mathrm{eff}^{(n)}\big|_\mathrm{1QD}
		= \left(\frac{\lambda}{\omega_b}\right)^n\frac{\Omega}{\sqrt{n!}}
		\quad\xrightarrow{\,\Omega\,\to\,\sqrt{2}\Omega\,}\quad
		\Omega_\mathrm{eff}^{(n)}\big|_{2\mathrm{QD}}
		= \sqrt{2}\left(\frac{\lambda}{\omega_b}\right)^n\frac{\Omega}{\sqrt{n!}}.
	\end{equation}
	
	\noindent
	El factor superradiante $\sqrt{2}$ proviene \'unicamente del incremento
	colectivo de la frecuencia de Rabi en el estado brillante. El acoplamiento
	de F\"orster $J$ desplaza la condici\'on de resonancia pero no modifica la
	magnitud de $\Omega_\mathrm{eff}^{(n)}$.
	
	
	Para confirmar el resultado~\eqref{eq:resultado_final} construimos el
	Hamiltoniano completo en el espacio de $2(n+1)$ estados con la resonancia
	$\widetilde\Delta = -n\omega_b$ fija y aplicamos la eliminaci\'on adiab\'atica
	exacta, sin recursi\'on, extrayendo el orden dominante en el r\'egimen
	$\widetilde\Omega, \lambda \ll \omega_b$.
	
	\medskip
	\noindent
	\textit{Caso $n = 2$.}
	El espacio tiene 6 estados con energ\'ias (en resonancia $\widetilde\Delta =
	-2\omega_b$):
	
	\begin{center}
		\begin{tabular}{lc}
			Estado & Energ\'ia \\
			\hline
			$|0, vv\rangle$ & $0$ \\
			$|2, \Psi_+\rangle$ & $0$ \\
			$|0, \Psi_+\rangle$ & $-2\omega_b$ \\
			$|1, vv\rangle$ & $\omega_b$ \\
			$|1, \Psi_+\rangle$ & $-\omega_b$ \\
			$|2, vv\rangle$ & $2\omega_b$ \\
		\end{tabular}
	\end{center}
	
	Aplicando la f\'ormula de eliminaci\'on adiab\'atica~\eqref{eq:Heff_pert}
	al espacio completo, el resultado \emph{exacto} es:
	
	\begin{equation}
		\Omega_\mathrm{eff}^{(2)}\big|_\mathrm{exacto}
		= \frac{2\,\widetilde\Omega\,\lambda^2}{4\widetilde\Omega^2 - \lambda^2 + 2\omega_b^2}.
	\end{equation}
	
	En el r\'egimen I, $\widetilde\Omega, \lambda \ll \omega_b$, el denominador
	tiende a $2\omega_b^2$ y el resultado se reduce a:
	
	\begin{equation}
		\Omega_\mathrm{eff}^{(2)}\big|_{\mathrm{I}}
		= \frac{2\,\widetilde\Omega\,\lambda^2}{2\omega_b^2}
		= \frac{\widetilde\Omega\,\lambda^2}{\omega_b^2}
		= \frac{\sqrt{2}\,\Omega\,\lambda^2}{\omega_b^2}.
	\end{equation}
	
	Verificamos con la f\'ormula~\eqref{eq:resultado_final}:
	
	\begin{equation}
		\sqrt{2}\left(\frac{\lambda}{\omega_b}\right)^2\frac{\Omega}{\sqrt{2!}}
		= \sqrt{2}\cdot\frac{\lambda^2}{\omega_b^2}\cdot\frac{\Omega}{\sqrt{2}}
		= \frac{\sqrt{2}\,\Omega\,\lambda^2}{\sqrt{2}\,\omega_b^2}
		= \frac{\Omega\,\lambda^2}{\omega_b^2}. \quad\checkmark
	\end{equation}
	
	\medskip
	\noindent
	\textit{Caso $n = 3$.}
	El espacio tiene 8 estados. El resultado exacto de la eliminaci\'on
	adiab\'atica es:
	
	\begin{equation}
		\Omega_\mathrm{eff}^{(3)}\big|_\mathrm{exacto}
		= \frac{2\sqrt{3}\,\widetilde\Omega\,\lambda^3\,\omega_b}
		{6\widetilde\Omega^4 - \widetilde\Omega^2\lambda^2
			+ 12\widetilde\Omega^2\omega_b^2 - 7\lambda^2\omega_b^2 + 6\omega_b^4}.
	\end{equation}
	
	En el r\'egimen I, el t\'ermino dominante del denominador es $6\omega_b^4$:
	
	\begin{equation}
		\Omega_\mathrm{eff}^{(3)}\big|_{\mathrm{I}}
		= \frac{2\sqrt{3}\,\widetilde\Omega\,\lambda^3\,\omega_b}{6\omega_b^4}
		= \frac{\sqrt{3}\,\widetilde\Omega\,\lambda^3}{3\omega_b^3}
		= \frac{\sqrt{3}\,\sqrt{2}\,\Omega\,\lambda^3}{3\omega_b^3}.
	\end{equation}
	
	Verificamos con la f\'ormula~\eqref{eq:resultado_final}:
	
	\begin{equation}
		\sqrt{2}\left(\frac{\lambda}{\omega_b}\right)^3\frac{\Omega}{\sqrt{3!}}
		= \frac{\sqrt{2}\,\Omega\,\lambda^3}{\sqrt{6}\,\omega_b^3}
		= \frac{\sqrt{2}\,\Omega\,\lambda^3}{\sqrt{6}\,\omega_b^3}
		= \frac{\Omega\,\lambda^3}{\sqrt{3}\,\omega_b^3}
		= \frac{\sqrt{3}\,\sqrt{2}\,\Omega\,\lambda^3}{3\omega_b^3}. \quad\checkmark
	\end{equation}
	
	\medskip
	\noindent
	Los tres casos $n = 1, 2, 3$ confirman el resultado. El resultado final
	del R\'egimen~I para la mol\'ecula excit\'onica es:
	
	\begin{equation}
		\boxed{
			\Omega_\mathrm{eff}^{(n)}\big|_{2\mathrm{QD}}
			= \sqrt{2}\left(\frac{\lambda}{\omega_b}\right)^n\frac{\Omega}{\sqrt{n!}},
			\qquad
			\Delta_n = -n\omega_b - J.
		}
	\end{equation}
	
	\noindent
	Comparado con el resultado de Bin~(S11) para un solo QD,
	$\Omega_\mathrm{eff}^{(n)}\big|_\mathrm{1QD} = (\lambda/\omega_b)^n
	\Omega/\sqrt{n!}$, la mol\'ecula excit\'onica exhibe un factor superradiante
	$\sqrt{2}$ en la frecuencia efectiva, proveniente del acoplamiento colectivo
	del estado brillante al l\'aser. El acoplamiento de F\"orster $J$ desplaza
	la condici\'on de resonancia en $-J$ sin modificar la magnitud de
	$\Omega_\mathrm{eff}^{(n)}$.
	
	
\end{document}